%Standard packages
\documentclass{article}
\usepackage{amsmath}
\usepackage{mathtools}
\usepackage{amsfonts}
\usepackage{amssymb}
\usepackage{amsthm}
\usepackage{fancyhdr}
\usepackage{float}
\usepackage{epigraph}
\usepackage{caption}
\usepackage{esint}

%Page formatting
\lhead{Eric Du}
\chead{Project Ideas}
\rhead{\today}
\pagestyle{fancy}
\cfoot{\thepage}
\title{Physics 5BL Capstone Project Ideas}
\author{Eric Du}
\date{\today}

%.sty file handling
\usepackage[sexy]{evan}
\usepackage{tcolorbox}
\usepackage{xcolor}
\renewcommand{\labelitemi}{\textendash}
\renewcommand{\abstractname}{}
\theoremstyle{definition}
\newtheorem*{solution}{\color{blue}Solution}
\numberwithin{equation}{section}
\numberwithin{definition}{section}

%Paragraph Formatting
\setlength{\epigraphwidth}{148pt}
\setlength{\parindent}{0pt}
\linespread{1.3}
\allowdisplaybreaks

%TikZ special settings
\usepackage{circuitikz}
\usetikzlibrary{shapes.geometric}
\usetikzlibrary{decorations.markings}


\begin{document}

\maketitle

\section{Where does Hooke's Law fail?}

Ever since that one day when we were in lab stretching the spring as far as possible, I was wondering how well does Hooke's law hold for springs. Or for instance, with harmonic oscillators, how large does the amplitude have to get (relative to the spring constant, for instance), do I have to release the system so that the spring no longer follows smiple harmonic motion? 

I think this is an interesting question that is likely to produce interesting results, since it essentially tests the limits of our current, simplified model for simple harmonic motion.

\section{Harmonic Oscillators with large amplitudes}

In lab we've previously looked at harmonic oscillators with pendulums, but we've only restricted them to small oscillations, so we may use the approximation $\sin \theta \approx \theta$. However, we would like to investigate the behaviour of larger periods of oscillation, and investigate the dynamics of the system when such an approxmiation cannot be made.

There is already enough research out there for this type of oscillator, and the system is already essentially solved, our investigations would be aimed at trying to experimentally verify the theory.

\subsection*{Expected Values}

There are already existing theories concerning the period of pendulums given any initial angle $\theta_0$, and they're generally defined by an infinite series of terms: 

\[ T = 2\pi \sqrt{\frac{L}{g}}\left[\sum_{n = 0}^\infty \left(\frac{(2n)!}{2^{2n}(n!)^2}\right)^2 \sin^{2n}\left(\frac{\theta_0}{2}\right)\right]\]

\section{Investigating Faraday's law}

Faraday's law is something that we've learned in in 5B, that we can probably build on in this class. 


\section{Using a spring as an inductor in a series LC circuit}



\end{document}