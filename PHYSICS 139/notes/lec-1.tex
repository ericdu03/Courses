\section{Introduction}
\begin{itemize}
	\item Let's go back to 1974, e
	\item No classical theory permits this gradual energy loss, except in General Relativity!
	\item Speaking of General relativity, one thing it predicted was the present of gravitational waves (GW), and 
		this was where the lost energy was going. Specifically, we can calculate its power:
		\[
		P = -\frac{2}{5}\frac{G^{4}M^{5}}{R^{5}c^{5}}
		\] 
	\item We commonly think of \( G = 6.67 \times 10^{-11}\), but later in the course we're going to work in units 
		where \( G = 1 \), to simplify things. 
	\item In electrodynamics, an charge \( q \) that experiences an acceleration also emits electromagnetic waves. This
		is called synchrotron radiation:
		\[
			P = -\frac{2}{3} \left( \frac{q^2}{4\pi \epsilon_0} \right) \frac{a^2}{c^3}
		\] 
		To get a sense of the scale of this power, the amount that our solar system is losing due to the sun and 
		Jupiter is around 200 W. But Hudson and Taylor found a power of \( P = -7 \cdot 10^{34} \) W!
	\item In 1983, they measured this, in 1993 they won the Nobel prize for their indirect detection of 
		gravitational waves. In 2015, LIGO detected these waves directly, and found a power 
		\( P = 3.6 \times 10^{49} \) W. In 2017, they won the Nobel prize for this discovery. 
\end{itemize}
\subsection{Why General Relativity?}
\begin{itemize}
	\item In Newton's gravity, we have the equation \( \vec F_i = \dv[2]{\vec r_i}{t} \), and for universal 
		gravitation, we had:
		\[
		\vec F_i = \sum_{j\neq i}\frac{G m_i m_j}{|\vec r_i - \vec r_j|^3}(\vec r_j - \vec r_i)
		\] 
		\comment{Note that it's only formatted like this so that we can talk about vectors.}
	\item One problem with this interpretation is that things are instantaneous: this is an issue because objects
		don't react instantaneously to changes (information 
		can't travel faster than the speed of light), which Newton's equations seem to imply. 

		We can say the same about Coulomb's law: and the solution there was to replace the notion of a force 
		with \textit{fields}. Now, the force can be written as: 
		\[
		\vec F_i = q(\vec E + \frac{1}{c}\vec v \times \vec B)
		\] 
		\comment{We'll use Gaussian units, mainly because \( \vec E \) and \( \vec B \) now have the 
		same units. With this,
		\begin{align*}
			\vec \div E &= 4 \pi \rho \\
			\vec \div B &=  0 \\
			\vec \curl E &= -\frac{1}{c}\pdv{\vec B}{t}\\
			\vec \curl B &= \frac{1}{c}\pdv{\vec E}{t} + \frac{4\pi}{c}\vec J 
		\end{align*}		}
		So the question then becomes: why didn't we do this for Gravitation? Well, this is a thing, but it's 
		only an approximation. 
	\item Let's talk about energy conservation: in E\&M, the energy is written as:
		\[
		\mathcal E = \frac{1}{8\pi}\int (\vec E^2 + \vec B ^2) dv + \sum_i k_i
		\] 
		What happens when we change the sign on everything to accomodate for gravitation? Then, we introduce instability 
		into the system! 
\end{itemize}
