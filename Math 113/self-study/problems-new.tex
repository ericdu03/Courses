\documentclass[10pt]{article}
\usepackage{../../local}
\urlstyle{same}

\newcommand{\classcode}{Math 113}
\newcommand{\classname}{Selected Problems for Abstract Algebra}
\renewcommand{\maketitle}{%
\hrule height4pt
\large{\phantom{a} \hfill \classcode}
\newline
\Large{Eric Du}\Large{\hfill \classname \hfill} \large{\today}
\hrule height4pt \vskip .7em
\small{Header styling inspired by CS 70: \url{https://www.eecs70.org/}}
\normalsize
}
\linespread{1.2}
\newenvironment{problem}{\textbf{Problem:}}{}
\DeclareMathOperator{\im}{im}

\begin{document}

\section*{Introduction}
This document contains solutions to selected problems from Artin's abstract 
algebra book. 

\section{Groups}

\subsection{Laws of Composition}
\begin{problem}
	Let \( \N \) denote the set \( \{ 1, 2, 3, \dots\}  \) of natural nubmers, 
	and let \( s: \N \to \N \) be the \textit{shift} map, defined by \( s(n) = n +1 \). 
	Prove that \( s \) has no right inverse, but that it has 
	infinitely many left inverses. 
\end{problem}

\subsection{Groups and Subgroups}
\begin{problem}
	Let \( x, y, z \) and \( w \) be elements of a group \( G \). 
	\begin{enumerate}[label=\alph*)]
		\item Solve for \( y \), given that \( xyz^{-1}w = 1 \).
			
			\begin{solution}
				\( x, y, z, w \) are all elements of \( G \), so therefore an inverse 
				exists for all of them. Given that \( xyz^{-1} w = 1 \), then we can
				write \( y = x^{-1} w^{-1}z \) by left and right 
				multiplying out inverses.
			\end{solution}
		\item Suppose that \( xyz = 1 \). Does it follow that 
			\( yzx = 1 \)? Does it follow that \( yxz = 1 \)?

			\begin{solution}
				The first of the two equations follows, since we conclude that 
				\( x^{-1} = yz \), and therefore \( yzx = (yz)x = x^{-1} x = 1 \). 
				The second equation does not follow, since although we
				 conclude that \( z^{-1} = xy \) it is not necessarily 
				 true that \( xy = yx \). 
			\end{solution}
	\end{enumerate}
\end{problem}


\begin{problem}
	In the definition of a subgroup, the identity element in \( H \) is required 
	to be the identity of \( G \). One might require only that \( H \) have an identity 
	element, not that it need be the same identity in \( G \). Show that 
	if \( H \) has an identity at all, then it is the identity in \( G \). 
	Show that the analogous statement is true for inverses. 
\end{problem}

\begin{solution}
	The fact about identities is very simple: \( G \) has a single identity 
	element, and since \( H \) is a subgroup of \( G \), then \( H \subset G \), 
	hence the identity element of \( H \) must equal the identity element of \( G \).  

	To be a bit more rigorous, suppose \( 1_H \neq 1_G\). Then, \( 1_H \) is defined 
	such that for all \( h \in H \), \( h 1_H = h \), and since \( h \in G\), then 
	we also have \( h 1_G = h \), so therefore \( h 1_H = h 1_G \), which can 
	only be true if \( 1_H = 1_G \). 

	Now consider an element \( h \in H \). Since \( H \) is a subgroup, then 
	\( h^{-1} \in H \), and \( h^{-1} \) is defined to be the element 
	that \( hh^{-1} = 1 \), which also holds in \( G \). Therefore, 
	the inverse of \( h \) is the same inverse of \( h \) in \( G \).
\end{solution}

\subsection{Subgroups of the Additive Group of Integers}
\begin{problem}
	Prove that if \( a \) and \( b \) are positive integers whose sum is a prime 
	\( p \), their greatest common divisor is 1. 
\end{problem}

\begin{solution}
	We have the relation \( a + b = p \). Applying modulus, we get:
	\[
		a \equiv p \pmod b
	\] 
	Since \( p \) is prime, then \( \gcd(p, b) = 1 \) so therefore \( p^{-1} \)
	exists, and we can write \( ap ^{-1} \equiv 1 \pmod b \). This then 
	implies that \( a^{-1} \equiv p^{-1} \pmod b \), and since a modular inverse 
	exists iff \( \gcd(a, b) = 1 \), we conclude that \( \gcd(a, b) = 1 \) from 
	here. 
\end{solution}

\subsection{Cyclic groups}
\begin{problem}
	An \( n \) th root of unity is a complex number \( z \) such that \( z^{n} = 1 \). 
	\begin{enumerate}[label=\alph*)]
		\item Prove that the \( n \)-th roots of unity form a cyclic subgroup 
			\( \C^{\times} \) of order \( n \). 

			\begin{solution}
				By the fundamental theorem of algebra, we know that 
				the equation \( z^{n} = 1 \) has \( n \) values. Specifically, 
				we can write the \( i \)-th root of unity as:
				\[
				z_i = e^{i\frac{2\pi}{n}}
				\] 
				It's clear that each one of these satisfy \( z_i^{n} = 1 \), and all 
				\( n \) values are distinct, so therefore this set has \( n \) 
				elements.  

				For the properties of a group, starting with closedness:
				\[
				z_iz_j = e^{i \frac{2\pi}{n}} e^{j \frac{2\pi}{n}} 
				= e^{(i + j) \frac{2\pi}{n}}
				\] 
				which is another element of the group (if \( i, j \) exceed \( n \), 
				it is possible to factor out \( e^{2 \pi} = 1\) out). 

				The identity in \( C^{\times} = 1 \), which also exists 
				in this subgroup since \( 1^{n} = 1 \). 

				As for inverses, for any element \( z_i \), we have:
				\[
				z_i z_{n - i} = e^{i \frac{2 \pi}{n}}e^{(n - i) \frac{2 \pi}{n}}
				= e^{n \frac{2 \pi}{n}} = 1
				\] 
				so the inverse exists. 
			\end{solution}
		\item Determine the product of all the \( n \)-th roots of unity. 

			\begin{solution}
				Since this is a subgroup, each element can be paired up with its 
				inverse, except the identity (whose inverse is itself). Since 
				multiplication is commutative, then the entire product collapses
				and we get 1. 
			\end{solution}
	\end{enumerate}
\end{problem}

\begin{problem}
	Let \( x \) and \( y \) be elements of a group \( G \). Assume that each of the 
	elements \( x, y \) and \( xy \) has order 2. Prove that the set 
	\( H = \{1, x, y, xy\}  \) is a subgroup of \( G \), and that it 
	has order 4.
\end{problem}

\begin{solution}
	We first show closedness: we know that each element has order 2, so \( x^2 = 1 \), 
	\( y^2 = 1 \), \( (xy)^2 = 1 \), all of which are in \( H \).  

	\question{Since \( x, y \in H\), but \( yx \not \in H \)? How is this a subgroup?}
\end{solution}

\subsection{Homomorphisms}

\begin{problem}
	Let \( \varphi: G \to G' \) be a group homomorphism. 
	Prove that if \( G \) is cyclic, 
	then \( G' \) is cyclic, and if \( G \) is abelian, then \( G' \) is 
	abelian. 
\end{problem}

\begin{solution}
	Let \( x \) be the generator of \( G \), so \( \varphi(x) \in G' \) and 
	\( \varphi(x^{-1}) = \varphi(x)^{-1} \in G' \). To prove that 
	\( G' \) is cyclic, we have \( \varphi(x^{m}) = \varphi(x)^{m} \) (this 
	follows inductively from the property that 
	\( \varphi(ab) = \varphi(a) \varphi(b) \)), where \( m \) 
	is any integer. Therefore, \( G' \) is generated by the element \( \varphi(x) \), 
	so therefore \( G' \) is cyclic. 

	Now, if \( G \) is abelian, it means that \( ab = ba \) for all \( a, b \in G \). 
	Therefore, we have \( \varphi(ab) = \varphi(a) \varphi(b) \) and 
	\( \varphi(ba) = \varphi(b) \varphi(a) \), so we see that \( \varphi(a) \varphi(b)
	= \varphi(b) \varphi(a)\), therefore \( G' \) is abelian. 
\end{solution}

\begin{problem}
	Let \( f: \R^{+} \to \C^{\times} \) be the map \( f(x) = e^{ix} \). Prove that 
	\( f \) is a homomorphism, and determine its kernel and image. 
\end{problem}

\begin{solution}
	To prove that \( f \) is a homomorphism, we prove that \( f(ab) = f(a) f(b) \) 
	for \( a, b \in \R^{+} \). Therefore, we're being asked to prove that
	\[
	e^{i(a + b)} = e^{ia} e^{ib}
	\] 
	this is obviously true from the laws of exponents, so \( f \) is indeed a 
	homomorphism. 

	The image set is the set \( e^{i \theta} \), which is a circle of radius 1 in the 
	complex plane. The kernel is the set of elements in \( \R^{+} \) such that 
	\( f(x) = 1 \), so this would correspond to the elements \( 2\pi n \), where 
	\( n  \) is an integer, since \( e^{i2 \pi n} = (e^{i2\pi})^{n} = 1^{n} = 1 \). 
\end{solution}

\begin{problem}
	\begin{enumerate}[label=\alph*)]
		\item Let \( G \) be a cyclic group of order 6. How many of its elements 
			generate \( G \)? Answer the same question for cyclic groups of orders 
			5 and 8. 

			\begin{solution}
				\( G \) is a cyclic group of order 6, so if \( x \) generates
				 \( G \), then we know that \( x^{6} = 1 \). Now, we claim that 
				 if \( x^{n} \) generates \( G \), then \( \gcd(n, 6) = 1 \). This 
				 is required since we need there to exist a value of \( k \) such that 
				 for all \( a \in \{0, \dots, 5\}  \), we have
				 \( kn \equiv a \pmod 6 \), which is only achievable if 
				 \( n \) has an inverse.\footnote{
					 This is true because \( k \) can be any number, and \( a \) can be 
					 any number, so we can choose \( k, a \) to share factors with 6,
					 at which point the only way a solution exists is 
				 \( k \equiv a n^{-1} \pmod 6 \).}
				 This last condition is true if and only 
				 if \( \gcd(n, 6) = 1 \), so therefore this is a requirement. 
				
				 Then, this means that all powers coprime to 6 will generate \( G \), 
				 so this would be the elements \( x, x^{5} \).  

				 Since 5 is prime, then it is coprime to all values 
				 \( \{1, \dots, 4\}  \), so therefore all elements except the identity
				 will generate a cyclic group of order 5. 

				 For order 8, the elements are \( \{x, x^3, x^{5}, x^{7}\}  \). 
			\end{solution}
		\item Describe the number of elements that generate a cyclic group of 
			arbitrary order \( n \). 

			\begin{solution}
				Building off the previous part, the number of elements is 
				given by \( \phi(n) \), where \( \phi \) is Euler's totient function. 
			\end{solution}
	\end{enumerate}
\end{problem}


\begin{problem}
	Prove that the \( n \times n \) matries that have the block form 
	\( M = \begin{bmatrix} A & B \\ 0 & D  \end{bmatrix} \), with 
	\( A \) in \( GL_r(\R) \) and \( D \) in \( GL_{n - r}(\R) \), form a subgroup 
	\( H \) of \( GL_n(\R) \), and that the map \( H \to GL_r(\R)\) that 
	sends \( M \rightsquigarrow  A \) is a homomorphism. What is its kernel?
\end{problem}

\begin{solution}
	First, we check that \( M \in GL_n(\R) \). This is easily shown, by first 
	noticing that we can row reduce \( D \) to the identity in the bottom right block, 
	then use it to reduce \( B \) to the zero matrix. Then, we reduce \( A \) to its 
	identity. Therefore, \( M \) is invertible, so \( M \in GL_n(\R) \). 

	Next, given \( M_1 = \begin{bmatrix} A_1 & B_1 \\ 0 & D_1 \end{bmatrix}  \) 
	and \( M_2 = \begin{bmatrix} A_2 & B_2\\ 0 & D_2\end{bmatrix}  \), 
	we compute \( M_1M_2 \):
	\[
		M_1M_2 = \begin{bmatrix} A_1A_2 & A_1 B_2 + B_1D_2\\0 & D_1D_2 \end{bmatrix}
		\in GL_n(\R)
	\] 
	it exists in \( GL_n(\R) \) by the same argument we did for \( M \). 
	Then, we can also see that \( \varphi(M_1M_2) = A_1A_2 = 
	\varphi(M_1) \varphi(M_2) \), so \( \varphi \) is indeed a homomorphism. The 
	kernel is the set of elements in \( H \) that are sent to the 
	identity -- this is simply the matrices where \( A = I \).   
\end{solution}

\subsection{Isomorphisms}
\begin{problem}
	Let \( G' \) be the group of real matrices of the form 
	\( \begin{bmatrix} 1 & x \\  & 1 \end{bmatrix}  \). Is the map 
	\( \R^{+} \to G' \) that sends \( x \) to this matrix an isomorphism?
\end{problem}

\begin{solution}
	An isomorphism is a bijective map between \( G \) and \( G' \). This map 
	is indeed bijective, since the unique number \( x \) maps to a unique matrix. We 
	now check whether this map \( \varphi \) is a homomorphism:
	\[
		\varphi(x_1) \varphi(x_2) = \begin{bmatrix} 1 & x_1 \\ & 1 \end{bmatrix} 
		\begin{bmatrix} 1 & x_2 \\ & 1 \end{bmatrix}  = 
		\begin{bmatrix} 1 & x_1 + x_2\\ & 1 \end{bmatrix} 
	\] 
	so we've confirmed that \( \varphi(x_1x_2) = \varphi(x_1) \varphi(x_2) \), 
	confirming that \( \varphi \) is an isomorphism. 
\end{solution}

\begin{problem}
	Prove that in a group, the products \( ab \) and \( ba \) are conjugate elements. 
\end{problem}

\begin{solution}
	Our goal is just to show that there exists some \( g \in G \) such that 
	\( g(ab) g^{-1} = ba \), which can be done by letting \( g = a^{-1} \). This 
	is guaranteed to exist in \( G \) since \( a \in G \). 
\end{solution}

\begin{problem}
	Let \( H \) be a subgroup of \( G \), and let \( g \) be a fixed element of \( G \).
	The \textit{conjugate subgroup} \( gHg^{-1} \) is defined to be the set of all 
	conjugates \( ghg^{-1} \), with \( h \) in \( H \). Prove that \( gHg^{-1} \) is a 
	subgroup of \( G \). 
\end{problem}

\begin{solution}
	Firstly, \( ghg^{-1} \in G \) since all elements are in \( G \), and \( G \) is 
	closed under composition. Now, we show closure in \( H \). Let 
	\( gh_1g^{-1} \) and \( gh_2g^{-1} \) be two elements in \( H \). Then:
	\[
		(gh_1g^{-1})(gh_2g^{-1}) = gh_1(g^{-1}g)h_2g^{-1} = g(h_1h_2)g^{-1} 
	\] 
	Since \( H \) is a subgroup, then  \( h_1h_2 \in H \), so closedness is guaranteed. 

	The identity is guaranteed by letting \( h = 1_H = 1_G\), so we have 
	\( g 1_G g^{-1} = 1_G\). 

	Now, let \( ghg^{-1} \in gHg^{-1} \). Then, the inverse is:
	\[
		(ghg^{-1})^{-1} = gh^{-1}g^{-1}
	\] 
	and \( h^{-1} \in H \), so the inverse is in \( gHg^{-1} \) as well. This concludes
	the proof. 
\end{solution}

\subsection{Equivalence Relations and Partitions}

\begin{problem}
	An equivalence relation on \( S \) is determined by the subset \( R \) of 
	the set \( S \times S \) consisting of those pairs \( (a, b) \) such that 
	\( a \sim b \). Write the axioms for an equivalence relation in 
	terms of the subset \( R \). 
\end{problem}

\begin{solution}
	The subset \( R \) consists of points \( (a,b) \) such that \( a \sim b \). 
	Transivity can be expressed as if \( (a, b) \in R \) and \( (b, c) \in R \) then 
	\( (a, c) \in R \), symmetry is \( (a, b) \in R \implies (b, a) \in R \), and 
	\( \forall a \ (a, a) \in R \). 
\end{solution}

\subsection{Cosets}
\begin{problem}
	Does every group whose order is a power of a prime \( p \) contain 
	an element of order \( p \)? 
\end{problem}

\begin{solution}
	No. If \( x \) is an element of order \( p \), then it implies that 
	the set \( \{1, x, \dots, x^{p - 1}\}  \) consists of only unique 
	elements, at which point the group is the cyclic group \( \left< x \right> \). 
	Not every group is cyclic, so therefore this is not a requirement. 
\end{solution}

\begin{problem}
	Does a group of order 35 contain an element of order 5? of order 7?
\end{problem}

\begin{solution}
	Due to Lagrange's theorem, the order of every element divides the order of the 
	group, meaning that every element has order 1, 5, 7, 35. If \( G \) contains 
	an element \( x \) of order 35, then \( x^{7}  \) has order 5 and 
	\( x^{7} \) has order 5. The identity is the only element with order 1. 

	Now, suppose \( G \) does not have any elements order 7. Then, since every 
	element is order 5, then \( G \) consists of \( n \) cyclic subgroups 
	each of order 5. Since they all share the identity, then each subgroup contains 
	4 unique elements.\footnote{This uses the principle that groups with prime 
		order are cyclic, so if they had any element other than the identity in common,
	they would be the same cyclic subgroup.}
	Combining all these subgroups, we get \( 35 = 4n + 1 \), but 
	there is no integer value of \( n \) that solves this, so therefore 
	this cannot exist. 

	A similar argument is made for 7, where we have \( 6n + 1 = 35 \), also with 
	no satisfying value of \( n \). Therefore, \( G \) must contain 
	an element of order 5 and 7. 
\end{solution}

\begin{problem}
	Let \( \varphi: G \to G' \) be a group homomorphism. Suppose that 
	\( |G| = 18, |G'| = 15 \), and that \( \varphi \) is not the trivial homomorphism.
	What is the order of the kernel?
\end{problem}

\begin{solution}
	We know that \( |\ker \varphi| \) divides \( |G| \), and also that \( |G| = 
	\left| \ker \varphi \right| \left| \im \varphi \right| \), and 
	\( \left| \im \varphi \right|  \) divides \( |G| \) and \( |G'| \). Since  
	\( |G'| = 15 \), then the only common factor between 18 and 15 is 3, so therefore 
	\( |\im \varphi| = 3 \). From this, we get \( |\ker \varphi| = 6 \). 
\end{solution}

\addtocounter{subsection}{1}
\subsection{The Correspondence Theorem}

\begin{problem}
	Let \( H \) and \( K \) be subgroups of a group \( G \). 
	\begin{enumerate}[label=\alph*)]
		\item Prove that the intersection \( xH \cap yK \) of two cosets 
			\( H \) and \( K \) is either empty or else is a coset of the subgroup 
			\( H \cap K \). 

			\begin{solution}
				If \( xH \cap yK \) is nonempty, then it contains some element \( z \). 
				By (2.8.5), we have that \( zH = xH \) and also \( zK = yK \) (this 
				is becuase \( z \) is in \( xH \) and also \( yK \)). So, 
				we can simplify our life by only considering WLOG 
				the case where \( x = y = z \). Then, we have \( z(H \cap K) 
				\subseteq zH \cap zK\), so therefore  \( zH \cap zK \) contains 
				at least one coset \( z(H \cap K) \). Also, 
				\( zh = zk \) for some \( h \in H \) and \( k \in K \), so the 
				only way this can be true is if \( h = k \in H \cap K \), 
				so \( zH \cap zK \subseteq z(H \cap K) \). Therefore, 
				\( zH \cap zK = z(H \cap K) \). 

				\question{This solution is heavily drawn from stackexchange, 
				make sure you understand it.}
			\end{solution}
		\item Prove that if \( H \) and \( K \) have finite index in \( G \) then 
			\( H\cap K \) also has finite index in \( G \). 

			\begin{solution}
				We know that \( H \cap K \subset H \) and \( H \cap K \subset K \), 
				so \( [G : H \cap K]  = [G : H][H : H \cap K]  \), and 
				also \( [G : H \cap K] = [G : K][K : H \cap K] \). We know 
				\( [G : H] \) and \( [G : K] \) are finite, so rearranging:
				\[
					[G : H] = \frac{[G : H \cap K]}{[H : H \cap K]} \quad 
					[G : K] = \frac{[G : H \cap K]}{[K : H \cap K]}
				\] 
				this implies that the right hand side must be finite, or in 
				other words \( [G : H \cap K] \) is finite, so \( H \cap K \) is finite
				in \( G \).

				\comment{I wonder if there's a way to make this proof slightly better}
			\end{solution}
	\end{enumerate}
\end{problem}

\begin{problem}
	With the notation of the Correspondence Theorem, let \( H \) and \( H' \) be 
	corresponding subgroups. Prove that \( [G : H] = [G':H'] \). 
\end{problem}

\begin{solution}
	The Correspondence theorem gives us a bijection between subgroups of \( H \) and 
	\( H' \), but more importantly it gives us the identity \( |H| = |H'| |K| \). 
	Further, we know from Lagrange's theorem that:
	\[
		[G : H] = \frac{|G|}{|H|} \quad [G' : H'] = \frac{|G'|}{|H'|}
	\] 
	substituting the identity to the right we have:
	\[
		[G' : H'] = \frac{|G'| |K|}{|H|}
	\] 
	Now, we claim that \( |G| = |G'| |K| \), but this is a direct result 
	of \( |G| = |\im \varphi| |\ker \varphi| \), so the desired result 
	immediately follows.
	\footnote{Another way to see this is that every element in \(g' \in  G' \) that has 
		a non-kernel pre-image \( g \in G \) is also mapped to \( g' \) by 
		\( gk \), where \( k \in K \), so for every element in \( G' \) there are 
	\( |K| \) kernel elements that are collapsed, hence the multiplication.}
\end{solution}

\addtocounter{subsection}{1}
\subsection{Quotient Groups}
\begin{problem}
	Show that if a subgroup \( H \) of a group \( G \) is not normal, there are 
	left cosets \( aH \) and \( bH \) whose product is not a coset. 
\end{problem}

\begin{solution}
	Since \( H \) is not normal, there exists some  \( g \in G \) and \( h \in H \) 
	such that \( ghg^{-1} \not \in H \). Also, \( (aH)(bH) \) is the set 
	of elements \( ahbh'\) for \( h, h' \in H \). 

	Now let \( a = g, b = g^{-1} \), our product set is \( (aH)(bH) = 
	ghg^{-1} h'\) for all \( h, h' \in H \). Now, let \( h_0 \in G \) be the element 
	such that \( gh_0g^{-1} \not \in H \), then we can let \( h' = 1 \), 
	implying that \( gh_0g^{-1} \in (gH)(g^{-1} H) \). Now, letting \( h = 1 \)
	leaves us with \( h'\), so \( H \subset (gH)(g^{-1}H) \).

	If \( (gH)(g^{-1} H) \) is a coset of \( H \) (since cosets partition \( G \) 
	and \( ab \in (aH)(bH) \)), then it has \( |H| \) elements, 
	since all cosets have the same order. Combining this with 
	\( H \subset (gH)(g^{-1} H) \) allows us to conclude that 
	\( H = (gH)(g^{-1} H) \), but then we get that 
	\( gh_0g^{-1} \in H \), which is a contradiction since \( H \) is not normal.  
\end{solution}

\begin{problem}
	In the general linear group \( GL_3(\R) \), consider the subsets
	\[
		H = \begin{bmatrix} 1 & * & * \\0 & 1 & * \\ 0 & 0 & 1  \end{bmatrix} , 
		\text{and } \ 
		K = \begin{bmatrix} 1 & 0 & * \\ 0 & 1 & 0\\ 0 & 0 & 1 \end{bmatrix} 
	\] 
	where \( * \) represents an arbitrary real number. Show that \( H \) is a 
	subgroup of \( GL_3 \), that \( K \) is a normal subgroup of \( H \), and 
	identify the quotient group \( H / K \). Determine the center of \( H \). 
\end{problem}

\begin{solution}
	\( H \) is the set of upper triangular matrices with entries 1 on the diagonal. We
	know that two upper triangular matrices multiplied together give an upper triangular
	matrix, so \( H \) is closed under composition. The identity element is in \( H \),
	where all non-diagonal elements are zero. The inverse is also upper triangular 
	(this is not hard but I really don't want to prove it), all of which combined 
	means that \( H \) is a subgroup. 

	Now, consider any element of \( H \), and an element \( h \in H \), 
	written as follows:
	\[
		h = \begin{bmatrix} 1 & x & y\\0 & 1 & z\\0& 0 & 1 \end{bmatrix} \quad
		h^{-1} = \begin{bmatrix} 1 & -x & -y + xz\\0 & 1 & -z\\0 & 0 & 1 \end{bmatrix} 
	\] 
	Then, doing the computation:
	\[
		hkh^{-1} = 	\begin{bmatrix} 1 & x & y\\0 & 1 & z\\0& 0 & 1 \end{bmatrix} 
		\begin{bmatrix} 1 & 0 & w\\ 0 & 1 & 0\\0& 0& 1 \end{bmatrix} 
			\begin{bmatrix} 1 & -x & -y + xz\\0 & 1 & -z\\0 & 0 & 1 \end{bmatrix} 
			= \begin{bmatrix} 1 & 0 & w\\0 & 1 & 0 \\ 0 & 0 & 1 \end{bmatrix} \in K
	\] 
	so therefore \( K \) is a normal subgroup of \( H \). The quotient group 
	\( H / K \) is the set of cosets of \( K \), so it's the set of matrices
	of the form:
	\[
		\begin{bmatrix} 1 & * & * \\ 0 & 1 & * \\ 0 & 0 & 1 \end{bmatrix} 
		\begin{bmatrix} 1 & 0 & *\\ 0 & 1 & 0\\0 & 0 & 1 \end{bmatrix} 
		= \begin{bmatrix} 1 & * & *\\ 0 & 1 & *\\ 0 & 0 & 1 \end{bmatrix} 
	\] 
	so \( H / K \) is actually just \( H \) itself. The center of \( H \) are the 
	matrices in \( GL_3(\R) \) that do commute with matrices in \( H \); since 
	matrices don't commute in general, there are two matrices that come to mind: the 
	identity and the zero matrix.  
\end{solution}


\begin{problem}
	Let \( G \) be the group of upper triangular real matrices 
	\( \begin{bmatrix} a & b \\ 0 & d \end{bmatrix}  \) with 
	\( a \) and \( d \) different from zero. For each of the following subsets, 
	determine whether or not  \( S \) is a subgroup, and whether or not 
	\( S \) is a normal subgroup. If \( S \) is a normal subgroup, identify the 
	quotient group \( G / S \). 
	\begin{enumerate}[label=\roman*)]
		\item \( S \) is the subset defined by \( b = 0 \). 
		\item \( S \) is the subset defined by \( d = 1 \).
		\item \( S \) is the subset defined by \( a = d \). 
	\end{enumerate}
\end{problem}
\end{document}



