\documentclass[10pt]{article}
\usepackage{../../local}
\urlstyle{same}

\newcommand{\classcode}{Math 113}
\newcommand{\classname}{Selected Problems for Abstract Algebra}
\renewcommand{\maketitle}{%
\hrule height4pt
\large{\phantom{a} \hfill \classcode}
\newline
\Large{Eric Du}\Large{\hfill \classname \hfill} \large{\today}
\hrule height4pt \vskip .7em
\small{Header styling inspired by CS 70: \url{https://www.eecs70.org/}}
\normalsize
}
\linespread{1.2}
\newenvironment{problem}{\textbf{Problem:}}{}

\begin{document}
	\maketitle
	\section{Introduction}
	A collection of selected problems from the abstract algebra books by Dummit \& Foote, Artin, and 
	Hungerford. 

	\section{Introduction to Groups}

	\subsection{Basic Axioms and Examples}
	\begin{problem}
		Determine which of the following binary operations are associative:
			\begin{enumerate}[label=\alph*)]
				\item the operation \( * \) on \( \Z \) defined by \( a * b = a - b \).

					\begin{solution}
						In all these problems, assosiativity means that \( (a * b) * c = a * (b * c) \). This is 
						not guaranteed by this operation, as \( (a - b) - c \neq a - (b - c) \), it's not hard 
						to find a counterexample.
					\end{solution}
				\item the operation \( * \) on \( \R \) defined by \( a * b = a + b + ab \).

					\begin{solution}
						So here we have \( (a * b) * c = (a + b + ab) * c = a + b + ab + c + c(a + b + ab) 
						= a + b + ab + c + ac + bc + abc\). 
						On the other hand, we have \( a * (b * c) = a * (b + c + bc) = a + b + c + bc + a(b + c + bc) 
						= a + b + c + bc + ab + ac + abc\). These two are in fact equal, so this operation 
						is associative. 
					\end{solution}
				\item the operation \( * \) on \( \Q \) defined by \( a * b = \frac{a + b}{5} \).

					\begin{solution}
						Here, we have \( (a * b) * c = \frac{a + b}{5} * c = \frac{\frac{a + b}{5} + c}{5} = 
						\frac{a + b + 5c}{25}\), and on the other side we have \( a * (b * c) = a * \frac{b + c}{5} 
						= \frac{5a + b + c}{25}\). These two are not equal, so this operation is not associtative.  
					\end{solution}
			\end{enumerate}
	\end{problem}


	\begin{problem}
		Prove for all \( n > 1 \) that \( \Z / n\Z \) is not a group under multiplication of residue classes. 
	\end{problem}

	\begin{solution}
		The requirements of a group are that the operation must be associative, there must exist an identity, and an 
		inverse must also exist. These conditions are satisfied, except for the residue class \( 0 \), for which 
		there exists no inverse. Therefore, \( \Z / n\Z  \) is not a group under multiplication. 
	\end{solution}

	\begin{problem}
		Prove that a finite group is abelian if and only if its group table is a symmetric matrix. 
	\end{problem}

	\begin{solution}
		A finite group is abelian if \( a * b = b* a \) for all \( a, b \in G\), and the group table is the 
		\( n \times n \) matrix such that the entry in \( (i, j) \) is the group element \( g_i g_j \). Therefore, 
		the group table must be a symmetric matrix, since we require \( g_i g_j= g_j g_i \). The if and only if is 
		easy from here. 
	\end{solution}

	\begin{problem}
		For an element \( x \) in \( G \) show that \( x \) and \( x^{-1} \) have the same order. 
	\end{problem}

	\begin{solution}
		The order of an element \( x \) is defined as the value of \( n \) such that \( x^{n} = 1 \). So, given 
		that the order of \( x \) is \( n \), now let's raise \( x^{-1} \) to the \( n \)-th power. Then, we 
		have:
		\[
			(x^{-1})^{n} = x^{-n} = (x^{n})^{-1} = 1^{-1} = 1
		\] 
		Now, we need to show that \( n \) is the smallest exponent we can raise \( x^{-1} \) by. Suppose there's 
		a smaller \( n' < n \) such that \( x^{-n'} = 1 \). Then, multiplying through by \( x^{n'} \), we get
		\( x^{n'} = x^{0} = 1 \), which is a contradiction since \( n \) is the order of \( x \). Therefore, \( n \)
		is indeed the order of \( x^{-1} \). 
	\end{solution}

	\begin{problem}
		Let \( G \) be a finite group and let \( x \) be an element of \( G \) of order \( n \). Prove that 
		if \( n \) is odd, then \( x = (x^2)^{k} \) for some \( k \). 
	\end{problem}

	\begin{solution}
		Since \( x \) has order \( n \), then we know that \( x^{n} = 1 \). Now, if \( n \) is odd, 
		we may write \( n = 2k + 1 \), then we may write \( x^{2k + 1} = 1 \), so 
		\( x^{2k + 1} \cdot x = x^{2(k + 2)} = (x^2)^{k + 2} = x \), which is of the desired form. 
	\end{solution}

	\begin{problem}
		If \( a \) and \( b \) are \textit{commuting} elements of \( G \), prove that \( (ab)^{n} = a^{n}b^{n} \) 
		for all \( n \in \Z \). [Do this by induction for positive \( n \) first.]
	\end{problem}

	\begin{solution}
		Since \( a \) and \( b \) are commuting elements, it means that \( a * b = b * a \), whatever the \( * \) 
		operation is. We first prove this by induction for positive \( n \), with a base case of \( n = 0 \):
		
		By definition, at \( n = 0 \) we have \( (ab)^{0} = a^{0}b^{0} = 1\), which is verified. Now, assume that 
		for some \( k \), we have \( (ab)^{k} = a^{k}b^{k} \), and we prove the claim for \( k + 1 \), where 
		we have \( (ab)^{k + 1} = (ab)^{k} * ab = a^{k}b^{k} * a * b\). Now, because \( a, b \) commute, then 
		we may write \( a^{k}* b^{k} * a * b = a^{k} * a * b^{k} * b = a^{k + 1}b^{k + 1}\), as desired. 

		Now we handle the negative integers, also starting at a base case of \( n = 0 \). Now for our 
		inductive hypothesis we have that for some negative \( k \), we have \( (ab)^{-k} = a^{-k}b^{-k} \), and 
		we prove this for \( (ab)^{-k - 1} \) now. Using the inductive hypothesis, we may write 
		\( (ab)^{-k - 1} = (ab)^{-k} (ab)^{-1} = a^{-k}b^{-k} (ab)^{-1} = a^{-k}b^{-k} * a^{-1} * b^{-1}\) (by 
		Proposition 1). Then, since \( a, b \) commute then the approach is the same as the positive
		integers, where we recover \( (ab)^{-k - 1} = a^{-k -1}b^{-k-1}\), as desired. Combining these two parts 
		we recover all \( n \in \Z \). 
	\end{solution}

	\begin{problem}
		Prove that if \( x^2 = 1 \) for all \( x \in G \) then \( G \) is abelian. 
	\end{problem}

	\begin{solution}
		We need to prove that \( a * b = b * a \) for all \( a, b \). Since \( x^2 = 1 \) for all elements, then 
		we know that \( a^2 = 1 \) and \( b^2 = 1 \), so \( a = a^{-1} \) and \( b = b^{-1} \), and by extension 
		we know that \( ab = a^{-1}b^{-1} \). Further, we also know that \( (ab)^{-1} = b^{-1} a^{-1} = ab \), so 
		therefore we have \( a^{-1} b^{-1} = b^{-1} a^{-1} \), and therefore \( ab = ba \), as desired. 
	\end{solution}

	\begin{problem}
		Assume \( H \) is a nonempty subset of \( (G, *) \) which is closed under the binary operation 
		on \( G \) and is closed under inverses, i.e., for all \( h \) and \( k \in H \), \( hk \) and 
		\( h^{-1} \in H \). Prove that \( H \) is a group under the operation \( * \) restricted to \( H \) (such 
		a subset \( H \) is called a \textit{subgroup} of \( G \)).
	\end{problem}

	\begin{solution}
		To prove that \( H \) is a group, we need to prove the group axioms; namely, that \( H \) is 
		associative, the identity of \( G \) exists in \( H \), and every element also contains its 
		inverse. The point about inverses is already guaranteed by the construction of \( H \), since \( H \) 
		is closed under inverses.

		We first prove that \( H \) is associative, so we need to prove that given \( a, b, c \in H \), we 
		have \( (ab)c = a(bc) \). Since \( G \) is a group, we know that they are indeed equal, so all we need to 
		really show is that \( (ab)c \in H \). This is also guaranteed, since \( ab \in H \), and since 
		 \( c \in H \), then \( (ab)c \in H \), so \( a(bc) \in H \) as well. 

		Now we prove the identity element must also exist in \( H \). For any element \( a \in H \), 
		since \( a^{-1} \in H \) and \( a a^{-1} = 1 \) by definition, then the identity element 
		must exist in \( H \) also. 
	\end{solution}

	\begin{problem}
		Prove that if \( x \) is an element of the group \( G \) then \( \{x^{n} \mid n \in \Z\}  \) is a 
		subgroup (cf. the preceding exercise) of \( G \) (called the \textit{cyclic subgroup} of \( G \) generated
		by \( x \)). 
	\end{problem}

	\begin{solution}
		Let \( S \) be the set in question.
		Clearly, the set is associative, and the identity also exists as \( x^{0} \in S \). The inverse 
		also exists, since \( x^{-1} \in S \). So, all we need to show is that this is closed, but this is also 
		obvious, since given two elements in \( S \), we have \( x^{k}x^{m} = x^{k + m} \in S \). 
	\end{solution}


	\begin{problem}
		Prove that any finite group \( G \) of even order contains an element of order 2. [Let 
		\( t(G) \) be the set \( \{g \in G \mid g \neq g^{-1}\}  \). Show that \( t(G) \) has an 
		even number of elements and every nonidentity element of \( G - t(G) \) has order 2.]
	\end{problem}

	\begin{solution}
		Following the hint, let \( t(G) \) be as defined, and we want to show that \( t(G) \) has 
		an even number of elements. Consider an element \( g \in t(G) \), with order 
		\( 2k \) (that is, \( g^{2k} = 1 \)). Then, \( g^2 \in t(G) \) as well, since the order of 
		\( g^2  \) must be \( k \). So, for every element in \( g \), there are actually two elements, meaning 
		that there are an even number of elements in \( t(G) \). Further, by definition of \( t(G) \), 
		we know that every nonidentity element of \( G - t(G) \) has order 2, and \( t(G) \) cannot equal 
		the set \( G \) itself since the identity element must exist in \( G \). Therefore, \( G \) 
		must have an element of order 2. 

		\question{verify this solution.}
	\end{solution}

	\begin{problem}
		If \( x \) is an element of infinite order in \( G \), prove that the elements \( x^{n}, n \in \Z \) 
		are all distinct. 
	\end{problem}

	\begin{solution}
		Since \( x \) is an element of infinite order, then we know that there is no value of \( n \) such that 
		\( x^{n} = 1 \). Further, suppose for the sake of contradiction that there are two values \( n_1 < n_2 \) 
		(WLOG) such that \( x^{n_1} = x^{n_2} \). Then, we can write \( x^{n_2} = x^{n_1 + n_2 - n_1} = 
		x^{n_1} x^{n_2 - n_1}\), but the fact that \( x^{n_1} = x^{n_2} \) implies that \( x^{n_2 - n_1} = 1 \) 
		meaning there was an inverse somewhere in this multiplication. 
		Therefore, all values \( x^{n} \) must be distinct. 
	\end{solution}
	\section{Subgroups}
	\subsection{Definition and Examples}

	\begin{problem}
		Give an explicit example of a group \( G \) and an infinite subset \( H \) of \( G \) that is 
		closed under the group operation but is not a subgroup of \( G \). 
	\end{problem}

	\begin{solution}
		Consider the vector space \( \R \), and now consider the infinite subset formed by the set of points 
		\( \{(x, 0) \mid x \in \R_+\}  \), and consider the operation of addition on this set defined as 
		\( (x_1, y_1) + (x_2, y_2) = (x_1 + y_1, x_2 + y_2) \). This set is certainly closed under the group 
		operation, but given any identity element \( (x_0, 0) \), then for any other element \( (x, 0) \) its 
		inverse is \( (x_0 - x, 0) \), but since \( x_0 - x \) could be negative, this inverse element does not 
		necessarily exist within the set. Thus, this subset is not a subgroup. 
	\end{solution}

	\begin{problem}
		Prove that \( G \) cannot have a subgroup \( H \) with \( |H| = n - 1 \), where \( n = |G| > 2 \). 
	\end{problem}

	\begin{problem}
		Let \( G \) be an abelian group. Prove that \( \{g \in G \mid |g| < \infty\}  \) is a subgroup of \( G \) 
		(called the \textit{torsion subgroup} of \( G \)). 
		Give an explicit example where this set is not a subgroup when 
		\( G \) is non-abelian.
	\end{problem}

	\begin{problem}
		Let \( H \) and \( K \) be subgroups of \( G \). Prove that \( H \cup K \) is a subgroup if and only if 
		either \( H \subseteq K \) or \( K \subseteq H \). 
	\end{problem}
\end{document}
