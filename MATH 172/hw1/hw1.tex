\documentclass[10pt]{article}
\usepackage{../../local}
\urlstyle{same}

\newcommand{\classcode}{Math 172}
\newcommand{\classname}{Combinatorics}
\renewcommand{\maketitle}{%
\hrule height4pt
\large{Eric Du \hfill \classcode}
\newline
\large{HW 01} \Large{\hfill \classname \hfill} \large{\today}
\hrule height4pt \vskip .7em
\small{Header styling inspired by Berkeley EECS Department: \url{https://eecs.berkeley.edu/}}
\normalsize
}
\linespread{1.2}
\begin{document}
	\maketitle

	\begin{problem}
		Show that an \( m \)-by-\( n \) chessboard has a perfect cover by dominoes if and only if at least one of
		\( m \) and \( n \) is even. 
	\end{problem}

	\begin{solution}
		We begin with the forward direction first. If we have a perfect cover, then the number of
		squares covered must be a multiple of 2, since each domino occupies exactly 2 squares. This would be
		possible if one of \( m \) or \( n \) is even, but not possible when \( m \) and \( n \) are odd,
		since \( mn \) would be odd. As such, one of \( m \) or \( n \) must be even. 

		With the reverse direction, suppose WLOG that \( m \) is even. Then, we may lay dominoes along \( m
		\) perfectly (that is, arrange the long edge along the side with length \( m \)), and then we repeat
		this tiling for all columns \( n \). This ensures a perfect tiling, as desired. 
	\end{solution}

	\begin{problem}
		Imagine a prison consisting of 64 cells arranged like the squares of an 8-by-8 chessboard. There are
		doors between all adjoining cells. A prisoner in one of the corner cells is told that he will be
		released, provided he can get into the diagonally opposite corner cell after passing through every
		other cell exactly once. Can the prisoner obtain his freedom? 
	\end{problem}

	\begin{solution}
		This is not possible. Whenever the prisoner moves from one cell to another, the color of the tile he
		stands on changes. Because he cannot repeat tiles, this means that he must reach the opposite corner
		cell in a total of 63 moves, which means that he makes an odd number of color changes. However, the
		opposite tile has the same cell color as the original square, so such a path is impossible.    
	\end{solution}

	\begin{problem}
		A game is played between two players, alternating turns as follows: The game starts with an empty
		pile: When it is his turn, a player may add either 1, 2, 3 or 4 coins to the pile. The person who
		adds the 100th coin to the pile is the winner. Determine whether it is the first or second player who
		can guarantee a win in this game. What is the winning strategy?  
	\end{problem}

	\begin{solution}
		The strategy is to notice that the player who leaves the pile at either 96, 97, 98, 99 coins will end
		up losing, since the subsequent player can place up to 4 to reach the 100th coin. As such, this means
		that the player who reaches 95 coins wins, since the next player is guaranteed to place an amount
		that lands them on one of 96 through 99. Continuing this logic, we find that the winning strategy is
		to place an amount of coins that leaves the overall pile at a multiple of 5. This means that player
		two can guarantee a win, since player 1 starts and can only reach a maximum of 4 coins. 
	\end{solution}

	\begin{problem}
		Suppose that in the previous exercise, the player who adds the 100th coin loses. Now who wins, and
		how? 
	\end{problem}

	\begin{solution}
		We can use a similar logic, except now the "key" coin counts are 99, 94, 89, etc., all the way down
		to 4. Thus, player one now has a guaranteed winning strategy.  
	\end{solution}

	\begin{problem}
		Eight people are at a party and pair off to form four teams of two. In how many ways can this be
		done?
	\end{problem}

	\begin{solution}
		Order does not matter here so we just have \( {8 \choose 2}{6 \choose 2}{4 \choose 2}{2 \choose 2} =
		2520\). Then, the order in which the teams are picked doesn't matter, so we have to divide by \( 4!
		\) to get rid of the duplicates. Thus, we have 105 ways in total. 
	\end{solution}

	\begin{problem}
		For each of the four subsets of the two properties (a) and (b), count the number of four-digit
		numbers whose digits are either 1, 2, 3, 4 or 5:

		\begin{enumerate}[label=(\alph*)]
			\item The digits are distinct
			\item The number is even
		\end{enumerate}

		\begin{solution}
			If the digits are distinct, then we have \( 5! = 120 \) different ways. If the number is strictly
			even, then we only have the restriction that the number must end in 2 or 4. There are \( 5^{4} =
			625 \) total four-digit numbers that can be made from these digits, and we note that there are an
			equal number of numbers that must end in 2 and 4, so exactly \( \frac{2}{5} \) of these numbers
			satisfy condition (b), which gives us 250.

			For numbers that are distinct and even, then we do the same thing as above, except the total
			space is now \( 5! = 120 \) numbers. Again, an equal number of numbers end in 2 than in 4, so we
			have \( \frac{2}{5} \times 120 = 48 \) total numbers. 

			To find the number of numbers that are neither distinct nor even, we use principle of
			inclusion-exclusion. The total space is \( 5^{4} = 625 \) numbers; those that satisfy (a) is \(
			5! = 120 \), those that satisfy (b) is \( \frac{2}{5} \times 625 = 250 \), and those that satisfy
			(a) and (b) is \( \frac{2}{5} \times 120 = 48 \), so by PIE:
			\[
				N = 625 - 250 - 120 + 48 = 303
			\]
		\end{solution}
	\end{problem}

	\begin{problem}
		How many orderings are there for a deck of 52 cards if all the cards of the same suit are together? 
	\end{problem}

	\begin{solution}
		There are \( 4! \) ways to arrange the suits together, and then within each suit there are \( 13! \)
		ways of arranging the cards of the same suit. This gives us \( 4! \times (13!)^{4} \) total ways.   
	\end{solution}

	\begin{problem}
		How many distinct positive divisors does each of the following numbers have?
		\begin{enumerate}[label=(\alph*)]
			\item \( 3^{4} \times 5^2 \times 7^{6} \times 11 \)

				\begin{solution}
					The prime factorization can be written in general as:
					\[
						N = p_1^{\alpha_1} p_2^{\alpha_2} \cdots p_n^{\alpha_n}
					\]
					where the number of factors is given by \( \sigma(n) = \prod_{i = 1}^{n}(\alpha_i + 1)
					\). Thus, we get 210 in this case. 
				\end{solution}
			\item 620

				\begin{solution}
					This factors into \( 2^2 \times 5 \times 31 \), which gives 12 divisors. 
				\end{solution}
			\item \( 10^{10} \)

				\begin{solution}
					This is \( 2^{10} \times 5^{10} \), so this gives 121 divisors. 
				\end{solution}
		\end{enumerate}
	\end{problem}

	\begin{problem}
		Determine the largest power of 10 that is a factor of the following numbers
		\begin{enumerate}[label=(\alph*)]
			\item 50!

				\begin{solution}
					Here, we count the number of multiples of 10 we can generate by decomposing the numbers
					in 50!. We first note that there are an abundance of factors of 2, so we need only count
					the factors of 5. There are 12 factors of 5 here, with 25 contributing 2 and 50 also
					contributing 2. Thus \( 10^{12} \) is the largest power. 
				\end{solution}
			\item 1000!

				\begin{solution}
					We do the same as the previous problem: there are 200 factors of 5, 40 factors of 25, 8
					factors of 125, and 1 factor of 625, so in total there are 249 total zeros.
				\end{solution}
		\end{enumerate}
	\end{problem}

	\begin{problem}
		In how many ways can four men and eight women be seated at a round table if there are to be two women
		between consecutive men around the table? 
	\end{problem}

	\begin{solution}
		There are \( 4! \) ways to arrange the men, and \( 8! \) ways to arrange the women. There are then
		three arrangements accounting for the overall arrangement of the table that we consider, so there are 
		\( 3 \times 4! \times 8! \) total ways. 
	\end{solution}

	
\end{document}
