\documentclass[10pt]{article}
\usepackage{../../local}


\newcommand{\classcode}{Physics 110A}
\newcommand{\classname}{Electromagnetism and Optics}
\renewcommand{\maketitle}{%
\hrule height4pt
\large{Eric Du \hfill \classcode}
\newline
\large{HW 10} \Large{\hfill \classname \hfill} \large{\today}
\hrule height4pt \vskip .7em
\normalsize
}
\linespread{1.1}
\begin{document}
	\maketitle
	\section*{Problem 1}
	An infinite slab of thickness $d$ with a uniform magnetization $\mathbf M = M_x \hat{x} + M_y \hat{y} + 
	M_z \hat{z}$ ($M_x, M_y, M_z$ are constants) extends along the $xy$-plane. Find the magnetic field 
	$\mathbf B$ for a point at a distance $L$ from the infinite slab. 

	\begin{solution}
		From the magnetization $M$, we can calculate the volume and bound currents. Since $M$ is constant,
		then we know that $J_b = \curl M = 0$. For the surface current, we need $K_b = M \times \hat{n}$. For 
		the top plate, we have $\hat{n} = \hat{z}$, so therefore: 
		\[
			K_{\text{b, top}} = \begin{vmatrix} \hat{i} & \hat{j} & \hat{k} \\ M_x & M_y & M_z \\ 0 & 0 & 1
			\end{vmatrix} = M_y \hat{x} - M_x \hat{y}
		\] 
		Now, since for the bottom slab we'll get $\hat{n} = -\hat{z}$, then $K_\text{b, bottom}$ will flow
		in the opposite direction of $K_{\text{b, top}}$, so therefore 
		\[
			K_{\text{b, bottom}} = -M_y \hat{x} + M_x \hat{y}
		\] 
		This means that we can treat our system essentially as if we had two infinitely large parallel plates,
		with one plate having current flowing in one direction and the other in the opposite direction. Then, 
		we can use Example 5.8 from Griffiths to motivate our solution. In that example, we can see that for
		a single slab with surface current $\vec K$, then we have: 
		\[
		\mathbf B = \begin{cases}
			- \frac{\mu_0}{2}K \hat{y} & \text{above}\\
			\frac{\mu_0}{2}K \hat{y} & \text{below}
		\end{cases}
		\] 
		Naturally, the signs will be reversed for the bottom plate. Therefore, when summing up the vector 
		components, we find that the magnetic field outside the slabs is 0 everywhere. Therefore, to answer 
		the original problem, we get that $B = 0$ everywhere outside the slab. 

		Alternatively, this argument can also be made via Ampere's law. Consider an Amperian loop that runs 
		through tne entire slab, to a distance $L$ above and below the slab. We know that here, the enclosed
		current is 0 (since they flow in exact opposite directions), so therefore we require that 
		\[
		\oint \mathbf B \cdot dl = 0
		\] 
		But in this case, since we expect the $B$ field to be constant in magnitude at a constant distance $L$ 
		above and below and also that their directions should be opposite of each other, this requires that 
		$B = 0$ everywhere as well. 
	\end{solution}

	\pagebreak
	\section*{Problem 2}
	A long cylinder of radius $R$ carries a magnetization parallel to the axis $\mathbf M = ks^2 \hat{z}$, where
	$k$ is a constant and $s$ is the distance from the axis. 
	\begin{enumerate}[label=\alph*)]
		\item Find the magnetic field inside and outside the cylinder.

			\begin{solution}
				Just like the previous problem, we calculate the surface and volume currents for this 
				situation. Here, we have $J_b = \curl M$, so therefore:
				\[
				\mathbf J_b = \curl M = -2ks \hat{\phi}
				\] 
				Similarly, we can calculate the bound current using $K_b = M \times n$:
				\[
				\mathbf K_b = M \times \hat{s} = kR^2 \hat{\phi}
				\] 
				These quantities will be
				useful when computing the $B$ field inside the cylinder. Outside the cylinder, we can draw an 
				Amperian loop along the vertical axis - here, due to the symmetry in the problem, the $B$ field
				on opposite sides of the loop are equal. Then, because there is no enclosed current, then 
				naturally we conclude that $B = 0$ outside the cylinder. To calculate the $B$ field inside the 
				cylinder, consider an Amperian loop that penetrates the cylinder to a depth $R - s$. 

				Using Ampere's law, we get: 
				\begin{align*}
					\mathbf B \cdot dl &= \mu_0 I_{enc}\\
					\mathbf Bl &= \mu_0\left(\int_s^R \mathbf J_b \cdot da + K_b l\right)\\
					\mathbf Bl &= \mu_0\left( -\int_s^R 2ks \cdot l ds + K_bl  \right) \\
					\therefore \mathbf B &= \mu_0\left( -k(R^2 - s^2) + kR^2 \right) \\
					&= \mu_0 ks^2 
				\end{align*}
				From the right hand rule, we can then figure out the direction of the $B$ field must be pointing
				in the $\hat{z}$ direction, so therefore
				\[
				B = \mu_0 ks^2 \hat{z}
				\] 
			\end{solution}
		\item Find $\mathbf H$ inside and outside the cylinder

			\begin{solution}
				We use the definition of $\mathbf H = \frac{1}{\mu_0}\mathbf B - \mathbf M$. Since $\mathbf B
				= 0$ outside the cylinder and $\mathbf M = 0$ as well, then we know that outside the cylinder, 
				$\mathbf H = 0$. Inside the cylinder, we get:
				\[
				\mathbf H = \frac{1}{\mu_0}(\mu_0 ks^2 \hat{z}) - ks^2 \hat{z} = 0
				\] 
				Therefore, $\mathbf H = 0$ both inside and outside the cylinder. 
			\end{solution}
		\item Check that the Ampere's law (6.20) in Griffiths is satisfied

			\begin{solution}
				Equation 6.20 reads 
				\[
					\oint \mathbf H \cdot d\mathbf l = \mu_0 I_{f_{enc}}
				\] 
				Since $\mathbf H = 0$ over the entire space, then we know that the left hand side of the equation
				evaluates to 0. Furthermore, since there is no free current (since the only current is from the 
				magnetization, which only gives bound currents), then the right hand side also evaluates to 
				0, regardless of the Amperian loop we choose. 
			\end{solution}
	\end{enumerate}


	\pagebreak
	\section*{Problem 3}
	An infinitely long cylinder of radius $R$ carries a current with a \textit{free} current density 
	$\mathbf J_f(s) = J_f(s) \hat{z}$ along its axis, within $s < R$. There is no current outside the cylinder.
	The cylinder is made by linear material with permeability $\mu$, and the magnetic field is found to be 
	\[
	\mathbf B = \begin{cases}
		ks^2 \hat{\phi} & \text{for $s < R$}\\
		0 & \text{for $s > R$}
	\end{cases}
	\] 
	Find the free current density $\mathbf J_f(s)$ in the bulk of the cylinder, as well as the free current 
	density $\mathbf K_f$ flowing over the side surface of the cylinder. 


	\begin{solution}
		Since we know that $\mathbf B = \mu \mathbf H$, then we can derive $H$ by dividing both sides by $\mu$:
		\[
		\mathbf H = \frac{\mathbf B}{\mu} = \frac{ks^2}{\mu}\hat{\phi}
		\] 
		Then, we can determine $\mathbf J_f$ by using $\mathbf J_f = \curl \mathbf H$:
		\begin{align*}
			\curl \mathbf H &= \frac{1}{s}\left[\pdv{s}\left(s \frac{ks^2}{\mu}\right)\right]\hat{z}\\
			\therefore \mathbf J_f &= \frac{3ks}{\mu}\hat{z}
		\end{align*}
		To find the surface current, we use the fact that
		$\mathbf H_{\text{above}} - \mathbf H_{\text{below}} = \mathbf K_f \times \hat{n}$. Since the current outside the 
		cylinder is 0, then this means that $\mathbf H_{\text{above}} = 0$. Below the cylinder, we have 
		$\mathbf H_{\text{below}} = \frac{kR^2}{\mu}\hat{\phi}$. Therefore:
		\begin{align*}
			-\frac{kR^2}{\mu}\hat{\phi} &= \mathbf K_f \times \hat{r}\\
			\therefore \mathbf K_f &= \frac{kR^2}{\mu}\hat{z}
		\end{align*}
	\end{solution}
	
\end{document}
