\documentclass[10pt]{article}
\usepackage{../../local}


\newcommand{\classcode}{Physics 110A}
\newcommand{\classname}{Electromagnetism and Optics}
\renewcommand{\maketitle}{%
\hrule height4pt
\large{Eric Du \hfill \classcode}
\newline
\large{Review Questions} \Large{\hfill \classname \hfill} \large{\today}
\hrule height4pt \vskip .7em
\normalsize
}
\linespread{1.1}
\begin{document}
	\maketitle
	\section*{Chapter 4}
	\begin{itemize}
		\item (Griffiths p.189) Book mentions that $\div \mathbf D = \rho_f$ and $\curl \mathbf D = 0$ only
			when the space is 
			entirely filled with a homogeneous dielectric -- can this argument be extended to say that within
			a homogeneous dielectric (that is large enough, but $E$ field is nonzero outside), that the same
			equations hold true?
		\item (Griffiths p.189) How does $\mathbf D = \epsilon_0\mathbf E_{\text{vac}}$ come from the two
			expressions 
			above it?
		\item (Griffiths p.192) I thought that $\rho_b$ had to do with polarization, which does not require
			the presence of free charge. Why must it be the case that if $\rho_f = 0$ then $\rho_b = 0$?
		\item (Griffiths p.193) Where does the $-E_0r \cos \theta$ term in Equation 4.45 come from? 
			\\
			\textbf{Answer:} Comes from the fact that $\mathbf E = E_0 \hat{z}$ so therefore using $V = 
			- \int \mathbf E \cdot d\mathbf l$ you get that $V = -E_0z = -E_0r\cos \theta$, since $z = r
			\cos \theta$. 
		\item (Griffiths p.195) When only given $\chi_e$, should the approach always be to use $\sigma_b = 
			\epsilon_0\chi_e \mathbf E$, then use the $\mathbf E$ field at the boundary (due to the bound charge) to solve for 
			$\sigma_b$, which then allows us to find $\mathbf E$?
	\end{itemize}

	\section*{Chapter 7}
	\begin{itemize}
		\item (Griffiths p.304) Why is it true that if $\sigma = \infty$, then the net force on the charges
			equals zero? Is this becuase we require $\mathbf J$ to be finite and since $\mathbf J = \sigma 
			\mathbf f $, then if $\sigma = \infty$ then in order for $\mathbf J$ to be finite then 
			we require that $\mathbf f = 0$?
	\end{itemize}

	\section*{Chapter 8}
	\begin{itemize}
		\item (Griffiths p. 359) Why is it that when $\dv{W}{t} = 0$, then we can conclude that 
			\[
				\int\pdv{u}{t}d\tau = - \oint \mathbf S \cdot d\mathbf a?
			\] 
	\end{itemize}
\end{document}
