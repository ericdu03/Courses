\documentclass[10pt]{article}
\usepackage{../local}


\newcommand{\classcode}{Physics 110A}
\newcommand{\classname}{Electrodynamics and Optics}
\renewcommand{\maketitle}{%
\hrule height4pt
\large{Eric Du \hfill \classcode}
\newline
\large{HW 04} \Large{\hfill \classname \hfill} \large{\today}
\hrule height4pt \vskip .7em
\normalsize
}
\linespread{1.1}
\begin{document}
    \maketitle
	\section*{Collaborators}
	I worked with \textbf{Andrew Binder} to complete this assignment. 
	\section*{Problem 1}
	Consider a spherical thin \textit{shell} of radius $R$ with charges uniformly distributed on its surface. The
	uniform surface charge density is denoted as $\sigma$. 
	\begin{enumerate}[label=\alph*)]
			\item Setting the potential $V$ to be zero at infinity, find the electric potential due to the 
					charged shell at points both inside and outside the shell directly using
					\[ V = \frac{1}{4\pi \epsilon_0}\int \frac{dq}{\rcurs}\]
					\begin{solution}
							Using the formula, we write: $dq = \sigma R^2 \sin \theta d \theta d\phi$, and since
							we integrate $\phi$ from $0$ to $2\pi$ so our $\phi$ integral becomes $2\pi$. 
							Therefore, we compute the integral:
							\begin{align*}
									V &= \frac{1}{4\pi \epsilon_0} \int 
									\frac{\sigma \cdot 2\pi R^2 \sin \theta d\theta}
									{\sqrt{ r^2 + R^2 - 2rR \cos \theta}}\\
									&= \frac{\sigma R^2}{2\epsilon_0}\int \frac{\sin \theta d\theta}{\sqrt{r^2 
									R^2 - 2rR \cos \theta} }\\
							\end{align*}
							From here we perform the substitution $u = r^2 + R^2 - 2rR\cos \theta$ with $du = 
							2rR \cos \theta$. Therefore, we now have the integral: 
							\begin{align*}
									V &= -\frac{\sigma R^2}{2\epsilon_0} \frac{1}{2rR}\int \frac{du}{\sqrt{u} }\\
									  &= \frac{\sigma R}{2\epsilon_0 r}\left[(r^2 + R^2 + 2rR)^{1/2} - 
									  (r^2 + R^2 - 2rR)^{1/2}\right]\\
									  &= \frac{\sigma R}{2\epsilon_0 r}\left[ r + R - |r - R|\right]\\
							\end{align*}
							If $r < R$, then we have $|r - R| = -r + R$, and so therefore if $r < R$, then we
							have:
							\[ V = \frac{\sigma R}{2\epsilon_0 r} (2r) = \frac{\sigma R}{\epsilon_0}\]
							And if $r > R$, then we can drop the absolute value: 
							\[ V = \frac{\sigma R}{2\epsilon_0 r}(2R) = \frac{\sigma R^2}{\epsilon_0 r}\]
					\end{solution}
			\item Using $V$, find the electric field $\mathbf E$ created by the shell.

					\begin{solution}
						We have $E = -\nabla V$, so therefore:
						\[ E = \begin{cases}
								0 & r < R\\
								-\frac{\sigma R^2}{\epsilon_0r^2} & r > R
						\end{cases}\]
					\end{solution}

			\item Find the energy $W$ stored in this system.

					\begin{solution}
							The energy stored in the system can be written as: 
							\[ U = \frac{\epsilon_0}{2}\int_{\mathbb R^3} |E|^2 \ d\tau\]
							And so inside the sphere, we have $W = 0$, since $E = 0$. Outside the sphere, we 
							compute the integral as normal:
							\begin{align*}
									W &= \frac{\epsilon_0}{2}\int_R^\infty \int_0^{2\pi} \int_0^\pi 
									\frac{\sigma^2 R^4}{\epsilon_0^2r^4} r^2 \sin \theta d \theta d\phi dr\\
									  &= (2\pi) \frac{\sigma^2R^4}{2\epsilon_0}\int_R^\infty \frac{1}{r^2} dr 
									  \int_0^\pi \sin \theta d \theta\\
									  &= \frac{\sigma^2 R^4}{\epsilon_0}\left( -\frac{1}{r} \right)\bigg|_R^\infty
									  (-\sin \theta)\bigg|_0^\pi\\
									  &= \frac{\sigma^2 R^4}{\epsilon_0}\frac{1}{R}(2)\\
									  &= \frac{2\sigma^2R^3}{\epsilon_0}
							\end{align*}
					\end{solution}
	\end{enumerate}

	\pagebreak

	\section*{Problem 2}
	Continue from Problem 1, consider the situation where the radius of the shell expands by $\delta R$ while the 
	total charge on the shell remains constant, and the energy stored in the system changes by $\delta W$ 
	accordingly. 
	\begin{enumerate}[label=\alph*)]
			\item Find the expression for $\delta W$

					\begin{solution}
							We use the expression that we obtained from part c). To preserve the charge, we 
							rewrite $\sigma = \frac{Q}{4 \pi R^2}$, and now introduce the $\delta R$ expansion:
							\[ \sigma = \frac{Q}{4 \pi (R + \delta R)^2}\]
							Therefore, our equation for $\delta W$ now becomes: 
							\begin{align*}
									\delta W &= W' - W \\
									&= \frac{2}{\epsilon_0}\left( \frac{Q}{4\pi (R + \delta R)^2} \right)^2
									(R+\delta R)^3 - \frac{2}{\epsilon_0}\left( \frac{Q}{4\pi R^2} \right)^2 R^3\\
									&= \frac{2}{\epsilon_0}\frac{Q^2}{16\pi^2(R + \delta R)} -
									\frac{2}{\epsilon_0}\frac{Q^2}{16\pi^2R}\\
									&= \frac{Q^2}{8\pi^2\epsilon_0}\left( \frac{1}{R + \delta R} - 
									\frac{1}{R} \right)  \\
									&= -\frac{Q^2\delta R}{8\pi^2\epsilon_0 R(R + \delta R)}
							\end{align*}
					\end{solution}
			\item The change in energy must be equal to the work done by the pressure $f$ (force per unit area) 
					exerted by the electric force as the shell expands:
					\[ \delta W = -f(4\pi R^2)\delta R\]
					There is a negative sign as the energy stored in the system is being used by the electric 
					force to do work. Use this relation to find $f$. Does the electric field try to squeeze the 
					shell or stretch the shell?

					\begin{solution}
							Now we can find the force using our relation from before:
							\begin{align*}
									-\frac{Q^2\delta R}{8\pi^2\epsilon_0R(R + \delta R)} &= -f(4 \pi R^2) 
									\delta R\\
									\therefore f &= \frac{Q^2}{32 \pi^2 \epsilon_0R^3(R + \delta R)}
							\end{align*}
							Since this sign is positive, then we conclude that the force here is radially 
							outward, meaning that the electric field actively tries to stretch the shell.
							Intuitively, this also makes sense since we expect that the charges on the shell
							repel each other, and thus want to spread apart as much as possible, so they'd 
							naturally push on the boundary, expanding the shell. 
					\end{solution}
	\end{enumerate}

	\pagebreak

	\section*{Problem 3}
	A parallel-plate capacitor has a separation distance $d$ between the two plates and is connected to a battery
	to main a potential difference $V$ between the two plates. The plates have a width $w$ and length $L$ 
	extending  into the page. We gradually insert a metal slab of thickness $t$ into the space between the two 
	plates. Along the way, the slab is parallel with the two plates. Find the electric force acting on the 
	metal slab when it is inserted a distance $x$ into the capacitor. (Don't forget that the external source 
	that maintains the potential difference also does work on the moving charges. You should find the force
	tends to draw the slab into the capacitor.)
	\begin{solution}
			We know that the force required is: \[
					F_E = -\pdv{U_E}{x} + V \dv{Q}{x}
			\] However, since the potential is constant (as the capacitor is hooked up to a battery), then the 
			first term is equal to zero. Since the electric fields are (roughly) constant, then we can assume
			that the charge distributions are also constant from 0 to $x$ and constant from $x$ to $w$. 

			Now we compute the magnitude of the electric field in two regions: the region where the slab is 
			inserted, and the region where the slab is absent (in other words, from 0 to $x$ and from $x$ to $w$.
			Since $E = -\nabla V$, then we can compute the electric field:
			\[
				 E = -\nabla V = -\dv{V}{y} = -\frac{\frac{V}{2}}{\frac{d-t}{2}} = \frac{V}{d-t}
			\] 
			Similarly, the electric field in the region where the slab is not present is:
			\[
			E = -\frac{V}{d}
			\] 
			Now consider a Gaussian pillbox on the top plate that extends from 0 to $x$, and from $0$ to $L$ 
			width-wise. In this region, the electric field above the capacitor is zero (since there is no 
			capacitor). Therefore, for the region where the slab is inserted: 
			\begin{align*}
					E_{above} - E_{below} &= 0 - \left( -\frac{V}{d-t} \right) = \frac{\sigma_1}{\epsilon_0} \\
					\therefore \sigma_1&= \frac{V\epsilon_0}{d-t} \\
			\end{align*}
			And similarly for the region where the slab isn't inserted:
			\begin{align*}
					E_{above} - E_{below} &= 0 - \left( -\frac{V}{d} \right) = \frac{\sigma_2}{\epsilon_0} \\
					\therefore \sigma_2 &= \frac{V \epsilon_0}{d}
			\end{align*}
			So the total charge $Q$ can be written as:
			\[
			Q = \sigma_1A_1+\sigma_2A_2 = \frac{V\epsilon_0}{d-t}xL + \frac{V\epsilon_0}{d}(w-x)L
			\] 
			Now computing $\dv{Q}{x}$:
			\[
					\dv{Q}{x} = \frac{V\epsilon_0L}{d-t} + \frac{V\epsilon_0L}{d}(-1) = \frac{V\epsilon_0L}{d-t}
					- \frac{V\epsilon_0L}{d}
			\] 
			Now finally using the relation that $F_E = V\dv{Q}{x}$, we get:
			\begin{align*}
					F_E &= V \dv{Q}{x}\\
					&= V^2\epsilon_0L\left( \frac{1}{d-t}- \frac{1}{d} \right)  \\
					&= V^2\epsilon_0L\left( \frac{d - (d-t)}{d(d-t)} \right)  \\
					&= V^2\epsilon_0L\left( \frac{t}{d(d-t)} \right)
			\end{align*}
			Note that the direction of the force is also consistent with what we expect to obtain based on the 
			problem statement, as it has positive sign. 
	\end{solution}
	

\end{document}
