\documentclass[10pt]{article}
\usepackage{../local}


\newcommand{\classcode}{Physics 110A}
\newcommand{\classname}{Electromagnetism and Optics}
\renewcommand{\maketitle}{%
\hrule height4pt
\large{Eric Du \hfill \classcode}
\newline
\large{HW } \Large{\hfill \classname \hfill} \large{\today}
\hrule height4pt \vskip .7em
\normalsize
}
\linespread{1.1}
\begin{document}
	\maketitle
	\section*{Collaborators}
	I worked with \textbf{Andrew Binder, Teja Nivarthi, Nathan Song, Christine Zhang} and \textbf{Nikhil Maserang} to complete this homework assignment.
	\section*{Problem 1}
	On the right shows a system where current $I$ flowing upward along the central wire, spreading uniformly at
	the upper cap in the radial direction, flowing down vertically on the cylindrical wall, and then flowing 
	radially inward back to the central wire. The radius and height of the system are $R$ and $L$ 
	respectively.

	\begin{enumerate}[label=\alph*)]
		\item Argue that the magnetic field everywhere has the form of $\mathbf B = B(s, z) \mathbf{\hat{\phi}}$

			\begin{solution}
				In terms of the dependences, we see that the only symmetry in the problem is azimuthal symmetry,
				so we don't expect $\phi$ dependence. No such symmetry exists in the other two coordinates, so 
				we of course expect $B$ to depend on them. In terms of the direction, if there were an $\hat{s}$
				or $\hat{z}$ direction, this would imply that the magnetic field wouldn't form closed loops 
				all the time -- something we know can't happen. Thus, we only expect it to point in the $\hat{
				\phi}$ direction
			\end{solution}
		\item Find $B(s, z)$ at every point in the space. 

			\begin{solution}
				For the sake of convenience, let the center of the bottom plate denote the origin. Then, we know
				that for regions $0 < z < L$ and $0 < r < R$, that the magnetic field can be given by 
				Ampere's law:
				\[
				B(2\pi s) = \mu_0I \implies B = \frac{\mu_0 I}{2\pi s}
				\] 
				On the other hand, when $0 < z < L$ and $s > R$, we can see that an Amperian loop here would 
				enclose the current through the center but also the current flowing down the walls. Due to
				the conservation of current, we know that the currents here must sum up equally. Thus, the 
				enclosed current is 0, and so $B = 0$ at $s > R$. In the region $z > L$, we see
				that an Amperian loop drawn here has no current piercing it either, so $B = 0$ as well. Thus,
				we're left with: 
				\[
				B = \begin{cases}
					\frac{\mu_0 I}{2\pi s} & 0 < z < L \text{ and } 0 < s < R\\
					0 & \text{otherwise}
				\end{cases}
				\] 
			\end{solution}
		\item Check that the boundary condition $B_{above} - B_{below} = \mu_0(\mathbf K \times \hat{\mathbf n})$
			is satsified at the cylindrical wall and upper cap. 

			\begin{solution}
				Firstly, since the magnetic field above the cap is 0, then we know that $B_{above} - B_{below} =
				-\frac{\mu_0I}{2\pi s} \hat{\phi}$. Now we need to check the right hand side. Firstly, we check 
				the direction of $K \times n$. $K$ points in the $\hat{s}$ direction and $\hat{n}$ points in 
				the $\hat{ z}$ direction, we know that $\hat{s} \times \hat{n} = -\hat{\phi}$, so the direction
				is verified. 

				In terms of magnitude, we know that we can determine $K$ by looking at an angular width $ r
				d\theta$, giving us the integral: 
				\[
				\int K r d\theta = I \implies K = \frac{I}{2\pi s}
				\] 
				so therefore
				\[
				\mu_0(K \times \hat{n}) = -\frac{I}{2\pi s}\hat{ \phi}
				\] 
				as expected. 
			\end{solution}
	\end{enumerate}
	\pagebreak

	\section*{Problem 2}
	Show that in a current-free volume of space, a static magnetic field can never have a local maximum by 
	considering 
	\[
		\oint_S \nabla(B \cdot B) da
	\] 
	where $\mathcal S$ is the surface of a small volume $\mathcal V$ containing a point $P$ in space. 

	\begin{solution}
		In order for a local max to occur, we require that $\nabla(\vec B \cdot \vec B) = 0$ and also that
		$\nabla^2 (\vec B \cdot \vec B) < 0$. We prove that this is an impossible condition to have, given 
		the fact that $\div B = 0$, and also $\curl B = 0$, given that there is no current.  

		Firstly, we turn this into a volume integral by using the divergence theorem:
		\[
			\oint_{\mathcal S} \nabla (\vec B \cdot \vec B) da = \int_{\mathcal V} \nabla^2 (\vec B \cdot \vec B)
			d\mathcal V
		\] 
		Now we focus on the integrand, and rewrite it as:
		\[
			\nabla^2(\vec B \cdot \vec B) = 2(\nabla^2 \vec B) B + (\nabla B)^2
		\] 
		We also know from an equation in the formula sheet that $\curl(\curl A) = \nabla(\div A) - \nabla^2 A$. 
		Rearranging for $\nabla^2 A$, we get:
		\[
		\nabla^2 A = \nabla(\div A) - \nabla(\curl A)
		\] 
		But since $\div A = 0$ and $\curl A = 0$ as mentioned above, then $\nabla^2 A = 0$ as well. Thus, 
		\[
		\nabla(\vec B \times \vec B) = (\nabla B)^2
		\]
		which is a strictly positive quantity. Therefore, it cannot possibly be true that $\nabla^2 (\vec B 
		\cdot \vec B) < 0$, so $B$ cannot have a local maximum. 

		Like the others that I worked with, I'm also confused as to why we need the integral to begin with, unless its purpose 
		was to get us to use the divergence theorem to see that we have to analyze 
		$\nabla^2 (\vec B \cdot \vec B)$, since we didn't use the integral at all.
	\end{solution}

	\pagebreak
	\section*{Problem 3}

	A uniformly charged solid sphere of radius $R$ carries a total charge $Q$, and is spinning with an angular
	velocity $\omega$ about the $z$-axis. 

	\begin{enumerate}[label=\alph*)]
		\item What is the magnetic dipole moment of the sphere?

			\begin{solution}
				Griffiths equation 5.90 is of particular use here since it gives an expression of the magnetic
				dipole moment as a function of the volume current density: 
				\[
				\vec m = \frac{1}{2}\int \vec r \times J d\tau
				\] 
				We know that the volume current density is $J = \rho (r \times \omega)$, so therefore
				\[
				J = \frac{Qr \omega}{\frac{4}{3}\pi R^2}\hat{\phi}
				\] 
				Now, we integrate this:
				\begin{align*}
					\vec m &= \frac{1}{2}\int (\vec r \times \frac{Qr \omega}{\frac{4}{3}\pi R^2}\hat{ \phi}) 
					d\tau\\
						   &= -\frac{3 \hat{ \theta}}{8\pi R^2} \int_0^R \int_0^{2\pi} \int_0^\pi Q r^2 \omega
						   r^2 \sin \theta d\theta d\phi dr\\
						   &= -\frac{3Q \omega R^2}{4}\hat{\theta} 
				\end{align*}
			\end{solution}

		\item At the large distance limit $r \gg R$, find the approximate vector potential at a point 
			$(r, \theta)$.

			\begin{solution}
				Here, we use the fact that the dipole term dominates, and that 
				\[
					A_{dip}(r) = \frac{\mu_0}{4\pi}\frac{\vec m \times \vec r}{r^2}
				\] 
				Using the value of $m$ that we got from the previous part, we get
				\begin{align*}
					A_{dip}(r) &= \frac{\mu_0}{4\pi}\frac{1}{r^2}\left( -\frac{3Q \omega R^2 \hat{ \theta}}{4}
					\times \vec r\right) \\
					&= \frac{\mu_0}{4\pi r} \left( -\frac{3Q\omega R^2}{4}\hat{ \phi} \right)
				\end{align*} 
				where I've used the fact that $\hat{\theta} \times \hat{r} = \hat{\phi}$ to get the result. 
			\end{solution}

		\item Find the \textit{exact} vector potential outside the sphere.

			\begin{solution}
				Here, we use the relation that 
				\[
				A(r) = \frac{\mu_0}{4\pi}\int \frac{J(r)}{\rcurs} d\tau'
				\] 
				Using the fact that $J = \frac{3Qr \omega}{4\pi R^2}$, we get: 
				\begin{align*}
					A(r) &= \frac{\mu_0}{4\pi}\frac{3Q \omega}{4\pi R^2}\int \frac{r}{\rcurs} d\tau'\\
						 &= \frac{\mu_0}{4\pi}\frac{3Q \omega}{4\pi R^2} \int_0^R \int_0^{2\pi} \int_0^\pi 
						 \frac{r^3 \sin \theta}{|r - r'|} d\theta d\phi dr\\
						 &= \frac{3 \mu_0 Q \omega}{4 \pi R^2}\int_0^R \int_0^\pi \frac{r^3 \sin \theta}{
						 |r - r'|} d\theta dr' 
				\end{align*}
				And from here, I'm not really sure how to do the integral. While it is true that we can write  $|r - r'| =
				\sqrt{r^2 + r'^2 - 2r r' \cos \theta}$, this relation isn't exactly useful since it still gives
				an awful integral to solve. 

				When I was solving this problem with friends, another approach came up: Griffiths problem 5.11 gives that 
				for a shell, the vector potential $A(r)$ is given by: 
				\[
				A(r) = \begin{cases}
					\frac{\mu_0 R \omega \sigma r}{3}\sin \theta \hat{ \phi} & r \le R\\
					\frac{\mu_0 R^4 \omega \sigma}{3}\frac{\sin \theta}{r^2}\hat{ \phi} & r > R
				\end{cases}
				\] 
				so to find the vector potential here, we just superimpose a bunch of these shells of radius $R$
				on top of each other, in other words we are computing an integral from 0 to $R$ of the shells.
				Doing so, we get: 
				\begin{align*}
					A &= \int dA_{\text{shell}} \\
						&= \frac{\mu_0 R^4 \omega}{3} \frac{\sin \theta}{r^2} \int_0^R \frac{Q}{4\pi R^3} dr\\
						&= \frac{\mu_0 R^4 \omega}{3} \frac{\sin \theta}{r^2} \frac{Q}{\frac{4}{3}\pi R^2}\\
						&= \frac{\mu_0 Q R^2 \omega \sin \theta}{4\pi r^2}\hat{ \phi}
				\end{align*}
				While I can't find a direct mistake in this line of reasoning, I can't shake the fact that 
				it just doesn't seem right, since the integral expression that I derived seems to suggest 
				that the form of $A(r)$ is much more complicated than the result this approach yields.  
			\end{solution}
	\end{enumerate}
\end{document}
