\documentclass[10pt]{article}
\usepackage{../../local}


\newcommand{\classcode}{Physics 110A}
\newcommand{\classname}{Electrostatics and Optics}
\renewcommand{\maketitle}{%
\hrule height4pt
\large{Eric Du \hfill \classcode}
\newline
\large{HW 07} \Large{\hfill \classname \hfill} \large{\today}
\hrule height4pt \vskip .7em
\normalsize
}
\linespread{1.1}
\begin{document}
	\maketitle
	\section*{Collaborators}

	I worked with \textbf{Andrew Binder, Teja Nivarthi, Nathan Song, Christine Zhang} and \textbf{Nikhil
	Maserang} to complete this homework. 
	\section*{Problem 1}
	Show that the quadrupole term in the multipole expansion can be written as
	\[
		V_{\text{quad}}(r) = \frac{1}{4\pi \epsilon_0}\frac{1}{r^3}\sum_{i, j = 1}^3 \hat{r_i}\hat{r_j}Q_{ij}
	\] 
	where 
	\[
		Q_{ij} = \int \left[\frac{3}{2}r_i'r_j' - \frac{1}{2}(r')^2 \delta_{ij}\right]\rho(\mathbf r') d\tau'
	\] 
	Note that $Q_{ij}$ is a two-rank tensor, so it is possible to express it as a matrix. Also show that $Q_{ij}$
	is traceless.


	\begin{solution}
		Here we will compare this expression to the quadrupole expansion term we are normally used to show 
		equivalence. Recall that the actual way we write the quadrupole term is:
		\[
			V_{\text{quad}}(r) = \frac{1}{4\pi \epsilon_0}\frac{1}{r^3}\int (r')^2\left( \frac{3}{2}\cos^2 \alpha
			- \frac{1}{2}\right) \rho(r') d\tau
		\] 
		Comparing this with our expression, it's clear that we only need to show that:
		\[
			\sum_{i, j = 1}^3 \hat{r_i}\hat{r_j}\int \left(\frac{3}{2}r_i' r_j' -
			\frac{1}{2}(r')^2 \delta_{ij}\right)\rho(r') d\tau' = \int (r')^2\left( \frac{3}{2}\cos^2 \alpha - 
			\frac{1}{2}\right) \rho(r') d\tau'
		\] 
		Moving the integral in the sum, it means we need to show that:
		\begin{equation}
		\label{eq1}
			\sum_{i, j = 1}^3 \hat{r_i}\hat{ r_j}\left( \frac{3}{2}r_i' r_j' - 
			\frac{1}{2}(r')^2 \delta_{ij}\right)
			= (r')^2 \left( \frac{3}{2}\cos^2 \alpha - \frac{1}{2} \right)
		\end{equation} 
		Looking at the first term in particular, we want to show that:
		\[
			\frac{3}{2}\sum_{i, j = 1}^3 \hat{r_i}\hat{r_j}r_i'r_j' = (r')^2 \cos^2 \alpha
		\] 
		We can then split the left hand side into $\frac{3}{2}(\sum_i \hat{r_i}r_i' \sum_j \hat{r_j}r_j')$ and
		each term $\sum_i \hat{r_i}r_i' = r'\cos \alpha$, then we get that the total term:
		\[
			\frac{3}{2}\sum_{i, j = 1}^3 \hat{r_i}\hat{r_j}r_i'r_j' = \frac{3}{2}\sum_{i = 1}^3 \hat{r_i}r_i'
			\sum_{j = 1}^3 \hat{r_j}r_j' =  \frac{3}{2}(r'\cos \alpha)(r' \cos \alpha) = \frac{3}{2}(r')^2 
			\cos^2 \alpha
		\] 
		Now for the second term, we can see that: 
		\[
			\frac{1}{2}\sum_{i, j = 1}^3 \hat{r_i}\hat{r_j}(r')^2 \delta_{ij} = \frac{1}{2}\sum_{i = 1}^3
			(\hat{r_i})^2 (r')^2 = \frac{1}{2}(r')^2
		\] 
		So therefore, the two terms on either side of equation \ref{eq1} are the same, and thus this is an 
		equivalent way to write the quadrupole term. Furthermore, to show that $Q_{ij}$ is traceless, notice 
		that when $i = j$, then the term we get is:
		\[
			\sum_{i = 1}^3 \frac{3}{2}(r_i)^2 - \frac{1}{2}(r')^2 = \frac{3}{2}(r')^2 - \frac{3}{2}(r')^2 = 0
		\] 
		Since the diagonal elements are zero, $Q_{ij}$ is traceless. 
	\end{solution}
	\pagebreak
	\section*{Problem 2}
	Show that the quadrupole moment $Q_{ij}$ is independent of origin if the monopole and dipole moments both 
	vanish.

	\begin{solution}
		If the monopole and dipole terms vanish, this means that $p = 0$ and $Q = 0$. Suppose we picked another
		origin point $O' = O + \mathbf R$. Then, we have the following: 
		\begin{align*}
			\mathbf{r_{o'}} &= \mathbf{r_o} + \mathbf R\\
			\mathbf{r_{i, o'}} &= \mathbf{r_{i, o}} + \mathbf{R_i} \\
			\mathbf{r_{j, o'}} &= \mathbf{r_{j, o}} + \mathbf{R_j} 
		\end{align*}
		Now we can expand the quadrupole term:
		\begin{align*}
			Q_{ij}' &= \int \left[ \frac{3}{2}(r_{i, o} + Ri)(r_{j, o} + R_j) - \frac{1}{2}(\mathbf{r_o} 
				+\mathbf R)^2
			\delta_{ij}\right]\rho(\mathbf{r_0} + \mathbf R) d\tau'\\
					&= \int \left[\frac{3}{2}r_{i, o}r_{j, o} - \frac{1}{2}(\mathbf r_o)^2\right]\rho d\tau
				+ \frac{3}{2} R_j \int r_{i, o} \rho d\tau - \mathbf r_0 \cdot \mathbf R \delta_{ij} \int 
				\rho d\tau \\&\phantom{aaaaaaaa} + \frac{3}{2}R_i \int r_{i, o} \rho d\tau -
				\int \mathbf r_o \cdot \mathbf R \rho
					d\tau + \int \left[\frac{3}{2}R_i R_j  -
					\frac{1}{2}|\mathbf R|^2 \delta_{ij} \right]\rho d\tau\\
							 &= Q_{ij} + \frac{3}{2}R_j p_i + \frac{3}{2}R_ip_j - 2\delta_{ij}\mathbf r_o \cdot 
							 \mathbf R Q + \left[\frac{3}{2}R_i R_j - \frac{1}{2}|\mathbf R|^2\delta_{ij}\right]
							 Q
		\end{align*}
		And since $Q = 0$ and $p = 0$, then all the terms vanish except the first term, implying that 
		\[
			Q_{ij}' = Q_{ij}
		\] 
		And thus the quadrupole moment is independent of the origin. 
	\end{solution}

	\pagebreak
	\section*{Problem 3}
	A circular disk has a radius $R$ and uniform charge density $\sigma$. The disk is lying on the $x-y$ plane, 
	with its center fixed at the origin. Find the potential $V(\mathbf r)$ of the disk for large $r$, up to 
	the $1/r^3$ term. 

	\begin{solution}
		This problem essentially asks for the multipole expansion for this disk of radius $R$. Considering the 
		monopole term, we have:
		\[
			V_{\text{mon}}(r) = \frac{1}{4\pi\epsilon_0}\frac{Q}{r} = 
			\frac{1}{4\pi \epsilon_0}\frac{\pi R^2 \sigma}{r}
		\] 
		For the dipole term, we have:
		\[
			V_{\text{dip}}(r) = \frac{1}{4\pi\epsilon_0}\frac{1}{r^2} \int r' \cos \alpha \rho(r') d\tau'
		\] 
		Since the origin is placed at the center of the disc, then this integral becomes: 
		\begin{align*}
			\int r' \cos \alpha \sigma(r') dA &= \sigma \cos \alpha \int_0^R \int_0^{2\pi} r' r' d\phi dr'\\
											  &= 2\pi \sigma \cos \alpha \frac{R^3}{3}\\
			&= \frac{2\pi \sigma \cos \alpha R^3}{3} 
		\end{align*}
		Giving us: 
		\[
			V_{\text{dip}}(r) = \frac{1}{4\pi \epsilon_0}\frac{2\pi \sigma \cos \alpha R^2}{3r^2}
		\] 
		Similarly for the quadrupole, we need to solve:
		\[
			V_{\text{quad}}(r) = \frac{1}{4\pi \epsilon_0}\frac{1}{r^3}\int (r')^2 \left(\frac{3}{2}\cos^2 \alpha
			- \frac{1}{2}\right)\sigma(r') da
		\] 
		Again, computing the integral:
		\begin{align*}
			\int (r')^2 \left( \frac{3}{2}\cos^2 \alpha - \frac{1}{2} \right) \sigma(r') dr' &= \sigma \left[ 
			\frac{3}{2}\cos^2 \alpha \int (r')^2 r' d\phi dr'- 
		\frac{1}{2} \sigma \int(r')^2 r' d\phi dr'\right] \\
			&= \frac{3\pi \sigma \cos^2 \alpha R^4}{4} - \frac{\pi \sigma R^4}{4}
		\end{align*}
		So therefore the quadrupole term becomes: 
		\[
			V_{\text{quad}}(r) = \frac{1}{4\pi \epsilon_0}\left( \frac{3\pi \sigma \cos^2 \alpha  R^4}{4r^3} -
			\frac{\pi \sigma R^4}{4r^3} \right) 
		\] 
		Finally, since $\alpha$ is defined as the angle between the plane and the radius vector $\mathbf r$, 
		we can write $\alpha = \frac{\pi}{2} - \theta$, therefore we can express the potential in terms of the
		natural spherical coordinates $V(r, \theta)$. Combining all three terms, we get:
		\[
		V(r, \theta) = \frac{1}{4\pi \epsilon_0}\frac{\pi R^2 \sigma}{r} + \frac{1}{4\pi \epsilon_0}\frac{2\pi\sigma}{3r^2}\cos\left( \frac{\pi}{2} - \theta \right) + \frac{1}{4\pi \epsilon_0}\left( \frac{3\pi\sigma R^4}{4r^3}\cos^2\left( \frac{\pi}{2} - \theta \right) - \frac{\pi \sigma R^4}{4r^3} \right) 
		\] 
		We can cancel off the $\pi$ terms:
		\[
		 V(r, \theta) = \frac{R^2 \sigma}{4 \epsilon_0r} +
		 \frac{\sigma}{6 \epsilon_0 r^2}\cos\left( \frac{\pi}{2} -
			\theta \right)  + \frac{1}{4\epsilon_0 r^3}\left(\frac{3 \sigma R^4}{4} \cos^2\left( \frac{\pi}{2}- 
			\theta \right) - \frac{\sigma R^4}{4}\right)  \\
		\]
	\end{solution}
\end{document}
