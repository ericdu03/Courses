\documentclass[10pt]{article}
\usepackage{../local}


\newcommand{\classcode}{Physics 110A}
\newcommand{\classname}{Electromagnetism and Optics}
\renewcommand{\maketitle}{%
\hrule height4pt
\large{Eric Du \hfill \classcode}
\newline
\large{HW 08} \Large{\hfill \classname \hfill} \large{\today}
\hrule height4pt \vskip .7em
\normalsize
}
\linespread{1.1}
\begin{document}
	\maketitle
	\section*{Problem 1}
	Consider a parallel capacitor where the plates have area $A$ and are separated by a distance $d$. The gap is
	filled with a linear dielectric material of electric susceptibility $\chi_e$. Suppose the capacitor is 
	charged so that the plates carry free charges $\pm Q_f$.
	\begin{enumerate}[label=\alph*), start]
		\item Find the bound charge $Q_b$ originated from the polarization in terms of $Q_f$ and $\chi_e$.

			\begin{solution}
				We can calculate the bound charge from the polarization using $\rho_b = - \div P$ and since 
				$P = \epsilon_0 \chi_e E = \epsilon_0 \chi_e \frac{D}{\epsilon}$, then we get:
				\[
				\rho_b = -\epsilon \chi_e \left( \div \frac{D}{\epsilon} \right) = -\frac{\epsilon_0 \chi_e}{
				\epsilon_0(1 + \chi_e)}(\div D)
				\] 
				Then, since $D$ is related to $Q_f$ via $\div D = \frac{Q_f}{A}$, we get: 
				\[
				Q_b = -\frac{\chi_e}{1 + \chi_e}Q_f
				\] 
			\end{solution}
		\item Find the \textit{net} electric field (caused by both the free and bound charges) between the 
			plates. Express your answer in terms of $Q_f, A$ and $\epsilon$, where $\epsilon = (1 + \chi_e)
			\epsilon_0$

			\begin{solution}
				First, we calculate the $D$ field:
				\[
				\oint D \cdot da = Q_f = DA \implies D = \frac{Q_f}{A}
				\] 
				It's obvious that $D$ points in the $x$ direction, so therefore $\vec D = \frac{Q_f}{A}\hat{x}$.
				Furthermore, we know that $\vec E = \frac{\vec D}{\epsilon}$, so therefore:
				\[
				E = \frac{Q_f}{\epsilon A} \hat{x}
				\] 
			\end{solution}
		\item Compute the work needed to establish the system from zero free charge to a final free charge 
			$Q_f$ by gradually moving infinitesimal free charges $dq_f$ from the right plate to the left plate,
			overcoming the electric potential of the net field. Express your answer in terms of $Q_f$, 
			$\epsilon, A$ and $d$. 
			
			\begin{solution}
				We derive the capacitance of a parallel plate capacitor as: $C = \frac{\epsilon A}{d}$, and 
				we also know that $C = \frac{Q}{V}$. Since the total charge initially is only due to the bound 
				charges $q_b$, then can write the potential as: 
				\[
				V = \frac{Q_b}{C} = \frac{Q_b d}{\epsilon A}
				\]
				Furthermore, at any point, the free charges that are added on also contribute to the total charge
				on the capacitor, so in fact, we have a general expression for the potential as a function of 
				the charge $q$ on the capacitor: 
				\[
				V(q) = \frac{qd}{\epsilon A}
				\] 
				Then, we now just have to integrate over the total charge: 
				\[
					W = \int_0^{Q_f} V(q) dq = \int_0^{Q_f} \frac{qd}{\epsilon A} dq = \frac{Q_f^2 d}{2 
					\epsilon A}
				\] 
			\end{solution}
		\item Calculate $W = \dfrac 12 \int D \cdot E d\tau$ for this system. It should agree with your result 
			in part (c)

			\begin{solution}
				We have expressions for $D$ and $E$, so we just have to do the integral:
				\begin{align*}
					W &= \frac{1}{2}\int D \cdot E d\tau \\
					&= \frac{1}{2} \int \frac{Q_f^2}{\epsilon A^2} d\tau \\
					&= \frac{1}{2}\frac{Q_f^2}{\epsilon A^2}(Ad) \\
					&= \frac{Q_f^2 d}{2\epsilon A}
				\end{align*}
			\end{solution}
	\end{enumerate}
	The difference between $W = \dfrac 12 D \cdot E d\tau$ and $W = \dfrac{\epsilon_0}{2}\int E \cdot E d\tau$ 
	is that the latter only takes into account the energy needed to put the charges in place, while ignoring
	the potential energy associated with \textit{microscopic} interactions between the molecules, or within the
	molecules, which is wt we neeed to overcome when we polarize the material. Specifically, consider the 
	following two-step process. First, we separate the free charges $\pm Q_f$ from each other as shown in 
	the right figure. 

	\begin{enumerate}[label=\alph*), resume]
		\item What is the energy needed to separate the $\pm Q_f$ charges at a distance $d$ apart as shown above?

			\begin{solution}
				To separate just the free charges, we look at the total potential energy of the capacitor with 
				charges $Q_f$. We know that $U = \frac{1}{2}CV^2$ with $C = \frac{Q_f}{V}$, so therefore: 
				\begin{align*}
					U = \frac{1}{2}\frac{Q_F}{V} V^2 &= \frac{1}{2} Q_f V\\
					&= \frac{1}{2}Q_f \frac{Q_fd}{\epsilon_0A} \\
					&= \frac{Q_f^2 d}{2 \epsilon_0 A}
				\end{align*}
			\end{solution}
	\end{enumerate}

	We now separate the bound charges by pulling the $\pm Q_b$ apart from each other within the two charged 
	plates. Along the way, we need to overcome the force between the bound charges, and the force exerted
	on $+Q_b$ due to $\pm Q_f$

	\begin{enumerate}[label=\alph*),resume]
		\item What is the energy needed to separate the $\pm Q_b$ charges a distance $d$ apart as shown? 

			\begin{solution}
				The energy needed is equal to the difference in energy of the final and the initial 
				configurations. In the initial system, the charges $\pm Q_b$ reside on the same plate, so the 
				potential energy here is just equal to the potential energy of the configuration with $Q_f$: 
				\[
				U_i = \frac{Q_f^2d}{2 \epsilon_0 A}
				\] 
				After the charges are moved to their locations, the net charge on each plate is now $Q_f - Q_b$,
				so the final potential is: 
				\[
				U_f = \frac{(Q_f - Q_b)^2d}{2 \epsilon_0 A}
				\] 
				Therefore, the energy required is the difference of these two: 
				\[
				\Delta U = U_f - U_i = \frac{d}{2\epsilon A}((Q_f - Q_b)^2 - Q_f^2) = \frac{d}{2\epsilon_0A}
				(Q_b^2 - 2Q_bQ_f)
				\] 
			\end{solution}
		\item Compute the $W = \frac{\epsilon_0}{2}\int E \cdot E d\tau$ for the final configuration. It should
			equal to the sum of part (e) and (f). 

			\begin{solution}
				Here, we know the $E$ field is equal to $E = \frac{Q_f - Q_b}{\epsilon_0A}$ since the total 
				charge on the plates is $Q_f - Q_b$, so therefore: 
				\begin{align*}
					W &= \frac{\epsilon_0}{2} \int \frac{(Q_f - Q_b)^2}{\epsilon_0^2A^2}d\tau \\
					  &= \frac{\epsilon_0}{2} \frac{(Q_f - Q_b)^2}{\epsilon_0^2A^2} Ad \\
					  &= \frac{(Q_f - Q_b)^2d}{2\epsilon A}
				\end{align*}
				Summing up parts (e) and (f), we get: 
				\[
				U_i + \Delta U = \frac{Q_f^2d}{2\epsilon_0A} + \frac{(Q_b^2 - 2Q_b Q_f)d}{2\epsilon_0A} = 
				\frac{(Q_f^2 - 2Q_fQ_b + Q_b^2)d}{2\epsilon_0A} = \frac{(Q_f^2 - Q_b^2)}{2\epsilon_0A}
				\] 
				This matches what we get from computing $W = \frac{\epsilon_0}{2} \int E \cdot E d\tau$, as 
				desired. 
			\end{solution}
	\end{enumerate}
	
	\pagebreak
	\section*{Problem 2}
	A thick spherical shell (inner radius $a$, outer radius $b$) is made of dielectric material with a permanent
	polarization
	\[
		\mathbf{P(r)} = \frac{k}{r}\hat{\mathbf r}
	\] 
	where $k$ is a constant and $r$ is the distance from the center. (There is no free charge in the problem.)
	Find the electric field in all three regions, $r < a$, $a < r < b$, $r > b$. 


	\begin{solution}
		Because there is no free charge in the problem, it's convenient here to use Gauss' law: 
		\[
		Q_f = \oint \mathbf D \cdot da = 0
		\] 
		Because this integral is zero then, this also implies that $\mathbf D = 0$ everywhere. 
		Then, since $\mathbf D = \epsilon \mathbf E + \mathbf P$, we can rearrange this to get $\mathbf E =
		-\frac{\mathbf P}{\epsilon_0}$ within the 
		region of the spherical shell. Inside the shell, since there is no charge, we know that $\mathbf E = 0$. 
		Outside
		the shell, we know that there is no polarization (since there isn't any polarized material), so 
		$\mathbf E = 0$
		outside the shell as well. Therefore, combining this together we get: 
		\[
		E(r) = \begin{cases}
			0 & r < a\\
			-\dfrac{k}{\epsilon_0 r} \hat{\mathbf r}&  a < r < b \\
			0 & r > b 
		\end{cases}
		\] 
	\end{solution}

	\pagebreak	

	\section*{Problem 3}
	An uncharged conducting sphere of radius $a$ is coated with a thick insulating shell out to radius $b$. The 
	insulating shell has a permanent polarization $\mathbf P = P_0 \hat{z}$. Find the electric potential in the 
	three regions, $r < a$. $a < r < b$, and $r > b$. 

	\begin{solution}
		Since we are dealing with a spherical conductor (I assume that we have a solid sphere here), then we know
		that the potential within the conductor must be zero simply due to the properties of a conductor. 

		For the region $a < r < b$, we need to be more careful. Firstly, we can calculate the surface and bound
		charges:
		\begin{align*}
			\rho_b &= \div P = \frac{P_0}{r \sin \theta}\pdv{\theta} (-\sin^2 \theta) + \frac{1}{r^2}\pdv{r}
			(r^2 \sin \theta)\\
			&= -\frac{2P_0 \cos \theta}{r} + \frac{2 \cos \theta P_0}{r} \\
			&= 0 \\
			\sigma_b &= \vec P \cdot \hat{n} = \vec P \cdot r \\
			&= P_0 \cos \theta \hat{n}
		\end{align*}
		Further, if we consider the surface of the insulating region, we get that the normal vector $\hat{n}$
		points inwards at $r = a$ and outwards at $r = b$, so at $r = a$, $\sigma_b = -P_0 \cos \theta$ and 
		at $r = b$, $\sigma_b = P_0 \cos \theta$. Qualitatively speaking, this means that the potential 
		within this region is the same potential as two concentric shells of bound charges with a distribution
		of $P_0 \cos \theta$, with the outer shell consisting of positive charges and the inner one consisting
		of negative charges. To solve this potential, we solve each case separately using the general 
		solution for $V(r)$, then use the superposition principle to find the total potential. 

		Let's start with the shell of radius $b$, which has charge density $\sigma_b = P_0 \cos \theta$. Recall
		that the general form for the potential is: 
		\[
			V(r) = \sum_{l = 0}^\infty \left( A_l r^l + \frac{B_l}{r^{l+1}} \right) P_l(\cos \theta)
		\] 
		where $P_l$ denotes the $l$-th Legendre polynomial. We want to find the coefficients $A_l$ and $B_l$.
		Inside the shell, we know that the potential cannot blow up to infinity, so none of the $B_l$ terms
		can exist. Similarly, outside the shell, we require that the potential is 0 at infinity, so the $A_l$
		terms cannot exist outside. Therefore, the potential is of the form: 
		\[
		V(r) = \begin{cases}
			\displaystyle	\sum_{ l= 0}^\infty \frac{B_l}{r^{l+1}}P_l(\cos \theta) & r > b\\
			\\
			\displaystyle \sum_{l = 0}^\infty A_l r^l P_l (\cos \theta) & r < b
		\end{cases}
		\] 
		Now, there are two conditions we need to satisfy with this potential. Firstly, we require that
		\[
			\pdv{V}{n}\bigg|_{\text{below}}^{\text{above}} = \frac{\sigma}{\epsilon_0} = \frac{P_0 \cos \theta}{
			\epsilon_0}
		\] 
		So, computing the derivative with respect to $r$ for $V(r)$ at $r = b$ and equating them to each other, 
		we get:
		\begin{align*}
			\pdv{V_{out}}{r} - \pdv{V_{in}}{r} &= \sum_{l = 0}^\infty - \frac{(l + 1)B_l}{b^{l + 2}}P_l (\cos 
			\theta) - \sum_{l' = 0}^\infty l' A_{l'} b^{l' -1} P_{l'}(\cos \theta)\\
			\frac{P_0 \cos \theta}{\epsilon_0} &=  \sum_{l = 0}^\infty - \frac{(l + 1)B_l}{b^{l + 2}}P_l
			(\cos \theta) - \sum_{l' = 0}^\infty l' A_{l'} b^{l' -1} P_{l'}(\cos \theta)
		\end{align*}
		The left hand side only has a linear $\cos \theta$ term which corresponds to the first legendre
		polynomial $P_1(x) = x$, so therefore the right hand simplifies massively, since $l = l' = 1$ is 
		the only term that is nonzero. Therefore, we're left with the equation:
		\begin{align*}
			\frac{P_0 \cos \theta}{\epsilon_0} &=  -\frac{2B_1}{b^3}\cos \theta - A_1 \cos \theta \\
			\frac{P_0}{\epsilon_0} &= -\frac{2B_1}{b^3} - A_1
		\end{align*}
		To solve for $B_1$ and $A_1$, we now use the fact that $V(r)$ must be differentiable at $r = b$, so 
		at $r = b$, we require that $V_{out}(b) = V_{in}(b)$: 
		\begin{align*}
			\frac{B_1}{b^2}\cos \theta &= A_1 b \cos \theta\\
			\therefore A_1 &= \frac{B_1}{b^3}
		\end{align*}
		Substituting this back into the previous expression, we have: 
		\begin{align*}
			\frac{P_0}{\epsilon_0} &= -\frac{2B_1}{b^3} - \frac{B_1}{b^3} \\
			&= -\frac{3B_1}{b^3} \\
			\therefore B_1 &= -\frac{P_0b^3}{3\epsilon_0}, \ \ A_1 = -\frac{P_0}{3 \epsilon 0}
		\end{align*}
		This approach is identical for the shell of radius $r = a$ except we have the opposite charge density, 
		so from there we get:
		\begin{align*}
			A_1 &= \frac{P_0}{3 \epsilon_0} \\
			B_1 &= \frac{P_0a^3}{3\epsilon_0}
		\end{align*}
		Now we can begin the process of joining them together. To do so, notice that for $a < r < b$, we want
		the solution outside the shell of radius $a$, but inside the shell for radius $b$, Therefore, 
		\begin{align*}
			V(r) &= V_{\text{out, a}}(r) + V_{\text{in, b}}(r) \\
			&= \frac{P_0}{3\epsilon_0}r \cos \theta + \frac{1}{r^2}\left( -\frac{P_0b^3}{3 \epsilon_0} \right) 
			\cos \theta \\
			&= \frac{P_0 \cos \theta}{3\epsilon_0}\left( r - \frac{b^3}{r^2} \right)  \\
			&= \frac{P_0 r \cos \theta }{3\epsilon_0}\left( 1 - \frac{b^3}{r^3} \right) 
		\end{align*}
		For the region $r > b$, we want the solution outside both shells:
		\[
		 V(r) = \frac{P_0a^3}{3\epsilon 0 r^2}\cos \theta - \frac{P_0b^3}{3\epsilon_0r^2}\cos \theta 
		 = \frac{P_0 \cos \theta}{3\epsilon_0r^2}(a^3 - b^3)
		\] 
		Interestingly enough, since we use the solution inside both shells for $r < a$, we get that in this
		region: 
		\[
		V(r) = -\frac{P_0}{3\epsilon_0}r \cos \theta - \left( -\frac{P_0}{3\epsilon_0} \right) r \cos \theta = 0
		\] 
		which matches the result we got earlier. However, we shouldn't use this approach here since at $r < a$
		we have a conductor rather than empty space, which is what solving for the potential at $r < a$ 
		in this fashion describes. Finally, we can write:
		\[
		V(r) = \begin{cases}
			 0 & r< a\\
			 \displaystyle \frac{P_0 r \cos \theta}{3\epsilon_0}\left( 1 - \frac{b^3}{r^3} \right) & a < r < b\\
			 \displaystyle \frac{P_0 \cos \theta}{3\epsilon_0r^2}(a^3 - b^3) & r > b
		\end{cases}
		\] 
	\end{solution}
\end{document}
