\chapter{Lecture 14}
Lecture 14 was held on \textbf{Wednesday Feb 22nd, 2022}. It finished the example problem that we started last
lecture and also covered the \textbf{Laplace Equation in Spherical Coordinates}

\section{Last Time: Two Planes}
Recall from last time that we have the solution 
\[ V(x, y) = \sum_n c_n e^{-\frac{n \pi x}{a}} \sin \left( \frac{n \pi y}{a} \right) \] 
which we've obtained by imposing the following conditions:
\begin{enumerate}
		\item $V = 0$ at $y = 0$, getting rid of the cosine term
		\item $V =0$ at $y =a$, giving us the quantization of the wavelength
		\item $V \to 0$ as $x \to \infty$, giving us no exponential growth.
\end{enumerate}
Now we will impose our final boundary condition: at $x = 0$, we have $V = V_0(y)$. So plugging in $x = 0$, we see
that: 
\[ V(0, y) = \sum_n c_n \sin\left( \frac{n \pi y}{a} \right) \]
\begin{insight*}{}
	Note that this is actually only possible because the set of functions form a complete basis, and they are 
	orthonormal, so the sum of sines can represent any arbitrary function $V_0(y)$. Again, recall the 
	orthonormality condition: 
	\[ \braket{\sin\left( \frac{n \pi y}{a} \right)}{\sin\left( \frac{m \pi y}{a} \right) } \equiv \int_0^a
	dy \sin\left( \frac{n \pi y}{a} \right) \sin\left( \frac{m \pi y}{a} \right) = \begin{cases}
			0 & n \neq m\\
	\frac{a}{2} & n = m
\end{cases}\]
\end{insight*}
and so we can see from this that we can grab each coefficient by writing: 
\[ c_n = \frac{2}{a}\braket{\sin\left( \frac{n \pi y}{a} \right) }{V_0(y)} = \frac{2}{a}\int_0^a c_n 
e^{-\frac{n\pi x}{a}} \sin\left( \frac{n \pi y}{a} \right) \]
\begin{insight*}{}
		We also see this sinusoidal behavior with other partial differential equations as well: 
		\begin{align*}
				\left(\partial_x^2 - \frac{1}{v^2}\partial_t^2\right) \phi &= 0 & \text{(Wave Equation)}\\
				\partial_t \phi &= \laplace \phi & \text{(Flow Equation)}
		\end{align*}
		These equations all share the property that they have a second order differential which equals itself, 
		which generally means that the solution is some form of complex exponential ($e^{ikx}$ of some kind). This
		is also useful because  we know that $e^{ikx}$ terms form a complete orthonormal basis, so we can model any
		arbitrary function with these exponentials as well.  
\end{insight*}
\section{Laplace's Equation in Spherical Coordinates}
Here, the differential form of Laplace's equation remains the same $\laplace \phi = 0$, but the Laplacian is 
different since we're now working in spherical coordinates:
\[ \laplace V = \frac{1}{r}\pdv{r}\left( r^2 \pdv{V}{r} \right) + \frac{1}{r^2 \sin \theta}\pdv{\theta}
\left( \sin \theta \pdv{V}{\theta} \right)  + \frac{1}{r^2 \sin \theta}\pdv[2]{V}{\phi}\]
For simplicity, we assume azimuthal symmetry, so $V$ does not depend on the $\phi$ coordinate. If it did, then 
we'd have to impose the boundary condition that $\phi(0) = \phi(2\pi)$, which would imply a solution of the form
$e^{in\phi}$. Because there is no $\phi$ coordinate, then we know that $\pdv{V}{\phi} =0$. Writing out the other
two terms: 
\[ \frac{1}{r^2} \pdv{r} \left( r^2 \pdv{V}{r} \right) + \frac{1}{r^2 \sin \theta}\pdv{\theta}
\left( \sin \theta \pdv{V}{\theta} \right) \]
We now consider the separable solution  $V(r, \theta) = R(r) \Theta(\theta)$. Plug this ansatz back into the 
the equation and multiplying by $r^2/R\Theta$, we get: 
\[ \frac{1}{R}\pdv{r}\left(r^2 \pdv{R}{r}\right) + \frac{1}{\sin^2 \theta \Theta}\pdv{\theta}
\left( \sin \theta \pdv{\Theta}{\theta} \right) = 0\]
And so just like last time, each one of these terms must be constant, otherwise we can fix one and vary the other
which wouldn't give us 0 in general. For later convenience, we will define the constant as $l (l+1)$, so we write:
\begin{align*}
		\frac{1}{R}\pdv{r}\left( r^2 \pdv{R}{r} \right) &= l(l+1)\\
		\frac{1}{\sin^2 \theta}\pdv{\theta}\left( \sin \theta \pdv{\Theta}{\theta} \right) &= -l(l+1)
\end{align*}
Looking at the radial equation, we have:
\[ \dv{r}\left( r^2 \dv{R}{r} \right) = l(l+1)R\]
This differential equation asks us to take the function $R$ and take the derivative twice, multiply by $r^2$ and 
it's supposed to give us $R$ back. This hints at a solution of the form $R = r^m$, since the exponent is returned
after differentiating twice. So plugging this back in: 
\[ \dv{r}\left( r^2 \dv{r} r^m \right) = m(m+1)r^m\] 
Giving us 
\begin{align*}
		m(m+1)r^m &= l(l+1)r^m\\
		m^2 + m - l(l+1)&= 0\\
		(m-l)(m+(l+1)) = 0 \implies m &= l \text{ or } m = -(l+1)
\end{align*}
And so therefore we can write $R$ as a linear combination of these two values for $m$: 
\[ R(r) = \sum_l \left( A_l r^l + B_l \frac{1}{r^{l+1}} \right) \]
\begin{insight*}{}
		So far, $l$ could still take on any value, so even though this is a summation notation we haven't set any
		restriction on $l$. It's the angular equation that will set the restriction on $l$.
\end{insight*}
So now let's look at the angular equation:
\[ \dv{\theta} \left( \sin \theta \dv{\Theta}{\theta} \right) = -l(l+1) \sin \theta \Theta \]
This differential equation, unlike the previous one, is highly nontrivial to solve. However, this differential 
equation has been solved previously and is well known as the Legendre equation, which has the form: 
\[ \dv{x}\left[(1-x^2) \dv{P_n(x)}{x}\right] + n(n+1)P_n(x) = 0\]
the solution to this is given by Rodrigues' formula, where 
\[ P_n(x) = \frac{1}{2^n n!}\left( \dv{x} \right)^n (x^2-1)^n\]
\begin{notation*}{Legendre Polynomials}
		We denote $P_n(x)$ as the $n$-th Legendre polynoimal.
\end{notation*}
In our case, we have $x = \cos \theta$ and $\dv{\theta}{x} = \frac{1}{-\sin \theta}$, so therefore the Legendre
equation reads:
\[ \frac{1}{\sin \theta}\dv{\theta} \left[ \sin \theta \dv{P_n(\cos \theta)}{\theta}\right] + n(n+1)P_n(\cos
\theta) = 0\] 
which is exactly what our original differential equation looks like! Therefore, we now know that $\Theta(\theta) =
P_l(\cos \theta)$. 
\begin{insight*}{}
		As we can see, it is merely the fact that we would like non-singular solutions (i.e. non-constant) 
		solutions between $\theta =0$ and $\theta = \pi$ that constrains $l$ to be a nonnegative integer. Note 
		that $0$ is allowable in this case because the Rodrigues' formula allows us to generate $P_0(x)$.
\end{insight*}
Combining this with our solution to the radial equation, we get the full solution as:
\[ V(r, \theta) = \sum_{l = 0}^\infty \left(A_lr^l + B_l \frac{1}{r^{l+1}}\right) P_l(\cos \theta)\]
This is the solution to the electrostatic potential of a charge distribution, given azimuthal symmetry. As
a recap of Legendre polynomials, we have:
\begin{definition*}{}
		The Legendre polynomials are solutions to the Legendre equation, and are given by the formula:
		\[ P_l(x) = \frac{1}{2^ll!}\left( \dv{x} \right)^l (x^2 - 1)^l\]
		The first few Legendre polynomials are: 
		\begin{align*}
				P_0(x) &= 1\\
				P_1(x) &= x\\
				P_2(x) &= \frac{1}{2}(3x^2 -1)\\
				P_3(x) &= \frac{1}{2}(5x^3 - 3x)\\
				P_4(x) &= \frac{1}{8}(35x^4 - 30x^2 + 3)
		\end{align*}
		A cool thing to note is that when $l$ is odd, we only get odd exponents and we also only get even ones 
		when $l$ is even. This means that odd values of $l$ correspond to antisymmetric solutions and even ones
		correspond to symmetric solutions, a property we will explore further in next lecture.
\end{definition*}


