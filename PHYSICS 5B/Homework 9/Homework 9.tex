\documentclass{article}
\usepackage{amsmath}
\usepackage{mathtools}
\usepackage{amsfonts}
\usepackage{amssymb}
\usepackage{amsthm}
\usepackage{fancyhdr}
\usepackage{float}
\usepackage{epigraph}
\usepackage{caption}
\usepackage{esint}

%Page formatting
\lhead{Eric Du}
\chead{Homework 9}
\rhead{\today}
\pagestyle{fancy}
\cfoot{\thepage}
\title{Homework 9}
\author{Eric Du}
\date{\today}

%.sty file handling
\usepackage[sexy]{evan}
\usepackage{tcolorbox}
\usepackage{xcolor}
\renewcommand{\labelitemi}{\textendash}
\renewcommand{\abstractname}{}
\theoremstyle{definition}
\newtheorem*{solution}{\color{blue}Solution}
\numberwithin{equation}{section}
\numberwithin{definition}{section}

%Paragraph Formatting
\setlength{\epigraphwidth}{148pt}
\setlength{\parindent}{0pt}
\linespread{1.3}
\allowdisplaybreaks

%TikZ special settings
\usepackage{circuitikz}
\usetikzlibrary{patterns}
\usetikzlibrary{shapes.geometric}
\usetikzlibrary{decorations.markings}


\begin{document}
\maketitle


\begin{abstract}
\noindent \textbf{[NOTE]:} To complete this homework I worked with \textbf{Andrew Binder} and \textbf{Aren Martinian}.
\end{abstract}


\section{Problem 1}


\subsection*{Part a}

Refer to the following for the to-scale version, provided by Andrew. Though this diagram is drawn using computer graphics software, we have drawn the diagram according to the rules that were taught in class. 

% Insert tikz code here


\subsection*{Part b}

To calculate these values algebraically, we use the thin lens equation:

\begin{align*}
    \frac{1}{f} &= \frac{1}{d_o} + \frac{1}{d_i}\\
    \frac{1}{d_i} &= \frac{1}{f} - \frac{1}{d_o}\\
    \therefore d_i &= \frac{fd_o}{d_o - f}\\
    &= -6.67
\end{align*}

Looking at our diagram, this calculation checks out, and it's confirmed by the diagram that we drew above. We can see that the position of the image relative to the lens is less than -10, but also greater than 5, which matches pretty well with the result we got from algebra alone. Calculating the magnification:

\begin{align*}
    M &= -\frac{d_i}{d_o}\\
    &= \frac{1}{3}
\end{align*}

Again, we can see that this value also makes sense in terms of our diagram - the arrow height seems to have been reduced to approximately one third of its original height, hence the negative value. It's important to note that we can still claim our diagram confirms our calculations, since even though it is digitally created it was created according to the rules of optics that we learned in class.


\subsection*{Part c}

In order to generate these corrective lenses, we need to make an object that is actually 25 cm away to appear as if it were 60 cm away or further. Thus, we can set $d_o = 25$ cm and $d_i = -60$ cm (remember that $d_o$ is always chosen to be positive by convention), and via the thin lens equation we get:


\begin{align*}
    \frac{1}{f} &= \frac{1}{25} - \frac{1}{60}\\
    \therefore f &= \frac{300}{7} = 42.96 \ \text{cm}
\end{align*}

This result makes sense, since we expect that for hyperopic people, we would require lenses with high focal lengths, and the exact opposite would be true for those who are nearsighted. 

\subsection*{Part d}


% Write this out later

\section{Problem 2}

We split this section into two parts: first we tackle the image created by the converging lens, then we calculate it from the diverging lens. From the converging lens, we have the thin lens equation:

\[ \frac{1}{d_o} + \frac{1}{d_i} = \frac{1}{f} \]

Note that we also have the magnification equation that we can use, since we know that the image size is exactly twice the size of the actual candle.

\[ M = -\frac{d_i}{d_o} = -2\]

Solving this system of equations isn't too difficult - all we need to do is substitute the latter equation into the former, then we have an equation in terms of a single variable, which can be solved using a multitude of methods. At the end of all this, we obtain the following relationship:

\[ \begin{cases}
    d_o &= 15 \text{cm}\\
    d_i &= 30 \text{cm}
\end{cases}\]


% There's some symmetry argument here that I don't udnerstand?

\section{Problem 3}

\subsection*{Part a}

To achieve this we would place the light source at the very left, then followed by the condenser lens and finaly the field lens before the light hits the screen. The field lens must be placed past the focal point of the condenser lens, so that light rays that hit at different angles are slightly scattered. Then, when the light passes thhrough the field lens, light rays at differnet angles are accordingly scattered, producing the pattern we want.


\subsection{Part b}

In order to produce a sharp area, it makes sense that we need to focus the light somewhere so that it can be detected. As a result, if the iris were a converging lens and placed between the field lens and the image, it would focus the incoming light in such a way that it would focus onto a single point, which it can than be observed.

\section{Problem 4}


Here we use a diagram to illustarate 

\end{document}