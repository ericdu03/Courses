\documentclass{article}
\title{Physics 5B Homework}
\author{Eric Du}
\date{\today}
\usepackage[cm]{sfmath}
\usepackage{amsmath}
\usepackage{mathtools}
\usepackage{amsfonts}
\usepackage{amssymb}
\usepackage{amsthm}
\setlength{\parindent}{0pt}
\linespread{1.3}
\allowdisplaybreaks
\usepackage{fancyhdr}
\pagestyle{fancy}
\cfoot{\thepage}
\usepackage{float}
\lhead{Eric Du}
\chead{Physics 5A Homework}
\rhead{\today}
\usepackage{epigraph}
\setlength{\epigraphwidth}{148pt}
\usepackage{color}
\renewcommand{\labelitemi}{\textendash}
\renewcommand{\abstractname}{}
\renewcommand{\familydefault}{\sfdefault}
\usepackage[sexy]{evan}
\theoremstyle{definition}
\newtheorem*{solution}{\color{blue}Solution}
\usepackage{caption}
\numberwithin{equation}{section}
\numberwithin{definition}{section}
\begin{document}

\maketitle
\begin{abstract}
    \noindent \textbf{[NOTE:]} I primarily worked with \textbf{Andrew Binder}, \textbf{Rebecca Feng} and \textbf{Aren Martinian} to complete this assignment.
\end{abstract}


\section{Problem 1}

\subsection{Part a}
It's important to note that due to symmetry, our electric field 
We take two cases: namely $r > R$ and $r < R$.



\end{document}