\documentclass{article}
\usepackage{amsmath}
\usepackage{mathtools}
\usepackage{amsfonts}
\usepackage{amssymb}
\usepackage{amsthm}
\usepackage{fancyhdr}
\usepackage{float}
\usepackage{epigraph}
\usepackage{caption}
\usepackage{esint}

% Page formatting
\lhead{Eric Du}
\chead{Prelab 0}
\rhead{\today}
\pagestyle{fancy}
\cfoot{\thepage}
\title{Physics 5CL Prelab 0}
\author{Eric Du}
\date{\today}

%.sty file handling
\usepackage[sexy]{evan}
\usepackage{tcolorbox}
\usepackage{xcolor}
\renewcommand{\labelitemi}{\textendash}
\renewcommand{\abstractname}{}
\theoremstyle{definition}
\newtheorem*{solution}{\color{blue}Solution}
\numberwithin{equation}{section}
\numberwithin{definition}{section}

%Paragraph Formatting
\setlength{\epigraphwidth}{148pt}
\setlength{\parindent}{0pt}
\linespread{1.3}
\allowdisplaybreaks

%TikZ special settings
\usepackage{circuitikz}
\usetikzlibrary{patterns}
\usetikzlibrary{shapes.geometric}
\usetikzlibrary{decorations.markings}

\begin{document} 
\begin{abstract}
    \textbf{[NOTE:]} Although we have separate prelab submissions, I worked extensively with \textbf{Andrew Binder} to complete this prelab, so that is the reason why our work shares relatively similar styles.
\end{abstract}
\section{Problem 1: Brick Geometry}

\textit{Consider the geometry of a ray of light incident on a rectangular brick of thickness t as shown. You may assume that the index of refraction of the surrounding air is 1.}
\begin{parts}
\Part \textit{Determine $z$ in terms of the lenghts $t$ and $y$ (you just need geometry for this part - no Snell's law is allowed)}

\Part \textit{Use propagation of errors to determine $\alpha_x$ as a function of $\theta_1$ and $\alpha_\theta_1$. Then determine $\alpha_z$ as a function of $t, y,\alpha_t$ and $\alpha_y$.} 

\Part \textit{Show that the geometry of reflection and refraction implies that $\theta_1 = \theta_3 = \theta_4 = \theta_5$}

\Part \textit{Use Snell's law and your result from (a) to determine the index of refraction of the brick $n_{brick}$ first in terms of $x$ and $z$ and then in terms of $t, y$ and $\theta$.}
\end{parts}
\end{document}
