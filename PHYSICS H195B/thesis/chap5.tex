\chapter{Discussion}
This chapter is dedicated to a brief discussion about the results we obtained in \cref{results}. In
particular, the conclusion that we arrived that large rounded bends result in a worse transmission than that
of a smaller bend is extremely interesting, and opens the door to questions about what other large-scale
features in the design of a CPW we should be avoiding to achieve an ideal transmission.    
Furthermore, there are also other questions yet to be answered, such as the effect on the CPW when we start
to add more than one bend. In addition to fully simulating the slow-wave structure presented in 
\cite{hosaengkimWireBondFreeTechnique2009}, these are ultimately questions that we didn't have time to get
to, which we will definitely investigate in the future following the conclusion of this honors thesis.

There is also the question of how these modifications we investigate behave in the real world. As the results
we've concluded here are only based on idealized simulation results, it would be interesting, 
and also worthwhile, to     
fabricate a real device and prove that the effects we observe here are actually real. Results obtained from
physical device testing will be invaluable to furthering our research into the design of the feedline, as it will be
able to show which modifications among the ones we simulated have the greatest effect experimentally.        

I want to conclude with some thoughts about my progress over the past two semesters. Overall, despite having
failed in our initial goal of investigating the slow-wave structure, I am nevertheless extremely proud of
what I've been able to accomplish over the past year. Throughout the past year, being involved in this
project has really taught me \textit{how} to tackle problems in the field of research. In particular, it's
given me lots of practice in identifying problems, forming hypotheses, then having the ability to devise a
plan to test that hypothesis. Mastery of these skills are essential to become a good researcher, and I am
thankful to have had this opportunity.

I want to particularly thank Dr. Yen-Yung Chang and Prof. Matt Pyle for agreeing to sponsor me and supervise me
through this project, especially given the short time frame I gave you when I decided to move forward with an
honors thesis. Without your support this thesis would not have been possible, and for that I am infinitely
grateful. 





