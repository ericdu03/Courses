\chapter{Introduction}
Here, we introduce the origins of the theory of dark matter. For this section, most of the history is sourced
from \cite{bertoneHistoryDarkMatter2018}. 

\section{History}
To begin, we should ask ourselves the question: what is dark matter? The very origins of dark matter stem all
the way back to Lord Kelvin, who came up with a way to calculate the velocity distribution of stars in the
Milky Way galaxy. In this calculation, Kelvin noted that the number of stars that could reasonably exist in
our Milky Way is around \( 10^{9} \), but since the observed number of stars is far less than that, he
concluded that many of the stars in our galaxy must be "dark bodies". This then prompted Henri Poincare to
coin the term "dark matter". At the time, it was believed by notable astronomers
that dark matter was composed mainly of faint stars and other "meteoric matter" -- matter composed primarily
of bodies like asteroids that don't radiate light, and are too faint to detect with instruments at the time.
Moreover, people generally believed that though dark matter exists, it was less prevalent than ordinary
(baryonic) matter. 

In the 20th century, this hypothesis was about to change. In 1933, Fritz Zwicky published the first modern
interpretation of dark matter: while studying the velocity distribution of galaxies in the Coma cluster, he
found a velocity dispersion (effectively a measure of the "spread" of the distribution) of 1000 km/s, far
greater than the expected 80 km/s that would consistent assuming the cluster only contained visible matter.   
This led to the now widely-accepted conclusion: that dark matter exists in "much greater quantity than
luminous matter" \cite{andernachEnglishSpanishTranslation2017}. 
Further, Zwicky also concluded the same paper that based on the observed speeds and
the amount of visible matter, galaxy clusters like the Coma system should "fly apart", implying that the
amount of visible matter present is not nearly enough to support angular velocities as observed.       
  
At the time, people had reason to doubt Zwicky's conclusion -- the conclusion was extremely
counter-intuitive, as it suggested that a large majority of a galaxy's mass (and consequently the universe's)
is comprised mostly of matter that is effectively invisible to us. People obviously thought this was
unlikely, so they looked for alternative explanations for why this may have occurred. One interesting
theory was proposed by Erik Holmberg, who suggested that the galaxies used in Zwicky's calculation were not
permanent cluster members but instead galaxies on hyperbolic orbits that just happened to come across one
another at the time of observation. This hypothesis was later ruled out by Martin Swarzchild, who
redid the calculation whilst being careful about which galaxies to include, and still arrived at an
abnormally high velocity dispersion. 

This is just one example of the many ways astronomers sought to discredit the dark matter theory, but to no
avail as every opposing hypothesis was eventually discredited. Eventually, by the late 1960s and 1970s,
astronomers had no choice but to accept that dark matter may in fact be a real phenomenon, and began
investigating what it may be composed of. Early on, Zwicky hypothesized that it could be made up of "cool and
cold stars, macroscopic and microscopic solid bodies, and gases". However, this turned out to be false as
Herbert Rood discovered in 1974 that the "missing mass" had to come from the space between galaxies, meaning
that it could not be coming from faint stars or rocky bodies. Interstellar gas was also proposed since it
fits the criterion of existing between galaxies, however subsequent X-ray analyses showed that the amount of
intergalactic gas contributed only 2\% of what is necessary to fully account for the required mass.   

Slowly but surely, more and more "ordinary" forms of matter were ruled out as sources of dark matter.
Theories which pointed to planets, dwarf stars, and other forms of less luminous matter were ultimately ruled
out as subsequent observations to assess their prevalence in our own galaxy failed to produce the quantities
necessary to fully account for the missing mass. Furthermore, by the late 1990s and early 2000s, calculations
using the cosmic microwave background showed that the majority of the universe's mass is non-baryonic,
effectively forcing dark matter to be in the form of some subatomic particle.     

And that concludes our overview of the history of dark matter to the present day. Most scientists agree
that dark matter is probably made of some currently undiscovered subatomic particle. Generally, people refer
to these particles as Weakly Interacting Massive Particles (WIMPs); they are the leading theory for two primary
reasons: first, the quantity of WIMPs that would have been produced in the early universe seems to be consistent
with the estimated amount of dark matter in the universe today (around 83\%). Second, most theories beyond the
standard model -- which, importantly, were not devised in the interest of explaining dark matter -- includes
the introduction of some sort of new particle, which automatically becomes a candidate for a WIMP.   


\section{Alternatives to Dark Matter}
This section is not very crucial to the focus of this thesis, but it is worthwhile to note that even though
\textit{most} astronomers agree that dark matter is a real phenomenon, there are still proponents who
strongly advocate for alternative theories to dark matter. One such known example is called MOdified
Newtonian Dynamics (MOND), and has gained traction in recent years as the leading alternative to the theory of
dark matter. Pioneered by Moredehai Milgrom in 1983, MOND stands out as a theory from the rest as it is one of 
the only alternative theories that manages to explain the rotational
anoamalies in galaxies as well as make some correct predictions. For instance, MOND was able predict the
dynamics of low-mass spiral galaxies \cite{milgromMONDRotationCurves2007}, and also account for the Tully-Fisher formula which relates the
rotation curve of a galaxy to its intrinsic luminosity. Beyond this though, MOND has been largely
unsuccessful in conforming to observations, which is why it still remains a "fringe" theory that is not
widely accepted. 
  
  
  
  



