\documentclass[12pt]{article}
\usepackage{../local}
\urlstyle{same}
\usepackage[style=numeric,sorting=none,maxnames=3]{biblatex}
\addbibresource{ref.bib}
%\newcommand{\classcode}{Italian 130A}
%\newcommand{\classname}{Dante's Inferno, Love and Justice}
\setlength{\parindent}{25pt}
\author{Yutong Du}
\date{December 3, 2023}
\title{Italian 130A Final Paper}
\linespread{2}
\begin{document}
	\maketitle

	Throughout this semester, one of the things that always stuck out to me was Dante's incorporation of mythological creatures 
	into \textit{Inferno}. There have been many instances of Greek and Roman figures throughout the work, 
	and I wanted to take this opportunity to analyze the differences between Greek and Roman representation 
	in Dante's \textit{Inferno.} Specifically, Dante seems to favor Roman traditions far over Greek ones, and consequently
	Roman Catholicism as a beacon of absolute truth, which is reinforced through a combination of Roman figures being 
	portrayed as legendary, while simultaneously degrading Greek figures by punishing their sins within Inferno. 

	Perhaps the most notable example of roman favoritism is Dante's choice of Virgil, a Roman poet, 
	as his guide throughout \textit{Inferno.} When
	we are introduced to Virgil in Canto I, Dante (the character) expresses his devotion to Roman literature by endlessly praising
	Virgil by calling him a "wide stream of eloquence" and immediately credits Virgil as the figure that allowed Dante to "acquire 
	the beautiful style that has won [Dante] Honor." (Inferno I, 79, 86-87). From then on, Virgil's words to Dante are interpreted as 
	absolute truth, and his words are never challenged by Dante throughout the journey. This is exemplified in countless places, 
	perhaps most notably in the forest of suicides, where Virgil convinces Dante to "pluck [\dots] a little branch from [the] trees"
	(Inferno XIII, 28-29), prompting Dante to immediately "stretch out [his] hand" and "snap off a twig" from a nearby brush. (Inferno
	XIII, 31-32). This example highlights Virgil's authority as Dante does not even question Virgil's commands for a moment, 
	but rather
	immediately does as he is asked. Virgil's authority over Dante's actions throughout \textit{Inferno} highlights Dante's 
	subservience not only to Virgil but also towards the roman tradition that Virgil represents. 
	% page 163 
	Not only does Dante refrain from challenging Virgil's words, many figures encountered throughout \textit{Inferno} do not either.
	For instance, when Dante and Virgil encounter Plutus at the entrance of the fourth circle, Virgil silences the beast by telling 
	it to "gnaw on your own rage until you choke" (Inferno VII, 8-9), to which Plutus obeys without question. When Dante and Virgil 
	meet Ulysses in the 8th Bolgia, Virgil tells Dante to "let [Virgil] do the talking," (Inferno XXVI, 73) and upon Virgil's 
	request for the flame to tell their story, they oblige immediately. Virgil's authority 
	over other characters is further demonstrated during the transition into the Malebolge, when Virgil alone convinces Geryon to 
	give Dante and Virgil "a ride upon his strong shoulders" into the eighth circle of Hell (Inferno XVII, 42). These displays 
	of authority elevate Virgil's status far above simply a guide for Dante, as it demonstrates that his command is absolute, 
	highlighting his superiority as a Roman figure.  

	That said, there are instances where Virgil's word was certainly challenged, yet despite this his wishes are always 
	granted in the end. Most notably, before the entry of the 6th circle, Dante and 
	Virgil are prevented from entering the city of Dis, and are forced to wait for an unknown figure "sent from Heaven" 
	(Inferno IX, 85) to help 
	open the gates for them. While at first this appears to be an instance where Virgil's authority was not absolute, the appearance 
	of the heavenly figure reinforces the opposite, as it further demonstrates how Virgil's word cannot be opposed, and that 
	their descent through the Inferno, facilitated by Virgil, is a journey that no entity can prevent. As mentioned, this notion 
	that Virgil will succeed in bringing Dante through hell no matter the cost reinforces his authority and serves to highlight 
	Dante's favorable characterization of Roman characters. 

	Aside from Virgil, there are also other Roman figures found within Inferno. When exploring the first circle of Limbo, where 
	Virgil also resides, Dante comes across the poets Homer, Horace, Ovid and Lucan. Upon encountering them, Dante describes 
	them as "lord[s] of highest poetry that soar like an eagle above the rest" (Inferno IV, 95-96), clearly highlighting that Dante 
	has the utmost respect for these legendary poets. Interestingly, Homer is the only Greek poet mentioned in this passage, 
	showing how even within the realm of poetry, Dante seems to favor Roman poets over Greek ones. After Dante and the poets become 
	acquainted with one another, Dante joins them as the "sixth in that flight of wisdom" and proceed "on into the light, 
	talking of things better left in silence" (Inferno IV, 102-104). This description of Dante joining them "into the light" 
	highlights Dante's desire to join this league of legendary (primarily Roman) poets, further showing Dante's bias towards Roman 
	traditions. The placement of these characters in Limbo also further highlights Dante's preference toward Roman characters, as 
	sinners are only placed here as a result of being born before Christ, implying that had they been born after Christ, they would be
	found in Heaven rather than Limbo. Therefore, in a sense, the poets found within this Canto, who are primarily Roman, are 
	deserving of a spot in 
	Paradise, but are instead found here through no fault of their own. 

	In addition to the Roman figures found within Limbo, Dante as an author severely punishes characters who directly challenge 
	Roman authority. Most notably, the two leaders who led the conspiracy to assassinate Julius Caesar, Brutus and Cassius, are 
	found within the deepest circle of Hell, being "chewed upon" and "kept [\dots] in constant agony" by Dis himself (Inferno 
	XXXIV, 55-57). Dante's placement of them within the circle meant for traitors against masters, and also the fact that here 
	reside the souls that are "punished the most" (Inferno XXXIV, 61) highlights that Dante perceives the sin Brutus and Cassius 
	committed -- the assassination of Julius Caesar -- is the sin that merits the most severe punishment of all. This is 
	then contrasted further with the fact that Caesar himself is found within Limbo, which as mentioned earlier is effectively 
	Paradise for those who could never obtain it. Combining these two ideas together 
	highlights Dante's belief that Caesar's Roman empire was the ultimate civilization to exist, and those that facilitated
	its downfall are deserving of the utmost punishment. Furthermore, Brutus and Cassius are also punished alongside Judas, who is
	guilty of betraying Jesus Christ, indicating that the betrayal of the Roman empire is a sin equal 
	in magnitude to the betrayal of Christ himself, further highlighting how Dante favors the Roman empire and believes it to be 
	equally important as Roman Catholicism. 
	
	In stark contrast to the Roman figures found within \textit{Inferno}, Dante does not hesitate in punishing many Greek figures
	throughout the Comedy. Firstly, many of the Guardian figures for each circle of Hell: Charon, Minos, Cerberus, Plutus, 
	Phlegyas, and many more are all of Greek origin. Not only are they of Greek origin, but because many of them symbolically 
	represent the sin found within this circle, it serves to degrade these figures as sinners themselves and not worthy of 
	Heaven. For instance, when Dante and Virgil are confronted Cerberus while entering the third circle, Virgil tames the beast 
	by "scoop[ing] up earth, and threw fistfuls of it into the monster's ravenous gullets," ultimately causing Cerberus to 
	put "all his effort into swallowing" (Inferno VI, 26-30). Here, Cerberus' reaction to swallow and consume what Virgil threw 
	at him is representative of gluttony, as ithe third circle punishes sinners for their inability to control their desire to 
	endlessly consume. This highlights Cerberus' nature as a gluttonous beast himself, and as a result degrades him to be 
	eternally bound to reside within Hell itself. This is further reinforced through the use of the Greek figure Geryon as the 
	guardian of the eighth
	circle, being described directly as a "loathsome image of treachery" (Inferno XVII, 7), another example where a Greek 
	mythology figure is degraded and used solely to guard the entrance to a circle within \textit{Inferno}. Finally, and 
	perhaps most importantly, Dante comes across Ulysses in the eighth circle of Hell, which not only is another example of a
	Greek hero being placed within Hell, but is also significant for another reason entirely. As described in Canto 
	XXVII (Canto 27), the sinners in this Bolgia of the eighth circle are punished here as a result of their misleading counsel; 
	in the case of Ulysses, this refers to his "burning desire for experience of the wide world," (Inferno XXVI, 97) and
	convinced his crew to 
	"steel [their] hearts and made them eager for the voyage ahead" (Inferno XXVI, 122), which ultimately led to their demise. 
	Ironically however, Ulysses is not punished for his role in creating the Trojan horse, which "opened the gates of Troy thorugh 
	which the noble seed of Rome set forth" (Inferno XXVI, 60). This choice for punishing Ulysses for his false counseling of 
	his crew instead of the Trojan horse demonstrates Dante's predisposition for Roman traditions, since in his view, the rise of 
	Rome is not a sin that needs to be punished as he holds the Roman empire in such a high regard. To further supplement this 
	idea, Dante's description of Ulysses' downfall is a continuation of the story following 
	the conclusion of Homer \textit{Odyssey}, highlighting not only that Dante is willing to rewrite history to justify Ulysses' 
	placement but also specifically rewrite the writings of a Greek poet, further showing Dante's unsatisfactory views toward 
	Greek figures. 

	Throughout Dante's \textit{Inferno}, he makes use of many Greek and Roman figures as representative sinners, and uses them to 
	highlight the sins that are contained within each circle of Hell. Through the unending placement of Greek figures into 
	Hell whether as a guardian or a representative sinner, combined with his continual praise of Roman traditions and characters,
	Dante clearly shows that he has a preference for Roman symbols, and believes their ideology to be superior and consequently the 
	representation of absolute truth within his universe. As a result, Dante's \textit{Inferno} is not only a reflection of 
	sins and punishment, but also a reflection of Dante's philosophical belief that the Roman heritage is the ultimate 
	symbol of puritiy.
	\nocite{*}
    \printbibliography
	% mention other notable roman figures in hell 
	% mention the plot to assassinate Ceasar and Brutus' subsequent placement in Hell as a result 
	% contrast this with the greek figures  
\end{document}
