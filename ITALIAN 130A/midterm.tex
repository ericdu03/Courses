\documentclass[10pt]{article}
\usepackage{../local}


\newcommand{\classcode}{Italian 130A}
\newcommand{\classname}{Dante's Inferno: Love and Justice}
\renewcommand{\maketitle}{%
\hrule height4pt
\large{Eric Du \hfill \classcode}
\newline
\large{Midterm Review} \Large{\hfill \classname \hfill} \large{\today}
\hrule height4pt \vskip .7em
\normalsize
}
\linespread{1.1}
\begin{document}
	\maketitle
	\section{Canto I}
	\begin{itemize}
		\item Opening phrase ``midway through the journey of our life''
		\item Meets three beasts: a leopard, a lion, then a she-wolf. These beasts represent something within 
			the human experience, but it's unclear what exactly.
		\item As a result of these three beasts, Dante is unable to move forward and is instead forced to 
			retreat back, during which he meets Virgil
		\item Here, Dante gives overwhelming praise to Virgil: ``O glory and light of all other poets... 
			you are my master, you are my author. It is from you that I have acquired the beautiful style 
			that has won me honor" (lines 85-90)
		\item Virgil says that ``You will have to go by another road" (line 91), and convinces Dante to 
			go on the journey with Virgil through the Inferno. Here, he explains how 
			the Virgin Mary sent Beatrice (Dante's lover) to go tell Virgil to save Dante.   
	\end{itemize}

	\section{Canto II}
	\begin{itemize}
		\item Where Dante questions whether he truly is fit to go on the mission set out for him by Virgil.
	\end{itemize}
	\section{Canto III}
	\begin{itemize}
		\item Before they enter the first circle, they encounter the neutrals. They are souls that are neither
			wanted by Hell or Heaven, since they did not choose sides in their lives. 
		\item For this, they are punished by being constantly stung by insects, while also pursuing a 
			banner that leads nowhere.
		\item In some sense, Dante is pointing to the fact that being neutral is arguably worse than picking a 
			side. 
		\item Dante falls asleep at the end of this Canto. 
	\end{itemize}
	\section{Canto IV}
	\begin{itemize}
		\item First circle of inferno: Limbo
		\item Reserved for souls whose only sin is that they did not worship God in their past life. These 
			include those who predated God, great Roman and Greek poets reside here.
		\item Dante is essentially calling into question: what happens to these people? If the rules of 
			the bible are that you must worship God in order to be allowed into Heaven, then those who 
			were born before Christ would have no choice but to reside in Hell (through no fault 
			of their own). 
		\item The only punishment given to those in Limbo is the fact that they cannot exit. Other than that, 
			this seems like a pretty chill place. 
		\item Virgil reveals that there has been an instance where people were ``saved" from Limbo by 
			christian figures, but it won't happen again.
		\item Dante is invited as the sixth into the group of legendary poets, and falls asleep there.
	\end{itemize}
	\section{Canto V}
	\begin{itemize}
		\item Second circle of Inferno: Lust
		\item Guardian: Minos. Judge souls and determines where they go. The soul goes to the layer determined 
			by the number of times Minos spins his tail.
		\item Punishment: infernal wind that never rests that whips them around. They are never allowed to rest. 
		\item People who were blown about by the passion of love are punished by being blown about by winds.
		\item Notable Figure: Francesca who was reading Lancelot with a lover and then ended up 
			making out. 
		\item Dante falls asleep listening to this story
	\end{itemize}
	\section{Canto VI}
	\begin{itemize}
		\item Third circle of Inferno: Gluttony
		\item Guardian: Cerberus, which Virgil just shoves dirt in his mouth to get him to shut up.
		\item Notable Figure: Ciacco -- a fellow Florentine.
		\item Punishment: Laying in a pool and being rained on. 
		\item Ciacco gives Dante a prophecy about what will happen to Florence
			\begin{itemize}
				\item It's important to learn about the conflict that exists here: there is divide in Florence
					among the Guelphs and the Ghibellines, while the Guelphs are further divided into 
					white and black guelphs. 
				\item Guelphs supported the pope, and Guibellines supported the emperor.  
				\item Notice Dante injecting his own political views into this story.
			\end{itemize}	
		\item Dante also mentions that those who were well respected in their normal life are condemned to 
			``hell's blackest souls". He makes clear that what you are respected for is not all that 
			you are being measured for.
	\end{itemize}
	\section{Canto VII}
	\begin{itemize}
		\item Fourth circle: Avaricious and the Prodigal
		\item Guardian: Plutus 
		\item Here we see Dante echo Aristotle's \textit{Ethics}, where he describes a balance between 
			two extremes: hoarding and wasting.  
		\item Their punishment is to scream at one another about hoarding and wasting. 
		\item Fifth Circle: Wrathful and Sullen
		\item Not much mention of this circle at all really, the wrathful are stuck fighting each other 
			and the sullen are found below the river styx. They are condemned to repeat the same words, but 
			in their throats because they cannot speak the words fully.
	\end{itemize}
	\section{Canto VIII}
	\begin{itemize}
		\item Guardian: Phlegyas
		\item Dante recognizes somebody from florence (Filippo Argenti), and the two get into an altercation where 
			Dante says that the soul should ``be damned right here where 
			it steeps". 
		\item To this Virgil gives Dante a kiss on the cheeks, and congratulates him on what he's done.
		\item They come to the outer wall of the city of Dis, and this is the first point where Virgil 
			fails to get through the guardians of the city. They will have to go 
			by another path as a result. 
	\end{itemize}
	\section{Canto IX}
	\begin{itemize}
		\item Virgil says that this isn't the first time he's made this journey, assures Dante that he's 
			in good hands
		\item They are stopped by three furies who threaten to use Medusa and turn Dante to stone, Virgil 
			then puts a hand to Dante's eyes so that he doesn't look.
		\item Somebody from above comes and helps them out, but is unnamed. 
		\item They enter the Sixth circle, where the heretics (those who don't believe in the 
			bible) reside. 
	\end{itemize}
	\section{Canto X}
	\begin{itemize}
		\item Still in circle 6.
		\item Notable Figure: Farinata
			\begin{itemize}
				\item He's a Ghibelline, so obviously he has to be in this circle of Hell
				\item It's fitting that he's placed in Heresy, too, given that it's a circle meant for 
					those who ``have strayed from God"
				\item This section goes into detail about the political conflict, and it's just a lot of 
					Farinata asking why his party wasn't welcomed in Florence.
			\end{itemize}
		\item Cavalcante is also here, asks Dante whether his son is still alive, to which Dante does not reply.
			Then at the end of the canto he tells Farinata to tell Cavalcante that his son is still living.  
	\end{itemize}
	\section{Canto XI}
	\begin{itemize}
		\item Very important Canto, goes over the structure of the entire inferno.
		\item They couldn't continue the journey because of the smell -- think about the ways that Dante (the 
			author) tries to insert moments of dialogue into the poems.  
		\item Structure of the last three circles:
			\begin{itemize}
				\item First is violence, separated into three rings: to God, to themself, and to neighbours.
				\item Violence against neighbours (those who have killed others) are found first in violence, 
					those who were violent toward 
					themselves (sodomy, usury) and their possessions are found second, and those who were 
					violent toward God (blasphemy) are found last. 
			\end{itemize}
		\item The fraudulent are found next, in the 8th circle, for hypocrites, sorcerers and parasites. The 
			traitorous is the worst type of fraud, and is therefore found at the very bottom of inferno.
		\item Aristotle's Ethics: the three things rejected by Heaven: Incontinence, malice, and beastial 
			rage. Incontinence is the lightest of the three, so they aren't punished as much.
		\item As Dante puts it, art and your creations are in some sense a grandchild of God, but since 
			the usurer takes advantage of others, it is effectively a sin against God.  
	\end{itemize}
	\section{Canto XII}
	\begin{itemize}
		\item Seventh Circle, 1st section: Violence against others
		\item Virgil explains that when Christ came to Limbo and saved souls from there, it caused this 
			landslide to occur. 
		\item They approach the river of blood, where these souls lie.
		\item They are eternally punished to reside in the river, guarded by Centaurs who shoot them down 
			whenever they get too close.
	\end{itemize}
	\section{Canto XIII}
	\begin{itemize}
		\item Seventh Circle, 2nd section: Violence against self
		\item Virgil tells Dante to break off a branch, which actively hurts one of the trees. 
		\item Virgil then apologizes to the tree, but says that it needed to be done since Dante wouldn't have 
			believed him. 
		\item The more we go into the Inferno, the more Dante (the author) seems willing to highlight 
			Virgil's limitations as a poet and figure.
		\item The tree Dante pulled is Pier de la Vigna, who was an advisor for emperor Frederick. He was 
			accused of treason, to which he thought he could escape by killing himself. This was not true, and 
			instead he's punished for thinking this in the 7th circle of hell.
		\item These people are punished by being reincarnated into a tree, from which they grow fruit 
			for harpies to come eat. 
		\item The suicides are in a sense also violence against God, against life, which is why it's punished 
			harshly as well. 
	\end{itemize}
	\section{Canto XIV}
	\begin{itemize}
		\item Seventh Circle, 3rd section: Violence against God
		\item They are punished by the hot sand and hot rain
		\item Capaneus: a figure whose pride is his punishment, and Virgil points out that ``no torment 
			in Hell would be painful enough to match your fury''.
		\item Here it is revealed that the river of blood is also called the phleglothon, and mentions the 
			river Lethe as well. Virgil replies that the Lethe will not be found in Inferno, but 
			instead resides in Purgaotry.  
	\end{itemize}
	\section{Canto XV}
	\begin{itemize}
		\item Continuing on with violence against God, meets his teacher Brunetto Latini
		\item Brunetto comes to Dante, and walks alongside him, while Dante is in the boat. He is positioned 
			higher than Brunetto is, showing how the student has surpassed the master in a sense. 
		\item Brunetto praises Dante and reassures him that he will be a great poet. Quite elegant 
			of him to do that.
	\end{itemize}
	\section{Canto XVI}
	\begin{itemize}
		\item Here Dante comes across the souls that he asked Ciacco about back in Canto VI (Jacopo, Tegghiaio
			and Guido).
		\item These people ask about Florence's future, and their presence here highlights the fact that 
			what you were regarded in your past life is not all that is considered when you are thrown 
			into Inferno
	\end{itemize}
\end{document}
