\documentclass[10pt]{article}
\usepackage{../../local}
\urlstyle{same}

\newcommand{\classcode}{Physics C191}
\newcommand{\classname}{Introduction to Quantum Computing}
\renewcommand{\maketitle}{%
\hrule height4pt
\large{Eric Du \hfill \classcode}
\newline
\large{HW 01} \Large{\hfill \classname \hfill} \large{\today}
\hrule height4pt \vskip .7em
\small{Header styling inspired by CS 70: \url{https://www.eecs70.org/}}
\normalsize
}
\linespread{1.1}
\begin{document}
	\maketitle
	\section*{Problem 1}
	\begin{enumerate}[label=\alph*)]
		\item Find the eigenvectors, eigenvalues, and diagonal representations of the Pauli operators \( I, X, Y, Z \),
			where the corresponding matrices are given by:
			\begin{align*}
				I &\equiv \sigma_0 \equiv \begin{pmatrix} 1 & 0 \\ 0 & 1 \end{pmatrix} & X &\equiv \sigma_x \equiv 
				\begin{pmatrix} 0 & 1 \\1 & 0  \end{pmatrix} \\
				Y &\equiv \sigma_y \equiv \begin{pmatrix} 0 & -i\\i & 0 \end{pmatrix}  & Z &\equiv \sigma_z \equiv 
				\begin{pmatrix} 1 & 0 \\ 0 & -1 \end{pmatrix} 
			\end{align*}
			Show that \( X^2 = Y^2 = Z^2 = I \) (calculating \( X^{n}, Y^{n}, \) or \( Z^{n} \) thus becomes very 
			simple).

			\begin{solution}
				I'll work down the list, starting with \( I \). Because it's already a diagonal matrix, we know its 
				eigenvectors are on the diagonal. Moreover, 
			\end{solution}
		\item Find the points on the Bloch sphere that corresponds to the normalized eigenvectors of the Pauli 
			matrices. 
		\item Find the action of the \( Z \) operator on a general qubit state \( \ket*{\psi} = \alpha\ket*{0} + 
			\beta\ket*{1}\) and describe this actino on the Bloch sphere vector, i.e., how does the vector connecting 
			the origin to the point on the surface representing \( \ket*{\psi} \) in Bloch sphere coordinates
			get rotated on the Bloch sphere?
		\item If \( x \) is a real number and \( A \) a matrix with the property that \( A^2 = I \), show that the 
			exponentiated operator \( e^{iAx} \) can be written as 
			\[
			e^{iAx} = \cos(x)I + i \sin(x) A
			\] 
		\item Use the result in d) to form the matrix representation of \( R_z(\gamma) = e^{-i \gamma Z / 2} \), and 
			show how this exponentiated operator acts on the Bloch sphere vector for \( \ket*{\Psi} \). 
			Evaluate the explicit form of \( R_z(\gamma) \) for \( \gamma = 0, 2\pi, 4\pi \), and comment 
			on anything surprising. This operator gives rise to several well-known signal-qubit quantum gates, 
			naely the phase gate \( S \) (\( \gamma = \pi / 2 \)) and the \( T \) gate (\( \gamma = \pi / 4 \)), up to 
			a global phase factor in each case. Check these for yourself. 
	\end{enumerate}
	\pagebreak
	\section*{Problem 2}
	\begin{enumerate}[label=\alph*)]
		\item Show that the Pauli operators and Hadamard gate satisfy the following identities:
			\[
			HXH = Z, HYH = -Y, HZH = X
			\] 
		\item Show that if \( U \) and \( V \) are unitary, then \( U \otimes V \) is also unitary. 
		\item Write the \( 4 \times 4 \) matrix of the unitary operation on two qubits resulting from 
			performing a Hadamard transform on the first qubit and a phase flip on the second qubit. 
	\end{enumerate}
	\pagebreak
	\section*{Problem 3}
	Compare the result of measuring the state \( \ket*{\psi} = \frac{1}{\sqrt{2} }\ket*{0} +
	\frac{e^{i \theta}}{\sqrt{2} }\ket*{1} \) in the computational basis \( \ket*{0}, \ket*{1} \), 
	and in the Hadamard basis \( \ket*{+}, \ket*{-} \). Hint: You will have to transform the state into the Hadamard
	basis before making the second measurement. Can either of these two measurements give information about the value of
	the relative phase \( \phi \)? Specify what can be learned. 
	\pagebreak
	\section*{Problem 4}
	Start with two quantum bits in the state \( \alpha\ket*{00} + \beta\ket*{11} \). Subject the first bit to 
	a Hadamard gate. Now measure the first bit. What is the state of the second bit?
\end{document}
