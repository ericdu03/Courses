\documentclass[10pt]{article}
\usepackage{../../local}
\urlstyle{same}

\newcommand{\classcode}{Physics C191}
\newcommand{\classname}{Introduction to Quantum Information}
\renewcommand{\maketitle}{%
\hrule height4pt
\large{Eric Du \hfill \classcode}
\newline
\large{HW 04} \Large{\hfill \classname \hfill} \large{\today}
\hrule height4pt \vskip .7em
\small{Header styling inspired by CS 70: \url{https://www.eecs70.org/}}
\normalsize
}
\linespread{1.1}
\begin{document}
	\maketitle
	\section*{Problem 1}
	This problem will demonstrate an important aspect in designing quantum algorithms. Suppose we have some 
	"oracle" that implements a unitary \( O_{x, \pm} \) based on a classical 2-bit string \( x = x_0x_1 \). The 
	action of the unitary on the computational basis is 
	\[
	O_{x, \pm}: \ket*{b} \rightarrow (-1)^{x_b}\ket*{b} \ \text{for \( b \in \{0, 1\}  \)}
	\] 
	\begin{enumerate}[label=\alph*)]
		\item Say that we run the 1-qubit circuit \( HO_{x, \pm}H \) on the initial state \( \ket*{0} \) and then 
			measure in the computational basis. What is the probability distribution on the output bit as a 
			function of \( x \)?

			\begin{solution}
				Carrying out the calculation:
				\begin{align*}
					HO_{x, \pm}H\ket*{0} &= HO_{x, \pm}\left( \frac{\ket*{0} + \ket*{1}}{\sqrt{2} } \right) \\
					&= H \cdot \left( \frac{(-1)^{x_0} \ket*{0} + (-1)^{x_1}\ket*{1}}{\sqrt{2} } \right)  
				\end{align*}
				Now, depending on the identity of \( x \), this state before acting \( H \) will change. So, 
				we need to treat this on a case-by-case basis: 
				\begin{align*}
					x = 00 \longrightarrow H\left( \frac{\ket*{0} + \ket*{1}}{\sqrt{2} } \right) 
					&= \frac{1}{2}\left[\ket*{0} + \ket*{1}
				+ \ket*{0} - \ket*{1}\right] = \ket*{0}\\
					x = 01 \longrightarrow H\left( \frac{\ket*{0} - \ket*{1}}{\sqrt{2} } \right) 
					&= \frac{1}{2}\left[\ket*{0} + \ket*{1} - (\ket*{0} - \ket*{1})\right] = \ket*{1} \\
					x = 10 \longrightarrow H\left( \frac{-\ket*{0} + \ket*{1}}{\sqrt{2} } \right) 
					&= \frac{1}{2}\left[ -(\ket*{0} + \ket*{1}) + \ket*{0} - \ket*{1} \right] = -\ket*{1} \\
					x = 11 \longrightarrow H\left( \frac{-\ket*{0} - \ket*{1}}{\sqrt{2} } \right) 
					&= \frac{1}{2}\left[ -(\ket*{0} + \ket*{1} - (\ket*{0} - \ket*{1}) \right] = -\ket*{0} 
				\end{align*}
				So, the output bit from this process is entirely determined by the value of \( x \). 
			\end{solution}
		\item Now suppose we have a second implementation \( O'_{x, \pm} \) that copies the value of the 
			classical bit \( b \) over to a second register. That is, \( O'_{x, \pm} \) is implemented 
			as 
			\[
			O;_{x, \pm} : \ket*{b, 0} \to (-1)^{x_b}\ket*{b, b}
			\] 
			Suppose we ignore the second qubit and run the algorithm of (a) on the first qubit with 
			\( O'_{x, \pm} \) instead of \( O_{x, \pm} \) (and \( H \otimes I \) instead of \( H \) and initial state
			\( \ket*{00} \) ). What is now the probability distribution on the output bit (i.e. if we measure 
			the first of the two qubits).

	\end{enumerate}	
	The point is to show the importance of "cleaning up" additional computational resources that were used in the 
	computation. By leaving "garbage" in an ancillary register that is coupled to a sub-register one 
	wishes to measure, the residual entnalgement between the two registers can corrupt 
	measurement outcome distributions. 

			\begin{solution}
				Applying the first two gates to the state, we get: 
				\[
					(H \otimes I) O'_{x, \pm} (H\otimes I) \ket*{00} = \frac{1}{\sqrt{2} }(H \otimes I) 
					\left[ (-1)^{x_0}\ket*{00} + (-1)^{x_1}\ket*{11} \right] 
				\] 
				(I'm too lazy to write out the intermediate steps). Again, we do the same process as the previous 
				problem:
				\begin{align*}
					x = 00 \longrightarrow \frac{1}{\sqrt{2} }(H \otimes I)(\ket*{00} + \ket*{11}) 
					&= \frac{1}{2}(\ket*{00} + \ket*{10} + \ket*{01} + \ket*{11})\\
					x = 01 \longrightarrow \frac{1}{\sqrt{2} }(H \otimes I)(\ket*{00} - \ket*{11})
					&= \frac{1}{2}(\ket*{00} + \ket*{10} - \ket*{01} - \ket*{11}) \\
					x = 10 \longrightarrow \frac{1}{\sqrt{2} }(H \otimes I)(-\ket*{00} + \ket*{11})
					&= \frac{1}{2}(-\ket*{00} - \ket*{10} + \ket*{01} + \ket*{11}) \\
					x = 11 \longrightarrow \frac{1}{\sqrt{2} }(H \otimes I)(-\ket*{00} - \ket*{11})
					&= \frac{1}{2}(-\ket*{00} - \ket*{10} - \ket*{01} - \ket*{10}) 
				\end{align*}
				In all of these states, the probability of measuring 0 from the first qubit is \( \frac{1}{4} 
				+ \frac{1}{4} = \frac{1}{2}\), which is different than what we have in part (a). 
			\end{solution}

	\pagebreak
	\section*{Problem 2}
	\begin{enumerate}[label=\alph*)]
		\item Prove that any classical algorithm requires at least two calls to \( f \) to solve the Deutsch 
			Josza problem with probability greater than \( 1 / 2 \). 

			\begin{solution}
				A simple randomized algorithm, where we pick a random input to \( f \), will be able to 
				determine the identity of \( f \) with a success rate of \( \frac{2}{3} \) after two calls, this 
				fact was proven in lecture. 

				We can then make the argument that \textit{any} classical algorithm must also do this becuase 
				regardless of how our classical algorithm inquires about \( f \), the probability of it receiving 
				either 0 or 1 is \( \frac{1}{2} \) if \( f \) is balanced. Therefore, the probability of error 
				after two calls
				is going to be 
				\[
				P(\text{error}) = \frac{1}{2^2} = \frac{1}{4}
				\] 
				so the probability of success is greater than \( \frac{1}{2} \), as desired.
			\end{solution}
		\item Prove that any classical algorithm requires \( 2^{n - 1} + 1 \) calls to \( f \) to 
			solve the Deutsch-Josza problem with certainty.

			\begin{solution}
				A classical algorithm needs to check more than half the 
				values of \( f \) to determine \( f \) with certainty, due to the pigeonhole principle (once 
				we've covered more than half the values, then if \( f \) is balanced then both 0 and 1 would have 
				been seen.). Therefore, for a bitstring of length \( n \), the search space is \( 2^{n} \), so 
				we require at least \( 2^{n - 1}  + 1 \) queries. 
			\end{solution}
		\item Describe a classical approach to the Deutsch Josza problem that solves it with 
			high probability using fewer than \( 2^{n - 1} + 1 \) calls. Calculate the success probability of 
			your approach as a function of the number of calls.

			\begin{solution}
				Consider a randomized algorithm	that selects a random bitstring out of the \( 2^{n} \) possible 
				ones and inquires about \( f \). If \( f \) is balanced, then the probability that \( f \) outputs 
				either 0 or 1 is \( \frac{1}{2} \) on every call. Therefore, after \( k \) calls, the probability 
				of error (i.e. we output that \( f \) is constant when in reality \( f \) is balanced) is: 
				\[
				P(\text{error}) = \frac{1}{2^{k}}
				\] 
				Note that this probability of error drops off very quickly; after just 10 queries, the probability
				of error is already less than 0.1\%. 
			\end{solution}
	\end{enumerate}
	\pagebreak
	\section*{Problem 3} 
	The (classical) Fourier transform mod \( N \) is the \( N \times N \) matrix given by 
	\[
		FT_{N} = \frac{1}{\sqrt{N} }\begin{pmatrix} 1 & 1 & 1 & \dots & 1\\
			1 & \omega & \omega^2 & \dots & \omega^{N - 1}\\
			1 & \omega^2 & \omega^{4} & \dots & \omega^{2(N - 1)}\\
			\vdots & \vdots & \vdots & \ddots & \vdots\\
			1 & \omega^{N - 1} & \omega^{2(N - 1)} & \dots & \omega^{(N - 1)^2}\end{pmatrix} 
	\] 
	where \( \omega = e^{2 \pi i / N} \) has a primitive \( N \) th root of unity. So the \( i, j \) 'th 
	element of \( FT_N \) is \( \frac{1}{\sqrt{N} }\omega^{ij} \), for \( i, j = 0, \dots, N - 1 \). Show that 
	\( FT_N \) is unitary by evaluating the inner product between the \( i \) and \( j \) th columns of 
	\( FT_N \), i.e., show that 
	\[
		\mel{i}{FT_N^{\dagger}FT_N}{j} = \delta_{ij} = \begin{cases}
			1 & \text{if \( i = j \)}\\
			0 & \text{if \( i \neq j \)}
		\end{cases}
	\] 
	Your calculation will demonstrate a very important and useful result, namely that for \( \omega \) a primitive 
	\( N \)-th root of unity (i.e.,  \( \omega^{N} = 1 \) but \( \omega^{m} \neq 1 \) for \( 0 < m < N \)), 
	\[
	\sum_{k=0}^{N - 1} \omega^{kj} = \begin{cases}
		N & \text{if \( j = 0 \pmod N \)}\\
		0 & \text{otherwise}
	\end{cases}
	\]

	\begin{solution}
		Notice that the \( (i, j) \)-th element of the product \( FT_N^{\dagger}FT_N \) is given by 
		the dot product of the \( i \)-th row of \( FT_N^{\dagger} \) and \( j \)-th column of 
		\( FT_N \). An element in the \( i \)-th row of \( FT_N^{\dagger} \) is written as 
		\( \frac{1}{\sqrt{N} }\omega^{-ik} \), and an element in the \( j \)-th column of \( FT_N \) is 
		written as \( \frac{1}{\sqrt{N} }\omega kj \). 
		Therefore, we can write the product as: 
		\[
			M_{ij} = \frac{1}{N}\sum_{k=1}^{N - 1} \omega^{-ik} \omega^{kj} = \frac{1}{N} \sum_{k=1}^{N-1} 
			\omega^{(j - i) k}
		\] 
		Now, let \( m = j - i \), so we have 
		\[
		\frac{1}{N}\sum_{k=1}^{N-1} \omega^{km}
		\] 
		Since \( \omega \) is an \( N \)-th root of unity, then it satisfies the equation \( \omega^{N} - 1 = 0\).
		Factoring this, we get
		\begin{equation}
			\label{omega}
			(\omega - 1)(\omega^{N - 1} + \omega^{N-2} + \cdots + 1) = 0
		\end{equation} 
		If \( \omega - 1 = 0 \), then this means that \( m = 0\) or \( i = j \),
		so then \( \omega = 1 \). This means that 
		\[
		\frac{1}{N} \sum_{k=1}^{N-1} \omega^{km} = \frac{1}{N}\sum_{k=1}^{N-1} N = 1
		\] 
		If the second term in eq. \ref{omega} is equal to zero, then this implies that \( j \neq i \), and 
		we notice that 
		\[ \sum_{k=1}^{N-1} \omega^{km} = \omega^{N - 1} + \omega^{N-2} + \cdots + 1 = 0\]
		so 
		we immediately have that if \( j \neq i \), then the summation is also zero. Therefore, we have:
		\[
		M_{ij} = \frac{1}{N}\sum_{k=1}^{N-1} \omega^{km} = \begin{cases}
			1 & i = j\\
			0 & i \neq j 
		\end{cases} \implies M_{ij} = \delta_{ij}
		\] 
		as desired. 
	\end{solution}
	\pagebreak
	\section*{Problem 4}
	The uncertainty principle bounds how well a quantum state can be localized simultaneously in the standard 
	basis and the Fourier basis. In this question, we will derive an uncertainty princpile for a discrete system 
	of \( n \)-qubits. 

	Let \( \ket*{\psi} = \sum_{x \in \{0, 1\} ^{n}}\alpha_x\ket*{x} \) be the state of an \( n\)-qubit system. 
	A measure of the spread of \( \ket*{\psi} \) is \( S(\ket*{\psi}) \equiv \sum_x |\alpha_x| \). For example, 
	a completely localized state \( \ket*{\psi} = \ket*{y} (y \in \{0, 1\} ^{n}) \), the spread of 
	\( S(\ket*{\psi}) = 1 \). For a maximally spread state \( \ket*{\psi} = \frac{1}{\sqrt{2^{n}} }\sum_x 
	\ket*{x}\), \( S(\ket*{\psi}) = 2^{n} \cdot \frac{1}{\sqrt{2^{n}} } = \sqrt{2^{n}}  \). 

	\begin{enumerate}[label=\alph*)]
		\item Prove that for any quantum state \( \ket*{\psi} \) on \( n \) qubits, 
			\( S(\ket*{\psi}) \le  2^{n / 2} \). (Hint: use the Cauchy-Schwarz inequality, 
			\( \braket*{v}{w} \le \|v\| \cdot \|w\| \). 

			\begin{solution}
				First, we use the triangle inequality to conclude:
				\[
				\sum_x |\alpha_x| \le \left| \sum_x \alpha_x \right|  = \left| \sum_x \braket*{x}{\psi} \right| 
				\] 
				Then, we apply Cauchy-Schwarz on the right hand side: 
				\[
				\left| \sum_x \braket*{x}{\psi} \right| \le \left| \sum_x \|x\| \cdot \|\psi\| \right| 
				\] 
				We know that \( \|\psi\| = 1 \), and since there are \( 2^{n} \) states, then we can write: 
				\[
				\sum_x |\alpha_x| \le \sqrt{2^{n}} = 2^{n / 2}
				\] 
				as desired. 
			\end{solution}
		\item Suppose that \( |\alpha_x| \le  a \) for all \( x \). Prove that \( S(\ket*{\psi}) \ge \frac{1}{a} \). 
			(Hint: think about normalization...)

			\begin{solution}
				Firstly, it's simpler to prove that \( a S(\ket*{\psi}) \ge 1 \). This means we want to 
				show:
				\[
				aS(\ket*{\psi}) = a \sum_x |\alpha_x| = \sum_x a |\alpha_x| \ge 1
				\] 
				Now, notice that from the normalization condition, we have:
				\[
				\sum_x |\alpha_x|^2 \le  \sum_x a \cdot |\alpha_x| \le  \sum_x a^2
				\] 
				The first half of this equation is what we want, since \( \sum_x|\alpha_x|^2 = 1 \), so we have:
				\[
				1 \le \sum_x a \cdot |\alpha_x| = aS(\ket*{\psi})
				\] 
				as desired.
			\end{solution}
		\item Argue that \( H^{\otimes n} = \frac{1}{\sqrt{2^{n}} }\sum_y (-1)^{x \cdot y} \ket*{y} \) 
			(\( x \cdot y \equiv \sum_{i = 1}^{n}x_i y_i \) ). (Hint: you can use results from previous 
			homeworks.)

			\begin{solution}
				We did this exact problem on the previous homework, so I'm going to copy-paste my solution 
				from there: 

				First, consider the Hadamard on one qubit: 
				\begin{align*}
					H \ket*{0} &= \frac{\ket*{0} + \ket*{1}}{\sqrt{2} }\\
					H \ket*{1} &= \frac{\ket*{0} - \ket*{1}}{\sqrt{2} } 
				\end{align*}
				Now, consider acting \( H \) on an arbitrary state \( \ket*{x} \) :
				\[
				H\ket*{x} = \frac{\ket*{0} + (-1)^{x} \ket*{1}}{\sqrt{2} }
				\]  
				Writing it more suggestively to match the desired form:
				\[
				H \ket*{x} = \frac{(-1)^{0 \cdot x}\ket*{0} + (-1)^{1 \cdot x}\ket*{1}}{\sqrt{2} }
				\] 
				So, we can write it as follows: 
				\[
				H \ket*{x} = \frac{1}{\sqrt{2} }\sum_{y \in \{0, 1\}^{n} }(-1)^{x \cdot y} \ket*{y}
				\] 
				as desired. 
			\end{solution}
		\item Using c), the action of  \( H^{\otimes n} \) on \( \ket*{\psi} \) can be written as 
			\( H^{\otimes n}\ket*{\psi} = \sum_x \beta_x \ket*{x} \), where 
			\( \beta_x = \frac{1}{2^{n / 2}}\sum_y (-1)^{x \cdot y}\alpha_y \). 
			
			Use this to prove that for all \( y \), \( |\beta_y| \le \frac{1}{2^{n / 2}}S(\ket*{\psi}) \). 
			(Hint: use the triangle inequality.)

			\begin{solution}
				We want to show:
				\[
				|\beta_y| = \left|\frac{1}{\sqrt{2^{n}} }\sum_x (-1)^{x \cdot y}\alpha_x\right| \le \frac{1}{\sqrt{2^{n}} }
				S(\ket*{\psi})
				\] 
				Therefore, cancelling out the prefactors, we basically want to show:
				\[
				\left| \sum_x (-1)^{x \cdot y}\alpha_x \right| \le \sum_x |\alpha_x|
				\] 
				But this sum is immediately true due to the triangle inequality, which says that \( |a + b|
				\le |a| + |b|\), exactly what we have here.  
			\end{solution}
		\item Prove the uncertainty relation \( S(\ket*{\psi}) S(H^{\otimes n} \ket*{\psi}) \ge 2^{n / 2} \). 
			Justify why it makes sense to call this an uncertianty relation. 

			\begin{solution}
				Here, we leverage part (b) for this problem. In the earlier part, we proved that all 
				the coefficients of \( H^{\otimes n}\ket*{ \psi} \) are less than \( \frac{1}{2^{n / 2}}
				S(\ket*{\psi})\). From part (b), we know that if we have an upper bound on the coefficients, then 
				that gives us a lower bound on the spread of \( \ket*{\psi} \). Therefore: 
				\begin{align*}
					S(H^{\otimes n}\ket*{\psi}) & \ge \frac{1}{\frac{1}{2^{n / 2}}S(\ket*{\psi})}\\
					\therefore S(H^{\otimes n}\ket*{\psi}) \cdot S(\ket*{\psi}) &\ge 2^{n / 2}
				\end{align*}
				as desired. 
			\end{solution}
	\end{enumerate}
\end{document}
