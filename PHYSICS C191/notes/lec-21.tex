\section{Quantum Computing Platforms}
\begin{itemize}
	\item Trapped ions: qubits are single atoms, but we've removed one of the electrons so they're positively charged.
		We do this so that we have better control over them. This is the platform that's pursued by Honeywell/
		Quantinuum, AQT, among other companies.  

		\textbf{Pros:}
		\begin{itemize}
			\item These have long coherence times \( T_2 \sim 1 \) minute
			\item They operate at room temperature -- basically it's just a big vaccuum chamber sitting in a room 
				without the need for cryogenics. Nothing about the qubits require low temperature, we just happen 
				to involve cryogenics in order to achieve a better vaccuum. 
			\item Highest fidelity gates so far, and have been one of the first to discover quantum gates.  
		\end{itemize}

		\textbf{Cons:}
		\begin{itemize}
			\item Gates operate typically at 50 microseconds.  
			\item Requires lasers, optics
			\item The coulomb interaction between ions makes scaling more difficult
		\end{itemize}
	\item Neutral Ions: basically the same trapping techniques, except the atoms are neutral instead of 
		charged. We can't trap them using fields because they are neutrally charged. 

		\textbf{Pros:}
		\begin{itemize}
			\item Long qubit coherence times
			\item Room temperature operation 
			\item Inherently somewhat scalable, since neutral atoms don't interact
			\item Optical interface. 
		\end{itemize}
		\textbf{Cons:}
		\begin{itemize}
			\item Experiments usually looks like a mess (requires a lots of lasers), and much of the effort goes into 
				managing the lasers
			\item Requires an ultrahigh vaccuum (so we need cryogenics basically)
			\item Trapping is inherently more difficult and requires high laser power 
		\end{itemize}
	\item Superconducting qubits: these are man-made qubits instead of atoms. 

		\textbf{Pros}
		\begin{itemize}
			\item Chip-based architecture. It's something that we can imagine scaling up to a chip, and we interact 
				with it electronically
			\item Fast gate times (on the order of 50 nanoseconds)
			\item Previously leading the field commercially, but starting to recognize that superconducting 
				qubits might not be the way. (some comapnies are starting to invest in atomic-based approaches)  
		\end{itemize}

		\textbf{Cons:}
		\begin{itemize}
			\item Requires dilution refrigerators, so need cryogenic temperatures 
			\item Short coherence times, though this is getting much better 
			\item Control lines needed for every individual qubit -- every single qubit you want to add means an 
				extra set of lines you need to connect to your system. 
			\item Not very anharmonic. 
		\end{itemize}
	\item Quantum Dots: creating a 2D electron gas, and is possibly the most similar to the structure that we studied 
		in the previous lecture. 
		
		\textbf{Pros:}
		\begin{itemize}
			\item Semiconductor chip based architecture
			\item High fidelity, fast-qubit and two qubit gates. 
			\item Controlled by microwaves and electronics, there are no lasers required in this process. 
		\end{itemize}
		
		\textbf{Cons:}
		\begin{itemize}
			\item Requires cryogenic temperatures
			\item Short coherence times 
			\item Lots of tuning required for each device -- very sensitive architecture
			\item Scaling up to many qubits is still an open problem. We can do 2-qubit gates, but it's not clear 
				how we would scale up beyond that. 
		\end{itemize}
	\item Photonics: none of the stuff that we've talked about so far really applies here. States are single 
		photons, where the information is either stored in the polarization or which rail the photon 
		lives on. 
		
		\textbf{Pros:}
		\begin{itemize}
			\item Silicon chip based architecture
			\item Room temperature operation, what this basically means is that in principle we can do this 
				at room temperature, but the best photon detectors still require cryogenics. 
			\item Fairly easy to scale up, since photons naturally fly around
			\item "Measurement" based -- some of the gates are very easy to physically implement. For instance, 
				if you wanted to change the polarization then all you'd do is just introduce a wave plate
		\end{itemize}
		
		\textbf{Cons:}
		\begin{itemize}
			\item Gate operations are very difficult to achieve, and are inherently probabilistic. Preparing 
				the state itself is a challenge (trying again and figuring out when you succeed), then performing 
				the desired computation afterwards. 
			\item Requires identical photonic states and photonic elements
			\item Low gate fidelities
		\end{itemize}
\end{itemize}
\subsection{Trapped Ions}
\begin{itemize}
	\item Ideally we'd like to use the simplest atom possible, which in our case would be hydrogen-like 
		atoms. We can't use hydrogen itself, because transitions for hydrogen are in the UV spectrum, which poses 
		some challenges (what specifically?)
	\item Commercially, we normally use alkaline earth metals and then strip one electron off so that we end up with 
		a hydrogen-like atom. 
	\item Hydrogen-like atoms have a principal quantum number \( n \), whose energy scales with:
		\[
		E_n = -\frac{z^2 \frac{\mu}{m} E_H}{2n^2}
		\] 
		\( E_H \) is a physical constant, which is written as \( E_H = mc^2 \alpha^2 \)
	\item We also have angular momentum \( \ell \), which has possibilties between \( 0 \) to \( n - 1 \). 
	\item Transitions between these obey selection rules, which tells us that \( \Delta \ell = \pm 1 \) and 
		\( \Delta m = \pm 1 \). 
\end{itemize}
