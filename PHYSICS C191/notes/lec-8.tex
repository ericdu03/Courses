\section{Schor's Algorithm}
\subsection{Modular Exponentials}
\begin{itemize}
	\item A modular exponential is a function of the form \( f(x) = a^{x }\pmod N \), hwere \( a \) and \( N \) are assumed
		to be coprime (i.e. they only share 1 as a common factor).
	\item The order of \( f(x) \) is given by the smallest integer \( r \) such that \( f(r) = a^{r}\pmod N = 1 \). 
		Since \( r \) is defined this way, \( r \) can also be thought of as the period of \( f(x) \). 

		\begin{proof}
			Since \( f(r) \equiv 1 \pmod N \), we know that \( a^{r} = kN + 1 \), so 
			\[
			a^{r + 1} = kN a + a \equiv a \pmod N
			\] 
			And as a result, \( a^{r + x} \equiv a^{x} \pmod N \), so in general \( f(r +x) = f(x) \), so 
			\( r \) is indeed the period. 
		\end{proof}
	\item Classically, there are three possible solutions for the value of \( r \): 
		\begin{enumerate}[label=\roman*)]
			\item \( r \) is odd. This is not useful for finding nontrivial factors of \( N \). 
			\item \( r \) is even and \( a^{ r / 2} \pmod N \equiv -1 \). This implies that  \( a^{r / 1} + 1 
				= kN\); this is considered a "trivial factor" and isn't useful to us for finding factors.  
			\item \( r \) is even and \( a^{r / 2} \not \equiv 1 \). This implies (from classical number theory) 
				that at least one of \( N \) or \( a^{r / 2} \pm 1 \) (specifically the gcd of the two) is a 
				non-trivial factor of  \( N \). 
		\end{enumerate}
	\item Therefore, we want to basically find a good value of \( r \), but how do we find one in the first 
		place?
		
		Our strategy will be to evaluate \( f(x) \) at many values in parallel via quantum superposition, and 
		use QFT (quantum fourier transform) to detect the period in the sequence/distribution of values.   
\end{itemize}
\subsection{Quantum Period Finding}
\begin{itemize}
	\item We will make use of 2 registers: the first will store \( K \) qubits such that \( Q = 2^{K} \), and 
		that \( N^2 \le  Q \le  2N^2 \). The second qubit will store at least \( n = \log_2 N \) qubits.
	\item This is a six step process:
		\begin{enumerate}[label=\arabic*.]
			\item \textbf{Initialization:} we start with a state \( \ket*{0}^{\otimes K} \otimes \ket*{0}^{\otimes N} \) 
			\item \textbf{Transform to equal superposition:} Use the Hadamard gate to make everything into a
				a superposition: 
				\[
				H^{\otimes K}\ket*{0} = \frac{1}{\sqrt{2^{K}} }\sum_y  \ket*{y}  
				\] 
				\comment{The same superposition can also be constructed by using the quantum fourier transform:
					\[
						\ket*{q} \to \sum_{q = 0}^{Q - 1} \exp{\frac{2 \pi i q q'}{Q}}\ket*{q'}
					\] 
					then set \( q = 0 \) to get the same state. This confirms the earlier statement that QFT 
				and the Hadamard gate are very similar.}
			\item \textbf{Apply Quantum \( U_a \) that implements modular exponential:} In essence, it takes 
				\( q \to f(q) = a^{q} \pmod N \), and \( a \) is a random number between \( 1 \) and \( N \). 

				This function is easy to compute classically; it turns out that we can model every reversible 
				classical computation with a quantum circuit \question{Is this a result of the quantum 
				church-turing thesis?} 

				Note that all the values \( f(q) \) are distinct for all \( q \in [0, r - 1] \); this is a result 
				of \(a \) and \( N \) being coprime. 
				
				We now apply \( f \) to register 1, and store the result in register \( 2 \).
				This means our state is of the form: 
				\[
				\frac{1}{\sqrt{Q} } = \sum_{q = 0}^{Q - 1}\ket*{q} \ket*{a^{q} \pmod N}
				\] 
				So the second state is in the state given by \( f(q) \). 
			\item \textbf{Measure Register 2:} Note that even though we have \( Q \) values, there are only 
				\( r \) distinct ones due to the periodicity of \( f(q) \). 

				We'll simplify by assuming that \( Q \) is an integer multiple of \( r \). This tells us that there are 
				exactly \( m \)  differnet \( q \) with the same value for \( f(q) \).  

				\question{This is not a necesary condition?}. 

				Therefore, there are \( m = \frac{Q}{r} \) different states in register \( 1 \) which contribute 
				to the state after measuring register 2 (remember that we've collapsed the state into those 
				that are \( f(q) \equiv m \pmod N \), of which there are \( m \) states in regsiter 1 
				that contribute to this state. We can write this as:
				\[
				\frac{1}{\sqrt{Q / r} }\sum_{j = 0}^{Q - 1}\ket*{jr + q_0}\ket*{f(q_0)}
				\] 
				Note now that the periodicity has shifted to register 1, since register 2 now measures a defined 
				value \( f(q_0) \). At this point, we no longer need register 2, so we'll stop writing it down.  
			\item \textbf{Periodic Superposition of states in Register 1:} Now we apply the quantum Fourier 
				transform modulo \( Q \), so we get: 
				\[
				\frac{1}{\sqrt{Q / r} }\sum_{j = 0}^{Q - 1}\ket*{jr + q_0} \to \frac{1}{\sqrt{r} }\sum_{k = 0}^{r}
				\omega^{kq_i}\ket{\frac{kQ}{r}}
				\] 
				But now we've reduced the number of states from \( Q / r \) to \( r \) terms:
				\[
				\ket*{\Phi_{q_0}} = \frac{1}{\sqrt{Q} }\sum_{a = 0}^{Q - 1}g(a) \ket*{a}
				\] 
				where \( g(a) = \sqrt{\frac{r}{Q}}  \) if \( a - q_0 \) is a multiple of \( r \), and 0 otherwise. 
				Basically, although this looks like a superpositoin of \( Q \) states, it is in reality only 
				a superposition of \( r \) states, since a lot of them are zero. 

				We apply QFT again and we get: 
				\[
					\sum_{c}\frac{\sqrt{r} }{Q}\sum_{j = 0}^{Q / r - 1 }\exp{2 \pi i (jr + q_0)c} \ket*{c} 
					= \sum_c \frac{\sqrt{r} }{Q} \left[ \sum_{j = 0}^{Q / r - 1}\exp{2 \pi i j \left( \frac{rc}{Q} \right)} \exp{\frac{2 \pi i q_0c}{Q}} \right] \ket*{c}
				\] 
				Then, since \( Q = mr \) is measured, then the state is: 
				\[
					\sum_{c}\sqrt{\frac{r}{Q}} \frac{Q}{r}\exp{\frac{2 \pi i q_0c}{Q}} \ket*{c}
				\] 
				If we then write \( c = \frac{kQ}{r} \)
		\end{enumerate}
\end{itemize}
