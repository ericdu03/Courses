\section{Multiple Qubits, Entanglement}
\subsection{Multiple Qubits}
\begin{itemize}
	\item Suppose we have two qubits \( \ket*{\psi_1} = \alpha\ket*{0} + \beta \ket*{1} = 
		\begin{bmatrix} \alpha\\ \beta \end{bmatrix}  \) and \( \ket*{\psi_2} = x\ket*{0} + y\ket*{1} = 
		\begin{bmatrix} x\\y \end{bmatrix}  \)
	\item Then the combined state, if the two qubits live on their own, is given by \( \ket*{\Psi} = \ket*{\psi_1} 
		\otimes \ket*{\psi_2} \). The \( \otimes \) symbol denotes a tensor product.
		\[
		\ket*{\psi_1} \otimes \ket*{\psi_2} = (\alpha \ket*{0} + \beta \ket*{1}) \otimes (x \ket*{0} + y \ket*{1})
		\] 
		In matrix form, this is represented as: 
		\[
		\begin{bmatrix} \alpha\\ \beta \end{bmatrix} \otimes \begin{bmatrix} x\\y \end{bmatrix} = 
		\begin{bmatrix} \alpha x\\ \alpha y\\\beta x\\\beta y\end{bmatrix} 
		\] 
		the resulting vector lives in \( \C^{4} \), with the basis states \( \ket*{00} = \ket*{0}_1 \ket*{0}_2, 
		\ket*{10} = \ket*{1}_1 \ket*{0}_2, \ket*{01} = \ket*{0}_1 \ket*{1}_2, \ket*{11} = \ket*{1}_1 \ket*{1}_2\)
	\item In general given \( n \) qubits, there are \( 2^{n} \) basis states, and hence we will be working with 
		superpositions over these \( 2^{n} \) basis states. This fact underscores the power of quantum computers, 
		since they scale much more efficiently than classical computers. This is also sometimes referred to as 
		"quantum parallelism". 
	\item If we measure all qubits, then the outcome is just some sort of bitstring, so we have to be clever about 
		how we are measuring to get the information we want. 
	\item With multiple qubits, operators are also tensor products. Given the two operators:
		\begin{align*}
			A &= \begin{pmatrix} a_{11} & a_{12}\\ a_{21} & a_{22} \end{pmatrix} \\
			B&= \begin{pmatrix} b_{11} & b_{12}\\ b_{21} & b_{22} \end{pmatrix}  
		\end{align*}
		Then \( A \otimes B \), the operator that acts on the multi-qubit state, is given by 
		\[
			A \otimes B = \begin{pmatrix} a_{11}B & a_{12}b\\a_{21}B & a_{22}B \end{pmatrix} 
		\] 
		Note that \( A \otimes B \) is not the same as \( B \otimes A \). 
\end{itemize}
\subsection{Quantum Circuits}
\begin{itemize}
	\item A generic quantum circuit is written as:
		\begin{center}
			\begin{quantikz}
				\lstick{\( \ket*{\psi} \) } & \gate{H} & 
			\end{quantikz}
		\end{center}
		the box with an \( H \) denotes a gate (in this case, a Hadamard gate), which corresponds to a rotation 
		on the Bloch sphere.  
	\item Let's analyze the following quantum circuit:
		\begin{center}
			\begin{quantikz}
				\lstick{\( \ket*{\psi_1} \) }\slice{1} & \gate{H} \slice{2} & \ctrl{1} \slice{3} &  \slice{4} &
				\gate{Y}\slice{5} & \meter{} \slice{6}& \setwiretype{c}\\
				\lstick{\( \ket*{\psi_2} \) } & & \targ{} & \gate{X} & & \meter{} &\setwiretype{c}
			\end{quantikz}
		\end{center}
		Let's analyze this in stpes:
		\begin{itemize}
			\item Initially, we have \( \ket*{\psi_1} = \alpha_1 \ket*{0} + \beta_1 \ket*{1}\)  and \( \ket*{\psi_2} = 
				\alpha_2 \ket*{0} + \beta_2 \ket*{1}\), whose combination can be written as:
				\[
					\ket*{\psi_{12}^{(1)}} = \alpha_1 \alpha_2 \ket*{00} + \alpha_1 \beta_2 \ket*{01} + \beta_1 \alpha_2
					\ket*{10} + \beta_1 \beta_2 \ket*{11}
				\]
			\item At step 2, we run the first qubit through a Hadamard gate, and leave the second qubit untouched. This
				means we act the operator \( H \otimes I \) on the state:
				\begin{multline*}
					\ket*{\psi_{12}^{(2)}}H \otimes I \ket*{\psi_{12}} = \alpha_1 \alpha_2 \left( \frac{1}{\sqrt{2}} 
					\ket*{00} + \frac{1}{\sqrt{2} }\ket*{10}  \right) 
					+ \alpha_1 \beta_2 \left( \frac{1}{\sqrt{2} }\ket*{01} + \frac{1}{\sqrt{2} }\ket*{11} \right) \\
					+ \beta_1 \alpha_2 \left( \frac{1}{\sqrt{2} }\ket*{00} - \frac{1}{\sqrt{2} }\ket*{10} \right) +
					\beta_1\beta_2 \left( \frac{1}{\sqrt{2} }\ket*{01} - \frac{1}{\sqrt{2} }\ket*{11} \right) 
				\end{multline*} 
			\item At step 3, we apply a CNOT gate, which flips the state of the second bit if the value of the first 
				bit is 1. As a truth table:
				\begin{center}
					\begin{tabular}{c|c}
						Input & Output\\
						\hline\\
						00 & 00\\
						01 & 01\\
						10 & 11\\ 
						11 & 10
					\end{tabular}
				\end{center}
				As a matrix, it's written as;
				\[
					\text{CNOT} = 
					\begin{bmatrix} 1 & 0 & 0 & 0\\ 9 & 1 & 0 & 0\\ 0 & 0 & 0 & 1\\ 0 & 0 & 1 & 0 \end{bmatrix} 
				\] 
				We then apply this CNOT gate to each component of \( \ket*{\psi_{12}^{(2)}} \) to get 
				\( \ket*{\psi_{12}^{(3)}} \). 
			\item Apply \( I \otimes X \) to \( \ket*{\psi_{12}^{(3)}} \to \ket*{\psi_{12}^{(4)}}\) 
			\item Apply \( Y \otimes I \) tp \( \ket*{\psi_{12}}^{(4)}  \to \ket*{\psi_{12}^{(5)}}\)
			\item Measurement in the \( Z \) basis, by applying projection operators to the final resulting state.  
		\end{itemize}
\end{itemize}
\subsubsection{Other Common Gates}
\begin{itemize}
	\item There are many quantum gates that we'll study, here's a list of them that will be useful:
	\item CPHASE, or controlled \( Z \) gate
	\item Swap gate: swaps the 
	\item S-phase: rotation by 90 degrees, \( \begin{pmatrix} 1 & 0 \\ 0 & i \end{pmatrix}  \)
	\item P-phase: a general phase gate \( \begin{pmatrix} 1 & 0 \\ 0 & e^{i \theta} \end{pmatrix}  \)
	\item Toffoli gate: controlled-controlled NOT gate:
		\begin{center}
			\begin{quantikz}
				& \ctrl{2} & \\
				& \ctrl{1}& \\
				& \targ{}& 
			\end{quantikz}
		\end{center}
	\item T-gate: \( \begin{pmatrix} 1 & 0\\ 9 & e^{i \pi / 4j} \end{pmatrix}  \)
\end{itemize}
\subsubsection{Universal Gate Sets}
\begin{itemize}
	\item A set \( G \) of quantum gates is considered universal if for \( \epsilon > 0 \) and for any unitary matrix 
		\( U \) on \( n \) qubits, there is a sequence of gates from \( G \) such that 
		\[
			\|U - U_{g_{\ell}} \cdots U_{g_2}U_{g_1}\| < \epsilon
		\]
		In this definition, we define \( U_g = V \otimes I \), where \( V \) is an operator acting on \( k \) qubits, 
		and \( I \) acts on the remaining  \( n - k \) qubits. The double bar represnets an operator norm, defined as:
		\[
		\|U - U'\|= \max_{\text{\( \ket*{v} \) unit vectors}} \|(U - U') \ket*{v}\|
		\] 
		where \( \|w\| = \sqrt{\braket*{w}{w}} \).
	\item Examples of universal gate sets:
		\begin{itemize}
			\item Barenco et al. (1995): CNOT and all single qubit (continuous) gates. 
			\item CNOT, H, S, T gates
			\item Rotation operators \( R_x(\theta), R_y(\theta), R_z(\theta) \), the phase operator \( P_\phi \) and 
				CNOT.
		\end{itemize}
\end{itemize}
\subsection{Entanglement}
\begin{itemize}
	\item Consider 2 qubits:
		\begin{center}
			\begin{quantikz}
				\lstick{\( \ket*{0} \) } & \gate{H} & \ctrl{1} & \rstick[2]{\( \ket*{\psi} = ?\)}\\
				\lstick{\( \ket*{0} \)} & & \targ{} & 
			\end{quantikz}
		\end{center}
		Well, we first start with the state \( \ket*{00} \), and after passing the first bit through a Hadamard gate, 
		we get the state
		\[
			\frac{\ket*{00} + \ket*{10}}{\sqrt{2} }
		\] 
		Then, running it through the CNOT, then we have:
		\[
		\frac{\ket*{00} + \ket*{11}}{\sqrt{2} } = \Phi^{+}
		\] 
		This is one of four states called the "Bell states", because there is no way to express this state as a product 
		state of two individual qubits. 
\end{itemize}
