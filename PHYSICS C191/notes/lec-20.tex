\section{Guest Lecture: Superconducting Qubits}
\subsection{Josephson Junctions}
\begin{itemize}
	\item Harmonic oscillators are linear, so we can't access individual transitions, since it's hard to control 
		the energy levels if they're all spaced by \( \hbar \omega \). 
	\item So, we employ a Josephson junction, which makes the energy spacing highly nonlinear. 
	\item It's basically two superconductors that sandwich an insulator:
		\( S(\phi_1) - I - S(\phi_2) \), and \( \delta = \phi_1 - \phi_2 \). 
	\item Here, the current is modeled as \( I(\delta) = I_0 \sin \delta \), and \( V(\delta) = \frac{\hbar}{2e}
		\dv{\delta}{t}\), which is what we call a \textit{nonlinear inductor}.  
	\item This nonlinearity is so strong that we don't actually have to go to very high energies in order to 
		feel the nonlinearity. 
\end{itemize}
\subsection{DiVincenzo's Criteria}
\begin{itemize}
	\item Recall the five criteria, and how we achieve these criterion using superconducting qubits.  
		\begin{enumerate}[label=\arabic*)]
			\item Stability
				\begin{itemize}
					\item The quantum circuit is a combination of the qubit, a coupler and a cavity. The cavity 
						is a reasonator, and is connected to the qubit via a capacitor. This gives us a total 
						Hamiltonian:
					\item In the dispersive limit, or in other wrods wehn \( \Delta = \omega_q - \omega_c \gg g \), 
						then:
						\[
						H = \omega_c a^{\dagger} a + \frac{1}{2}\omega_q \sigma_z + \chi(\sigma_z a^{\dagger}a)
						\] 
						We can then group the terms:
						\[
						H = (\omega_c + \chi \sigma_z) a^{\dagger}a + \frac{1}{2}\omega_q \sigma_z
						\] 
						This actually allows us to discern between whether the qubit is in the \( \ket*{0} \) or 
						\( \ket*{1} \) state without destroying the qubit. We can also measure the state of multiple 
						qubits with a single pulse, which makes this very efficient. 
				\end{itemize}
			\item Initialization 
				\begin{itemize}
					\item Modern refrigerators get down to about 10 mK, 
						and by cooling the circuit down to such temperatures we can 
						initialize the qubits in the ground state  \( \ket*{g} \). 
				\end{itemize}
			\item Long decoherence

				\begin{itemize}
					\item To look at decoherence, it's useful to look at Fermi's golden rule:
						\[
							\Gamma_{i \to f} = \frac{2\pi}{\hbar} S(E_{if}) |\mel{f}{H}{i}|^2
						\] 
				\end{itemize}
			\item Universal quantum gates

				\begin{itemize}
					\item We can apply a controlled pulse to alter the state of a single qubit. Then, using what we 
						learned last lecture, this can be combined with another pulse in order to basically produce
						any qubit on the Bloch sphere.
					\item Our ability to do this demonstrates the capability of realizing a universal set of quantum 
						gates. 
					\item However, there is error in this: if we send in too short of a \( \pi \)-pulse (like a
						square wave), then this widens the spectrum in frequency space. This is bad, because this 
						could potentially lead to the qubit jumping from the ground state \( \ket*{g} \) into 
						higher energy states like \( \ket*{f} \), which we count as leakage. 
				\end{itemize}
			\item Qubit specific measurement
		\end{enumerate}
\end{itemize}
