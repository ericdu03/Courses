\section{Quantum Fourier Transform}
\begin{itemize}
	\item This is perhaps the most important lecture in all of quantum information. 
	\item Classically, the Fourier Transform is defined as: 
		\[
		y_k \equiv \frac{1}{\sqrt{2^{n}} }\sum_{j = 0}^{2^{n} - 1} x_j e^{2 \pi i j k / 2^{n}}
		\] 
	\item The important thing about this definition is the phase factors. If we let \( N = 2^{n} \), the form 
		looks a bit more familiar: =
		\[
		y_k = \frac{1}{\sqrt{N} }\sum_{k = 0}^{N - 1} \omega^{-jk}f_k
		\] 
		where \( \omega^{-jk} = \exp{-2 \pi i /N} \). 
	\item Quantumly, the Quantum fourier transform is:
		\[
		\ket*{j} \to \frac{1}{\sqrt{2^{n}} }\sum_{k = 0}^{2^{n} - 1} e^{2 \pi i j k / 2^{n}} \ket*{k}
		\] 
		The states \( \ket*{k} \) are analogous to the phases in the classcal Fourier transform, and are represented by
		an integer basis state.
	\item So an arbitrary state \( \ket*{\psi} \) can be written as: 
		\[
		\ket*{\psi} = \sum_{j = 0}^{2^{n} - 1} y_j \ket*{k}
		\] 
	\item Here, we'll represnet \( j \) as a binary number: \( j = j_{12}^{n - 1} + j_2 2^{n - 2} + \cdots + 
		j_n 2^{0}\), and let \( k \) be defined similarly.
	\item Now, we're going to write \( k \) out in terms of this binary expansion, and we get:
		\[
		\frac{1}{\sqrt{2^{n}} }\sum_{k = 0}^{2^{n} - 1}e^{2 \pi i j k_1 2^{-1}} \times e^{2 \pi i j k_2 2^{-2}} \times 
		\cdots \times e^{2 \pi i j k_n 2^{-n}}\ket*{k_1k_2\dots k_n}
		\] 
	\item After a bit more math (see notes), we ultimately get the following expansion:
		\[
		\frac{1}{\sqrt{2} }(\ket*{0} + e^{2 \pi ij 2^{-1}}\ket*{1})(\ket*{0} + e^{2 \pi i j 2^{-2}}\ket*{1}) 
		\cdots \left( \ket*{0} + e^{2 \pi i j 2^{-n}}\ket*{1} \right) 
		\] 
		So this is basically just a tensor product of a bunch of states, each with a different phase 
		factor \( e^{2 \pi i j / 2^{k}} \) where \( k \) ranges from 1 to \( n \). 
	\item To put this in a form which we can exploit, we'll rewrite the quantity \( \frac{j}{2^{k}} \) in binary 
		fractional form, so that it's easy to work with. Specifically, write \( \frac{j}{2^{k}} \) as a 
		"decimal":
		\begin{align*}
			\frac{j}{2^{k}} &= \sum_{\nu}^{n}j_{\nu}2^{n - \nu - k} \\
			&= j_1j_2\dots j_{n-k}.j_{n-k+1}\dots j_n 
		\end{align*}
	\item Now we use this representation on the phase factors we found earlier. For the values of 
		\( \frac{j}{2^{k}} \) that are larger than 1, notice we can just pull it out and instead represent it 
		entirely by a binary fraction:
		\[
		e^{2 \pi i \frac{j}{2^{k}}} = 1\times e^{2 \pi i (0.j_1j_2\dots j_n)}
		\] 
		Doing this for every term, we get:
		\[
		\frac{1}{\sqrt{2^{n}} }\left( \ket*{0} + e^{2 \pi i (0.j_n)}\ket*{1} \right) 
		\left( \ket*{0} + e^{2 \pi i (0.j_{n-1}j_n)}\ket*{1} \right) \cdots 
		\left( \ket*{0} + e^{2 \pi i (0.j_1j_2\dots j_n)} \right) 
		\] 
		\question{Why does the decimal representation get more complex as we go down the list?} 
	\item To see how to implement this, first notice that the \( n - l + 1 \)-th qubit is determined by 
		the expression: 
		\[
		\frac{1}{\sqrt{2} }\left( \ket*{0} + e^{2 \pi (0.j_l \dots j_n)} \right) 
		\] 
		We'll first pull out the first component of the phase:
		\[
		\frac{1}{\sqrt{2} }\left( \ket*{0} + e^{2 \pi i (0.j_l)} \times e^{ 2 \pi (0.0 j_{l + 1} \dots j_n)} \right) 
		\] 
		This first component can be rewritten as:
		\[
		\frac{1}{\sqrt{2} }(\ket*{0} + e^{2 \pi i (0.j_l)} = \frac{1}{\sqrt{2} }
		(\ket*{0} + e^{2 \pi i j_l / 2} \ket*{1}) = \frac{1}{\sqrt{2} }(\ket*{0} + (-1)^{j_l}\ket*{1})
		\] 
		where we've used the definition of the binary fraction and also the fact that \( e^{i\pi j_l} = (-1)^{j_l} \).
		This looks exactly like what we would get if we acted the Hadamard gate on \( \ket*{j_l} \) ! 

		\comment{This is becuase \( (-1)^{j_l} = 0 \) if \( j_l = 0 \), so we get the \( + \) in the middle, and 
		the \( - \) in the middle if we had \( j_l = 1 \).}
	\item With this first bit taken care of, we repeat the same process to get the second gate, and so on. Therefore,
		the rest of the components can be done by implementing a series of rotations: 
		\[
			R_k = \begin{bmatrix} 1& 0\\0 & e^{2 \pi i / 2^{k}} \end{bmatrix} 
		\] 
		with each value being controlled by the value of the \( j_k \)-th qubit. This is becuase we only want the 
		rotation to be applied when \( j_k = 1 \). 
	\item So in general, we just apply a bunch of controlled \( H \) gates to get the final state. 
	\item However, when the state is prepared, notice that the states are in reverse order of what we want them to be. 
		That's okay, however, since we have SWAP gates that allow us to swap the states so that we get them in the 
		right order. Recall that a SWAP gate is basically just three CNOT gates put alongside each other in 
		reversed order.  
	\item In terms of runtime complexity, the total number of \( R \) gates needed is \( n(n+1) / 2 \), which means 
		that the Quantum Fourier transform is \( O(n^2) \). Compared to the classical algorithm which completes
		the same task in \( O(n \log n) \), this is much faster. 
	\item The QFT is also unitary, just like the Classical FT. This makes sense, since had it not been unitary then 
		we'd have massive issues. 
\end{itemize}


