\documentclass[10pt]{article}
\usepackage{../../local}
\urlstyle{same}

\newcommand{\classcode}{Physics C191}
\newcommand{\classname}{Introduction to Quantum Computing}
\renewcommand{\maketitle}{%
\hrule height4pt
\large{Eric Du \hfill \classcode}
\newline
\large{HW 02} \Large{\hfill \classname \hfill} \large{\today}
\hrule height4pt \vskip .7em
\small{Header styling inspired by CS 70: \url{https://www.eecs70.org/}}
\normalsize
}
\linespread{1.1}
\begin{document}
	\maketitle
	\section*{Problem 1}
	\begin{enumerate}[label=\alph*)]
		\item Consider a general qubit basis \( \ket*{v}, \ket*{v^{\perp}} \) where \( \ket*{v} = a\ket*{0} + 
			b\ket*{1}\), and \( \ket*{v^{\perp}} = b^* \ket*{0} - a^* \ket*{1} \) are arbitrary 
			normalized vectors. Show that \( \ket*{v} \) and \( \ket*{v^{\perp}} \) are 
			orthogonal.

			\begin{solution}
				To show that they're orthogonal, we can take the inner product of the two:
				\begin{align*}
					\braket*{v^{\perp}}{v} &= (\bra*{0}b - \bra*{1}a)(a\ket*{0} + b\ket*{1})\\
					&=  ba \braket*{0}{0}  + b^2 \braket*{0}{1} - a^2\braket*{1}{0} + ab \braket*{1}{1}	\\
					&= ab - ba \\
					&= 0 
			\end{align*} 
			And since the inner product evaluates to 0, then these two vectors are orthogonal.
			\end{solution}
		\item Prove that the Bell state \( \ket*{\psi^{-}} = \frac{1}{\sqrt{2} }(\ket*{vv^{\perp}} - 
			\ket*{v^{\perp}v})\) with \( \ket*{v} \) annd \( \ket*{v^{\perp}} \) perpendicular 
			normalized vectors, is invariant under rotations of the two qubits (applying the same rotation on 
			both qubits). i.e., taking the form of \( \ket*{v} \) and \( \ket*{v^{\perp}} \) as in (a), show that 
			the state \( \ket*{\psi^{-}} = \frac{1}{\sqrt{2} }(\ket*{vv^{\perp}} - \ket*{v^{\perp}v}) \) will 
			always be equal to \( \frac{1}{\sqrt{2} }(\ket*{10} - \ket*{01}) \). 
	\end{enumerate}
	\pagebreak
	\section*{Problem 2}
	Consider the state \( \ket*{\phi} = \cos \phi \ket*{0} + \sin \phi \ket*{1} \). Suppose that with 1/2 probability 
	you are given the state \( \ket*{\phi} \) and with 1/2 probability you're given the state \( \ket*{0} \), but 
	you don't know which one you were given. What measurement basis is optimal to distinguish the two states, i.e., 
	to guess with the greatest likelihood which of the two states you have been give? 

	In the following you will prove that the basis \( \{\ket*{a}, \ket*{b}\} \) that maximizes the probability of 
	distinguishing the two states is given by 
	\begin{align}
		\ket*{a} &=  \cos(\pi / 4 + \phi / 2) \ket*{0} + \sin(\pi /4 + \phi / 2) \ket*{1} \\
		\ket*{b} &=  \sin(\pi / 4 + \phi / 2) \ket*{0} - \cos(\pi / 4 + \phi / 2) \ket*{1} 
	\end{align}
	Note that this basis sares the same angle bisector as te one between \( \ket*{0} \) and \( \ket*{\phi} \) 
	as shown in the drawing below. 
	\begin{center}
		\begin{tikzpicture}[scale=3]
			\draw(-1, 0) -- (1, 0);
			\draw(0, -1) -- (0, 1); 
			\draw[thick, -stealth, red] (0, 0) -- (1, 0) node[above] {\( \ket*{0} \) };
			\draw[thick, -stealth, red] (0, 0) -- (60:1) node[above right] {\( \ket*{\phi} \) };
			\draw[thick, -stealth, blue] (0, 0) -- (-20:1) node[below right] {\( \ket*{b} \) };
			\draw[thick, -stealth, blue] (0, 0) -- (70:1) node[above left] {\( \ket*{a} \) };
		\end{tikzpicture}
	\end{center}
	So how do we go about proving this? First, we have to quantify what we mean by distinguishing the two states
	 and also define the probability of doing this successfully. To do this, use basic concepts of probability theory 
	 to argue that the probability of guessing the correct state from a measurement with basis 
	 \( \{\ket*{a}, \ket*{b}\}  \) is 
	 \begin{equation}\label{prob} 
	 \frac{1}{2}|\braket*{a}{\phi}|^2 + \frac{1}{2}|\braket*{b}{0}|^2
	 \end{equation} 
	 where outcome \( a \) would imply the state is \( \ket*{\phi}\) and outcome \( b \) would imply the 
	 state is \( \ket*{0} \). 

	 Now show that the optimal basis to distinguish these 2 states is on the real plane. To do so consider 
	 a general parametrization of our measurement basis as 
	 \begin{align}
		 \ket*{a} &= \cos(\theta) \ket*{0}+ \sin(\theta) e^{i \gamma}\ket*{1} \\
		 \ket*{b} &= \sin(\theta) e^{-i \gamma}\ket*{0} - \cos(\theta) \ket*{1} 
	 \end{align}
	 By plugging this in, you should be able to deduce that the guessing probability is maximized when 
	 \( e^{i \gamma} \) is \( +1 \) or \( -1 \), so the optimal basis is indeed on the real plane. 

	 So the measurement basis parametrization simplifies to 
	 \begin{align*}
	 \ket*{a} &=  \cos(\theta) \ket*{0} + \sin(\theta) \ket*{1} \\
	 \ket*{b} &= \sin(\theta) \ket*{0} - \cos(\theta) \ket*{1} 
	 \end{align*} 
	 Now you can write out the corresponding probability of successfully distinguishing the states, Eq. 
	 (\ref{prob}), in terms of \( \theta \). Find the maximum value of this with respect to the angle \( \theta \) 
	 that defines the optimal measurement basis vectors \( \ket*{a} \) and \( \ket*{b} \). This should give 
	 you an equation defining one or more possible values of \( \theta \) in the relevant range 
	 \( [0, 2\pi) \). Insert each of your solutions back into the guessing function, 
	 eq. (\ref{prob}) to identify the value of \( \theta \) that gives the maximum probability.

	 \begin{solution}
	 	
	 \end{solution}
	 \pagebreak
	 \section*{Problem 3}
	 This problem will have you explore basic quantum operations on states. Consider the following situation: 
	 you start with the two qubit state
	 \[
		 \ket*{\psi_0} = \frac{1}{2}\ket*{00} + \frac{i}{2}\ket*{01} - \frac{1}{\sqrt{2}} \ket*{11}
	 \] 
	 Next you apply a Hadamard gate to the first qubit then a CNOT gate with the first qubit as the control 
	 and the second as the target. 
	 \begin{enumerate}[label=\alph*)]
	 	\item Verify that the initial state is normalized. 

			\begin{solution}
				To verify that it's normalized, we have: 
				\[
					\left( \frac{1}{2} \right) ^2 + \left( \frac{i}{2} \right) ^2 + \left( \frac{1}{\sqrt{2} } \right) ^2 = \frac{1}{4} + \frac{1}{4} + \frac{1}{2} = 1
				\] 
				Since the total sum is 1, then the state is fully normalized.  
			\end{solution}
		\item Draw a quantum circuit diagram representing this series of operations. 

			\begin{solution}
				The quantum cricuit can be drawn as:
				\begin{center}
					\begin{quantikz}
						\lstick[2]{\( \ket*{\psi_0} \) } & \gate{H} & \ctrl{1} & \\
													   & & \targ{} & 
					\end{quantikz}
				\end{center}
			\end{solution}
		\item Write the intermediate state after application of the Hadamard gate. 
			Argue that it is also noramlized (hint: this can be done without explicit calculation using properties 
			of unitary operations).

			\begin{solution}
				The Hadamdard gate does the following operation on the qubits:
				\begin{align*}
					\ket*{0} &\mapsto \frac{\ket*{0} + \ket*{1}}{\sqrt{2} }\\
					\ket*{1} &\mapsto \frac{\ket*{0} - \ket*{1}}{\sqrt{2} }
				\end{align*}
				Therefore, we have: 
				\begin{align*}
					\ket*{\psi_1}&= \frac{1}{2}\left( \frac{\ket*{0} - \ket*{1}}{\sqrt{2}} \right)\ket*{0}
					+ \frac{i}{2}\left( \frac{\ket*{0} + \ket*{1}}{\sqrt{2} } \right) \ket*{1}
					- \frac{1}{\sqrt{2} }\left( \frac{\ket*{0} -\ket*{1}}{\sqrt{2} } \right) \ket*{1}\\
					&= \frac{1}{2\sqrt{2} }\ket*{00} + \left( \frac{1}{2\sqrt{2} } - \frac{1}{2} \right) \ket*{01}
					- \frac{1}{2\sqrt{2} }\ket*{10} + \left( \frac{i}{2\sqrt{2} } + \frac{1}{2} \right) \ket*{11}
				\end{align*} 
				This state must be normalized becuase a unitary gate is a norm-preserving transformation.
			\end{solution}
		\item What is the final state at the end of the circuit. Is it normalized?

			\begin{solution}
				We then send the bit through a CNOT, so the final state can be written as:
				\[
				\ket*{\psi_2} = \frac{1}{2\sqrt{2} }\ket*{00} + \left( \frac{i}{2\sqrt{2} } - \frac{1}{2} \right) 
				\ket*{01} - \frac{1}{2\sqrt{2} }\ket*{11} + \left( \frac{i}{2\sqrt{2} } + \frac{1}{2} \right) \ket*{10}
				\] 
				The factors in front of all the states are the same, and since the previous part is normalized, 
				then so is this one.
			\end{solution}
	 \end{enumerate}
	 \pagebreak
	 \section*{Problem 4}
	 Circuit identities are mathematical equivalences between different operations on quantum registers. They can be 
	 useful in converting between desired algorithmic opreations and the restricted set of possible operations 
	 on a given processor, i.e., for compiling. This question will have you explore several different 
	 circuit idnetities. 
	 \begin{enumerate}[label=\alph*)]
	 	\item Prove the following 2-qubit identities: 
			\begin{center}
				\begin{quantikz}
					& \ctrl{1} & \\
					& \targ{} &
				\end{quantikz}
				$=$ 
				\begin{quantikz}
					& &\ctrl{1} & &\\
					& \gate{H} & \gate{Z} & \gate{H} & 
				\end{quantikz}
			\end{center}

			\begin{center}
				\begin{quantikz}
					& \ctrl{1} & \\
					& \gate{Z} &
				\end{quantikz}
				$=$
				\begin{quantikz}
					&\gate{Z} & \\ 
					& \ctrl{-1}& 
				\end{quantikz}
			\end{center}
			The second identity shows that the control and target in the controlled-Z gate are symmetric, and so 
			the gate is often denoted as: 
			\begin{quantikz}
				& \ctrl{1} &\\ 
				& \ctrl{-1}& 
			\end{quantikz}

			\begin{solution}
				We can do this by showing that the resulting matrices from the tensor product of the two 
				result in the same matrix:
			\end{solution}
		\item Prove that the direction of the CNOT is reversed in the Hadamard basis, i.e., show the following 
			circuit identity (hint: use the prior part):
			\begin{center}
				\begin{quantikz}
					& \ctrl{1} &\\
					& \targ{}& 
				\end{quantikz}
				\( = \) 
				\begin{quantikz}
					& \gate{H} & \targ{} & \gate{H}&\\
					&\gate{H} & \ctrl{-1} & \gate{H}
				\end{quantikz}
			\end{center}
	 \end{enumerate}
	 \pagebreak
	 \section*{Problem 5}
	 The following circuit identities will be useful when we study quantum error correction. The point is that 
	 single qubit operations can be "pushed through" conditional gates. Prove each one.
	 \begin{enumerate}[label=\alph*)]
	 	\item 
			\begin{quantikz}
				& \gate{X} & \ctrl{1} & \\
				& & \targ{}& 
			\end{quantikz} 
			$=$ 
			\begin{quantikz}
				& \ctrl{1} & \gate{X} & \\
				& \targ{} & \gate{X} & 
			\end{quantikz}
		\item 
			\begin{quantikz}
				& \gate{Z} & \ctrl{1} & \\
				& & \targ{}&
			\end{quantikz}
			\( = \) 
			\begin{quantikz}
				& \ctrl{1} & \gate{Z} & \\
				& \targ{} & &
			\end{quantikz}
		\item 
			\begin{quantikz}
				& & \ctrl{1} & \\
				& \gate{X} & \targ{} &
			\end{quantikz}
			\( = \) 
			\begin{quantikz}
				& \ctrl{1} & & \\
				& \targ{} & \gate{X}& 
			\end{quantikz}
		\item 
			\begin{quantikz}
				& & \ctrl{1} & \\
				& \gate{Z} & \targ{} &
			\end{quantikz}
			\( = \)
			\begin{quantikz}
				& \ctrl{1} & \gate{Z} & \\
				& \targ{} & \gate{Z} & 
			\end{quantikz}
	 \end{enumerate}
\end{document}
