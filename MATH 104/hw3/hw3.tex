\documentclass[10pt]{article}
\usepackage{../local}

\newcommand{\classcode}{Math 104}
\newcommand{\classname}{Real Analysis}
\renewcommand{\maketitle}{%
\hrule height4pt
\large{Eric Du \hfill  \classcode}
\newline
\large{HW 03} \Large{\hfill  \classname \hfill } \large{\today}
\hrule height4pt \vskip .7em
\normalsize
}
\linespread{1.1}
\begin{document}
		\maketitle

		\section*{Problem 1}

		Let $s_n = n!/n^n$. Prove that $s_n \to 0$ as $n \to \infty$.

		\begin{solution}
				First, we rewrite $s_n$: 
				\[ \frac{n!}{n^n} = \frac{n(n-1) \cdots 1}{n \cdot n \cdots n} = \frac{n-1}{n} \cdot 
				\frac{(n-2)(n-3)\cdots 1}{n \cdot n \cdots n} \le \frac{n-1}{n}\]
				We also know that this sequence is bounded below by 0 since $n$ is positive, so if we can prove
				that 
				\[ \lim_{n \to \infty} \frac{n-1}{n} = 0 \] 
				Then we've solved the problem. To do this, we look for a value of $N$ such that for all 
				$\epsilon > 0$: 
				\begin{align*}
						|\frac{n-1}{n} - 0| &< \epsilon\\
						\frac{n-1}{n} &< \epsilon\\
						n-1 &< n\epsilon\\
						\therefore n &> \frac{1}{1-\epsilon}
				\end{align*}
				So therefore if we let $N = \frac{1}{1-\epsilon}$ then we satisfy this inequality for all 
				$\epsilon > 0$. Therefore, we've proven the limit, so we now have
				\[ 0 \le \lim_{n \to \infty} s_n \le 0\]
				which implies that $\lim_{n \to \infty} s_n = 0$.
		\end{solution}

		\pagebreak
		\section*{Problem 2}
		Let $(t_n)$ be a bounded sequence, i.e. there exists $M$ such that $|t_n| \le M$ for all $n$, and let 
		$(s_n)$ be a sequence such that $\lim s_n = 0$. Prove $\lim(s_nt_n) = 0$.

		\begin{solution}
				Since $t_n$ is bounded, then we know that our sequence satisfies:
				\[ -Ms_ \le t_n s_n \le Ms_n \ \forall n \in \mathbb N\]
				
		\end{solution}

\end{document}
