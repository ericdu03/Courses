\documentclass[10pt]{article}
\usepackage{../local}


\newcommand{\classcode}{Math 104}
\newcommand{\classname}{Introduction to Analysis}
\renewcommand{\maketitle}{%
\hrule height4pt
\large{Eric Du \hfill \classcode}
\newline
\large{HW 07} \Large{\hfill \classname \hfill} \large{\today}
\hrule height4pt \vskip .7em
\normalsize
}
\linespread{1.1}
\begin{document}
	\maketitle
	\section*{Problem 1}
	Let $f(x) = x^2 \sin \frac{1}{x}$ for $x \neq 0$ and $f(0) = 0$.

	\begin{enumerate}[label=\alph*)]
		\item Use Theorems 28.3 and 28.4 to show $f$ is differentiable at each $a \neq 0$ and calculate $f'(a)$.
			Use, without proof, the fact that $\sin x$ is differentiable and that $\cos x$ is its derivative. 

			\begin{solution}
				At $a \neq 0$, we know that $x^2$ is differentiable at $a \neq 0$, and so is $\sin \frac{1}{x}$,
				so the composition (the product) of the two functions is also differentiable at $a \neq 0$. 
				Computing the derivative using regular product rule:
				\[
				f'(a) = 2a \sin \frac{1}{a} - a^2 \cos \frac{1}{a} \frac{1}{a^2} = 2a \sin \frac{1}{a} - \cos
				\frac{1}{a}
				\] 

			\end{solution}
		\item Use the definition to show that $f$ is differentiable at $x = 0$ and $f'(0) = 0$. 

			\begin{solution}
				At $x = 0$, the definition of differentiability is:
				\[
					\lim_{x \to c} \frac{f(x) - f(c)}{x - c} = f'(c)
				\] 
				So when approaching $x = 0$, we get:
				\[
					f'(0) = \lim_{x \to 0} \frac{f(x) - f(0)}{x - 0} = \lim_{x \to 0} \frac{f(x)}{x} = 
					\lim_{x \to 0} x \sin \frac{1}{x} = 0
				\] 
				The conclusion that it equals zero immediately follows from the fact that the sine function 
				oscillates between $-1$ and $1$, so therefore $\lim_{x \to 0} x$ dominates and gives us 0.  

			\end{solution}
		\item Show $f'$ is not continuous at $x = 0$

			\begin{solution}
				Looking at the function for the derivative:
				\[
				f'(a) = 2a \sin \frac{1}{a} - \cos \frac{1}{a}
				\]
				Using a result from a previous homework, we can switch the limit from $x \to 0$ to $x \to \infty$
				by making the substitution $x \to \frac{1}{x}$. Therefore, we are now looking for the limit: 
				\[
					\lim_{a \to \infty} \frac{2}{a} \sin a - \cos a
				\] 
				Now, we can choose two subsequences: first, choose the sequences at $a = 2n\pi$, giving us the
				sequence $a_n = -1$, we can then choose a second sequence at $a = n\pi$ which gives the sequence
				$b_n = (-1)^{n+1}$. Since these two sequences are different, then this limit does not converge,
				and thus the limit does not exist. As a result, this means that $f'$ cannot be continuous at
				$x = 0$, since the limit doesn't exist.  
			\end{solution}
	\end{enumerate}

	\pagebreak

	\section*{Problem 2}

	Let $f$ and $g$ be differentiable on an open interval $I$ and consider $a \in I$. Define $h$ on $I$ by 
	the rules: $h(x) = f(x)$ for $x <a$, and $h(x) = g(x)$ for $x \ge a$. Prove $h$ is differentiable at $a$
	if and only if both $f(a) = g(a)$ and $f'(a) = g'(a)$ hold. \textit{Suggestion:} Draw a picture to see 
	what is going on.

	\begin{solution}
		We prove the forward direction: $h$ is differentiable if $f(a) = g(a)$ and $f'(a) = g'(a)$. To do this, 
		note first that $h$ must be continuous at $a$, meaning: 
		\[
			\lim_{x \to a^-} h(x) = \lim_{x \to a^+} h(x) = h(a)
		\] 
		Since $h = f$ for $x < a$ and $h = g$ for $h \ge a$, we note that
		\[
			\lim_{x \to a^-} h(x) = \lim_{x \to a^-} f(x) = f(a)
		\] 
		Similarly, 
		\[
			\lim_{x \to a^+} h(x)  \lim_{x \to a^+} f(x) = g(a)
		\]
		If $f(a) = g(a)$, then $h(a)$ is continuous. Similarly, if $h$ is continuous then both these
		conditions must hold.  The derivative must also satisfy similar conditions:
		\[
			\lim_{x \to a^-} h'(x) = \lim_{x \to a^+} h'(x) = h'(a)
		\] 
		Again, since $h = f$ for $x < a$, then we have the following two equations:
		\begin{align*}
			\lim_{x\to a^+} h'(x) &= \lim_{x \to a^+} f'(x) = f'(a)\\
			\lim_{x \to a^-} h'(x) &= \lim_{x \to a^-} g'(x) = g'(a)
		\end{align*}
		Again, the left and right limits must equal, which is only true if $f'(a) = g'(a)$. Just like the proof
		of continuity, if $h'(a)$ were to exist, then $f'(a) = g'(a)$ must also hold by this equation. Here, 
		we've essentially proven both sides of the argument at the same time, by showing that both conditions
		result in if and only if conditions. 
	\end{solution}

	\pagebreak

	\section*{Problem 3}
	Let $f$ be differentiable on $\mathbb R$ with $a = \sup \{|f'(x)|: x \in \mathbb R\} < 1$. 

	\begin{enumerate}[label=\alph*)]
		\item Select $s_0 \in \mathbb R$ and define $s_n = f(s_{n-1})$ for $n \ge 1$. Thus $s_1 = f(s_0)$, 
			$s_2 = f(s_1)$, etc. Prove $(s_n)$ is a convergence sequence. \textit{Hint:} To show that $(s_n)$ is
			Cauchy, first show $|s_{n+1} - s_n| \le a|s_n - s_{n-1}|$ for $n \ge 1$. 

			\begin{solution}
				To prove a sequence is Cauchy, we need to prove that for all $\epsilon > 0$, there exists some
				$N \in \mathbb N$ such that for all $n, m > N$,
				\[
				|s_m - s_n| < \epsilon
				\] 
				Without loss of generality, let $s_m > s_n$ in our case. Then by the mean value theorem, we 
				know that there exists some $c$ on the interval $(s_n, s_m)$ such that:
				\[
					|f'(c)| = \frac{|f(s_m) - f(s_n)}{|s_m - s_n|} = \frac{|s_{m+1} - s_{n+1}|}{s_m - s_n}
				\] 
				This implies that $|s_{m+1} - s_{n+1}| = |f'(c)| |s_m - s_n|$, and now using $a$ as defined 
				in the problem statement, then we derive the inequality:
				\[
					|s_{m+1} - s_{n+1}| \le a|s_m - s_n| < |s_m - s_n|
				\] 
				From this equation, we can see that $|s_m - s_n|$ is a strictly decreasing sequence which is
				bounded below by 0 (due to the absolute value), so therefore there will be some $N \in \mathbb N$
				that satisfies our Cauchy condition. Therefore, $s_n$ is a convergent sequence. 
			\end{solution}
		\item Show $f$ has a \textit{fixed point}, i.e., $f(s) = s$ for some $s$ in $\mathbb R$

			\begin{solution}
				We proved earlier that $s_n$ is a convergent sequence. Notice that by definition of the sequence,
				$s_n = f(f(f(\cdots f(s_0))))$ which I will notate as $s_n = f^{(n)}(s_0)$. Then as $n \to 
				\infty$ we know that $f(s_n) = s$ for some $s$. Then, recall that we can write $s_n$ recursively:
				\[
					s = f(f^{(n-1)}(s_0))
				\] 
				and as $n \to \infty$, then we have $f^{(n-1)}(s_0) = f^{(n)}(s_0) = s$. Therefore:
				\[
				s = f(s)
				\] 
				in the limit, therefore proving that $f$ has some fixed point. 
			\end{solution}
	\end{enumerate}

	\pagebreak

	\section*{Problem 4}

	Let $f : \mathbb R \to \mathbb R$ be a twice differentiable function, where $f(0) = 0$ and $f''(x) \ge 0$ 
	for all $x > 0$. Prove that $f(x)/x$ is increasing for $x > 0$. 

	\begin{solution}
		Let $g(x) = \frac{f(x)}{x}$. Then, we know that: 
		\[
		g'(x) = \frac{xf'(x) - f(x)}{x^2}
		\] 
		Our goal is to show that $g'(x) \ge 0$. Firstly, notice that since we are only looking at $x > 0$, then
		showing this amounts to showing that $xf'(x) - f(x) \ge 0$, or more explicitly $f'(x) \ge f(x)/x$. By 
		the mean value theorem, we know that for all $x$, there exists some value $c$ in $0 < c < x$ such that:
		\[
		f'(c) = \frac{f(x) - f(0)}{x-0} = \frac{f(x)}{x}
		\] 
		Now, we use the relation that $f''(x) \ge 0$, so therefore $f'(x)$ is increasing. Since $c < x$, then 
		we know that $f'(x) \ge f'(c)$, so by extension we have: $f'(x) \ge f(x)/x$, which is the statement we
		were seeking to prove. 
	\end{solution}

	\pagebreak

	\section*{Problem 5}
	Prove: 

	\begin{theorem}
		Let $f$ be a differentiable function on $(a, b)$. If $a < x_1 < x_2< b$, and if $c$ lies between 
		$f'(x_1)$ and $f'(x_2)$, there exists [at least one] $x$ in $(x_1, x_2)$ such that $f'(x) = c$
	\end{theorem}

	Hint: consider $g(x)= f(x) -cx$

	\begin{solution}
		Following the hint, consider $g(x) = f(x) - cx$. Then, $g'(x) = f'(x) - c$. Without loss of generality, 
		we can assume that $f'(x_1) < f'(x_2)$, this means that $g'(x_1) < 0$, and $g'(x_2) > 0$. If we can 
		prove that there exists some $x$ such that $g'(x) = 0$, then we are done. Consider the mean value
		theorem for $g'$:
		\[
		g'(c) = \frac{g(x) - g(x_1)}{x - x_1}
		\] 
		Therefore, proving that $g'(c) = 0$ amounts to proving there exists some $x$ such that $g(x) = g(x_1)$. 
		We prove this must be the case that by contradiction. Suppose that this point $x$ does not exist. 
		Then, this means that $g(x)$ is either strictly increasing or decreasing. If $g(x)$ is increasing, then
		this contradicts $g'(x_1) = 0$, since when approaching $x_1$ from the right, we have:
		\[
			\lim_{x \to x_1^+} \frac{g(x) - g(x_1)}{x - x_1} < 0
		\] 
		But since $x > x_1$, then this implies that $g(x) - g(x_1) < 0$, which is impossible if $g(x)$ is 
		strictly increasing. Similarly, if $g(x)$ is strictly decreasing, then this would violate $g'(x_2) > 0$,
		when we analyze the limit approaching from the left:
		\[
			\lim_{x \to x_2^-} \frac{g(x) - g(x_2)}{x - x_2} < 0
		\] 
		Here, since $x < x_2$, then we require that $g(x) - g(x_2) < 0$ or equivalently that $g(x) < g(x_2)$. 
		Again, this is impossible if $g(x)$ were strictly decreasing. Therefore, $g(x)$ is neither strictly 
		increasing or decreasing, implying that there must be some $x$ such that $g(x) = g(x_1)$, completing
		the proof. 
	\end{solution}
\end{document}
