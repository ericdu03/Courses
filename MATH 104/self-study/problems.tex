\documentclass[10pt]{article}
\usepackage{../../local}
\urlstyle{same}

\newcommand{\classcode}{Math 104}
\newcommand{\classname}{Solutions to Selected Rudin Problems}
\renewcommand{\maketitle}{%
\hrule height4pt
\large{\phantom{Eric Du} \hfill \classcode}
\newline
Eric Du \Large{\hfill \classname \hfill} \large{\today}
\hrule height4pt \vskip .7em
\small{Header styling inspired by CS 70: \url{https://www.eecs70.org/}}
\normalsize
}
\linespread{1.2}

%\newenvironment{problem}{\textbf{Problem:}}{}
\newenvironment{definition}{\textbf{Definition:}}{}
\begin{document}
	\maketitle
	\section{The Real and Complex Number Systems} 

	\begin{problem}
		If \( r \) is rational (\( r \neq 0 \)) and \( x \) is irrational, prove that \( r + x \) and 
		\( rx \) are irrational. 
	\end{problem}

	\begin{solution}
		Suppose by contradiction that \( r  +x \) and \( rx \) are both rational. Then, since \( r \) is rational, 
		then we can write \( r = \frac{a}{b} \). Similarly, we write \( rx = \frac{c}{d} \), \( r + x = \frac{e}{f} \). 
		Then, we have:
		\[
		\frac{a}{b} + x = \frac{e}{f} \quad \frac{a}{b}x = \frac{c}{d}
		\] 
		Rearranging for \( x \) in both cases:
		\[
		x = \frac{e}{f} - \frac{a}{b} \quad x = \frac{c}{d} \cdot \frac{b}{a}
		\] 
		In both cases, since the rationals are closed under multiplication and addition, this implies that \( x \) is 
		rational, which is a contradiction. 
	\end{solution}
	\begin{problem}
		Let \( E \) be a nonempty subset of an ordered set; suppose \( \alpha \) is a lower bound of \( E \)
		and \( \beta  \) is an upper bound of \( E \). Prove that \( \alpha \le  \beta \). 
	\end{problem}

	\begin{solution}
		Let \( S \) be the set such that \( E \subset S \). Since \( \alpha \) is a lower bound of \( E \), then 
		this means that \( \alpha \in S \) such that \( \alpha \le x \) for every \( x \in E \), and \( \beta \ge x \) 
		for every \( x \in E \). 

		Now suppose for the sake of contradiction that \( \beta < \alpha \). Then, since \( \alpha \) is 
		the lower bound, this implies that for all \( x  \in E\), \( \alpha < x \). But since \( \alpha > \beta \), 
		then this means that  \( \beta < x \) for every \( x \) as well. This is impossible however, since 
		this contradicts the fact that \( \beta  \) is an upper bound.  
	\end{solution}

	\begin{problem}
		Let \( A \) be a nonempty set of real numbers which is bounded below. Let \( -A \) be 
		the set of all numbers \( -x \), where \( x \in A \). Prove that 
		\[
		\inf A = -\sup (-A)
		\] 
	\end{problem}

	\begin{solution}
		Since \( A \) is bounded below, then \( \beta = \inf A \) exists, such that \( \beta \le  x \) for all 
		\( x \in A \). Equivalently, we can write this as \( -\beta \ge -x \) for all \( x \in A \). Then, since 
		\( -x \) for \( x \in A \) is the set \( -A \), then we conclude that \( -\beta \) is an upper bound 
		for the set \( -A \). Hence, we have:
		\[
		\sup(-A) = - \beta = - \inf A \implies \inf A = - \sup (-A)
		\] 
		as desired. 
	\end{solution}
	\begin{problem}
		Fix \( b > 1 \). 
		\begin{enumerate}[label=\alph*)]
			\item If \( m, n, p, q \) are integers, \( n >0, q > 0 \) and \( r = m / n = p / q \), prove that 
				\[
					(b^{m})^{1 / n} = (b^{p})^{1 / q}
				\] 
				Hence it makes sense to define \( b^{r} = (b^{m})^{1 / n} \). 

				\begin{solution}
					We solve this by showing that we can exponentiate both sides to arrive at an equal value. If this 
					is the case, then we know that the original numbers must be the same, becuase Theorem 1.21
					guarantees us uniqueness in roots. 

					Notice that since \( m / n = p / q \), then we can write:
					\[
					b^{mq}	= \left(  (b^{m})^{1 / n}\right)^{n q}\quad \quad
						b^{np} = \left( (b^{p})^{1 / q} \right)^{nq} 
					\] 
					Then, since \( m / n = p / q \), we also have \( mq = np \), so the exopnents are the same 
					on both sides here. Thus, the two original quantities are the same. 
				\end{solution}
			\item Prove that \( b^{r + s} = b^{r} b^{s} \) if \( r \) and \( s \) are rational. 

				\begin{solution}
					Again, we aim to show that these are equal like in the previous case. Since \( r \) and \( 
					s\) are rational, let \( r = \frac{m}{n} \) and \( s = \frac{p}{q} \). Then, we have:
					\[
					b^{r + s} = b^{\frac{m}{n} + \frac{p}{q}} = b^{\frac{mq + np}{nq}}\quad 
					b^{r}b^{s} = b^{\frac{m}{n}}b^{\frac{p}{q}} 
					\]  
					Now, exponentiate both of these to the power of \( nq \), which is an integer:
					\[
						(b^{\frac{mq + np}{nq}})^{nq} = b^{mq + np} \quad \left(b^{\frac{m}{n}} b^{\frac{p}{q}}\right)
						^{nq} = b^{mq + np}
					\] 
					Then, since these values are the same, then it must follow that \( b^{r + s} = b^{r} b^{s} \). 
				\end{solution}
			\item If \( x \) is real, define \( B(x) \) to be the set of all numbers \( b^{t} \) where \( t \) 
				is rational and \( t \le x \). Prove that 
				\[
				b^{r} = \sup B(r)
				\] 
				when \( r \) is rational. Hence it makes sense to define 
				 \[
				b^{x} = \sup B(x)
				\] 
				for every real \( x \).

				\begin{solution}
					Here, it is immediately clear that \( b^{r} \) is an upper bound of \( B(r) \), since \( r \) is 
					the largest exponent, but we now need to show that \( b^{r} \) is the least upper bound of 
					\( B(r) \).

					We prove this by contradiction. Suppose there is some real number \( s \) such that 
					\( s < b^{r} \) and \( s \) is still an upper bound on \( B(r) \). Then, this implies that 
					\( b^{t} < s \) for all \( t < r \) as well. However, this cannot be the case, since 
					\( B(x) \) is defined to contain \( b^{r} \), so \( s \) cannot be an upper 
					bound on the entire set. Therefore, \( b^{r} \) is the least upper bound.  
				\end{solution}
		\end{enumerate}
	\end{problem}

	\begin{problem}
		Prove that no order can be defined in the complex field that turns it into an ordered field. \textit{Hint:}
		\( -1 \) is a square. 
	\end{problem}

	\begin{solution}
		By proposition 1.18 condition (d), we require that if \( x \neq 0 \), then \( x^2 > 0 \). However, 
		this is not true for the complex field, since \( i^2 = -1 < 0 \).  
	\end{solution}

	\begin{problem}
		If \( z \) is a complex numer, prove that there exists an \( r \ge  0 \) and a complex 
		number \( w \) with \( |w| = 1 \) such that \( z = rw \). Are \( w \) and \( r \) always uniquely 
		determined by \( z \)? 
	\end{problem}

	\begin{solution}
		This is essentially the construction of the polar coordinates. Let \( z = a + bi \). Then, we 
		may express \( z \) in terms of \( z = |z| \hat{z} \), where \( |z| = r\)  and \( \hat{z} = w \). 
		To find \( r \), one finds the norm of \( z \):
		\[
		r = \sqrt{a^2 + b^2} 
		\] 
		similarly, \( w \) takes the for \( e^{i \theta} \), where \( \theta \) is defined as the angle 
		counterclockwise from the \( x \) axis:
		\[
		\theta = \tan^{-1}\left( \frac{b}{a} \right) 
		\] 
		This is not a unique representation, since different ratios of \( b  \) and \( a \) can give rise 
		to the same value of \( \tan^{-1}(\theta) \). 
	\end{solution}

	\subsection{Challenge Problem from Bergman}
	On p. 21, Rudin mentions, but does not prove, that any two ordered fields with the least-upper bound property 
	are isomorphic. This exercise will sketch how that fact can be proved. For the benefit of students who 
	have not had a course in Abstract Algebra, I begin with some observations generally included in that course 
	(next paragraph and part (a) below). 

	If \( F \) is any field, let us define an element \( n_F \in F\) for each integer \( n \) as follows: let 
	\( 0_F \) and \( 1_F \) be the elements of \( F \) called ``0'' and ``1`` conditions (A4) and 
	(M4) of the definition of a field. (We add the subscript \( F \) to avoid confusion with elements \( 0, 1 
	\in \Z\)). for \( n = 1 \), once \( n_F \) is defined we recursively define \( (n+1)_F = n_F + 1_F \) ; in this 
	way \( n_F \) is defined for all nonnegative integers. Finally, for negative integers \( n \) we define 
	\( n_F = -(-n)_F \). (Note that in that expression, the ``inner'' minus is applied in \( Z \), the ``outer'' 
	minus in \( F \).) 
	\begin{enumerate}[label=\alph*)]
		\item Show that under the above conditions, we have \( (m + n)_F = m_F + n_F \) and 
			\( (mn)_F = m_F n_F \) for all \( n, n \in \Z \)
		\item Show that if \( F \) is an \textit{ordered} field, then we laso have \( m_F < n_F \iff m < n \)
	\end{enumerate}

	\section{Basic Topology}
	
	\subsection{Notes}
	\begin{definition}
		Consider two sets \( A \) and \( B \), whose elements may be any objects. Suppose for every element 
		\( x \in A \), there is an associated element in \( B \). Then, \( f \) is said to be a 
		function from  \( A \) to \( B \) (or a mapping). \( A \) is called the domain, and \( B \) is called
		the range of \( f \).  
	\end{definition}

	\subsection{Problems}
	\begin{problem}
		Prove that the empty set is a subset of every set. 
	\end{problem}

	\begin{solution}
		The definition of a subset is that every element of the subset is also a member of the superset. 
		Vacuously, the empty set contains elements of any set, and thus it is a subset of every set.  
	\end{solution}

	\begin{problem}
		A complex number \( z \) is said to be \textit{algebraic} if there are integers \( a_0, \dots, a_n \) not 
		all zero, such that
		 \[
		a_0z^{n} + a_1z^{n - 1} + \cdots + a_{n-1}z + a_n = 0
		\] 
		Prove that the set of algebraic numbers is countable. \textit{Hint:} For every positive 
		integer \( N \), there are only finitely many equations with 
		\[
		n + |a_0| + |a_1| + \cdots + |a_n| = N
		\] 
	\end{problem}

	\begin{solution}
		By the fundamental theorem of algebra, we know that a degree \( n \) polynomial with complex 
		coefficients (of which the integers are a subset) has exactly \( n \) roots, counting multiplicity. 
		Therefore, for every polynomial of degree \( n \), there are finitely many roots. 

		We now prove that there are countably many polynomials with integer coefficients. This is easy enough, 
		since for every \( n \), let \( E_n \) denote the set of integer coefficient of degree \( n \). Since 
		each coefficient is an integer, this set is countable. Then, the set of polynomials with integer coefficients 
		can be represented as:
		\[
		S = \bigcup_{i = 1}^{\infty}E_n
		\] 
		which by Theorem 2.12, \( S \) is countable. Combining this with the fact that for any given \( n \) there are
		finitely many roots, we conclude that the set of algebraic numbers is countable.  

		\question{Not sure how to use the hint to solve the problem.}  
	\end{solution}

	\begin{problem}
		Let \( E' \) be the set of all limit points of a set \( E \). Prove that \( E' \) is closed. Prove that 
		\( E \) and \( \overline E \) have the same limit points (Recall that \( \overline E = E \cup E'. \)). 
		Do \( E \) and \( E' \) always have the same limit points? 
	\end{problem}

	\begin{solution}
		First, we list off the relevant definitions:
		\begin{itemize}
			\item \textbf{Closedness:} A set \( E \) is closed if the limit points of \( E \) are contained in \( E \). 
			\item \textbf{Limit points:} A point \( p \) is a limit point of \( E \) if every neighborhood 
				of \( p \) contains a point \( q \neq p \) such that \( q \in E \). 
			\item \textbf{Neighborhood:} The set \( N_r(p) \) consisting of all points \( q \) such that 
				\( d(p, q) < r \) for some \( r > 0 \). Here, \( r \) is called the \textit{radius}. 
		\end{itemize}
		So, to prove that \( E' \) is closed, we aim to show that all the limit points of \( E' \) are contained
		within \( E' \). We prove this by contradiction. 
		Suppose \( E' \) is not closed. Then, there exists a point \( p \) such that \( p \) is a limit point of 
		\( E' \) but \( p \not \in E' \). This implies two things: 
		\begin{itemize}
			\item Since \( p \) is a limit point of \( E' \), it means that for any \( r > 0 \), there exists 
				a neighborhood \( N_r(p) \) that contains a point \( q \in E' \).  
			\item Since \( p \) is not a member of \( E' \), it means that \( p \) is not a limit point of \( E \). So, 
				there exists some radius \( r_0 > 0 \) such that \( N_{r_0}(p) \) does not contain 
				any points within \( E \).  
		\end{itemize}
		First, since \( q \in E' \), then \( q \) is a limit point of \( E \). Now, consider 
		the set \( N_\alpha(q) \), where \( \alpha = r_0 - d(p, q) \). This is the set of points 
		\( x \) defined by:
		\[
		\{x \mid d(x, q) < r_0 - d(p, q)\} 
		\] 
		The fact that \( q \) is a limit point of \( E \) implies that there exists some \( q' \in N_{\alpha}(q)\) 
		such that \( q \in E \). 
		Rearranging the inequality condition, 
		we get \( d(x, q) + d(p, q) < r_0 \). Now, we use the property that \( d(p, q)
		= d(q, p)\) and the triangle inequality to conclude that every point in \( N_{\alpha}(q) \) satisfies the 
		inequality \( d(x, p) < r_0 \). This condition is incidentally the same as that of \( N_{r_0}(p) \), 
		so the points in \( N_{\alpha}(q) \) form a (potentially proper) subset of \( N_{r_0}(p) \). By the 
		second bullet point, we then know no point within \( N_{\alpha}(q) \) is contained within \( E \). This is a 
		contradiction however, since \( q \) is defined to be a limit point. Thus, \( E' \) must 
		be closed. 

		Now, we prove that \( E \) and \( \overline E \) have the same limit points. First, the limit 
		points of \( E \) is just the set \( E' \) itself. 
		Then, since \( \overline E = E \cup E' \), then we know that the limit points of \( \overline E \) 
		are either limit points of \( E \) or \( E' \). If they're limit points of \( E \), they're contained
		in \( E' \), and if they're limit points of \( E' \), the fact that \( E' \) is closed means that they 
		are also a part of \( E' \). Therefore, since the limit points of both sets is just \( E' \) itself, 
		they have the same set of limit points. 

		\( E \) and \( E' \) must always have the same set of limit points. The set \( E' \) is the set of 
		limit points of \( E \), and since \( E' \) is closed, any limit point of \( E' \) is also contained in 
		\( E' \), hence the set of limit points for \( E \) and \( E' \) is just \( E' \) itself.  
	\end{solution}

	\begin{problem}
		Let \( A_1, A_2, \dots, A_n \) be subsets of a metric space. 
		\begin{enumerate}[label=\alph*)]
			\item If \( B_n = \bigcup_{i = 1}^{n}A_i \), prove that \( \overline B_n = 
				\bigcup_{i = 1}^{n}\overline A_i \), for \( n = 1, 2, 3, \dots \). 

				\begin{solution}
					Assuming that \( A_i = A_i \cup A_i' \), where \( A_i'\) is the set of limit 
					points of \( A_i \). Then, this equation becomes:
					\[
					\bigcup_{i = 1}^{n}\overline A_i = \bigcup_{i = 1} ^{n}(A_i \cup A_i')
					= \bigcup_{i=1} ^{n}A_i \cup \bigcup_{i = 1} ^{n}A_i' = B_n \cup B_n' = \overline B_n 
					\] 
					this leverages the associativity property of unions.  
				\end{solution}
			\item If \( B = \bigcup_{i = 1}^{n}A_i \), prove that \( \overline B \supset 
				\bigcup_{i = 1}^{\infty}\overline A_i \)
		\end{enumerate}
	\end{problem}

	\begin{problem}
		Let \( E^{\circ} \) denote the set of all interior points of a set \( E \) [See Definition 
		2.18(e); \( E^{\circ} \) is called the \textit{interior} of \( E \).]
		\begin{enumerate}[label=\alph*)]
			\item Prove that \( E^{\circ} \) is always open. 

				\begin{solution}
					We can prove that \( E^{\circ} \) is open by showing that every point of \( E^{\circ} \) 
					is an interior point of \( E^{\circ} \). Suppose for the sake of contradiction that 
					there exists a point \( p \) that is an interior point of \( E^{\circ} \), but 
					\( p \not \in E^{\circ} \).  

					We can prove that \( E^{\circ} \) is open by showing that every point of \( E^{\circ} \) is an 
					interior point of \( E^{\circ} \). Suppose for the sake of contradiction that there exists 
					a point \( p \in E^{\circ}\) that is not an interior point of \( E^{\circ} \). Then, this 
					implies that there is no neighborhood \( N \) around \( p \) such that \( N \subset E^{\circ} \). 
					Then, since \( p \in E^{\circ} \), \( p \) is an interior point of \( E \), so there 
					exists a neighborhood \( N \) around \( p \) of radius \( r \) such that \( N_r(p) \subset E \). 

					Now, let \( q \in N_r(p) \) and \( q \neq p \), and let the distance \( d = d(p, q) > 0\). Since 
					\( N_r(p) \subset E \), then this also implies that \( N_\alpha(q) \subset E \), 
					where \( \alpha = r - d(p, q) \), since \( N_\alpha(q) \subset N_r(p) \). Then, this 
					implies that \( q  \) is an interior point of \( E \), and since \( q \) was 
					arbitrarily chosen, this is true for every point within \( N_r(p) \). Therefore, 
					\( N_r(p) \subset E^{\circ} \) since all points 
					in \( N_r(p) \) are interior points of \( E^{\circ} \), which is a contradiction since we initially 
					assumed that such a neighborhood cannot exist. 
				\end{solution}
			\item Prove that \( E \) is open if and only if \( E^{\circ} = E \). 

				\begin{solution}
					Forward case: Since \( E^{\circ} = E \), then since \( E^{\circ} \) is always open (by part a), then
					\( E \) is also open. 

					Reverse case: If \( E \) is open, then every point of \( E \) is an interior point of \( E \) 
					hence, the set \( E^{\circ} \) must also be open since it's just the set \( E \) itself.  
				\end{solution}
			\item If \( G \subset E \) and \( G \) is open, prove that \( G\subset E^{\circ} \) 

				\begin{solution}
					Since \( G \) is open, then \( G^{\circ} = G \), and since \( G \subset E \), then 
					\( G^{\circ} \subset E^{\circ} \) (one can easily show this to be true), 
					and hence \( G \subset E^{\circ} \). 
				\end{solution}
			\item Prove that the complement of \( E^{\circ} \) is the closure of the complement of \( E \). 

				\begin{solution}
					The closure of a set \( E \) is the set \( \overline E = E \cup E' \), where \( E' \) 
					denotes the set of limit points of \( E \). So, the closure 
					of the complement of \( E \) is the union between \( E^{c} \cup \overline E^{c} \).  

					From part (c), we know that \( E^{\circ} \) is the largest open set contained within 
					\( E \), meaning 
					that it is the union of all open sets contained within \( E \). Therefore, its complement 
					is the intersection of all closed sets that contain the complement of \( E \). In other words, 
					for all sets \( F \) in the complement, \( E^{c} \subset F \). Therefore, by Theorem 
					2.27 (b), we see that \( \overline {E^{c}} \subset F \) for every set \( F \). Then, since 
					we are taking the intersection of all such sets, we get that the intersection leaves us 
					with only \( \overline{E^{c}} \), which is the closure of the complement of \( E \). 
				\end{solution}
			\item Do \( E \) and \( \overline E \) always have the same interiors?

				\begin{solution}
					No, take any finite set of points: then the interior of \( E \) is the null set, whereas
					the closure of \( E \) contains at least \( E \) itself and is nonempty.
				\end{solution}
			\item Do  \( E \) and \( E^{\circ} \) always have the same closures? 

				\begin{solution}
					No, again take any finite set of points. Then the closure of \( E \) contains at least \( E \), 
					whereas \( E^{\circ} = \varnothing\), so \( \overline{E^{\circ}} = \varnothing \). 
				\end{solution}
		\end{enumerate}
	\end{problem}

	\begin{problem}
		Let \( X \) be an infinite set. For \( p \in X \) and \( q \in X \), define
		 \[
		 d(p, q) = \begin{cases}
			 1 & \text{if \( p \neq q \)}\\
			 0 & \text{if \( p = q \)}
		 \end{cases}
		 \] 
		 Prove that this is a metric. Which subsets of the resulting metric space are open? Which are closed? 
		 Which are compact? 
	\end{problem}

	\begin{solution}
		To prove that this is a metric, it just needs to satisfy the three properties:
		\begin{itemize}
			\item Positivity: If \( p \neq q \), then \( d(p, q) = 1 > 0 \), and if \( p = q \) then
				\( d(p, q) = 0 \), which is satisfied. 
			\item Associativity: \( d(p, q) = d(q, p) \), this is rather obvious. 
			\item Triangle inequality: for any \( r \in X \), \( d(p, q) = 1 \), whereas \( d(p, r) + d(r, q) = 2 \) 
				if \( r \neq p \) and \( r \neq q \). If  \( r = p \) or \( r = q \), then 
				\( d(p, r) + d(r,q) = 1 \). In any case, the triangle inequality is satisfied. 
		\end{itemize}
		In terms of open subsets, we need to find subsets which only contain interior points. Recall that an 
		interior point is one where we can always find a neighborhood \( N \) around a point \( p \) such that 
		\( N \subset E \). Since the defined metric gives a distance of 1 for every point \( q  \) as long as 
		\( q \neq p \), then we know that every set containing only one point is open, since  \( N_{1/2}(p) = \{p\} \),
		and is hence a subset of itself. By extension, we can union these sets together, so every subset
		of \( X \) is also open. 

		Every set is also closed, since the complement of any subset is also a subset of \( X \).     

		A set is compact if every open cover of \( K \) contains a finite subcover. In other words, we basically 
		require that there is a finite number of open subsets \( G_\alpha \) such that 
		\( K \subset G_{\alpha_1} \cup \cdots \cup G_{\alpha_n} \). Based on this definition, every finite 
		subset satisfies this property, since every subset of \( X \) is open. However, the set \( X \) itself
		cannot be written in this way, since we cannot cover an infinitely-sized \( X \) with subsets of itself.  
	\end{solution}

	\begin{problem}
		Let \( K \subset R^{1} \) consist of \( 0 \) and the numbers \( 1/n \), for \( n = 1, 2, 3, \dots \). 
		Prove that \( K \) is compact directly from the definition (without using the Heine-Borel theorem). 
	\end{problem}

	\begin{solution}
		Recall that a set is compact if every open cover of \( K \) has a finite subcover, or in other words, we can 
		find a finite collection \( \{G_{\alpha}\}  \) of open subsets of \( R^{1} \) such that 
		\( K \subset \bigcup_\alpha G_\alpha \). In other words, our objective is to find such a collection 
		of open sets. 

		Consider an open cover \( G = \bigcup_\alpha G_\alpha  \). 
		Let \( G_0 \) be the open set containing 0. Since \( G_0 \) 
		is open, this implies there exists some neighborhood \( N_\epsilon(0) \subset G \) that exists, where 
		\( \epsilon > 0 \). Now, let \( N  \) be the largest value of \( n \) such that \( \frac{1}{N} \) is 
		not contianed within \( G_0 \). Then, let \( G_k \) denote the open set containing \( 1 / k \), which 
		we can then union together with \( G_0 \) to form the finite subcover of \( K \). 
	\end{solution}
	
	\begin{problem}
		Construct a compact set of real numbers whose limit points form a countable set.  
	\end{problem}

	\begin{solution}
		Take the set above: the set of 0 with the numbers \( 1 / n \): 0 is the only limit point in this set, since 
		it is the only point such that for every neighborhood \( N_r(0) \) can we find a point \( 1 / n \) that 
		is always in the set. 

		We can also show that 0 is the only limit point of the set; consider any point \( 1 / m \in K \). Then, 
		for any neighborhood \( N_r(1 / m) \) where \( r < \frac{1}{m} - \frac{1}{m+1} < \frac{1}{m-1} - 
		\frac{1}{m}\) will have no point 
		within \( K \) inside, so it cannot be a limit point. 
	\end{solution}

	\begin{problem}
		Show that Theorem 2.36 and its Corollary become false (in \( R^{1} \), for example) if the 
		word ``compact'' is replaced by ``closed'' or by ``bounded.''
	\end{problem}

	\begin{solution}
		I'll restate Theorem 2.36 and the Corollary for convenience:
		
		\begin{center}
			\begin{minipage}{0.9\textwidth}
				\textbf{Theorem:}
					If \( \{K_\alpha\}  \) is a collection of compact subsets of a metric space \( X \) such that the 
					intersection of every finite subcollection of \( \{K_\alpha\}  \) is nonempty, then 
				\( \bigcap K_\alpha \) is nonempty. 

				\medskip
				\textbf{Corollary:} If \( \{K_n\}  \) is a sequence of nonempty compact sets such that \( K_n \supset
				K_{n + 1}\) (\( n = 1,2, 3, \dots \)), then \( \bigcap_1^{\infty} K_n \) is not empty.  
			\end{minipage}
		\end{center}
		First, note that a set that is both closed and bounded is compact. We can prove this as follows: let 
		\( K \) be a closed and bounded set, and consider an open cover \( \{G_\alpha\}  \) of \( K \). 
		Let \( p \in K  \). Since \( \{G_\alpha\}  \) is an open cover, let \( G_1 \) be the open set 
		containing \( p \). Since \( G_1 \) is open, it has nonzero width (since it must contain some epsilon 
		ball). Then, consider points on the boundary of the closure of \( G_1 \) that are also 
		in \( K \). These points are not contained in \( p \), so there must be another open set \( G_2 \) that 
		contians these points, which also have nonzero width. We repeat this construction until no boundary points are 
		contained within \( K \).

		The closedness of \( K \) ensures that all limit points of \( K \) are covered, and the boundedness ensures 
		that we may complete this process in a finite number of steps. The union of the sets we selected, 
		\( \bigcup_1^{N}G_n \) is a finite subcover of \( K \). 

		With that in mind, we turn to the corollary: if \( \{K_\alpha\}  \) is bounded only, consider the set 
		of open sets \( \{K_\alpha\} = \{(0, x) \mid x \in \R \}\), then for any subcollection 
		of these sets \( \{G_\alpha\}  \), there exists a smallest set 
		\( K_\epsilon = (0, \epsilon) = \bigcap_\alpha G_\alpha  \). However, if we take the infinite intersection, 
		then for every point \( x \) there exists at least one set \( (0, x - \epsilon) \) that 
		does not contain \( x \), and hence the infinite interseciton is the empty set. 

		If \( \{K_\alpha\}  \) is unbounded, then consider the sets \( \{K_\alpha\}  = 
		\{(x, \infty) \mid x \in \R\} \), and the idea is the same -- any finite intersection of \( K_{x} \) will 
		be nonempty, but in the limit the intersection is the empty set, becuase for every point there exists 
		at least one set that does not contain that point. 

		Because the corollary is a special case of the theorem, the theorem also fails to these two 
		examples, and hence we are done. 
	\end{solution}

	\begin{problem}
		Let \( E \) be the set of all \( x \in [0, 1] \) whose decimal expansion contains only the digits 4 and 7.
		Is \( E \) countable? Is \( E \) dense in \( [0, 1] \)? Is \( E \) compact? Is \( E \) perfect?
	\end{problem}

	\begin{solution}
		\( E \) is clearly uncountable -- we can generate a decimal on the diagonal following the rule that if it's 
		a 4, then turn it into a 7, and vice versa. \( E \) is dense, but not dense 
		in \( [0, 1] \), since \( E \subset [0.4, 0.8] \). \( E \) is closed, since 
		every limit point, which are the points within \( E \), is obviously contained within \( E \). 
		\( E \) is also bounded, which means that it is also compact. 

		Since all the limit points of \( E \) are contained within \( E \), this implies that \( E \) is 
		also perfect, since every point of \( E \) is a limit point of \( E \). 
	\end{solution}

	\begin{problem}
		Is there a nonempty perfect set in \( R^{1} \) which contains no rational number?
	\end{problem}

	\begin{solution}
		I beleive the irrationals are considered a perfect set. Because the irrationals themselves are dense 
		(as the reals themselves are dense), then every point in the irrationals is a limit point, which is 
		all we need in order to ensure that the set is perfect. 
	\end{solution}

	\begin{problem}
		\begin{enumerate}[label=\alph*)]
			\item If \( A \) and \( B \) are disjoint closed sets in some metric space \( X \), prove 
				that they are separated. 

				\begin{solution}
					Two sets \( A \) and \( B \) are considered separated if both \( A \cap \overline B \) and 
					\( \overline A \cap B \) are empty. Since \( A \) and \( B \) are closed sets, 
					then \( A = \overline A \) and \( B = \overline B \), so therefore 
					\( A \cap B = \varnothing \) (disjoint) implies that \( A \cap \overline B = 
					\overline A \cap B = \varnothing \). 
				\end{solution}
			\item Prove the same for disjoint open sets. 

				\begin{solution}
					Let \( p \in A \), and consider the intersection \( A \cap \overline B \). Assume that 
					\( A \cap \overline B \) is nonempty. Therefore, we must ahve \( p \in \overline B \), and 
					since \( p \not \in B \), then \( p \) must be a limit point of \( B \). Since \( p \in A \)  
					and \( p \) is open, then consider the neighborhood of radius \( r \) such that 
					\( N_r(p) \subset A\). Since \( p \) is a limit point of \( B \), then \( N_r(p) \) must contain 
					a point \( q \in B \) as well. This is contradictory, however, since this implies that 
					\( A \cap B \neq \varnothing \). 
				\end{solution}
			\item Fix \( p \in X, \delta > 0 \), define \( A \) to be the set of all \( q \in X \) for 
				which \( d(p, q) < \delta \), define \( B \) similarly, with \( >  \) in place of \( < \). Prove 
				that \( A \) and \( B \) are separated. 

				\begin{solution}
					Based on the definition, \( A = N_{\delta}(p) \), which is open. Now, it is clear 
					that \( B = (\overline {A})^{c} \), since the closure is the set such that 
					\( d(p, q) \le \delta \), so \( B \) is also an open set. These two sets 
					are also disjoint (obviously), so by part (b) we conclude that they are separated. 
				\end{solution}
			\item Prove that any connected metric space with at least two points is uncountable. 
				\textit{Hint: Use (c).}

				\begin{solution}
					Consider a metric space \( X \). Let \( a, b \) be two points in \( X \), and \( \delta = 
					d(a, b) / 2\), and let \( A \) and \( B \) be defined as in part (c) with this value 
					of \( \delta \). Since \( X \) is connected, it cannot be the union of two (nonempty) 
					separated sets, and since we know from part (c) that \( A \) and \( B \) are separated, 
					then \( X \) cannot be the union of these two. Therefore, the points 
					\( e \in X \) such that \( d(a, e) = \delta \) must be in \( X \) as well.

					However, we can choose different values of \( \delta \) as well. For every 
					value \( r \in (0, 1) \), one can choose \( \delta_r = r \cdot d(a, b) \), 
					to which there is a correspodning \( e_r \in X \). Since there are uncountably many 
					\( r \), there are uncountably many points in \( X \). 
				\end{solution}
		\end{enumerate}
	\end{problem}

	\begin{problem}
		Let \( A \) and \( B \) be separated subsets of some \( R^{k} \), suppose \( \mathbf a \in A \), 
		\( \mathbf b \in B \), and define
		\[
		\mathbf p(t) = (1 - t) \mathbf a + t \mathbf b
		\] 
		for \( t \in R^{1} \). Put \( A_0 = \mathbf p^{-1}(A) \), \( B_0 = \mathbf p^{-1}(B) \). [Thus 
		\( t \in A_0 \) if and only if \( p(t) \in A \).]
		\begin{enumerate}[label=\alph*)]
			\item Prove that \( A_0 \) and \( B_0 \) are separated subsets of \( R^{1} \). 

				\begin{solution}
					Suppose for contradiction that \( A_0 \) and \( B_0 \) are not separated subsets of \( R^{1} \). 
					WLOG, this i=mplies that \( A_0 \cap \overline B_0 \) is nonempty. Consider
					a point \( x \) in this intersection. Since \( x \in A_0 \), then 
					we know that \( p(x) \in A \). We also have \( x \in \overline B_0 \), and we know that 
					\( x  \) must exist in the closure of \( B_0 \), since if \( x \in B_0 \), then this 
					implies that \( p(x) \in B \), but we know that \( A \) and \( B \) are separated
					subsets of \( R^{1} \). 

					Since \( x \in A \), then \( p(x) \) is an interior point of \( A \), 
					therefore there exists a neighborhood \( N_r(p(x)) \subset A \). 
					Since \( x \in B_0' \) then \( x \) is a limit point of \( B_0 \), so using the same 
					neighborhood \( N_r(p(x)) \) there exists another point \( q \neq p(x) \) such that 
					\( q \in B \). This last point then implies that \( A \cap B \) is nonempty, 
					which is a contradiction. 
				\end{solution}
			\item Prove that there exists \( t_0 \in (0, 1) \) such that \( p(t_0) \not \in A \cup B \). 

				\begin{solution}
					Suppose such a \( t_0 \) doesn't exist. Then, for every \( t \in (0, 1) \), it must be the 
					case that \( p(t) \in A \cup B \). Since \( A \) and \( B \) are separated 
					subsets and \( A \cap B = \varnothing \), then this means that each point \( t \in (0, 1) \) 
					can be placed into \( A_0  \) and \( B_0 \). Now, consider a point 
					\( t \) on the boundary of \( B_0 \), and consider any neighborhood \( N_r(t) \). \( t \) is 
					a limit point of \( B_0 \) by definition, but \( t \) must also be a limit point of \( A_0 \),
					since any point \( t - \epsilon \in A_0 \) for every \( \epsilon > 0 \). 

					Therefore, one of \( A_0 \cap \overline B_0 \) or \( \overline A_0 \cap B_0 \) is nonempty, 
					contradicting the assumption that \( A_0 \) and \( B_0 \) are separated. 
				\end{solution}
			\item Prove that every convex subset of \( R^{k} \) is connected. 

				\begin{solution}
					\question{What is a convex subset?}
				\end{solution}
		\end{enumerate}
	\end{problem}

	\begin{problem}
		A metric space is called \textit{separable} if it contains a countable dense subset. Show that \( R^{k} \) 
		is separable. \textit{Hint:} Consider the set of points which have only rational coordinates.  
	\end{problem}

	\begin{solution}
		Before we begin, we should also establish a different definition of density: given an open set \( S \subset 
		X\), the requirement that the set \( S \) be nonempty is equivalent to density. Part (j) of 
		Definition 2.18 gives the following definition of density:
		\[
			\text{\( E \) is \textit{dense} in \( X \) if every point of \( X \) is a limit point of \( E \), or 
			a point of \( E \) (or both).}
		\] 
		All we need to do is to show that the given definition implies that \( S \) is nonempty. Consider 
		a point \( x \in X \) and an open neighborhood \( N_r(x) \). Based on the given definition, either 
		\( x \in E\) or \( x \) is a limit point of \( E \). In the former case, then \( N_r(x) \) is obviously 
		nonemtpy since it contains \( x \). In the latter, then \( N_r(x) \) by definition must contain 
		a point \( q \neq x \) such that \( q \in E \), also satisfying that \( N_r(x) \) is nonempty. 

		With this established, consider the set \( Q^{k} \subset R^{k} \), the set of points with only rational 
		coordinates. Then, consider two points in this set, and consider these two points to the the boundary 
		of our open set. That is, given \( A = (a_1, a_2, \dots, a_k) \) and \( B = (b_1, b_2, \dots, b_k) \), 
		define the open set to be the points \( X = (x_1, x_2, \dots, x_k)  \) such that 
		\begin{align*}
			a_1 < \ &x_1 < b_1\\
			a_2 < \ &x_2 < b_2\\
				  &\vdots\\
			a_k < \ &x_k < b_k
		\end{align*}
		this set is clearly open (proof is very easy), and since the rationals are dense in \( R \), then we are 
		guaranteed to find an \( X \) that satisfies this inequality. Hence, the subset is dense. Further, this 
		subset is also countable, since the rationals \( Q \) are countable. Combining these two allows us 
		to conclude that \( R^{k} \) is separable. 
	\end{solution}
	

	\begin{problem}
		A collection \( \{V_\alpha\}  \) of open subsets of \( X \) is said to be a \textit{base} of \( X \) 
		if the following is true: For every \( x \in X \) and every open set \( G \subset X \) 
		such that \( x \in G \), we have \( x \in V_\alpha \subset G \) for some 
		\( \alpha \). In other words, every open set in  \( X \) is the union of a subcollection of 
		\( \{V_\alpha\}  \). 

		Prove that every separable metric space has a \textit{countable} base. \textit{Hint:} Take 
		all neighborhoods with rational radius and center in some countable dense subset of \( X \).   
	\end{problem}

	\begin{solution}
		The metric space \( X \) is separable, so it contains a countable dense subset. Then, following the hint, 
		we take all neighborhoods with rational radius and center within such a countable dense subset of \( X \). 
		Let the set \( S = \{x_1, x_2, \dots \}  \) represent this countable dense subset, and consider neighborhoods
		with rational radius and center just like the hint says. 

		Then, consider a point \( x \in G \). Because \( G \) is an open set, then \( N_r(x) \subset G\) exists 
		for some \( r \in \R \). Now, since \( S \) is dense in \( X \), then based on the previous problem 
		we know that every open set in \( X \) must contain an element of \( S \). Therefore, 
		there exist some \( x_k \in N_{\frac{r}{2}}(x) \). Now, let \( \alpha = d(x, x_k) \), and since the 
		rationals are dense in \( \R \), then we know that there exists some rational \( r' \) such that 
		\( \alpha < r' < \frac{r}{2} \). Then, \( x \in N_{r'}(x_k) \), and \( N_{r'}(x_k) \) is a neighborhood 
		with rational radius and center as required by our countable dense subset. 
	\end{solution}

	\begin{problem}
		Let \( X \) be a metric space for which every infinite subset has a limit point. Prove that 
		\( X \) is separable. \textit{Hint:} Fix \( \delta > 0 \), pick \( x_1 \in X \). Having chosen 
		\( x_1, \dots, x_j \in X \), choose \( x_{j+1} \in X \), if possible, so that \( d(x_i, x_{j+1})
		\ge  \delta \) for \( i = 1, \dots, j \). Show that this process must stop after a finite number of 
		steps, and that \( X \) can therefore be covered by finitely many neighborhoods of radius \( \delta \). 
		Take \( \delta = 1 / n \) ( \( n = 1, 2, 3, \dots \)), and consider the centers of the corresponding
		neighborhoods. 
	\end{problem}

	\begin{solution}
		To prove separability, we need to prove that it has a countable dense subset. 

		We approach this problem using the hint. Fix \( \delta > 0 \), and pick \( x_1 \in X \), and 
		upon choosing \( x_j \), we choose \( x_{j + 1} \in X \) such that \( d(x_i, x_{j + 1}) \ge \delta \) for all 
		\( i = 1, \dots, j \). Our aim is to show that this process can only proceed a finite number of times. 

		Suppose for contradiction that this sequence does go on infinitely, then the set
		\( S_\delta = \{x_1, x_2, \dots \}  \) is a countably infinite subset of \( X \), 
		and therefore must have a limit point. 
		That is, there exists a point \( x' \in X \) such that there exists some \( x_k \) in every 
		neighborhood \( N_r(x') \). Now, consider the neighborhood \( N_{\frac{\delta}{2}}(x') \). This neighborhood 
		can only contain one point \( x_k \), and therefore if \( \gamma = d(x', x_k) \), then 
		\( N_{\gamma}(x') \) contains no points in \( S_\delta \). The only way that \( N_{\gamma}(x') \) contains 
		a point is if \( x' \in S_\delta \), but then \( N_{\frac{\delta}{2}}(x') \) cannot contain any other 
		points in \( S \), so we are left with the same result. In either case, \( x' \) cannot be a limit 
		point, and therefore this process is guaranteed finite. 

		Then, we've proven that \( X \) can therefore be covered by finitely many neighborhoods of 
		radius \( \delta \). Now, we take \( \delta = \frac{1}{n} \) for \( n = 1, 2, 3, \dots \), which generates 
		a countably infinite set of points \( S = \bigcup_n S_{\frac{1}{n}} \). 
		Now, we need to show that this subset is dense, which is 
		the same as claiming that every open set \( G \subset X \) contains at least one point \( x_k \). 

		Consider a point \( p \in G \). \( G \) is an open set, so there exists a neighborhood \( N_r(p) \subset G \), 
		but this neighborhood must also contain a point \( x_k \), generated by \( \delta < r \), so 
		incidentally we prove that \( G \) must always contain a point \( x_k \in S\). Thus, 
		\( S \) is dense and countable, therefore \( X \) is separable. 

		\question{Perhaps this last argument can be made slightly more rigorous.} 
	\end{solution}

	\begin{problem}
		Define a point \( p \) in a metric space \( X \) to be a \textit{condensation point} of a set 
		\( E \subset X \) if every neighborhood of \( p \) contains uncountably many points of \( E \). 

		Suppose \( E \subset R^{k} \), \( E \) is uncountable, and let \( P \) be the set of all condensation 
		points of \( E \). Prove that \( P \) is perfect and that at most countably many points of \( E \) are not 
		in \( P \). In other words, show that \( P^{c}\cap E \) is at most countable. \textit{Hint:} Let 
		\( \{V_n\}  \) be a countable base of \( R^{k} \), let \( W \) be the union of those \( V_n \) 
		for which \( E \cap V_n \) is at most countable, and show that \( P = W^{c} \). 
	\end{problem}

	\begin{solution}
		Again, referring back to Definition 2.18, a set \( P \) is perfect if \( P \) is closed 
		and if every point of  \( P \) is a limit point of \( P \). We first show that the set \( P \) defined 
		in the problem is closed. For a set to be closed, it means that every limit point of \( P \) is 
		contained within \( P \). 

		For the sake of contradiction, assume that \( P \) is not closed; that is, there exists some \( q \not \in P \) 
		such that \( q \) is a limit point of \( P \), or every neighborhood of \( q \) contains at least one 
		point in \( P \). Consider a neighborhood \( N_r(q) \), and a point \( p \in N_r(q) \) such that 
		\( p \in P \). Then, since \( p \in P \), then every neighborhood around \( p \) contains uncountably 
		many points, so now let \( \delta = d(p, q) \), then \( N_{r - \delta}(p) \) is contained entirely in 
		\( N_r(q) \) (this was done in earlier problems), which implies that \( N_r(q) \) contains 
		uncountably many points. This can be done for every neighborhood, so we are forced to conclude that 
		\( q \in P \). 

		Further, we need to show that every point of \( P \) is a limit point of \( P \). Again, we consider 
		the contradiction be the fact that there exists some \( p \in P \) such that \( p \) is not a limit 
		point of \( P \). Since \( p \) is not a limit point of \( P \), it means that while \( N_r(p) \) is 
		uncountable, the neighborhoods of other points \( q \in N_r(p) \) are countable.      
		Now consider other points \( q \in N_{r}(p) \), and consider an infinite union of neighborhoods around \( q \)
		such that the union of such neighborhoods form \( N_r(p) \). By Theorem 2.12, this set is 
		countable, but then this contradicts the fact that \( N_r(p)  \) is uncountable. Therefore, there must exist 
		some neighborhood \( N_{r'}(q) \) that is uncountable, and therefore \( q \in P \), and \( p \) is 
		a limit point of \( P \). 

		Now we prove that there are countably many points of \( E \) that are not in \( P \), leveraging the hint. Let 
		\( \{V_n\}  \) be a countable base of \( R^{k} \) (this exists because \( R^{k} \) is separable), 
		and let \( W \) be the union of those \( V_n \) for which 
		\( E \cap V_n \) is at most countable. Our goal is to show that \( P = W^{c} \). 

		Based on the construction of \( W \), we can compute 
		\[
			E \cap W = E\cap \left(\bigcup_{\alpha \in \mathcal{A}} V_{\alpha}\right) = 
			\bigcup_{\alpha \in \mathcal{A}} \left( E \cap V_{\alpha} \right) 
		\]
		where \( \mathcal{A} \) is an index set of  \( n \) such that \( E \cap V_n \) is countable. This 
		implies that \( E \cap W \) is countable, and since \( E \) is uncountable, then \( W \) is countable. 
		Based on this, it is then impossible for any point \( p \in P \) to exist in \( W \), since we can 
		define a neighborhood \( N_r(p) \subset W \) that is countable. Thus, every point in \( P \) exists
		in \( W^{c} \). \comment{This is good enough for me, I'm not sure how to show explicit equality.}  
	\end{solution}

	\begin{problem}
		Prove that every closed set in a separable metric space is the union of a (possibly empty) perfect 
		set and a set which is at most countable. (\textit{Corollary:} Every countable closed 
		set in \( R^{k} \) has isolated points.) \textit{Hint:} use the previous part. 
	\end{problem}

	\begin{solution}
		Following the hint, let \( E \subset R^{k} \) be such a closed subset, and let \( P \) be the set of
		condensation points of \( E \). From the previous part, we know that \( P \) is a perfect set, and
		there are only countably many points in \( E \) that are not in \( P \). So, \( P \) is perfect and
		\( P^{c} \) is countable, and \( P \cup P^{c} \) covers \( E \), so we are done.   
	\end{solution}

	\begin{problem}
		Imitate the proof of Theorem 2.43 to obtain the following result: 
		\begin{center}
			\begin{minipage}{0.9\textwidth} 
				If \( R^{k} = \bigcup_1^{\infty} F_n \), where each \( F_n \) is a closed subset of \( R^{k} \), 
				then at least one \( F_n \) has a nonempty interior. 

				\textit{Equivalent statement:} If \( G_n \) is a dense open subset of \( R^{k} \), 
				for \( n = 1, 2, 3, \dots,  \) then \( \bigcap_1^{\infty}G_n \) is not empty 
				(in fact, it is dense in \( R^{k} \). 
			\end{minipage}
		\end{center}
		(This is a special case of Baire's theorem; see Exercise 22, Chap. 3, for the general case.)
	\end{problem}


	\section{Numerical Sequences and Series}
	\begin{problem}
		Let \( X \) be a metric space.\footnote{Persy wants me to suffer.}
		\begin{enumerate}[label=\alph*)]
			\item Call two Cauchy sequences \( \{p_n\} ,  \{q_n\} \) in \( X \) \textit{equivalent} if 
				\[
				\lim_{n \to \infty}d(p_n, q_n) = 0
				\] 
				Prove that this is an equivalence relation. 

				\begin{solution}
					For an equivalence relation, we have three satisfying properties: transitivity, reflexive, and 
					symmetric. 
					Consider three sequences \( \{p_n\}, \{q_n\}  \) and \( \{r_n\}  \), where 
					\( \{p_n\} \equiv \{q_n\}  \) and \( \{q_n\}  \equiv \{r_n\}  \). Then, we have 
					\( \lim_{n \to \infty}d(p_n, q_n) = 0 \) and \( \lim_{n \to \infty}d(q_n, r_n) = 0 \). Then, 
					we can write:
					\[
					d(p_n, r_n) \le d(p_n, q_n) + d(q_n, r_n)
					\] 
					due to the triangle inequality. Applying limits to both sides:
					\[
					\lim_{n \to \infty}d(p_n, r_n) \le \lim_{n\to \infty}d(p_n, q_n) + \lim_{n \to \infty}d(q_n, r_n)
					= 0
					\] 
					We can see this also by letting \( \{x_n\} = d(p_n, q_n) \) and \( \{y_n\} = d(q_n, r_n) \), 
					allowing us to separate the limit as above. 
					Since the metric is bounded below by zero, then we know that \( \lim_{n \to \infty}d(p_n, r_n) 
					= 0\), satisfying transitivity. The limit \( \lim_{n \to \infty}d(p_n, p_n) = 0 \) is 
					immediately obvious, so reflexivity is satisfied. Finally, 
					\( \lim_{n \to \infty}d(p_n, q_n) = 0 \) if and only if	\( \lim_{n \to \infty}d(q_n, p_n) = 0 \) 
					since \( d(a, b) = d(b, a) \), so symmetry is also satisfied. 
				\end{solution}
			\item Let \( X^{*} \) be the set of all equivalence classes so obtained. If \( P \in X^{*}, 
				Q \in X^{*}\), \( \{p_n\}  \in P \), \( \{q_n\}  \in Q \), define 
				\[
				\Delta(P, Q) = \lim_{n \to \infty}d(p_n, q_n)
				\] 
				by Exercise 23, this limit exists. Show that the number \( \Delta(P, Q) \) is unchanged if 
				\( \{p_n\}  \) and \( \{q_n\}  \) are replaced by equivalent sequences, and hence that 
				\( \Delta \) is a distance function on \( X^{*} \). 

				\begin{solution}
					Let \( \{r_n\} \) be an equivalent sequence to \( \{p_n\} \). Because of equivalence,
					then we know that \( \lim_{n \to \infty}d(p_n, r_n) = 0 \), and we are now asked to
					evaluate \( \lim_{n \to \infty}d(r_n, q_n) \). By the triangle inequality, we have
					\[
						d(p_n, q_n) \leq d(p_n, r_n) + d(r_n, q_n)
					\]
					Applying limits, we have:
					\[
						\lim_{n \to \infty}d(p_n, q_n) \leq \lim_{n \to \infty} d(p_n, r_n) + \lim_{n \to
						\infty}d(q_n, r_n)
					\]
					But since \( \{r_n\} \) and \( \{p_n\} \) are equivalent sequences, then the first term
					goes to zero, so 
					\[
						\lim_{n \to \infty} d(p_n, q_n) \leq \lim_{n \to \infty} d(q_n, r_n)
					\]
					But since \( X \) is a metric space, then we also have:
					\[
						\lim_{n \to \infty}d(q_n, r_n) \leq \lim_{n \to \infty}d(q_n, p_n) + \lim_{n \to
						\infty} d(p_n, r_n)
					\]
					from which we conclude
					\[
						\lim_{n \to \infty}d(q_n, r_n) \leq \lim_{n \to \infty}d(p_n, q_n)
					\]
					Combining these two inequalities, we have:
					\[
						\Delta(Q, R) \leq \Delta(Q, P) \leq \Delta Q(Q, R)
					\]
					so they are in fact equal. We need only show that replacing \( \{p_n\} \) with \( \{
					r_n\} \) to complete the proof, since the metric respects symmetry.
				\end{solution}
			\item Prove that the resulting metric space \( X^{*} \) is complete. 

				\comment{\textit{Note:} A
					metric space \( M \) is called complete if every Cauchy sequence of points in \( M \) has
				a limit that is also in \( M \).}

				\begin{solution}
					The trick is to leverage Theorem 3.11, which says that if \( X \) is compact then any
					Cauchy sequence converges to a point in \( X \), which is what we aim to show based on
					definition 3.12. 
				\end{solution}
			\item For each \( p\in X \), there is a Cauchy sequence all of whose terms are \( p \); let \( P_p \) 
				be the element of \( X^{*} \) which contains this sequence. Prove that 
				\[
				\Delta(P_p, P_q) = d(p, q)
				\] 
				for all \( p, q \in X \). In other words, the mapping \( \varphi \) defined by \( \varphi(p) = P_p \) 
				is an isometry (i.e., a distance-preserving mapping) of \( X \) into \( X^{*} \). 
			\item Prove that \( \varphi(X) \) is dense in \( X^{*} \), and that \( \varphi(X) = X^{*} \) if 
				\( X \) is complete. By (d), we may identify \( X \) and \( \varphi(X) \) and thus regard \( X \) 
				as embedded in the complete metric space \( X^{*} \). We call \( X^{*} \) the 
				\textit{completion} of \( X \).
		\end{enumerate}
	\end{problem}


\end{document}
