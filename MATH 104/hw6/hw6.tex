\documentclass[10pt]{article}
\usepackage{../../local}


\newcommand{\classcode}{Math 104}
\newcommand{\classname}{Real Analysis}
\newcommand{\sgn}{\mathrm{sgn}}
\renewcommand{\maketitle}{%
\hrule height4pt
\large{Eric Du \hfill \classcode}
\newline
\large{HW 06} \Large{\hfill \classname \hfill} \large{\today}
\hrule height4pt \vskip .7em
\normalsize
}
\linespread{1.1}
\begin{document}
	\maketitle
	\section*{Problem 1}
	Let $f$ be a continuous function from $(0, 1)$. Assume $f(x) < 1$ for any $x \in (0, 1)$, $\lim_{x \to 0} 
	f(x) < 1$ and $\lim_{x \to 1}f(x) < 1$. Prove $\sup_{x \in (0, 1)} f(x) < 1$.

	\begin{solution}
		Since $f(x) < 1$ for any $x \in (0, 1)$ then if we pick a subinterval $(a, b) \subset (0, 1)$, then we
		know that $f(x) < 1$ for any $x \in (a, b)$, and thus $\sup_{x \in (a, b)} f(x) < 1$ since $(a, b)$ is a subinterval of $(0, 1)$. Now, we need to 
		prove that this also holds for the intervals $(b, 1)$ and $(0, a)$. Let's first do it for $(b, 1)$. 

		Consider the sequence of intervals:
		\[
			\left(b, b + \frac{1 - b}{2}\right ), \left(b + \frac{1 - b}{2}, b + \frac{1-b}{2} + \frac{1 - \frac{1 - b}{2}}{2}\right), 
			\dots
		\] 
		where each next interval is defined recursively as $(b_{n-1}, b_{n-1} + \frac{1 - b_{n-1}}{2})$. Focusing
		just on the upper bound, we have the recursive relation $b_n = b_{n-1} + \frac{1-b_{n-1}}{2}$. If we
		can prove that as $n\to \infty$ that $b_n \to 1$, then we can prove that in the limit, this limit is 
		equivalent to the interval $(b, 1)$. Now let's prove this claim. This sequence $b_n$ is written as:
		\[
			b_n = b_{n-1} + \frac{1 - b_{n-1}}{2} = \frac{1}{2} + \frac{b_{n-1}}{2}
		\] 
		So in the limit as $n \to \infty$, then we can actually write: 
		\[
			b_\infty  = \frac{1}{2} + \frac{b_\infty}{2} \implies \frac{b_\infty}{2} = \frac{1}{2}, b_{\infty} 
			= 1
		\] 
		 so therefore we do conclude that $b_n \to 1$ as $n \to \infty$. Then, since each of these intervals $(
		 b_n, b_{n+1})$ is a subinterval of $(0,1)$, we know that $\sup_{x \in (b_n, b_{n+1})} f(x) <1$. This implies
		 that the supremum of the union of all these intervals:
		 \[
			 S = \bigcup_{n = 1}^\infty (b_n, b_{n+1})
		 \] 
		 is also less than 1, or equivalently $\sup_{x \in S} f(x) < 1$. The same logic 
		 holds for the interval $(0, a)$, where we construct a sequence $a_n = \frac{a_{n-1}}{2}$, and it's clear
		 that regardless of the choice of $a_0$ that this sequence goes to 0, so the supremum of the unions:
		 \[
			S' = \bigcup_{n = 1}^\infty (a_{n+1}, a_n)
		 \] 
		 is also less than 1, or $\sup_{x \in S'} f(x) < 1$. Now, we can combine all three intervals together. 
		 Since the supremum of each interval $S$, $S'$ and $(a, b)$ is less than 1, then we know that the 
		 supremum of the set $A = S \cup S' \cup (a, b)$ also has the property that $\sup_{x \in A} f(x) < 1$, 
		 and since $A = (0, 1)$, this implies that $\sup_{x \in (0, 1)} f(x) < 1$, as desired.
	\end{solution}

	\pagebreak

	\section*{Problem 2}
	Let $f$ be a continuous function from $[0, 1]$ to $[0, 1]$. Prove: There exists one (or more) fixed point
	$x$ such that $f(x) = x$. Hint: Consider $g(x) = f(x) - x$.

	\begin{solution}
		Consider the function $g(x) = f(x) - x$. The minimum difference that $f(x)$ can be from $x$ given the
		problem statement is $-1$ (just take the minimum value and maximum values), and the maximum difference
		is $1$. Therefore, $g(x)$ maps to a subset of the
		interval $[-1, 1]$. Now, if this interval that $g(x)$ maps to contains 0, then we know that the fixed
		point we are looking for is the value $x_0$ such that $g(x_0) = 0$. Now, we prove that it must always
		contain 0. We proceed by contradiction. 

		Suppose that the interval $g(x)$ maps onto does not contain $0$. Then, this means that the sign of $g(x)$
		is either strictly positive or strictly negative. In the case where $g(x) > 0$ for all $x \in [0, 1]$, 
		then this implies that $f(x) > x$ on this interval. However, this is impossible: consider the point $x =
		1$. The sign of $g(x)$ implies that $f(x) > 1$ for this specific point. This is a contradiction, since 
		the maximum value of $f$ is 1, as indicated by the question. 

		Now suppose that $g(x) < 0$ for all $x \in [0, 1]$. This would then imply that $f(x) < x$ for all $x$. 
		Now, consider $x = 0$. In this case, the sign of $g(x)$ implies that $f(x) < 0$, which is also impossible
		since the minimum value of $f$ is 0, given in the problem. Therefore, since neither case works, then 
		$g(x)$ must contain 0, and therefore must contain at least one fixed point. 
	\end{solution}

	\pagebreak
	\section*{Problem 3}
	Prove that a polynomial function $f$ of odd degree has at least one real root. \textit{Hint:} It may help 
	to consider the first case of a cubic, i.e. $f(x) = a_0 + a_1x + a_2x^2 + a_3x^3$ where $a_3 \neq 0$. 

	\begin{solution}
		Since the polynomial is an odd degree, then our goal is to show that regardless of the coefficients, 
		we can choose a value $x_1$ such that $f(x_1) < 0$ and another point $x_2$ such that $f(x_2) > 0$. Since
		$f$ is continuous\footnote{I assume that polynomials are continuous, this is also something that's 
		assumed in the lecture notes}, if we can prove the existence of these two points then we can invoke the
		intermediate
		value theorem to prove that there's some $x_1 < c < x_2$ such that $f(c) = 0$. 

		Now, we aim to prove the existence of $x_1$ and $x_2$. To do so, we look at the sign of the leading 
		coefficient. If $a_3 > 0$, then we know that $\lim_{x \to \infty} f(x) = \infty$ (see below for proof). Specifically, this 
		means that for any positive $M > 0$, we know that there exists some $x_2$ such that $f(x_2) > M$. 
		Therefore, there exists $x_2$ such that $f(x_2) > 0$. Likewise, we know that $\lim_{x \to -\infty} f(x) =
		-\infty$, so therefore for any $M < 0$, there exists a value $x_1$ such that $f(x_1) < M$, and thus
		$f(x_1) < 0$. Finally, since $x_1$ and $x_2$ exist, then by IVT there exists some real value $c$ where 
		$x_1 < c < x_2$ such that $f(c) = 0$. 

		The same logic works for the second case where $a < 0$. In this case, we know that $\lim_{x \to \infty} =
		-\infty$, so for any $M < 0$ there exists some $f(x_2) < M$, and hence $f(x_2) < 0$. Likewise, since
		$\lim_{x \to -\infty} f(x) = \infty$, then for any $M > 0$, there exists some $x_1$ such that $f(x_1) > M
		$, hence $f(x_1) > 0$. Therefore, since $f(x_1)>0$ and $f(x_2) < 0$, and $f$ is continuous, there must 
		exist a point $c$ such that $f(c) = 0$, implying that $f$ has a real root.

		I'm going to now prove that the limits as $x \to \pm \infty$, we have $f(x) \to \pm \infty$. Consider
		the function 
		\[
		 g(x) = \frac{\sgn(a_n)}{x^{n-1}}f(x) = \sgn(a_n)(a_{n-1} + a_n x)
		\]
		From here, it is clear that
		depending
		on the sign of $a_n$, that the limit as $x \to \pm \infty$ means that $g(x) \to \pm \infty$, and thus 
		$f(x) \to \pm \infty$ as well. 
	\end{solution}

	

	\pagebreak

	\section*{Problem 4}
	Suppose $f$ is continuous on $[0, 2]$ and $f(0) = f(2)$. Prove there exist $x, y$ in $[0, 2]$ such that 
	$|y - x| = 1$ and $f(x) = f(y)$. \textit{Hint:} Consider $g(x) = f(x+1) - f(x)$ on $[0, 1]$. 

	\begin{solution}
		Following the hint, consider $g(x) = f(x+1) - f(x)$. 
		Now we proceed to show that $g(x)
		= 0$ for some $x$ via contradiction. 

		Suppose that $g(x) \neq 0$ for all $x \in [0, 1]$. Since $g$ is continuous, then this means that either
		$g(x) > 0$ for all $x \in [0, 1]$ or $g(x) < 0$ for all $x \in [0, 1]$. Otherwise, if $g(x)$ flips sign
		in the interval $[0, 1]$, then the intermediate value theorem would tell us that there exists some $x_0$
		such that $g(x_0) = 0$, and hence $f(x) = f(y)$ when $|y - x| = 1$.

		First, suppose $g(x) > 0$ for all $x$. 
		Then, this means that for $x = 0$, this gives the relation that $f(1) > f(0)$, and plugging in $x = 1$
		gives $f(2) > f(1)$. Combining these two inequalities, this gives $f(2) > f(0)$, which is a 
		contradiction. 

		Likewise, suppose that $g(x) < 0$ for all $x$. Then, this gives $f(1) < f(0)$ for $x = 0$ and plugging in
		$x = 1$ gives $f(2) < f(1)$, so combining these two this gives $f(2) < f(0)$, which is also a 
		contradiction. 

		Therefore, since $g(x)$ cannot strictly be either positive or negative, there must exist a point where
		$g(x)$ switches sign. Therefore, there exists a point where $g(x) = 0$, and thus there exists $x, y \in 
		[0, 2]$ such that $|y - x| = 1$ and $f(x) = f(y)$.
	\end{solution}

	\pagebreak

	\section*{Problem 5}
	Prove the Cauchy condition for the limits of a function: Given $f: A \to \mathbb R$ and $c \in A$ is an 
	accumulation point of $A$. Then, $\lim_{x \to c} f$ exists if and only if the following Cauchy condition 
	holds. For any $\epsilon > 0$, there exists $\delta > 0$ such that for any $x_1, x_2 \in (c - \delta, c + 
	\delta)$, we have
	\[
	|f(x_1) - f(x_2)| < \epsilon
	\] 

	\begin{solution}
		First we prove the forward direction: that $\lim_{x \to c} f(x)$ exists if the Cauchy condition holds. 
		In other words, we prove that $\lim_{x \to c}$ exists if for any $x_1, x_2 \in (c - \delta, c + \delta)$
		then $|f(x_1) - f(x_2)| < \epsilon$. In this case, since $x_1, x_2$ are chosen from the interval $(c -
		\delta, c+\delta)$, then we know that $0 < |x_1 - c| < \delta$ and $0 < |x_2 - c| < \delta$. Then by 
		the triangle inequality: 
		\[
		|f(x_1) - f(c) - f(x_2) + f(c)| \le |f(x_1) - f(c)| + |f(x_2) - f(c)| < \epsilon
		\] 
		This implies that $|f(x_1) - f(c)| < \epsilon$ and $|f(x_2) - f(c)| < \epsilon$, which is the standard 
		statement for continuity. 

		Now for the reverse direction: if the limit exists, we prove the Cauchy condition holds. Recall the 
		standard definition of the limit: $\forall \epsilon >0$, $\exists \delta > 0$ such that $0 < |x - c| <
		\delta$ means that $|f(x) - f(c)| < \epsilon$. Therefore, the Cauchy condition holds if we just choose 
		two $x_1, x_2$ such that $x_1 \neq x_2$ that satisfy this continuity statement since $x_1, x_2$ are 
		within the interval $(c - \delta, c + \delta)$, and again by the triangle inequality, 
		\[
		|f(x_1) - f(c)| + |f(c) - f(x_2)| < 2\epsilon \implies |f(x_1) - f(x_2)| < 2 \epsilon 
		\]
		Since $\epsilon$ is arbitrary, we can just choose $\epsilon' = 2\epsilon$ and we get the Cauchy condition. 
	\end{solution}
	\pagebreak

	\section*{Problem 6}
	Prove: A set $A$ is compact (bounded and closed) if and only if $A$ is sequentially compact, meaning that for
	any sequences $(x_n)$ of $A$, there exists a subsequence $(x_{n_k})$ of $(x_n)$ such that $x_{n_k}$ converges
	to some point $a \in A$.

	\begin{solution}
		We prove the forward direction: $A$ is sequentially compact if $A$ is compact. Because $A$ is bounded,
		then every sequence $x_n \in A$ has a convergent subsequence (By Bolzano-Weierstrass). Furthermore,
		since $A$ is closed, then every sequence \textit{must} converge to a point $a \in A$, since $A$ contains
		all its limit points.

		Now the reverse direction: we prove that if $A$ is sequentially compact, then $A$ is compact. Using the
		hint from the notes, suppose that $A$ is not compact. This means that either $A$ is not bounded, not 
		closed, or both.

		Firstly, if $A$ is not closed, then this implies the existence of some sequence $x_n$ such that it 
		converges to a point $a \not\in A$. This contradicts the statement that $A$ is sequentially compact, 
		because we can just choose the subsequence $x_{n_k}$ to be the sequence $x_n$ itself, which does not
		converge to $a \in A$. 

		Secondly, if $A$ is not bounded, then this implies that $A$ is not bounded either from above or below. 
		Suppose without loss of generality that $A$ is not bounded from above. Now, define a sequence $x_n$ which
		is strictly increasing. Then, every subsequence of $x_n$ is also strictly increasing, and therefore there
		will not exist an $M$ which bounds \textit{any} sequence, because for any $M$ that bounds  
		$x_{n_k}$ up to some $n_k$, we know that there exists some $x_{n_m}$ where $n_m > n_k$ such that $x_{n_m}
		- x_{n_k} > M - x_{n_k}$. This implies that if we take the sequence up to $x_{n_m}$, $M$ no longer bounds
		the sequence. Therefore, since every subsequence of $x_n$ is unbounded, it must diverge, contradicting 
		the fact that there exists a sequence $x_{n_k}$ that converges to a value $a \in A$. This is a 
		contradiction, hence $A$ must be bounded. 

		Therefore, $A$ must be closed and bounded, and thus compact by definition. 
	\end{solution}


\end{document}
