\documentclass[10pt]{article}
\usepackage{../../local}


\newcommand{\classcode}{Math 104}
\newcommand{\classname}{Introduction to Analysis}
\renewcommand{\maketitle}{%
\hrule height4pt
\large{Eric Du \hfill \classcode}
\newline
\large{HW 10} \Large{\hfill \classname \hfill} \large{\today}
\hrule height4pt \vskip .7em
\normalsize
}
\linespread{1.1}
\begin{document}
	\maketitle
	\section{Problem 1}

	Suppose $f$ is a continuous function on $[a, b]$, and $f(x) \ge 0$ for all $x \in [a, b]$. Prove that if 
	$\int_a^b f = 0$, then $f(x) = 0$ for all $x \in [a, b]$.


	\begin{solution}
		Let $U(f; P)$ be the upper Riemann sum on a partition $P$ of $[a, b]$. We know that since $\int_a^b f = 
		0$, then $\int_a^b f = \inf_{P \in \Pi} U(f; P) = \sup_{P \in \Pi} L(f; P)$. Using the definition 
		of the lower Riemann sum, we have: 
		\[
			\sup_{P \in \Pi} \left( \sum_{k = 1}^n (x_k - x_{k-1}) \min_{I_k} f\right) = 0
		\] 
		where $I_k$ is the interval $[x_{k-1}, x_k]$ and $P = \{a = x_1 < x_2 < \dots < x_n = b\}$. Using the 
		definition of a supremum, we get:
		\[
			\sum_{k = 1}^n (x_k - x_{k-1}) \min_{I_k} f \le 0
		\] 
		Since we know that for any partition $x_k - x_{k-1} >  0$, then for this to be satisfied we must have 
		that $\min_{I_k} f \le 0$. So this means that on every interval, we have $\min_{I_k} f = 0$.
		
		Furthermore, for the same partition, we know that: 
		\[
			\inf_{P \in \Pi} \left( \sum_{k = 1}^n (x_k - x_{k-1}) \max_{I_k} f \right) = 0
		\] 
		So using the definition of supremum and the same logic as before, we can deduce that
		$\max_{I_k} f \ge 0$. However, these 


	\end{solution}


	\pagebreak

	\section*{Problem 2}
	Construct an example of a function where $f(x)^2$ is integrable on $[0, 1]$ but $f(x)$ is not. 

	\begin{solution}
		We use the Dirichlet function given in class, with one modification. Let it be defined instead as: 
		\[
		f = \begin{cases}
			1 & \mathbb Q\\
			-1 & x \in \mathbb R \setminus Q
		\end{cases}
		\] 
		Just like how the Dirichlet function is not integrable (the proof for this follows the same way, except
		here we have $\inf_{I_k} f = -1$ for all partitions instead of 0, and $\sup_{I_k} f = 1$ remains
		the same here. However, if we take $f^2$, then this function becomes:
		\[
		f = \begin{cases}
			1 & x \in \mathbb Q\\
			1 & x \in \mathbb R \setminus Q 
		\end{cases}
		\] 
		In this case, the lower Riemann sum reads:
		\[
			L(f; P) = \sum_{k = 1}^n (a_k - a_{k-1}) \inf_{I_k} f = 1
		\] 
		and the upper Riemann sum: 
		\[
			U(f; P) = \sum_{k = 1}^{n} (a_k - a_{k-1}) \sup_{I_k} f = 1
		\] 
		And since they agree, then this function is Riemann integrable. Another way we could have done this was 
		to notice that $f = 1$ for all $x \in \mathbb R$ in this interval, so clearly this function is Riemann
		integrable, since $f$ is a constant. 
	\end{solution}


	\pagebreak
	\section{Problem 3}
	Let $f$ be a bounded function on $[a, b]$, so there exists $B > 0$ such that $|f(x)| < B$ for all $x \in 
	[a, b]$.
	\begin{enumerate}[label=\alph*)]
		\item Show 
			\[
				U(f^2, P) - L(f^2, P) \le 2B[U(f, P) - L(f, P)]
			\] 
			For all partitions $P$ of $[a, b]$. \textit{Hint:} $f(x)^2 - f(y)^2 = [f(x) + f(y)][f(x) - f(y)]$

			\begin{solution}

			\end{solution}
		\item Show that if $f$ is integrable on $[a, b]$, then $f^2$ also is integrable on $[a, b]$. 
	\end{enumerate}




	\pagebreak
	\section*{Problem 4}
	Let $f$ and $g$ be integrable functions on $[a, b]$.
	\begin{enumerate}[label=\alph*)]
		\item Show $fg$ is integrable on $[a, b]$. \textit{Hint:} Use exercise 33.7 and $4fg = (f+g)^2 -
			(f-g)^2$.
		\item Show $\max(f, g)$ and $\min(f, g)$ are integrable on $[a, b]$. \textit{Hint:} Exercise 17.8.
	\end{enumerate}


	\pagebreak
	\section*{Problem 5}

	\begin{enumerate}[label=\alph*)]
		\item For any two numbers $u, v\in \mathbb R$, prove that $uv \le (u^2 + v^2)/2$. Let $f$ and $g$ be two
			integrable functions on $[a, b]$. Prove that if $\int_a^b f^2 = 1$ and $\int_a^b g^2 = 1$, then \[
			\int_a^b fg \le 1
			\] 

			\begin{solution}
				Rearranging the inequality, we get: 
				\begin{align*}
					2uv &\le u^2 + v^2\\
					\therefore (u-v)^2 &\ge 0
				\end{align*}
				which is a true statement for all $u, v$.

				With the integral, we can write the integral of the product as: 
				\begin{align*}
					\int_a^b fg &\le \int_a^b \frac{f^2 + g^2}{2}\\
					&\le \frac{1}{2}\left( \int_a^b f^2 + \int_a^b g^2\right)\\
					& \le 1
				\end{align*}
				as desired. 
			\end{solution}
		\item Prove the Schwarz inequality, that for any two integrable functions $f$ and $g$ on the interval
			$[a, b]$, 
			\[
				\left| \int_a^b fg\right| \le \left( \int_a^b f^2\right)^{1/2} \left( \int_a^b g^2 \right)^{1/2}
			\] 

			\begin{solution}
				
			\end{solution}
	\end{enumerate}
\end{document}
