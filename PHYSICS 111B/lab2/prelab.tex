\documentclass[10pt]{article}
\usepackage{../../local}
\urlstyle{same}

\newcommand{\classcode}{Physics 111B}
\newcommand{\classname}{Instrumentation Laboratory}
\renewcommand{\maketitle}{%
\hrule height4pt
\large{Eric Du \hfill \classcode}
\newline
\large{Prelab 2} \Large{\hfill \classname \hfill} \large{\today}
\hrule height4pt \vskip .7em
\small{Header styling inspired by CS 70: \url{https://www.eecs70.org/}}
\normalsize
}
\linespread{1.2}
\begin{document}
	\maketitle
	\begin{enumerate}[label=\arabic*.]
		\item What is the general principle of optical pumping? Go over your
			derivation of the Breit-Rabi formula and the values of the Lande
			g-factors of the hyperfine energy levels of \ch{^{85}RB} and \ch{^{87}RB}
			showing the fine, hyperfine, and Zeeman splittings. How do the Lande
			g-factors affect the ordering of the Zeeman levels? Show the transitions
			between these levels that are important to this experiment. Include these
			drawings in your write-up. For our rubidium system, what is the pumping
			process? Where is the pumped level? Where is the RF transition? 

			\begin{solution}
				The general idea of optical pumping is to observe the transitions
				between the ground and excited states of atoms, from \( \ket*{F, m_F}
				\) to \( \ket*{F, m_{F} + 1} \). \question{why are we only
				considering \( m_F + 1 \)? Why can't the increase be fractional?}

				Because simply exciting the electrons with the magnetic field has no
				effect on the system, and instead we need to spin polarize everything
				so that we see a change. 
			\end{solution}
		\item Why do we modulate (vary sinusoidally) the external magnetic field? How
			do we take data if the magnetic field were not modulated?
		\item In this experiment, how will we determine the resonance frequency? How
			can we best estimate the error? Will the modulation amplitude affect our
			result? What data will we take, and what plots will you make? 
	\end{enumerate}	
\end{document}
