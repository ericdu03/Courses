\documentclass[10pt]{article}
\usepackage{../../local}
\urlstyle{same}

\newcommand{\classcode}{Physics 111B}
\newcommand{\classname}{Experimentation Laboratory II}
\renewcommand{\maketitle}{%
\hrule height4pt
\large{Eric Du \hfill \classcode}
\newline
\large{Prelab 3} \Large{\hfill \classname \hfill} \large{\today}
\hrule height4pt \vskip .7em
\small{Header styling inspired by CS 70: \url{https://www.eecs70.org/}}
\normalsize
}
\linespread{1.2}
\begin{document}
	\maketitle
	\begin{enumerate}[label=\arabic*.]
		\item Why are there energy bands in materials? What is a valence band? A
			conduction band? A band gap?

			\begin{solution}
				Energy bands are the energies that electrons inside a material
				occupy. The valence band is defined to be the highest energy level
				that is occupied by electrons, the conduction band is the lowest
				energy level that is \textit{not} occupied. The band gap is the
				difference between these two energies.   

				The conduction band can be thought of as "freely moving" electrons.  
			\end{solution}
		\item How do conductors, insulators, and semiconductors differ in their
			energy-and structures?

			\begin{solution}
				Conductors have an overlapping conduction and valence band, giving
				rise to "free" electrons that move with an arbitrarily small voltage. 

				Insulators have a large band gap that is much larger than \( k_BT \),
				so small voltages don't cause any electrons to excite to the
				conduction band, hence no current. 

				Semiconductors are by definition also insulators, but where the band
				gap is on the order of \( k_BT \), so at \( k_BT \) due to the
				redistribution of electrons there are some electrons that do become
				promoted to the conduction band. 
			\end{solution}
		\item How do we explain the fact that there are free electorons in a metallic
			conductor? What is an extrinsic semiconductor? 

			\begin{solution}
				The electrostatic pull between electrons and the other nuclei give
				rise to free electrons in a metallic ocnductor. An extrinsic
				semiconductor is one that is "doped" with other atoms, in order to
				reduce the band gap. 
				
				\question{what do the holes do? Can electrons occupy those holes, and
				thereby we reduce the band gap?}
			\end{solution}
		\item What is the Hall effect? 

			\begin{solution}
				The Hall effect is created as a result of an imbalance in
				electrostatic charges when a magnetic field is placed across a
				current.   
			\end{solution}
		\item Explain the Van Der Pauw Technique. 

			\begin{solution}
				Uses four leads and the voltages across them to determine
				characteristics of the semiconductor.  

				The Hall voltage can be calculated using \( (R_{AC, BD} + R_{BD, AC})
				/ 2\). Combining this with the measurement of \( B_Z \) gives us the
				hall coefficient. We can determine it via:
				\[
					V_H = -\frac{I_x B_z}{end}
				\]
				Then, we get \( E_H = q V_H \), we have:
				\[
					R_H = \frac{qV_H}{J_x B_z} = \frac{1}{en}
				\]
				The quantity on the right is probably something that'll be given to
				us, and the one in the middle is the one we will experimentally solve
				for. 
			\end{solution}
		\item What measurements are needed for studying the Hall Effect? 

			\begin{solution}
				Given by the table, we need measurements \( I, V \) for all four
				corners. 
			\end{solution}
	\end{enumerate}
\end{document}
