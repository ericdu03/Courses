\documentclass[10pt]{article}
\usepackage{../../local}
\urlstyle{same}

\newcommand{\classcode}{Physics 111B}
\newcommand{\classname}{Advanced Experimentation Laboratory}
\renewcommand{\maketitle}{%
	\hrule height4pt
	\large{Eric Du \hfill \classcode}
	\newline
	\large{Prelab 1} \Large{\hfill \classname \hfill} \large{\today}
	\hrule height4pt \vskip .7em
	\small{Header styling inspired by CS 70: \url{https://www.eecs70.org/}}
	\normalsize
}
\linespread{1.2}
\begin{document}
\maketitle
\section*{Prelab Questions}
\begin{enumerate}[label=\arabic*.]
	\item What is Non-linear Dynamics (NLD)? Specifically what is it that's 
		non-linear? 

		\begin{solution}
			Nonlinear dynamics deals with the dynamics of nonlinear systems. 
			Namely, this means that the differential equation 
			that governs the motion of the system has a nonlinear term, so 
			a power of 2 or above in one of the \( x, \dot x, \ddot x, \dots \)
			terms. 

			For instance, the equation \( \ddot x + \frac{g}{L}\sin x = 0 \)
			is a nonlinear dynamical equation. 
		\end{solution}
	\item What is chaos? What are the defining characteristics of chaos?

		\begin{solution}
			The book defines it as follows:

			\textit{Chaos is \textbf{aperiodic long-term behavior} in a 
				\textbf{deterministic} system that exhibits 
			\textbf{sensitive dependence on initial conditions.}}
			Explained:
			\begin{itemize}
				\item Aperiodic just means that over the long term the system doesn't
					settle into a periodic state. 
				\item Deterministic means that the system doesn't have any 
					noisy inputs -- the irregular behavior arises as a result of the 
					equations rather than noise
				\item Sensitive dependence on initial conditions means that 
					trajectories which start off very similar diverge exponentially 
					quickly. 
			\end{itemize}
		\end{solution}

	\item Present and solve on the board the problems 9.3.1 and 11.4.2.
		\begin{enumerate}[label=\arabic*)]
			\item (Quasiperiodicity \( \neq \) chaos) 
				The trajectories of the quasiperiodic system 
				\( \dot \theta_1 = \omega_1, \dot \theta_2 = \omega_2 \) 
				(\( \omega_1 / \omega_2 \) is irrational) are not periodic. 

				\begin{enumerate}[label=(\alph*)]
					\item Why isn't this system considered chaotic?

						\begin{solution}
							This system is not considered chaotic because the distance 
							\( \|\delta(t)\| \) does not scale exponentially. 
						\end{solution}

					\item Without using a computer, find the largest Liapunov exponent 
						for the system. 

						\begin{solution}
							I imagine if we consider two trajectories separated 
							by some \( \| \delta\|\), that they won't diverge at all -- 
							they'll just be parallel to each other. There's no reason 
							for them to diverge if \(  \omega_1, \omega_2 \) are 
							identical for both systems. Thus, \( \|\delta(t)\| = 
							\| \delta_0\|\), so \( \max \lambda = 0 \). 
						\end{solution}
				\end{enumerate}
			\item Find the box dimension of the Sierpinski carpet.

				\begin{solution}
					The book gives the following definition for the box 
					dimension:
					\[
						d = \lim_{\epsilon \to 0}
						\frac{\ln N(\epsilon)}{\ln (1 / \epsilon)}
					\]
					Here \( N(\epsilon) \) is the number of \( \epsilon \)-sized
					squares are required to cover the entire carpet.

					Let the carpet have side length 1. Then, \( S_1 \) is covered 
					by 8 squares with side length \( \frac{1}{3} \). Then, \( S_2 \)
					is covered by \( 8^2 \) squares with side length 
					\( \epsilon = \left(\frac{1}{3}\right)^2 \). (this follows the 
					same process as Example 11.4.2 in the text) Then, we have:
					\[
						d = \lim_{\epsilon \to 0}\frac{\ln 8^{n}}{\ln 3^{n}}
						= \frac{\ln 8 }{\ln 3}
					\]
				\end{solution}
		\end{enumerate}
	\item What is the Feigenbaum ratio? What does it mean? See the 
		Wolfram Math World subjects for help. 

		\begin{solution}
			The Feigenbaum constant is a universal constant for systems which 
			approach chaos via the process known as \textbf{period 
			doubling}. Basically, for some chaotic systems where the period 
			appears to double, we can let \( \mu_n \) denote the point 
			where the period doubles to \( 2^{n} \), then we can 
			express the Feigenbaum ratio as:
			\[
				\delta = \lim_{n \to \infty} \frac{\mu_{n + 1} - \mu_{n}}{\mu_{n + 2}
				- \mu_{n + 1}}
			\]
			It is known that \( \delta > 1 \), and for a particular 
			structure called the logistic map, we know \( \delta \approx 4.669 \). 

			Really, it just tells us that we increase our parameter, 
			the time between subsequent period doublings gets faster and 
			faster, roughly 4.5x faster every time.  

			What's also interesting about this is that it's considered 
			\textbf{universal}, meaning that all period doubling systems, when 
			we perform the calculation above, yields the same constant. 
		\end{solution}
	\item What is a return map? What is a Poincare map? 

		\begin{solution}
			I honestly am not really sure. As for a Poincare section (which I imagine 
			to be very closely related), it basically a traced version of the 
			state space as time evolution is applied to the system. 


			After reading the textbook a bit more, basically the Poincare map 
			is a way for us to denote when a system comes back on itself. The textbook 
			has the following example: let \( \dot{\mathbf{x}} = f(\mathbf{x}) \) be 
			an \( n \)-dimensional system, and consider the \( n-1 \) dimensional 
			surface \( S \). Then, the evolution of the dynamical system passes
			perpendicular to \( S \) always. Now, let \( \mathbf{x}_k \) be the 
			\( k \)-th intersection, and \( \mathbf{x}_{k + 1} \) be the 
			\( (k + 1) \)-th intersection through \( S \). Then, the Poincare 
			map \( P: S \to S \) is definde by \( P(\mathbf{x}_k) = 
			\mathbf{x}_{k + 1}\)

			The Poincare map is also called the \textbf{first return map}, which 
			I think is what the first part of this problem is referring to. 
		\end{solution}
	\item What is a Fourier Transform? What does the power spectrum of a square 
		wave look like? What is aliasing?

		\begin{solution}
			A Fourier transform gives us a way to view a signal in frequency space 
			instead of temporal space. It's occasionally useful :)

			The Fourier transform is a sinc function, so the 
			power is a \( \mathrm{sinc}^2 \) function.

			Aliasing is the phenomenon when we undersample our points, and as a result 
			of our undersampling we get losses in the data. Bascially, the sampling 
			frequency must be at least half the bandwidth (in frequency space) 
			in order to prevent aliasing. (This is the Nyquist criterion) 
		\end{solution}
\end{enumerate}
\end{document}
