\documentclass[10pt]{article}
\usepackage{../../local}


\newcommand{\classcode}{Physics 112}
\newcommand{\classname}{Introduction to Statistical Mechanics}
\renewcommand{\maketitle}{%
\hrule height4pt
\large{Eric Du \hfill \classcode}
\newline
\large{HW 03} \Large{\hfill \classname \hfill} \large{\today}
\hrule height4pt \vskip .7em
\normalsize
}
\begin{document}
	\maketitle
	\section*{Collaborators}
	I worked with \textbf{Andrew Binder, Adarsh Iyer, Nathan Song} and \textbf{Teja Nivarthi} on this homework
	assignment. 


	\section*{Problem 1 (25 pts)}
	In this class (and in pretty much all of physics) it is key to be proficient at computing Gaussian integrals.
	Consider
	\[
		I = \iinf dx e^{-\frac{1}{2}x^2}
	\] 
	\begin{enumerate}[label=\alph*)]
		\item Show that
			\[
				I^2 = \int_{\mathbb R} dx dy e^{-\frac{1}{2}(x^2 + y^2)}
			\] 
			\begin{solution}
				The definition of $I^2$ is that we multiply two integrals together, with separate 
				integration variables. Therefore, we can write,
				\begin{align*}
					I^2 = \left( \iinf e^{-\frac{1}{2}x^2} dx \right) 
					\left( \iinf e^{-\frac{1}{2}y^2} dy \right) 
				\end{align*}
				Then notice that since these integrals are independent of each other, this is the same 
				as writing a double integral:
				\[
					I^2 = \iinf \iinf e^{-\frac{1}{2}(x^2 + y^2)} dx dy = \int_{\mathbb R^2} 
					e^{-\frac{1}{2}(x^2 + y^2)} dx dy
				\] 
				as desired.
			\end{solution}
		\item Compute $I^2$ by expressing this integral in polar coordinates. Conclude that
			\[
				I = \iinf dx e^{-\frac{1}{2}x^2} = \sqrt{2\pi} 
			\] 
			\begin{solution}
				Following the hint, we compute this integral in polar coordinates, where our 
				integration bounds now go from $r \in [0, \infty)$ and $\theta \in [0, 2\pi]$. Our 
				integral becomes:
				\[
					I^2 = \int_{0}^{2\pi}\int_0^\infty e^{-\frac{1}{2}r^2} r dr d\theta 
					= \int_0^{2\pi} d\theta \int_0^\infty e^{-\frac{1}{2}r^2} r dr
				\] 
				The second integral can be solved via a $u$-substitution of $u = \frac{r^2}{2}$ so 
				$du = r dr \implies dr = \frac{du}{r}$, hence:
				\[
					I^2 = 2\pi \int_0^\infty e^{-u} du = 2\pi \left[-e^{-u}\right]_{u = 0}^\infty = 2\pi
				\] 
				Since this is $I^2$, then we conclude that $I = \sqrt{2\pi}$.
			\end{solution}
		\item Show that 
			\[
				I = \iinf dx e^{-\frac{a}{2}x^2} = \sqrt{\frac{2\pi}{a}} 
			\] 
			\begin{solution}
				The same steps done to part (b) can solve this problem as well. We split into polar coordinates,
				and instead of letting $u = \frac{r^2}{2}$, we can let $u = \frac{ar^2}{2}$, meaning 
				$dr = \frac{du}{ar}$. Therefore, the integral becomes:
				\[
					I^2 = 2\pi \int_0^\infty e^{-u} \frac{du}{a} = \frac{1}{a}\underbrace{\left( 2\pi \int_0^\infty 
					e^{-u} du \right)}_{\text{same integral as part (b)}} = \frac{2\pi}{a}
				\] 
				Hence, $I = \sqrt{\frac{2\pi}{a}}$, as desired.
			\end{solution}
		\item Show that
			\[
				I = \iinf dx e^{-\frac{a}{2}x^2 + bx}  = e^{\frac{1}{2a}b^2}\sqrt{\frac{2\pi}{a}} 
			\] 
			\begin{solution}
				Here, we complete the square on the exponent. First, factor out $-\frac{a}{2}$ from the 
				exponent:
				\[
					I = \iinf e^{-\frac{a}{2}(x^2 + \frac{2b}{a}x)} dx
				\] 
				And now we complete the square:
				\[
					I = \iinf e^{-\frac{a}{2}\left(\left(x - \frac{b}{a}\right)^2 - \frac{b^2}{a^2}\right)} dx
					= \iinf e^{-\frac{a}{2}\left( x - \frac{b}{a} \right)^2 + \frac{b^2}{2a}} dx = e^{\frac{b^2}{2a}}
					\iinf e^{-\frac{a}{2}\left( x - \frac{b}{a} \right)^2} dx
				\] 
				This integral gives the same result as the previous integral, since we can perform a 
				$u$-substitution $u = x + \frac{b}{a}$ so $du = dx$. Therefore:
				\[
					I = e^{\frac{b^2}{2a}} \underbrace{\iinf e^{-\frac{a}{2}u^2} du}_{\text{same as part (c)}}
					= e^{\frac{b^2}{2a}} \sqrt{\frac{2\pi}{a}} 
				\] 
				as desired.
			\end{solution}
		\item Show that
			\[
				I = \iinf dx e^{-\frac{a}{2}x^2} x^2 = \frac{1}{a}\sqrt{\frac{2\pi}{a}} 
			\] 
			Hint: Compute
			\[
				\left[\pdv[2]{b} \iinf dx e^{-\frac{a}{2}x^2 + bx}\right]_{b = 0}
			\] 
			by both differentiating under the integral sign and explicitly computing its derivatives using 
			the result in d).

			\begin{solution}
				Following the hint, we know the result of the integral from the previous problem, so we're 
				basically just left with 
				\[
					I = \pdv[2]{b}\left[ e^{\frac{b^2}{2a}}\sqrt{\frac{2\pi}{a}}\right]_{b = 0} 
					= \sqrt{\frac{2\pi}{a}} \pdv{b}\left[\frac{b}{a}e^{\frac{b^2}{2a}}\right]_{b =0} 
					= \sqrt{\frac{2\pi}{a}} \left[ \frac{1}{a}e^{\frac{b^2}{2a}} + \frac{b^2}{a^2}e^{\frac{b^2}{2a}}\right]_{b =0}
						= \frac{1}{a}\sqrt{\frac{2\pi}{a}} 
				\] 
				as desired. 
			\end{solution}
	\end{enumerate}
	\pagebreak
	\section*{Problem 2 (20 pts)}
	Consider a system of $N$ particles with spin. Label the particles using an index $i = 1, \dots, N$ so 
	that the $i$-th particle has spin $s_i$. Unlike the systems we have encountered so far, in this system
	the possible values for each spin are $s_i = -1, 0, 1$. The energy of this system 
	is given by 
	\[
		E = D \sum_{i = 1}^N s_i^2
	\] 
	In other words, when the spin is $\pm 1$, this costs the system an energy $D$, while whenever 
	the spin is 0, this spin does not contribute to the energy of the system.
	\begin{enumerate}[label=\alph*)]
		\item Explain why the number of accessible states that the system when it has an energy $E$ is 
			\[
				\Omega(N, E) = {N \choose E/D} 2^{E / D}
			\] 
			\begin{solution}
				Since each particle contributes an energy $D$ to the system, if we have a total system 
				with energy $E$ then we can divide this up into $\frac{E}{D}$ quanta of energy, and we need 
				to count the number of ways to allocate $\frac{E}{D}$ quanta of energy among $N$ particles. This
				is a problem that we've done countless times before, where $q = \frac{E}{D}$, where we have 
				${N \choose q}$ possible ways. 

				Further, since a spin of $\pm 1$ contributes the same toward the total energy, then we have 
				2 choices for the spin of each particle among those that we've selected to carry 
				energy. This explains the $2^{E/D}$ term. Putting these two together, we get:
				\[
					\Omega(N, E) = {N \choose E / D} 2^{E / D}
				\] 
			\end{solution}
		\item Compute the entropy of the system $S(N, E)$ in the limit of many particles and high energies 
			$E \gg D$.

			\begin{solution}
				We use the definition of $S = k \ln \Omega$, so we have:
				\[
					S = k \ln \left[{N \choose E / D}\right] + k\frac{E}{D} \ln 2
				\] 
				For this, we expand the binomial coefficient into an expression involving factorials, and 
				use Stirling's formula of the form $\ln x! = x \ln x - x$, which gives us:
				\begin{align*}
					S &= k\left[ N \ln N - N - \frac{E}{D}\ln \frac{E}{D} + \frac{E}{D} - 
						\left( N - \frac{E}{D} \right) \ln \left( N - \frac{E}{D} \right) + 
					\left( N - \frac{E}{D}\right) + \frac{E}{D}\ln 2\right]\\
						&= k\left[N \ln N - \frac{E}{D}\ln \frac{E}{D}- \left( N - \frac{E}{D} \right) 
						\ln \left( N - \frac{E}{D} \right) + \frac{E}{D}\ln 2\right]
				\end{align*}
				We can't simplify this further, so that's where we'll stop. In the limit where $E \gg D$, 
				we can replace all $\frac{E}{D} = E$, so this gives us:
				\[
					S = k\left[N \ln N - E \ln E - \left( N - E \right) \ln (N - E) + E \ln 2\right]
				\] 
			\end{solution}
		\item Compute the temperature of this system as a function of the energy and the number of particles. 
			Can the temperature of this system be negative?

			\begin{solution}
				To calculate temperature, we use the definition that $\frac{1}{T} \equiv \dv{S}{U} = \dv{S}{E}$.
				Therefore, computing this derivative:
				\begin{align*}
					\frac{1}{T} &= k \left[-\frac{1}{D}\ln \frac{E}{D} - \frac{E}{D} \cdot \frac{1}{\frac{E}{D}}
						\frac{1}{D} -\left(-\frac{1}{D}\right) \ln \left(N - \frac{E}{D}\right) -
						\left( N - \frac{E}{D} \right) \frac{1}{\left( N - \frac{E}{D} \right) } 
					\left( -\frac{1}{D} \right)  + \frac{1}{D} \ln 2\right]\\
								&= \frac{k}{D}\left[\ln\left( N - \frac{E}{D} \right) + \ln 2 - \ln \frac{E}{D}\right]
				\end{align*}
				We can combine this into a single logarithm, giving us:
				\[
					\frac{1}{T} = \frac{k}{D}\ln\left[\frac{2\left( N - \frac{E}{D} \right)}{\frac{E}{D}}\right]
				\] 
				From here, we can see that it is indeed possible for $T$ to be negative, and precisely it 
				is negative when the logarithm returns a negative number, which occurs when:
				\[
					\frac{2\left( N - \frac{E}{D} \right) }{\frac{E}{D}} < e
				\] 
				Specifically in terms of $E$, this occurs when $E > \frac{2}{3}ND$.
			\end{solution}
		\item Obtain the energy for the system as a function of temperature. Discuss the low and high temperature
			limits. What happens to the entropy in these limits? What is the physical meaning of this?

			\begin{solution}
				We simply take the expression we got in the previous part and solve for $E$, which gives 
				us the following expression:
				\[
					E = \frac{ND}{1 + \frac{1}{2}e^{D / kT}}
				\] 
				Now to analyze the bounds. At high temperature, then $e^{D / kT} \to e^0 = 1$, so we have 
				$E \to \frac{2}{3}ND$ in this limit. At low temperature, $e^{D / kT} \to \infty$ so $E \to 0$
				in this limit. 

				Physically, the low temperature limit makes intuitive sense, since $E \to 0$
				at low temperature, matching our intuitive relationship between energy and temperature .
				At high temperature, we expect every single bit to be randomly 
				distributed, so $\frac{2}{3}$ of them will be in the states $\pm 1$, which contributes 
				to the total energy of the system. So, with this analysis, the high temperature limit 
				works as well.
				
				As for the entropy, we can substitute $E = \frac{2}{3}ND$ into the expression we 
				derived in part (b) for the high temperature limit, which gives us:
				\[
				S = kN \ln 3 = k \ln 3^N
				\] 
				At the low temperature limit, $E \to 0$, so all the terms with $E$ immediately disappear (note
				that $\lim_{x \to 0} x \ln x = 0$). This
				gives us that at low temperature, 
				\[
					S = k\left[N \ln N - N \ln N\right] = 0
				\] 
				The physical interpretation of the high temperature limit is that at high temperatures, each 
				particle has three possibilities for its spin, hence there are $3^N$ total microstates, which 
				matches our expression of $S = k \ln \Omega = k \ln 3^N$. At low temperatures, this limit also 
				makes sense (kind of), where we can think of the low temperature limit as a configuration where 
				we have very little information, hence $S$ approaches zero. 
			\end{solution}
	\end{enumerate}
	\pagebreak
	\section*{Problem 3 (20 pts)}
	Consider an ideal gas undergoing a process described by the fact that $pV^2$ is a constant. 
	What is the molar heat capacity of this process?

	\begin{solution}
		The heat capacity is defined as $C = \frac{Q}{\Delta T}$, and using the first law of Thermodynamics
		gets us
		\[
			C = \frac{Q}{\Delta T} = \frac{\Delta U - W}{\Delta T} = \frac{\Delta U}{\Delta T} -
			\frac{W}{\Delta T}
		\] 
		We assume that we're working with a monoatomic ideal gas so $f = 3$ (from Ed), so we have:
		\[
			U = \frac{3}{2}NkT = \frac{3}{2}nRT \implies \frac{\Delta U}{\Delta T} = \dv{U}{T} = \frac{3}{2}nR
		\] 
		Meanwhile, we can use $W = -P\Delta V$ to simplify the second term:
		\[
			\frac{W}{\Delta T} = -\frac{P \Delta V}{\Delta T} = -P \dv{V}{T}
		\] 
		To do this, we use the fact that $PV^2$ is constant, so $\dv{(PV^2)}{T} = 0$:
		\begin{align*}
			\dv{(PV^2)}{T} &= P \dv{V^2}{T} + V^2 \dv{P}{T}\\
			0 &= 2PV \dv{V}{T} + V^2 \dv{P}{T}\\
			V^2 \dv{P}{T} &= -2PV \dv{V}{T}
		\end{align*}
		Rearranging this to match the expression we have for $\frac{W}{\Delta T}$:
		\[
			-P  \dv{V}{T} = \frac{V}{2} \dv{P}{T} 
		\] 
		To calculate $\dv{P}{T}$, we use the ideal gas law:
		\[
			\dv{P}{T} = \dv{T}\left( \frac{nRT}{V} \right) = \frac{nR}{V} - \frac{nRT}{V^2} \dv{V}{T}
		\] 
		Therefore, we have:
		\begin{align*}
			-P \dv{V}{T} &= \frac{V}{2}\left( \frac{nR}{V} - \frac{nRT}{V^2}\dv{V}{T} \right) \\
						 &= \frac{nR}{2} - \frac{P}{2} \dv{V}{T} 
		\end{align*} 
		Noticing that we have the same term on the left hand side, we get:
		\[
			-\frac{P}{2} \dv{V}{T} = \frac{nR}{2} \implies -P \dv{V}{T} = nR = \frac{W}{\Delta T}
		\] 
		Therefore, putting them both together:
		\[
			C = \frac{Q}{\Delta T} = \frac{\Delta U - W}{\Delta T} = \frac{3}{2}nR - nR = \frac{1}{2}nR
		\] 
		Finally, computing the molar heat capacity:
		\[
		C_n = \frac{C}{n} = \frac{R}{2}
		\] 
	\end{solution}
\end{document}
