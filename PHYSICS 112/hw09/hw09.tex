\documentclass[10pt]{article}
\usepackage{../../local}
\urlstyle{same}

\newcommand{\classcode}{Physics 112}
\newcommand{\classname}{Introduction to Statistical and Thermal Physics}
\renewcommand{\maketitle}{%
\hrule height4pt
\large{Eric Du \hfill \classcode}
\newline
\large{HW 09} \Large{\hfill \classname \hfill} \large{\today}
\hrule height4pt \vskip .7em
\small{Header styling inspired by CS 70: \url{https://www.eecs70.org/}}
\normalsize
}
\linespread{1.1}
\begin{document}
	\maketitle
	\section*{Problem 1}
	Let \(  \Phi = E - TS - \mu N\) be the ``grand'' potential
	\begin{enumerate}[label=\alph*)]
		\item Derive the thermodynamic identity \( d\Phi = -S dT - P dV + N d \mu \) 
		\item Under what conditions does a system adjust so as to minimize \( \Phi \)?
		\item Suppose you have computed \( \Phi(T, V, \mu) \). How could you use it to determine 
			\( N(T, V, \mu)  \) and \( P(T, V, \mu) \)?
		\item Prove that \( \Phi(T, V, \mu) = -kT \ln \mathscr Z\), where \( \mathscr Z = \sum_{\alpha}
			e^{-\beta(E_\alpha - \mu N_\alpha})\)
	\end{enumerate}
	\pagebreak
	\section*{Schroeder 7.19}
	Each atom in a chunk of copper contributes one conduction electron. Look up the density and atomic mass 
	of copper, and calculate the Fermi energy, the Fermi temperature, the degeneracy pressure, and the 
	contribution of the degeneracy pressure to the bulk modulus. Is room temperature sufficiently low 
	to treat this system as a degenerate electron gas?

	\pagebreak
	\section*{Schroeder 7.22}
	Consider a degenerate electron gas in which essentially all of the electrons are highly relativistic 
	\( \epsilon \gg mc^2 \), so that their energies are \( \epsilon = pc \) (where \( p \) is the magnitude 
	of the momentum vector).
	\begin{enumerate}[label=\alph*)]
		\item Modify the derivation given above to show that for a relativistic electron gas at zero temperature,
			the chemical potential (or Fermi energy) is given by \( \mu = hc(3N / 8 \pi V)^{1 / 3} \).
		\item Find a formula for the total energy of this system in terms of \( N \) and \( \mu \).
	\end{enumerate}
	\pagebreak
	\section*{Schroeder 7.23}
	A \textbf{white dwarf} star (see Figure 7.12) is essentially a degenerate electron gas, with a bunch of 
	nuclei mixed in to balance the charge and to provide gravitational attraction that holds the start 
	together. In this problem you will derive a relation between the mass and the radius of a white dwarf 
	star, modeling the star as a uniform-density sphere. White dwarf stars tend to be extremely hot by our 
	standards; nevertheless, it is an excellent approximation in this problem to set \( T = 0 \).
	\begin{enumerate}[label=\alph*)]
		\item Use dimensional analysis to argue that the gravitational potential energy of a uniform-density 
			sphere (mass \( M \), radius \( R \) ) must equal 
			\[
				U_{\text{grav}} = -(\text{constant}) \frac{GM^2}{R}
			\] 
			where (constant) is some universal constant. Be sure to explain the minus sign. The constnat 
			turns out to equal 3/5; you can derive it by calculating the (negative) work needed to 
			assemble the sphere, shell by shell, from the inside out.
		\item Assuming that the star contains one proton adn one neutron for each electron, and that the 
			electrons are nonrelativistic, show that the total (kinetic) energy of the 
			degenerate electrons equals 
			\[
				U_{\text{kinetic}} = (0.0088) \frac{h^2 M^{5 / 3}}{m_e m_p^{5 / 3}R^2}
			\] 
			The numerical factor can be expressed exactly in terms of \( \pi \) and cube roots 
			and such, but it's not worth it.
		\item The equilibrium radius of the white dwarf is that which minimizes the total energy 
			\( U_{\text{grav}} + U_\text{kinetic} \). Sketch the total energy as a function of \( R \), 
			and find a formula for the equilibrium radius in terms of the mass. As the mass increases, does 
			the radius increase or decrease? Does this make sense?
		\item Evaluate the equilibrium radius for \( M = 2 \times 10^{30} \) kg, the mass of the sun. Also 
			evaluate the density. How does the density compare to that of water?
		\item Calculate the Fermi energy and the Fermi temperature, for the case considered in part (d). Discuss
			whether the approximation \( T = 0 \) is valid. 
		\item Suppose instead that the electrons in the white dwarf are highly relativistic. Using the result 
			from the previous problem, show that the total kinetic energy of the electrons is now 
			proportional to \( 1 / R \) instead of \( 1 / R^2 \). Argue that there is no stable 
			equilibrium radius for such a star.
		\item The transition from the nonrelativistic regime to the ultrarelativistic regime occurs 
			approximately where the average kinetic energy of an electron is equal to its rest energy, 
			\( mc^2 \). Is the nonrelativistic approximation valid for a one-solar-mass white dwarf? Above
			what mass would you expect a white dwarf to become relativistic and hence unstable?
	\end{enumerate}

	\pagebreak
	\section*{Schroeder 7.25}
	Use the results of this section to estimate the contribution of conduction electrons to the heat capacity 
	of one mole of copper at room temperature. How does this contribution compare to that of 
	lattice vibrations, assuming that these are not frozen out? (The electronic contribution has been measured
	at low temperatures, and turns out to be about 40\% more than predicted by the free electron model 
	used here.) 
\end{document}
