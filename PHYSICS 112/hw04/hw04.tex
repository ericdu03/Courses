\documentclass[10pt]{article}
\usepackage{../../local}


\newcommand{\classcode}{Physics 112}
\newcommand{\classname}{Introduction to Statistical Mechanics}
\renewcommand{\maketitle}{%
\hrule height4pt
\large{Eric Du \hfill \classcode}
\newline
\large{HW 04} \Large{\hfill \classname \hfill} \large{\today}
\hrule height4pt \vskip .7em
\normalsize
}
\linespread{1.1}
\begin{document}
	\maketitle
	\section*{Problem 1}
	The energy of a two-state paramagnet with $N$ spins $\sigma_i = \pm 1$ is
	\[
		U = -\mu B\sum_{i = 1}^N \sigma_i
	\] 
	Consider two such paramagnets with $N_A = 200$ and $N_B = 100$. 
	\begin{enumerate}[label=\alph*)]
		\item The two paramagnets are initially at temperatures $T_A = \mu B/ k_B$ and $T_B = -\mu B / k_B$
			respectively. Note the latter is at negative temperature! Using the formulas derived in class and 
			lecture, what are $U_A$ and $U_B$?  

			\begin{solution}
				From equation 3.31 in the book, we have 
				\[
				U = -N \mu B \tanh\left( \frac{\mu B}{kT}\right) 
				\] 
				so for the given values of $T_A$ and $T_B$, we have:
				\begin{align*}
					U_A &= -N_A \mu B \tanh \left( \frac{\mu B}{k (\mu B / k)} \right) = -N_A \mu B
					\tanh(1)\\
					U_B &= -N_B \mu_B \tanh\left( \frac{\mu B}{k (-\mu B / k)} \right) = -N_B \mu B 
					\tanh(-1)
				\end{align*}
			\end{solution}
		\item The two paramagenets are brought into contact, and energy flows between them until they reach 
			equilibrium. They are isolated from the environment. Once in equilibrium, what are $U_A, U_B$, 
			$T_A$ and $T_B$? Use Mathematica to obtain a numerical answer for $T$.

			\begin{solution}
				We're given the formula for $T^{-1}$:
				\[
				\frac{1}{T} = \frac{k}{2 \mu B} \ln\left( \frac{N - U /\mu B}{N + U / \mu B} \right) 
				\] 
				At thermal equilibrium, we have $T_A = T_B$, so 
				\begin{align*}
					\ln \left( \frac{N_A - U_A / \mu B}{N_A + U_A / \mu B} \right) &= 
					\ln \left( \frac{N_B - U_B / \mu B}{N_B + U_B/ \mu B} \right) \\
					\frac{\mu B N_A - U_A}{\mu B N_A + U_A} = \frac{\mu B N_B - U_B}{\mu B N_B + U_B}
				\end{align*}	
				Simplifying this expression all the way down gets us 
				\[
					U_B N_A - U_A N_B = 0 \implies \frac{U_B}{U_A} = \frac{N_B}{N_A} = \frac{1}{2}
				\] 
				Therefore, $U_B = 2U_A$. To numerically calculate $T$, we'll need to calculate $U_B$ 
				and $U_A$ first, which we can do using conservation of energy. We know the intiial energy is:
				\[
				U_i = -N_A \mu B \tanh(1) - N_B \mu B \tanh(-1) = (N_B - N_A) \mu B \tanh(1)
				\] 
				We know that the final energy can be written as $U_f = 3U_B$, so:
				\[
				U_f = 3 U_B \implies U_B = \frac{N_B - N_A}{3}\mu B \tanh (1) = -25.38 \mu B
				\] 
				$U_A$ is twice this value, so:
				\[
				U_A = \frac{2(N_B - N_A)}{3}\mu B \tanh(1) = -50.77 \mu B
				\] 
				We can then solve for $T_A$ algebraically by plugging our value for $U_A$ at equilibrium 
				and $N_A$ back into the expression for $U$ (I told Mathematica to simplify this expression 
				and this is what it gave me):
				\[
					T_A = \frac{2 B \mu}{k \ln \left( \frac{200 + \frac{200 \tanh(1)}{3}}{200
					- \frac{200\tanh(1)}{3}}\right) }
				\] 
				Similarly for $T_B$:
				\[
				T_B = \frac{2 B \mu}{k \ln \left( \frac{100 + \frac{100\tanh(1)}{3}}{100
					- \frac{100\tanh(1)}{3}}\right) }
				\] 
				Written in a nicer form, the algebraic
				\footnote{I wouldn't really call it an algebraic expression since we've already done 
					some simplifications already; I'd argue that the initial equation for $\frac{1}{T_A}$ 
				is the \textit{true} algebraic expression.}
				expression for $T_A$ and $T_B$ is:
				\[
				T_A = \frac{3.85 \mu B}{k} \ T_B = \frac{3.85 \mu B}{k}
				\] 
				Numerically, we have:
				\begin{align*}
					T_A &= 2.79 \times 10^{-21} \mu B\\
					T_B &= 2.79 \times 10^{-21} \mu B
				\end{align*}		
				These two values are identical, which makes sense, since from the start we claimed 
				that $T_A = T_B$.
			\end{solution}
		\item What is the change in the entropy $\Delta S_A$, $\Delta S_B$ and $\Delta S_{AB}$ during this 
			process? Is the 2nd law satisfied?

			\begin{solution}
				We can solve this problem by counting the number of microstates available to the 
				paramagnet. We know that initially, we have the following:
				\begin{align*}
					U_A &= -N_A \mu B \tanh(1) = -152.3 \mu B \\
					U_B &= -N_B \mu B \tanh(-1) = 76.15 \mu B
			\end{align*} 
				Since we're dealing with integer number of spins, I'm going to round down for both numbers, 
				so we have:
				\begin{align*}
					U_A &= -152 \\ 
					U_B &= 76
			\end{align*} 
				For magnet $A$, this means that there are 152 spin down states, and 48 spins that cancel each 
				other out. This means that half of the 48 spins are spin up, so there are a total of 24 spin 
				up states in magnet $A$. There are 200 spins in magnet $A$, so:
				\[
					S_A^i = k \ln {200 \choose 24}
				\] 
				Similarly, magnet $B$ has 76 spin up states, so 24 spins cancel, hence 12 spin down states.
				Therefore, the entropy of magnet $B$ is:
				\[
					S_B^i = k \ln {100 \choose 12}
				\] 
				Now recall our values for $U_A$ and $U_B$ at equilibrium:
				\begin{align*}
					U_A^f &= -50.77 \mu B\\
					U_B^f &= -25.38 \mu B 
				\end{align*}
				Again, we need to round these numbers to the nearest integer. Since we need an even number 
				of spins so that they cancel nicely, our rounding is going to be:
				\begin{align*}
					U_A^f &=  -52 \mu B \\
					U_B^f &= -24 \mu B
				\end{align*}
				The reason we round from $-50.77\mu B$ down to $-52\mu B$ is because initially the 
				system starts at a 
				lower energy of $-152 \mu B$ and it gains energy, since $-50 \mu B$ is higher energy 
				than $-50.77 \mu B$, this system cannot have that much energy. Therefore, we're forced
				to round down to $-52 \mu B$. Similarly, this means that system $B$ goes to $-24 \mu B$, 
				since it lost energy, and can't possibly give away more energy since that doesn't 
				contribute to the changing of a spin state in system $A$. 

				We then perform the same analysis as we did with the initial energy: there are 52 spin up
				states, and the remaining cancel 
				out, meaning that there are 74 spin down states, so 
				\[
					S_A^f = k \ln {200 \choose 74}\\
				\] 
				Similarly, for magnet $B$ we have 24 spin up states, so this corresponds to 38 spin down
				states, so 
				\[
					S_B^f = k \ln {100 \choose 38}
				\] 
				Now to calculate the entropy changes (done in Mathematica):
				\begin{align*}
					\Delta S_A &= S_A^f - S_A^i = 8.005 \times 10^{-22} \text{ J/K}\\
					\Delta S_B &= S_B^f - S_B^i = 4.045 \times 10^{-22} \text{ J/K}
				\end{align*}
				Then $\Delta S_{AB} = \Delta S_A + \Delta S_B = 1.205 \times 10^{-21}$ J/K. Clearly, this 
				value is positive, so the second law is satisfied.
			\end{solution}
	\end{enumerate}

	\pagebreak
	\section*{Problem 2}
	A liter of air, initially at room temperature and atmospheric pressure, is heated at constant pressure until 
	it doubles in volume. Calculate the increase in entropy during this process. 

	\begin{solution}
		We begin first by verifying the table. For diatomic gases, since $f = 5$, we expect a $C_P$ value 
		of $\frac{7}{2}R \approx 29.1$. Looking at the table, we get that for oxygen, 
		$C_P = 29.38$ and for Nitrogen, 
		$C_P = 29.12$, exactly as expected. 

		Now solving the problem:  We know that $Q = C_P dT$, and that:
		\[
			(\Delta S)_P = \int_{T_i}^{T_f} \frac{Q}{T} = C_P \int_{T_i}^{T_f} \frac{dT}{T} 
			= C_P \ln \left( \frac{T_f}{T_i} \right) 
		\] 
		So now all we need to do is calculate $T_f$ and $T_i$. This is relatively simple, we can use the 
		ideal gas law:
		\[
		T_i = \frac{P_0V_0}{Nk}, \ T_f = \frac{P_0(2V_0)}{Nk} = 2T_i
		\] 
		Therefore:
		\[
			(\Delta S)_P = C_P \ln \left( \frac{2T_i}{T_i} \right) = C_P \ln 2
		\] 
		This calculation for entropy gives us the change in entropy in one mole of air. We have one liter
		of air, which is $\frac{1}{22.4} = 0.044$ moles of air. (Here I used the fact that the molar volume
		is 22.4L). Therefore, the change in entropy is: 
		\[
			(\Delta S)_P = (29)(0.044)\ln 2 = 0.88 \text{ J/K}
		\] 
	\end{solution}
	\pagebreak
	\section*{Problem 3}
	Consider a monoatomic gas in which each particle experiences a constant potential energy $v$:
	\[
		U = \sum_{i = 1}^{N}\left[ \frac{\mathbf p_i^2}{2m} + v\right]
	\] 
	\begin{enumerate}[label=\alph*)]
		\item What is the entropy $S(U, N, V)$? Your result should be just a small modification of the 
			Sackur-Tetrode equation coming from the potential $v$; you don't need to derive the Sackur-Tetrode
			from scratch.

			\begin{solution}

				With the given potential, we can write this instead as:
				\[
					U_T = U_{\text{free}} + Nv
				\] 
				so rearranging, $U_{\text{free}} = U_T - Nv$. Since the free energy is really what the 
				Sackur-Tetrode equation uses as its value for $U$, then we can just substitute 
				this in for our entropy:
				\[
					S = Nk\left[ \ln\left( \frac{V}{N}\left( \frac{4 \pi m(U - Nv)}{3Nh^2} \right)^{3 / 2}\right) + \frac{5}{2} \right] 
				\] 
			\end{solution}
		\item What is $U(T, N, V)$?

			\begin{solution}
				We know that $U_{\text{free}} = \frac{3}{2}NkT$, so therefore:
				\[
				U(T, N, V) = \frac{3}{2}NkT - Nv
				\] 
			\end{solution}
		\item What is $P(T, N, V)$?

			\begin{solution}
				We have the relation $P = T\left( \pdv{S}{V} \right)_{U, N}$ so all we need to do 
				is take the derivative of $S$ with respect to $V$. Note that in the equation for $S$, there is 
				only one $V$ term, so once we split the logarithm only the $\frac{V}{N}$ term matters.
				Therefore, taking the derivative:
				\begin{align*}
					P &= T \pdv{v} \left[ Nk\left( \ln \left( 
					\frac{V}{N}\left( \frac{4 \pi m(U - Nv)}{3Nh^2} \right)^{3 / 2} \right) +
			\frac{5}{2} \right)  \right] \\
			&= NkT\left( \frac{1}{V} \right)
			\end{align*}
			Therefore, we have:  
			\[
				P(T, N, V) = \frac{NkT}{V}	
			\] 
			\end{solution}
		\item Prove that 
			\[
			\mu(N, V, T) = -k_B T \ln\left( \frac{V}{N \lambda_T^3} \right) + v
			\] 
			where $\lambda_T = 1 / \sqrt{2\pi m k_BT / h^2}$.

			\begin{solution}
				This problem has a lot of algebra so I won't bother going through it all, though I'll 
				highlight the major steps. We know that 
				to calculate the chemical potential, 
				\[
					\frac{\mu}{T} = - \pdv{S}{N}
				\] 
				Therefore, we have to take the derivative of $S$ with respect to $N$. Here comes the algebra:
				\begin{align*}
					-\frac{\mu}{T} &= \dv{N} \left[ Nk\left( \ln \left( \frac{V}{N} \right) + \frac{3}{2}
					\ln \left( \frac{4 \pi m (U - Nv)}{3Nh^2} \right)+ \frac{5}{2}\right) \right]\\
								   &= k \left[ \ln \left( \frac{V}{N} \right) +
								   \frac{3}{2}\left( \frac{4 \pi m(U - Nv)}{3Nh^2} \right) +
							   \frac{5}{2}\right] + 
								   Nk \left[ -\frac{1}{N} +
								   \frac{3}{2}\frac{-4 \pi m v}{4\pi m(U - Nv)} - \frac{3}{2}\frac{1}{3Nh^2}(3h^2) \right]  \\
								   &= k \ln \left( \frac{V}{N}\left( \frac{4 \pi m(U - Nv)}{3Nh^2} \right)^{3 / 2} + \frac{5}{2} \right) - Nk\left( \frac{1}{N} + \frac{3}{2}\frac{v}{U - Nv} + \frac{3}{2N} \right)
				\end{align*}
				I skipped quite a bit of from the second to third step, but I also can't be bothered to 
				write out all the algebra. From here, we can substitute $U= \frac{3}{2}NkT + Nv$
				\footnote{I'm omitting the $T$ subscript I used in 
				part (a), but it's the same $U$ I'm referring to}
				into both places, eventually getting us:
				\[
					-\frac{\mu}{T} = k \ln \left( \frac{V}{N}\left( \frac{4 \pi m}{3 N h^2} \frac{3}{2}NkT \right)^{3 / 2} \right) - \frac{3}{2}\frac{Nkv}{\frac{3}{2}NkT}
				\] 
				From here we multiply the $T$ and the negative sign over, and we get:
				\[
					\mu = -kT \ln \left( \frac{V}{N}\left( \frac{2 \pi m kT}{h^2} \right)^{3 / 2} \right) + v
				\] 
				We now recognize that everything inside the logarithm beside the $\frac{N}{V}$ is our $\lambda$
				term, so we finally get the desired equation:
				\[
				\mu(N, V, T)= -kT \ln \left( \frac{V}{N \lambda_T^3} \right) + v
				\] 
			\end{solution}
		\item Two boxes of gas with differing potentials $v_A \neq v_B$ are broiught into contact. Particles 
			and number flow to achieve chemical and thermal equilibrium, with their volumes $V_A, V_B$ fixed.
			They equilibriate to density $n_A = \frac{N_A}{V_A}$ and $\frac{N_B}{V_B}$ and temperature $T$.
			Find the expression for $\frac{n_A}{n_B}$ that depends on $v_A - v_B$ and $T$. 

			\begin{solution}
				We have the following expressions for the chemical potentials:
				\begin{align*}
					\mu_A &=  -kT_A\ln \left( \frac{V_A}{N_A \lambda_{T_A}^3} \right) + v_A \\
					\mu_B &= -kT_B\ln \left( \frac{V_B}{N_B \lambda_{T_B}^3} \right) + v_B 
				\end{align*}
				Our end goal is to set these two equal to each other to find an expression for $n_A$ and $n_B$.
				Note that at thermal equilibrium, $T_A = T_B = T$, and since $\lambda$ is only dependent on $T$,
				then $\lambda_{T_A} = \lambda_{T_B} = \lambda$. Now we set these equal to each other:
				\[
					kT \ln\left( \frac{V_A}{N_A \lambda^3}\right) - v_A - kT\ln\left( \frac{V_B}{N_B \lambda^3} \right)  + v_B = 0
				\]
				Simplify this by pulling $v_A$ and $v_B$ to one side and combine the logarithm:
				\begin{align*}
					kT \ln \left( \frac{V_A / (N_A \lambda^3)}{V_B / (N_B \lambda^3)} \right) &= v_A - v_B\\
					\ln \left( \frac{n_B}{n_A} \right) &= \frac{1}{kT}(v_A - v_B) \\
					\therefore \frac{n_A}{n_B} &= e^{-(v_A - v_B) / kT}
				\end{align*}
			\end{solution}
		\item How can this problem be used to give an alternative explanation to Schroder 1.16c from HW1? 

			\begin{solution}
				Problem 1.16c was discussing the relationship between the pressure and altitude, assuming
				that $T$ remained constant throughout the Earth's atmosphere. In that problem we 
				reasoned the equation for pressure by considering the volume of air at height $z$ and $z + dz$.

				Notice that in these two volumes of air, the particles have different
				potentials, just like the system described in part (e). Furthermore, notice that 
				the pressure of a gas is proportional to the density via the relationship:
				\[
					P = \frac{NkT}{V} = nkT
				\] 
				where $n$ is the density. Since $T$ is constant, $P \propto n$. Therefore, up to 
				some constants, we can write:
				\[
					\frac{P_A}{P_B} = e^{-(v_A - v_B) /kT} \implies P_A = P_B e^{-(v_A - v_B) / kT}
				\] 
				If we let $P_B$ be the lower of the two volumes, then this final equation gives us a 
				relationship between the pressure at a higher altitude in terms of the pressure 
				at a lower altitude. This matches exactly the equation we derived in 1.16c:
				\[
					P(z) = P(0) e^{-mgz / kT}
				\] 
				where the pressure at a height $z$ is related to the pressure at sea level $P(0)$. Note 
				that the potentials also match, since $v_A - v_B$ is the difference in potential energy 
				and if $B$ is at sea level then $v_B = 0$ so $v_A - v_B = mgz$, exactly as in the 
				equation for $P(z)$. 
			\end{solution}
	\end{enumerate}

\end{document}
