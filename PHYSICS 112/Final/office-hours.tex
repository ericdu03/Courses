\documentclass[10pt]{article}
\usepackage{../../local}
\urlstyle{same}

\newcommand{\classcode}{Physics 112}
\newcommand{\classname}{Introduction to Statistical and Thermal Physics}
\renewcommand{\maketitle}{%
\hrule height4pt
\large{Eric Du \hfill \classcode}
\newline
\large{Office Hours} \Large{\hfill \classname \hfill} \large{\today}
\hrule height4pt \vskip .7em
\small{Header styling inspired by CS 70: \url{https://www.eecs70.org/}}
\normalsize
}
\linespread{1.1}
\begin{document}
	\maketitle
	\begin{itemize}
		\item Study the occupancy stuff (energies larger than \( n_F \) go to zero).  
		\item Bose-Einstein Condensation: solve the number integral, then say that the number of excited particles is restricted 
			the max value given by the number integral. Then, the remainder of the particles must be in the ground state (we 
			get BEC from that.) 
		\item The issue arises in the fact that there are a large number of particles in the ground state, hence 
			\( g(\epsilon) \) isn't smooth, so we can't replace this with an integral. Therefore, only the number of particles 
			in the excited states can be transferred to an integral equation:
			\[
				N = \sum_j n_B(\epsilon_j - \mu) = \underbrace{n_B(\epsilon_0 - \mu)}_{N_0} + 
				\underbrace{\sum_{j > 0} n_B(\epsilon_j - \mu)}_{N_{\text{ex}}} = N_0 + \int_0^{\infty}g(\epsilon)
				n_B(\epsilon - \mu) d\epsilon
			\] 
			(The \( N_0 \) here is the source of the Bose-Einstein condensation. The proportion of particles between \( 0 \) 
			and \( \epsilon_1 \) goes to 0 in the thermodynamic limit, so this 
			approximation as an integral on \( [0, \infty) \) is good enough. 
		\item Stefan-Boltzmann Law: sphere of temperature \( T \) and radius \( R \), how much power does it radiate?

			\[
			P = (\text{surface area}) \cdot T^4 \cdot \sigma_B = 4\pi R^2 T^{4} \sigma_B 
			\] 
			\( \sigma_B \) is the Stefan-Boltzmann constant.
		\item Given \( g(\epsilon) \) (single particle density of states) and a temperature \( T \), how do we calculate the 
			occupancy?
		\item Relation between \( g(\epsilon) \) and momentum space:
	
			Given \( \epsilon(k) = \frac{\hbar^2 k^2}{2m} \), we can find the number of particles:
			\[
				N = \sum_{\vec k} n_{F/B}(\epsilon_k - \mu)
			\] 
			We converted this then to an integral and calculated \( N \). There's another way to calculate this:
			\[
			N = \int d\epsilon g(\epsilon) n_{F / B}(\epsilon - \mu)
			\] 
			where \( g(\epsilon)  \) is given by the number of single-particle with energy within the interval 
			\( [\epsilon, \epsilon + d\epsilon] \). For \( k^2 \) dispersion, read book to figure out \( g(\epsilon) \). 
			In lecture we did:
			\[
			N = \sum_{\vec k} n_{F / B}(\epsilon_k - \mu) = \left( \frac{L}{2\pi} \right) ^3 \int d^3k \cdot n
			_{F / B}(\epsilon_k - \mu)
			\] 
			We can replace this sum with an integral when the function is "smooth" around \( \epsilon_F \).  
		\item Learn how to do this: given a density of states, calculate the total number of particles in the system.  
			Is the solution to literally do the integral?
		\item Dispersion relation relates \( \epsilon \) to \( k \). 
	\end{itemize}	
\end{document}
