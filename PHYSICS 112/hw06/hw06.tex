\documentclass[10pt]{article}
\usepackage{../../local}
\newcommand{\classcode}{Physics 112}
\newcommand{\classname}{Introduction to Statistical Mechanics}
\renewcommand{\maketitle}{
	\hrule height4pt
	\large {Eric Du \hfill \classcode}
	\newline
	\large{HW 06} \Large{\hfill \classname \hfill} \large{\today}
	\hrule height4pt \vskip .7em
	\normalsize
}
\linespread{1.1}
\begin{document}
	\maketitle
	\section*{Collaborators}
	I worked with \textbf{Andrew Binder} and \textbf{Adarsh Iyer} on this assignment. asdjf;alsdfj kdsfj 
	\section*{Schroeder 5.6}
	A muscle can be thought of as a fuel cell, producing work from the metabolism of glucose:
	\[
		\ch{C6H12O6} + 6\ch{O2} \to 6\ch{CO2} + 6\ch{H2O}
	\]
	\begin{enumerate}[label=\alph*)]
		\item Use the data at the back of this book to determine the values of \( \Delta H \) and \( \Delta G \)
			for this reaction, for one mole of glucose. Assume that the reaction takes place at 
			room temperature and atmospheric pressure.

			\begin{solution}
				Looking at the back of the book, we have the following values:
				\begin{center}
					\begin{tabular}{c|c|c}
						Substance & \(\Delta H\) (kJ) & \(\Delta G\) (kJ)\\
						\hline
						\ch{C6H12O6} & -1275 & -910 \\
						6\ch{O2} & 0 & 0\\
						6\ch{CO2} & -2361.06 &-2366.16\\ 
						6\ch{H2O} & -1714.04 & -1422.78
					\end{tabular}
				\end{center}
				So to find $\Delta H$ of the entire reaction, we take \(\Delta H\) on the right and 
				subtract it from \(\Delta H \) on the left. This gives us \(\Delta H = -2803.04\) kJ/mol. We 
				can calculate $\Delta G$ using the same process, giving us $\Delta G = -2878.94$ kJ/ mol
			\end{solution}
		\item What is the maximum amount of work that a muscle can perform, for each mole of 
			glucose consumed, assuming ideal operation? 

			\begin{solution}
				Since $\Delta G$ is the quantity that takes into account the energy supplied by the
				environment, then assuming ideal operation, the muscle can perform 2879 kJ of work per mole 
				of glucose. 
			\end{solution}
		\item Still assuming ideal operation, how much heat is absorbed or expelled by the chemicals 
			during the metabolism of a mole of glucose? (Be sure to say which direction the heat flows.)

			\begin{solution}
				Recall that $G = H - TS$, meaning that $TS = H - G = 75.9$ kJ, and the positive sign indicates
				that we need to supply 75.9 kJ of energy, meaning that heat flows into the system.
			\end{solution}
		\item Use the concept of entropy to explain why the heat flows in the direction it does. 

			\begin{solution}
				Looking at the back of the book again, we have:
				\begin{center}
					\begin{tabular}{c|c}
						Substance & \(S\) \\
						\hline 
						\ch{C6H12O6} & 212\\
						6\ch{O2} & 1230.84\\
						6\ch{CO2} & 1282.44\\
						6\ch{H2O} & 1131.78
					\end{tabular}
				\end{center}
				So calculating the entropy of the reactants and products:
				\begin{align*}
					\Delta S_{\text{reactants}} &= 212 + 1230 = 1442.84 \\
					\Delta S_{\text{products}} &= 
					1282.44 + 1131.78 = 2414.22
				\end{align*} 
				Since the entropy of the products is higher than that of the reactants, this means that 
				heat must enter the system to raise the entropy of the reactants to a point where the reaction 
				can proceed, explaining why heat flows into the system.
			\end{solution}
		\item How would your answers to parts (b) and (c) change, if the operation of the 
			muscle is not ideal?

			\begin{solution}
				If the operation of the muscle is not ideal, then we wouldn't be able to use all the energy 
				that is produced via this reaction, so the values in part (b) would be lower. As a result 
				of this, some of the energy generated from the reaction will be lost to the environment 
				as heat, meaning that our answer to part (c) would also be lower since there we calculated the 
				net heat. 
			\end{solution}
	\end{enumerate}	
	\pagebreak

	\section*{Schroeder 5.7}
	The metabolism of a glucose molecule (see previous problem) occurs in many steps, resulting in the 
	synthesis of 39 molecules of ATP (adenosine triphosphate) out of ADP (adenosine diphosphate) and 
	phosphate ions. When the ATP splits back into ADP and phosphate, it liberates energy that is used in 
	a host of important processes including protein synthesis, active transport of molecules across cell 
	membranes, and muscle contraction. In a muscle, the reaction $\text{ATP} \to \text{ADP} + \text{phosphate}$
	is catalyzed by an enzyme called myosin that is attached to a muscle filament, causing the muscle to 
	contract. The force it exerts averages about 4 piconewtons and acts over a distance of about 11 nm. From 
	this data and the results of the previos problem, compute the ``efficiency'' of a muscle, that is, the 
	ratio of the actual work done to the maximum work that the laws of thermodynamics would allow. 

	\begin{solution}
		First, let's calculate the work done on average, which we can use $W = F \Delta x$ to get 
		$W = (4 \times 10^{-12})( 11 \times 10^{-9}) = 4.4 \times 10^{-20}$ joules. This means that 
		with one mole of this process, we get: $4.4 \times 10^{-20} \times 6.02 \times 10^{23} =  26.5$ kJ/mol. 

		By the problem statement, we see that a single mole of glucose gives us 39 moles of ATP, so 
		this means that per mole of glucose, we are doing 1033.5 kJ of work. Then, to compute the 
		efficiency, we divide this value by the maximum amount of work calculated in part (b) of the previous 
		problem, which gives us:
		\[
		\epsilon = \frac{1033.5}{2878.94} = 0.358 = 35.8 \%
		\] 
		So we get that the efficiency is about 35.8\%. 
	\end{solution}
	\pagebreak

	\section*{Schroeder 5.23}

	By subtracting $\mu N$ from $U, H, F$, or $G$, one can obtain four new thermodynamic potentials. Of the 
	four, the most useful is the \textbf{grand free energy} (or \textbf{grand potential})
	\[
		\Phi \equiv U - TS - \mu N
	.\]
	\begin{enumerate}[label=\alph*)]
		\item Derive the thermodynamic identity for $\Phi$ and the related formulas for the partial 
			derivatives of $\Phi$ with respect to $T, V$ and $\mu$.

			\begin{solution}
				We follow the approach we used in class, we first take a differential element $d \Phi$:
				\[
				d\Phi = dU - T dS - S dT - \mu dN 
				\] 
				Now we use the first law: $dU = T dS - PdV + \mu dN + N d \mu$:
				\[
				d\Phi = (T dS - PdV + \mu dN) - T dS - S dT - \mu dN = -PdV - SdT - N d \mu
				\]
				So we can get the identities:
				\begin{align*}
					\left( \pdv{\Phi}{V} \right)_{T,\mu} &= -P\\
					\left( \pdv{\Phi}{T} \right)_{V, \mu} &= -S\\
					\left( \dv{\Phi}{\mu} \right)_{T, V} &= -N
				\end{align*}
			\end{solution}
		\item Prove that, for a system in thermal and diffusive equilibrium (with a reservoir that can 
			supply both energy and particles), $\Phi$ tends to decrease. 

			\begin{solution}
				We'll follow the same approach as what was done in the textbook. Recall that 
				$dS$ can be calculated as:
				\[
				dS = \frac{1}{T}dU + \frac{P}{T}\ dV - \frac{\mu}{T}\ dN
				\] 
				Now consider a system $R$ and $E$ which are allowed to interact with one another, ensuring 
				thermal and diffusive equilibrium (so $T$ and $\mu$ are the same between both systems, which is
				why we're not subscripting them). Then, the total entropy is expressed as a sum of the 
				two individual entropies:
				\[
					dS_{\text{tot}} = dS_E + dS_R
				\] 
				In terms of the system $R$, then we can write, 
				\[
					dS_{R} = \frac{1}{T}\ dU_R - \frac{\mu}{T}\ dN_R
				\] 
				Now, in terms of the larger system, \(U\) and \(N\) flowing out of $R$ enters $E$, so in terms 
				of system $E$, we can write:
				\[
				dS_R = -\frac{1}{T}\ dU_E + \frac{\mu}{T }\ dN_E
				\] 
				Therefore, in total,
				\[
					dS_{\text{tot}} = dS_E - \frac{1}{T} dU_E + \frac{\mu}{T}dN_E = \frac{1}{T}(T dS_E - dU_E + 
					\mu dN_E)
				\] 
				Now, consider $\Phi$ under diffusive and thermal equilibrium. This means that $d \mu = 0$ and 
				$dN = 0$, so from the expression we got from part (a) with the differentials, we find that 
				\[
				d\Phi_E = dU_E - T dS_E - \mu dN_E
				\] 
				which is exactly what we have in parentheses in $dS_{\text{tot}}$. Therefore, we have:
				\[
					dS_{\text{tot}} = -\frac{1}{T}d\Phi_E
				\] 
				And since $S_{\text{tot}}$ tends to increase, then it must be the case that $\Phi$ decreases
				as a result. Note that while we have the subscript in $\Phi_E$, this is the same 
				as calculating $\Phi$ for any given system, since we really haven't specified anything about 
				system $E$ besides the fact that it's in thermal and diffusive equilibrium.
			\end{solution}
		\item Prove that $\Phi = -PV$. 

			\begin{solution}
				Note that $\Phi = U - TS - \mu N$, and we have from the textbook that $G = \mu N$, so therefore:
				\[
				\Phi = U - TS - (U - TS + PV) = -PV
				\] 
				and we're done.
			\end{solution}
		\item As a simple application, let the system be a single proton, which can be ``occupied" either 
			by a single electron (making a hydrogen atom, with energy -13.6 eV) or by none (with energy 
			zero). Neglect the excited states of the atom and the two spin states of the electron, so that 
			both the occupied and occupied states have zero entropy. Suppose that this proton is in the 
			atmosphere of the sun, a reservoir of temperature 5800 K and an electron concentration of 
			about $2 \times 10^{19}$ per cubic meter. Calculate \( \Phi\) for both the occupied and 
			unoccupied states, to determine which is more stable under these conditions. To compute the chemical
			potential of the electrons, treat them as an ideal gas. At about what 
			temperature would the occupied and unoccupied states be equally stable, for this value of the 
			electron concentration? (As in Problem 5.20, the prediction for such a system is only 
			a probabilistic one.)

			\begin{solution}
				First, let's start with the unoccupied state. Here, $S = 0$, $N = 0$ since we have no electron, 
				and $U = 0$ also because we don't have an electron. Therefore, all three terms 
				in $\Phi$ go to zero, so $\Phi = 0$ in the unoccupied state. 

				Now for the occupied state, we have $U = -13.6$ eV, $T = 5800$ K, $S = 0$, and $N = 1$. To 
				calculate $\mu$, we use the relation
				\[
					\mu= -T \pdv{S}{N}
				\] 
				so we have to find $\pdv{S}{N}$ first. To do this, we take the direction of the problem 
				statement and treat this system as an ideal gas, under which the entropy can be 
				expressed by the Sackur-Tetrode equation:
				\[
					S = Nk\left[ \ln\left( \frac{V}{N}\left( \frac{4\pi m U }{3Nh^2} \right)^{3 / 2} \right) + \frac{5}{2} \right] 
				\] 
				To make the differentiation with respect to $N$ easier, we can simplify this to:
				\[
					S = Nk\left[ \ln\left( V\left( \frac{4 \pi m U}{3h^2} \right)^{3 / 2} \right) 
					- \frac{5}{2} \ln N + \frac{5}{2}\right] 	
				\] 
				Therefore, this gives us: =
				\[
					\pdv{S}{N} = k\left[ \ln\left( V\left( \frac{4 \pi m U}{3h^2} \right)^{3 / 2} \right) 
					- \frac{5}{2} \ln N + \frac{5}{2}\right] + Nk \left( -\frac{5}{2}\frac{1}{N} \right) 
				\] 
				which, after some simplification, gets us:
				\[
					\pdv{S}{N} = k \ln \left( \frac{V}{N}\left( \frac{4 \pi m U}{3Nh^2} \right)^{3 / 2} \right) 
				\] 
				As a final point of simplification, we can use the fact that $U = \frac{3}{2}NkT$ to further 
				simplify the argument inside the log:
				\[
					\pdv{S}{N} = k\ln \left( \frac{V}{N} \left( \frac{2 \pi m k T}{h^2} \right)^{3/2} \right) 
				\] 
				Finally, we can calculate $\mu$:
				\[
					\mu = -T \pdv{S}{N} = -kT\ln\left( \frac{V}{N}\left( \frac{2 \pi m kT}{h^2} \right)^{3 / 2} \right) 
				\] 
				Plugging these values into a calculator, this gives us $\mu = 8.863$ eV. This means that for 
				the unoccupied state, we have:
				\[
					\Phi = -13.6 \text{ eV} + 8.863 \text{ eV} = -4.737 \text{ eV}
				\] 
				Now for the last part of the problem: calculating stability comes down to finding the 
				temperature at which $\Phi$ for the occupied state is equal to that of the unoccupied state. 
				Since $\Phi$ for the unoccupied state is always 0, then this just comes down to finding 
				when $\Phi$ for the occupied state approaches zero. Plugging this into Mathematica, we find 
				that this is reached at a temperature of: 
				\[
					T \approx 8559.16 \text{ K}
				\] 
				aaa
			\end{solution}
	\end{enumerate}
\end{document}
