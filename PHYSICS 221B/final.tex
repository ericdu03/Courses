\documentclass[10pt]{article}
\usepackage{../local}
\urlstyle{same}

\newcommand{\classcode}{Physics 221B}
\newcommand{\classname}{Quantum Mechanics II}
\renewcommand{\maketitle}{%
\hrule height4pt
\large{Eric Du \hfill \classcode}
\newline
\large{Final Exam} \Large{\hfill \classname \hfill} \large{\today}
\hrule height4pt \vskip .7em
\small{Header styling inspired by CS 70: \url{https://www.eecs70.org/}}
\normalsize
}
\linespread{1.1}
\begin{document}
	\maketitle
	\section*{Problem 1}
	We are given the Lagrangian density of the Dirac field 
	\[
		\mathcal L(\psi, \psi_{, \mu}) = \frac{i}{2} \overline \psi 
		\gamma ^{\mu}\psi_{, \mu} - \frac{i}{2}\overline\ \psi_{, \mu} 
		\psi - m \overline \psi \psi
	\] 
	\begin{enumerate}[label=\alph*)]
		\item Show that \( \mathcal L \) is not invariant under local 
			\( U(1) \) gauge transformations \( \psi \to \psi' = 
			e^{i \lambda(x)}\psi\). 
		\item In \( \mathcal L \), replace the ordinary derivatives \( \psi_{, \mu} = \partial_\mu \psi \) by 
			\textit{covariant derivatives} \( \psi_{; \mu} = D_\mu \psi = (\partial_\mu + i gA_\mu) \psi \), 
			with a new vector field \( A_\mu(x) \) and a constant \( g \). Find a condition on 
			\( A_\mu \) so that the new Lagrangian density \( \mathcal L' \) is invariant under local gauge 
			transformations. 
		\item Derive the Dirac equation from the new Lagrangian \( \mathcal L' \). Waht is the interpretation of 
			\( A_\mu \)? 
	\end{enumerate}
	\pagebreak
	\section*{Problem 2}
	Calculate the propagator \( \Delta(x) = \int \frac{d^{3}k}{(2\pi)^3} \frac{1}{\omega_k} \sin(kx) \) for both 
	spacelike and timelike \( x \) explicitly in terms of Bessel functions. 
	\pagebreak
	\section*{Problem 3}
	A particle of mass \( m \) and energy \( E \) scatters at a scattering center. At scattering resonance, 
	one of the partial amplitudes has a maximum. State the scattering cross section \( \sigma \) at that 
	\( E \), assuming that the other partial amplitudes are neglegible. 
	\pagebreak
	\section*{Problem 4}
	The exchange integral (\( e = \hbar = 1 \)) 
	\[
	K_{10}^{nl} = \int d^3r_1 d^3r_2 \frac{\psi_{100}^{*}(\mathbf r_1) \psi_{nl 0}^{*}(\mathbf r_2)
	\psi_{100}(\mathbf r_2) \psi_{nl0}(\mathbf r_1)}{|\mathbf r_1 - \mathbf r_2|}
	\] 
	where \( \psi_{nlm} \) is the usual hydrogenlike wave function, is responsible for the energy difference 
	between the ortho- and para-helium states. 
	\begin{enumerate}[label=\alph*)]
		\item Expand \( 1 / |\mathbf r_1 - \mathbf r_2| \) in terms of spherical harmonics and express \( K_{10}^{nl} \) 
			as a purely radial integral \( K_{10}^{nl} \int dr_1 r_1^2 \int dr_2 r_2^2 \dots \). Denote \( R_{nl} \) 
			are the radial functions corresponding to \( \psi_{nlm} \), \( r_> = \max(r_1, r_2) \), and 
			\( r_< = \min(r_1, r_2)  \). 
		\item For \( l = n -1 \), argue why \( K \) can't be negative. Hint: How many zeros do the Radial functions 
			have? 
	\end{enumerate}
	\pagebreak
	\section*{Problem 5}
	A charged, spinless particle with mass \( m \) (such as an ion) is trapped in a 3-D harmonic 
	potential \( V = \frac{1}{2}m\omega^2 r^2 \), where \( r \) is the distance from the center. Denote 
	\( \ket*{l, m, n} \) the energy eigenstates. The electromagnetic field is initially in the vacuum state. Give 
	an expression for the decay time constant of the state \( \ket*{1, 0, 0} \) into the state \( \ket*{0, 0, 0} \). 
	Please don't evaluate any matrix elements explicitly. Instead, state only whether they are zero, linear 
	in the electromagnetic field operator \( \mathbf A \), or quadratic. 
\end{document}
