\documentclass[10pt]{article}
\usepackage{local}


\linespread{1.1}
\begin{document}

    \section*{Discussion Section Time}
    My discussion time is with Jet Situ in Hearst Mining 310 on Wednesday and Friday 6-7pm.

    \section*{Discussion 3A: Modular Inverses}

    I'd ask students to take a moment and digest the problem statement. Specifically, the phrase that if $ax \equiv 1 \pmod m$ then $x$ is the modular inverse of $x$ modulo $m$. 
    \begin{itemize}[label={--}]
        \item Parts (a) and (b) familiarize students them with modular inverses using examples. Here, $3 \cdot 5 = 15 \not \equiv 1 \pmod{10}$, so $3$ is not an inverse of 5 mod 14. However, since $15 \equiv 1 \pmod{14}$, then 3 is an inverse of 5 mod 14.
        \item Part (c) is a natural extension of this concept, illustrating that we can multiply them together and then simplify the expression to see if we get 1. I'd first ask students to take a look again at how we verified inverses in the previous two parts, and see if we can generalize that process here.
        \item  To start part (d), I'd encourage students to write down an arithmetic equation that represents the modular equation, then arrive at a contradiction. Once this is solved, I want to encourage students to look back on the proof, and notice that we haven't used anything else besides the fact that $\gcd(a, m) \neq 1$ in the proof, meaning that our conclusion can be made much more general: if $\gcd(a, m) \neq 1$, then $a$ does not have an inverse modulo $m$. This is also the statement at the end of Theorem 6.2 in Note 6.
        \item For part (e), I'd encourage students to set up the equation involving $x$ and $x'$, and look at what operations we can do to combine these two equations to arrive at the fact that $x \equiv x' \pmod m$.
    \end{itemize} 


    \pagebreak

    \section*{Discussion 3B: Baby Fermat}


    \begin{enumerate}[label=(\alph*)]
        \item I'd ask students to consider what the pigeonhole principle is really saying, hopefully getting them to realize since the sequence is infinite and there are only $m$ possible values modulo $m$, that there must be repetitions.
        \item The first guidance I'd give is to notice that normally we'd write $a^{i-j} = a^i/a^j$, so I'd ask how do we achieve this in modular arithmetic. Here, the emphasis should be on the fact that $a^j$ is multiplied by $a^*$, that the exponent of $a^j$ is reduced, analogous to how it normally would be under division. Once we realize this then we can realize that if we multiply both sides by $(a^*)^j$ times, then the equation simplifies perfectly into $a^{i -j} \equiv 1 \pmod m$.
        \item This part just relies on noticing that we can take one from the exponent and write it out explicitly, giving us that $a^{i - j - 1}$ is the inverse of $a$ modulo $m$. The only guidance I can think of here is to ask students how else could they represent $a^{i-j}$ as the product of two powers of $a$ which look similar in form to the equation $ax \equiv 1 \pmod m$, hopefully leading them to realize this idea of bringing down a factor of $a$.
    \end{enumerate}





    

\end{document}