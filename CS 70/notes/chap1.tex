\chapter{Introduction to Sets}
\section{Introduction}
One thing mathematicians love doing is finding similarities between different objects, then 
grouping these objects based on their common properties. It is precisely this idea of clasification that 
gives rise to set theory, a field so important that without it many fields of mathematics 
would (quite literally) collapse. 

What use is set theory to computer science? Well, the idea of classifying things based on their 
similarities is such a natural habit that we do it here as well. For instance, consider the 
famous P vs NP problem: the entire premise of this problem relies on the notion that we can 
classify problems based on whether they're efficiently solvable -- that's an act of classification, 
and hence we need set theory here.\footnote{Math majors will probably hate me for calling this an application of 
set theory, but it \textit{somewhat works}, so please just let it slide.}

The basic definition of a set is as follows:
\begin{definition}{Set}{}
	A \textit{set} is a well-defined collection of objects. 
\end{definition}
A couple things about this definition here: firstly, there really is no restriction on what the ``objects'', 
also more commonly called the ``elements'' of a set, referred
to in this definition can be: it can be numbers, variables, problem statements (as with the P vs. NP 
example), and even other sets.\footnote{One should be careful about this last point since it's caused 
a lot of headache for mathematicians over the past 200 years.} 

Secondly, the notion of what it means to be ``well-defined" for our purposes 
is that there is a clear way to tell whether an element 
belongs or doesn't belong to a set.\footnote{If you want a more rigorous definition 
of what it means to be well-defined, this book really isn't for you.} As an example, the set of even numbers is 
well defined, since given a number \( x \), we know it belongs to the set if \( x \) is even, and doesn't belong 
if \( x \) is odd. The following box describes how we would talk about set membership mathematically. 

\begin{notation}{Membership}{}
	If an element \( x \) is a member of a set \( S \), then we write \( x \in S \). If \( x \) is not a 
	member of \( S \), then we write \( x \not\in S \). 
\end{notation}
\section{Describing Sets}
Now that we know what a set is, what are some ways we can write them down? The easiest and perhaps the most 
obvious one is to just list out the elements in the set one by one. If we wanted to write down the set of 
numbers between 1 and 10 inclusive, then we could write:
\[
S = \{1, 2, 3, 4, 5, 6, 7, 8, 9, 10\} 
\] 
This is certainly a valid way of writing this set, but you can imagine that if you were to write down the set 
of numbers from 1 to 1000 like this, it would take forever. So, there is indeed a better way to write this: what we 
do instead is we can define \( S \) as belonging to a larger set of items, but with certain conditions. In this 
way, the set \( S \) can be rewritten as:
\begin{equation}
	\label{setbuild}
S = \{x \in \Z \mid 1 \le x \le 10\} 
\end{equation} 
In English, we read this as: ``\( x \) belongs to the set of integers
such that \( x \) is between 1 and 10". Notice how in this 
way, we've defined \( x \) to first belong to the set of integers (a larger set of numbers), but then 
we added the restriction that \( x \) is between 1 and 10 to construct our set \( S \). Note that we're also 
not restricted by the number of conditions we can put, as long as all restrictions are satisfied by every 
element in the set. For instance, if we wanted to describe the set of numbers between 1 and 10 (inclusive) except 5, 
then we could write:
\[
S = \{x \in \Z \mid 1 \le x\le 10, x \neq 5\} = \{1, 2, 3, 4, 6, 7, 8, 9, 10\} 
\] 
In English, we read this as ``\( x \) belongs to the set of integers such that \( x \) is between 1 and 10, 
and \( x \) is not equal to 5." Notice that all the elements in the set satisfy both restrictions: they're 
between 1 and 10, and they're not equal to 5. 
\subsection{Notable Sets}
There are some sets that are so commonly used throughout the rest of this course that it's in your best 
interest to just go ahead and memorize them: 
\begin{itemize}
	\item \textbf{Natural Numbers:} Denoted by \(\mathbb N =  \{0, 1, 2, \dots \}  \)  
	\item \textbf{Integers:} Denoted by \( \Z = \{\dots, -2, -1, 0, 1, 2, \dots\}  \)
	\item \textbf{Rationals:} fractional numbers, denoted by \( \mathbb Q = \{\frac{x}{y} | y \in \Z, 
		y \neq 0 \}  \) 
	\item \textbf{Real Numbers}: any (potentially infinite) decimal number, denoted by \( \R \) 
	\item \textbf{Complex Numbers:} Denoted by \( \C = \{a + bi \mid a, b \in \R\} \)
	\item \textbf{Empty set:} The set that contains nothing, denoted by \( \varnothing \)\footnote{You might 
			be wondering: didn't we just say 
			that a set is defined to have elements in it? Doesn't an empty set contradict that very statement? 
		Short answer: don't worry about it. Long answer: go read a set theory textbook.} 
\end{itemize}
\section{Subsets}
When we were describing a set, I implicitly introduced the concept of describing a set \( S \) as being 
\textit{part of} another larger set. Rigorously, what this means is that the elements contained in \( S \) 
can be found within a larger set, in which case \( S \) would be considered a subset of that larger set. 
\begin{definition}{Subset}{}
	Given two sets \( A \) and \( B \), if every element of \( A \) is also a member of \( B \), then 
	\( A \) is a \textit{subset} of \( B \), which we write as 
	\( A \subseteq B \). 
\end{definition}
Sometimes, there is also the subtle distinction of whether \( B \) contains elements not contained in \( A \). If 
this is the case, then we sometimes call \( A \) a \textit{proper subset} of \( B \):
\begin{definition}{Proper Subset}{}
	Given two sets \( A \) and \( B \), if every element of \( A \) is also a member of \( B \) and \( B \) 
	contains elements not contained in \( A \), then \( A \) is a \textit{proper subset} of \( B \), 
	which we write as \( A \subset  B \). 
\end{definition}
This distinction is not really explored that much in CS70, so you won't have to worry too much about it. Another 
thing you won't have to worry about but I'll include here anyways is the notion of a superset: 
\begin{definition}{Superset}{}
	Given two sets \( A \) and \( B \), if every element of \( A \) is also a member of \( B \), then 
	\( B \) is a \textit{superset} of \( A \), denoted as \( A \supseteq B \). 
\end{definition}
And analogous to what a proper subset is to a subset, there is also the notion of a proper superset:
\begin{definition}{Proper Superset}{}
	Given two sets \( A \) and \( B \), if every element of \( A \) is also a member of \( B \) and \( B \) 
	contains elements not contained in \( A \), then 
	\( B \) is a \textit{proper superset} of \( A \), denoted as \( A \supset B \). 
\end{definition}
Again, these definitions are really just for completness sake and aren't really things we explore 
much in CS70. The next section, however, is extremely important.
\section{Combining Sets}
So far, we've covered what a set is, how to describe them, and also how to classify them in terms of one 
being a subset of another set. What I haven't explained yet is how we can mathematically talk about the relationship
between sets, which is done through unions and intersections of sets. 

Let's say that you have a set of numbers \( A  \), and another set of numbers \( B \).
Now, suppose you wanted to describe a set \( S \), which is formed by taking elements in either \( A \) or 
\( B \). Mathematically, we'd write that as
\[
S = A \cup B
\] 
where the \( \cup \) symbol denotes a \textbf{union}, defined below:
\begin{definition}{Union}{}
	Given two sets \( A \) and \( B \), the \textit{union} of \( A \) and \( B \) is the 
	set formed by all elements in \( A \) or \( B \), written as \( A \cup B \). 
\end{definition}
To fully illustrate the ``or'' condition here, it's helpful to look at an example:
\begin{example}{}{}
	Given a set \( A = \{1, 2, 3\}  \) and a set \( B = \{3, 4, 5\}  \), then the \textit{union} 
	is denoted by \( S = A \cup B = \{1, 2, 3, 4, 5\}  \).  

	Notice that here, even though the element 3 was contained in both \( A \) and \( B \), it only appears
	once in \( S \). This is the subtlety of the ``or'' condition: if the same element exists in both \( A \)
	and \( B \), only one copy of it is retained in \( A \cup B \). 
\end{example}
What if instead of looking at elements in \( A \) or \( B \), you wanted to look at the elements that are common 
between \( A \) and \( B \)? Then, we'd write it as:
\[
S = A \cap B
\] 
where the \( \cap \) symbol denotes a \textbf{intersection}, defined below:
\begin{definition}{Intersection}{}
	Given two sets \( A \) and \( B \), the \textit{intersection} of \( A \) and \( B \) 
	is the set formed by the elements common between \( A \) and \( B \), written as \( A \cap B \). 
\end{definition}
Like the union, let's illustrate this with an example:
\begin{example}{}{}
	Given a set \( A = \{1, 2, 3\}  \) and a set \( B = \{3, 4, 5\}  \), then the \textit{intersection}
	is denoted by \( S = A \cap B = \{3\}  \)

	Since 3 is the only common element between  \( A \) and \( B \), this is the only element that the 
	intersection picks out. 
\end{example}
Note that the intersection of two sets could potentially contain zero elements: this is what we call a 
disjoint set:
\begin{definition}{Disjoint Sets}{}
	Two sets \( A \) and \( B \) are said to be \textit{disjoint} if they contain no elements in common, or 
	equivalently, \( A \cap B = \varnothing \).
\end{definition}
\section{Complements}
Aside from combining two sets, another thing we would like to do is to learn how to subtract elements from the set. 
How do we express the act of \textit{taking away} elements from a set? This is where we introduce set complements 
and set differences. 

You might wonder why an operation like this is even useful to begin with. Well, on the surface, with sets
you can easily write down, it really isn't that useful. However, when it comes to infinitely sized sets, 
where describing a rule for them is quite complicated, this idea of subtracting or taking away elements actually 
comes in very useful.

Consider the set of irrational numbers for example. Because irrational numbers by definition go on infinitely 
and also don't repeat, you can't ever find a way of writing this set down in the way we described 
in equation \ref{setbuild}. However, you \textit{can} define the set of irrationals in terms of a difference 
between two known sets: the reals and the rationals. By	``difference", what we really mean is to take the set of 
reals \( \R \), take away all the rational numbers (members of \( \Q \)), and the set we're left with is the 
set of irrational numbers. Mathematically, we write that as:
\[
\Q' = \R \setminus \Q
\] 
Here, the set of irrationals is denoted by \( \Q' \). I formalize this notion in the box below:
\begin{definition}{Set Differences}{}
	Given two sets \( A \) and \( B \), the set \( S = A \setminus B \) consists only of elements in \( A \) that 
	are not elements of \( B \). 
\end{definition}
\begin{remark}{}{}
	There's a couple different ways we write set differences, which I've listed below:
	\begin{itemize}
		\item \( A \setminus B = \{x \in A \mid x \not \in B \}  \)
		\item \( A \setminus B = A - A \cap B \)
		\item \( x \in A \setminus B \iff x \in A \land x \not \in B \)
	\end{itemize}
	The last bullet here might not make sense at the moment; we'll revisit this one later so don't worry. See if you
	can convince yourself that these are indeed equivalent ways to write \( A \setminus B \).  
\end{remark}
This idea of set differences also allows us to also introduce the idea of \textit{complements}. Let's return to the 
irrational numbers. Because we've defined the irrationals to be constructed from taking the rationals \( \Q \) 
away from the set of reals \( \R \), it should make sense that if we tried to take the union and intersection  
between the two sets: 
\[
\Q \cap \Q' = \varnothing \quad \Q \cup \Q' = \R
\] 
Let's think about why this makes sense: if the intersection was not the empty set, then this would mean there 
is some element \( x \) which is both rational and irrational. However, we know that's impossible, hence the 
intersection must be the empty set.
On the other hand, if we take the union between the two, we can think of this process as adding the 
irrationals \( \Q' \) back into the set of rationals, and since we constructed \( \Q' \) by taking \( \Q \) away 
from \( \R \), it makes sense that adding them back would give us \( \R \) back as well. 
 
Because \( \Q  \) and \( \Q' \) have the property that their combination makes \( \R \) and they share no 
common elements, we say that they are complements of one another.\footnote{Implicitly, we've defined the universe
\( \mathbb U \) to be \( \R \) here. For the purposes of this book, we won't worry too much about what a universe is; 
it'll be very clear whether two sets are complements of each other.} 
\begin{notation}{Complement}{}
	If two sets \( A \) and \( B \) are complements of each other, we write \( \overline A = B \) or \( A^{c} = B \). 
	The little \( c \) here stands for ``complement."
\end{notation}

Complements will be useful to us later in the probability section of this book, 
so for now, keep this in the back of your mind. 
\section{Cartesian Products}
Now we've come to the second last thing about sets: the Cartesian product. Even though it may not seem like it, 
this is actually something you're familiar with already: you know how in algebra you write \( \R^2 \) to denote 
the real plane? That is a Cartesian product in disguise! I'll give the definition below:

\begin{definition}{Cartesian Product}{}
	Given two sets \( A \) and \( B \), the \textit{Cartesian Product}, denoted \( A \times B \), is the set of 
	all pairs \( (x, y) \) such that \( x \in A \) and \( y \in B \). Equivalently:
	\[
	A \times B = \{(a, b) \mid x \in A \land y \in B\} 
	\] 
\end{definition}
Looking back now at \( \R^2 \), hopefully you can see why we say that this defines the 2D-plane of numbers. We 
write \( \R^2 = \R \times \R \), and based on the definition above we know that it means all pairs 
\( (x, y) \) such that both \( x \) and \( y \) are real numbers. That's exactly the definition of the 
real plane!
\begin{example}{}{}
	Given the set \( A = \{1, 2\}  \) and \( B = \{2, 3\} \), the set \( A \times B \) is written as:
	\[
	A \times B = \{(1, 2), (1, 3), (2, 2), (2, 3)\} 
	\] 
	Notice the difference between this and a union: the point \( (2, 2) \) exists here because we're pulling 
	the first 2 from set \( A \) and the second 2 from set \( B \). If we were to union the two sets, 
	we'd get \( \{1, 2, 3\}  \), and not the pairs. 
\end{example}
 

%Note the differenceds between this and an intersection 
 
 


\section{Power Sets}
Now we come to the last thing about sets: the power set. This is not a very important concept since it appears 
only once elsewhere in this book, so let's go over the definition, and we'll discuss it further when the time 
comes.
\begin{definition}{Power Set}{}
	Given a set \( S \), the power set, denoted by \( \mathcal P(S) \), is the set of all subsets of \( S \). Note that 
	the empty set is also in \( \mathcal P(S) \), as it's a valid subset of \( S \). 
\end{definition}
\begin{example}{}{}
	Given a set \( A = \{1, 2, 3\}  \), the power set is written as:
	\[
		\mathcal P(A) = \{ \{\} , \{1\} , \{2\} , \{3\} , \{1, 2\} , \{1, 3\} , \{2, 3\} , \{1, 2, 3\} \} 
	\] 
	You can check that each one of these are indeed subsets of \( A \).  
\end{example}
That's all for sets! If you don't understand everything that we've talked about here just yet, that's okay! Sets are
a fairly abstract concept, intentionally so because they're meant to be a very general kind of mathematical object. 
Hopefully, when we start applying these concepts, things will make more sense.
