\chapter{Propositional Logic}
Now that we've covered sets and how to construct them, let's take a look at one of the many ways we use them in 
propositional logic. In my opinion, propositional logic is one of the most important topics of CS70, because of how 
universal this language is throughout all areas of math and theoretical computer science. If you take any 
kind of theoretical upper division course (like CS170, EE126, EE127, etc.), you'll definitely encounter 
this language in those classes as well.  

\section{Propositions}
At the heart of propositional logic, the basic building block that we're concerned with is what we call a 
\textbf{proposition}. A proposition is basically a sentence that is either true or false; the way you 
could think about is that it's a statement you can ``propose" as a potential truth. 
\begin{example}{}{props}
	The following are some examples of valid propositions:
	\begin{itemize}
		\item \( \pi \) is an irrational number.
		\item There are no real solutions \( x \) to the equation \( x^2 + x + 1 = 0\). 
		\item \( 1 + 1 = 5 \).
		\item \( n \) is less than 10. 
	\end{itemize}
\end{example}
\begin{remark}{}{}
	Notice that a proposition doesn't need to be true at all! The statement \( 1 + 1 = 5 \) is a perfectly valid 
	proposition, depsite it being blatantly false.  
\end{remark}
More rigorously, we label a proposition as a statement \( P(n) \), which has some truth value 
based on the input to \( n \).\footnote{In this sense, you can almost think of \( P(n) \) as a function 
based on the input \( n \).} In the first three statements above, even though they don't explicitly contain 
\( n \) as an input, they \textit{implicitly} do -- in the sense that they don't care about what the value of 
\( n \) is. 

Because \( P(n) \)'s truth value is dependent on the input \( n \), it should also make sense that 
\( P(n) \) could be true for some values of \( n \), and false for some others. Take the last proposition 
mentioned in the example:
\[
P(n) = \text{\( n \) is less than 10}
\] 
We know that when \( n < 10 \), then \( P(n) \) is true, but when \( n \ge  10 \), then \( P(n) \) is false. This is 
perfectly allowable in propositional logic. However, notice that there isn't a value of \( n \) for which 
\( P(n) \) is true and false at the same time. This is known as the \textbf{law of the excluded middle}:
\begin{theorem}{Law of the Excluded Middle}{law of excluded middle}
	Given a proposition \( P(n) \), for every value of \( n \), \( P(n) \) is either true or false, but not both.	
\end{theorem}
This principle should make intuitive sense: the whole point of coming up with propositions is so that we can talk 
about whether it's true or false for values of \( n \), so if \( P(n) \) could be true \textit{and} false for some 
values of \( n \), then how could we ever talk about the truth value of \( P(n) \) rigorously?

\section{Combining Propositions}
Now that we know how to make propositions, let's talk about how we can combine propositions together to create 
new, more complex ones. In this vein, there are only three ways that we really combine propositions:
\begin{itemize}
	\item \textbf{Conjunction:} combining two propositions with an ``and", written like 
		\( R(n) = P(n) \land Q(n) \). 
		Here, \( R(n) \) is only true when \( P(n) \) is true \textit{and} \( Q(n) \) is true. 
	\item \textbf{Disjunction:} combining two propositions with an ``or", written like \( R(n) = P(n) \lor Q(n) \). 
		Here, \( R(n) \) is true when either \( P(n) \) is true \textit{or} \( Q(n) \) is true, or both. 
	\item \textbf{Negation:} flipping the truth value of \( P(n) \), written like \( R(n) = \neg P(n) \). 
		Here, \( R(n) \) is true when \( P(n) \) is false. 
\end{itemize}
With these three methods, we can make any proposition we want!
\subsection{Propositional Sentences}
And now we come to the most important concept in propositional logic: the act of generating logical sentences from 
propositional statements. In a sense, constructing propositions \( P(n) \) alone without combining them into a sentence 
is rather pointless, since we can't really do much with them. However, when we combine them into sentences, 
that's where we get our system of logic from. 
%Fix this introduction 
What do we mean by combining propositional sentences together? Let's consider the statement: 
\[
	\text{If \( n \) is an integer, then \( n^2 \) is an integer.}
\] 
What do you notice here? This is basically the combination of two propositions! Specifically, if we let 
\( P(n) \) be the statement ``\( n \) is an integer" and \( Q(n) \) be the statement ``\( n^2 \) is an integer", then 
this statement basically simplifies to ``If  \( P(n) \) holds, then \( Q(n) \) holds". Mathematically, what 
we've just constructed is what's known as an \textbf{implication}, defined below:
\begin{definition}{Implication}{}
	An \textit{implication} between two propositional statements \( P(n) \) and \( Q(n) \) is equivalent to 
	the statement ``If  \( P(n) \), then \( Q(n) \)". We write that as \( P(n) \implies Q(n) \). 
\end{definition}
Before we move on, let's fully understand what an implication is doing: remember that \( P(n) \implies 
Q(n)\) is the same as saying ``If \( P(n) \) is true, then \( Q(n) \) is true". So, another way you can understand 
that as is that \textit{under the condition} that \( P(n) \) is true, then \( Q(n) \) is also true. Using the example
we had above, we can say that \textit{under the condition} that \( n \) is an integer, then \( n^2 \) is also 
an integer. 
\begin{remark}{}{}
	In terms of mathematical theorems that follow the form \( P(n) \implies Q(n) \), this is what we mean as well. We 
	first assume that \( P(n) \) is true, then the theorem tells us that \( Q(n) \) is true as well.  
\end{remark}
Along with \( P \implies Q \), there's also two other sentences with \( P \) and \( Q \) that are commonly 
introduced here:
\begin{definition}{Contrapositive, Converse}{}
	Given an implication \( P \implies Q \), the 
	\textit{contrapositive} and \textit{converse} are written as 
	follows:
	\begin{itemize}
		\item \textbf{Contrapositive:} \( \neg Q \implies \neg P \)
		\item \textbf{Converse:} \( Q \implies P \)
	\end{itemize}
\end{definition}	
\begin{warning}{}{}
	Note that contrapositive and converse are not the same thing! For an implication \( P \implies Q \), 
	the contrapositive is \( \neg Q \implies \neg P \), while the converse is \( Q \implies P \). As we'll 
	see in the next section, the contrapositive is logically equivalent to the original implication, 
	whereas the converse is not. 
\end{warning}
There's one last detail about implications you need to know: 
if both \( P(n) \implies Q(n) \) and \( Q(n) \implies P(n) \) is also true, then this is what we call an 
\textbf{if and only if} statement. In English, we'd say ``\( P(n) \) if and only if \( Q(n) \)". 
\begin{notation}{If and only if}{}
	Given two propositional statements \( P(n) \) and \( Q(n) \), if \( P(n) \) is true if and only if 
	\( Q(n) \) is true, then we write \( P(n) \iff Q(n) \). 
\end{notation}
An if and only if (also referred to as iff) statement is powerful because it gives us the ability to say 
for certain that if \textit{either} \( P(n) \) or \( Q(n) \) is true, then the other is true automatically. 
Later when we visit proofs, you'll hopefully appreciate why this is such a powerful condition.

\section{Logical Equivalence}
\label{Logical Equivalence}
Now that we can build an infinite number of propositional sentences, how can we tell whether two of these sentences
are saying the same thing? As you'll discover, there \textit{are} some 
propositional forms that initialy look very different, but are in fact are logically equivalent 
(i.e. mean the same thing). These are particularly useful becuase it can sometimes drastically simplify how we 
go about proving statements.     

How do we determine logical equivalence? Let's look at the theorem below:
\begin{theorem}{Logical Equivalence}{}
	If two propositional sentences \( P \) and \( Q \) have the same truth table, then they are \textit{logically equivalent}.
\end{theorem}
Now what is a truth table? It's basically a table that summarizes the truth values that the sentence formed by 
\( P \) and \( Q \) can take on. Suppose we have the statements \( P \) and \( Q \), and we look at the 
sentence \( P \implies Q \):
\begin{center}
	\begin{tabular}{c|c|c}
		\( P \) & \( Q \) & \( P \implies Q \) \\
		\hline
		T & T & T\\
		T & F & F\\
		F & T & T\\
		F & F & T
	\end{tabular}
\end{center}
This shows us that given the truth values of \( P \) and \( Q \), what the truth value of \( P \implies Q  \) is. 
How did I come up with this table? Let's go through each combination of \( P \) and \( Q \) and see if 
\( P \implies Q \) makes sense:
\begin{itemize}
	\item \( P = \text{T}, Q = \text{T} \). In this case, it is indeed true that a true statement implies a true 
		statement, so this one makes sense. 
	\item \( P = \text{T}, Q = \text{F} \). True statements should only imply other true statements, since we 
		want our mathematics to be logically consistent (i.e. truth implies truth). Therefore, a true statement cannot 
		imply a false one, hence the truth value being false. 
	\item \( P = \text{F}, Q = \text{T/F} \). The last two cases here both fall under the same category: 
		both of these fall under the 
		category where our initial statement \( P \) was false. Imagine this: if you started off a proof with a 
		false premise, then you can ostensibly prove anything you wanted, even false statements. Therefore, 
		both the implications here are assigned true values. 
\end{itemize}
Now that we've looked at \( P \implies Q \), let's also take a look at the truth table for \( \neg P \lor Q \):
\begin{center}
	\begin{tabular}{c|c|c}
		\( P \) & \( Q \) & \( \neg P \lor Q \) \\
		\hline
		T & T & T\\
		T & F & F\\
		F & T & T\\
		F & F & F
	\end{tabular}
\end{center}
Notice that \( P \implies Q \) and \( \neg P \lor Q \) take on the same truth values given the same truth 
values to \( P \) and \( Q \)! Because they have the same truth table, then we say that \( P \implies Q \) 
and \( \neg P \lor Q \) are \textit{logically equivalent}. 
\begin{notation}{Logical Equivalence}{}
	If two propositional sentences  \( A \) and \( B \) are logically equivalent, then we write that as 
	\( A \equiv B \).
\end{notation}
In our case, we'd write \( P \implies Q \equiv \neg P \lor Q \). Another sentence which is logically equivalent 
to \( P \implies Q \) is the contrapositive:
\[
\neg Q \implies \neg P
\] 
This fact is particularly useful since in proofs it's sometimes easier to prove \( \neg Q \implies \neg P \) 
than \( P \implies Q  \), and because they're logically equivalent statements, proving one proves the other 
as well!
\begin{exercise}{}{contrapositive}
	Verify that \( P \implies Q \) and \( \neg Q \implies \neg P \) have the same truth tables. 
\end{exercise}
\section{Quantifiers}
We've covered how to make propositional sentences \( P(n) \), now it's time to focus on the part with \( n \): 
how do we specify to the reader what values can \( n \) take on? This is the role of quantifiers in
propositional phrases. 

There are only two quantifiers in propositional logic:
\begin{itemize}
	\item \textbf{Universal:} Denoted by \( \forall \), it refers to all the elements of a particular set 
		we define. 
	\item \textbf{Existential:} Denoted by \( \exists \), it refers to the existence of an object within a set of 
		our choice. 
\end{itemize}
Now we are finally ready to fully create our first propositional logic statement: we're going to transform the 
sentence:
\begin{center}
	If \( n \) is an integer, then \( n^2 \) is an integer
\end{center}
completely into propositional language. To do this, the first thing we'll want to do is to tell the reader what set 
of numbers \( n \) lives in: in this case, it's the integers. Next, the statement makes no reference to 
the existence of a particular \( n \), so we'll want to 
use the universal quantifier \( \forall \) here. Finally, as before, let \( P(n) \) be the statement that 
\( n \) is an integer, and \( Q(n) \) be the statement that \( Q(n) \) is an integer. Then, we can write:
\[
	(\forall x \in \Z) (P(n) \implies Q(n))
\] 
And that's the complete transformation of our sentence into mathematical terms! The parentheses here aren't absolutely
necessary; you'll see some textbooks that don't use them at all, but I use them here becuase I think they're useful to 
highlight the different parts of our phrase. 

The box below shows a more complex sentence translation, and I encourage you to study it because it involves 
a technique that is very common in propositional logic. 
\begin{example}{}{}
	Let's try converting the following phrase into propositional logic:
	\begin{center}
		There exists only two distinct real solutions to the equation \( x^2 - 1 = 0 \).
	\end{center}
	Where do we even start with this one? First, let's handle the fact that there are two real solutions, and 
	worry about the ``only" keyword later. To write the fact that there are two distinct solutions, what we can do 
	is say instead is that there are two numbers \( x,y \) such that \( x^2 - 1 = 0 \) and \( y^2 - 1 = 0 \), 
	and \( x \neq y \). Written in propositional logic, this is what the phrase looks like so far:
	\[
		(\exists x,y \in \R) (x^2 - 1 = 0 \land y^2 - 1 = 0 \land x \neq y)
	\] 
	Now, how do we say that there are ``only" two solutions? The trick is to first \textit{suppose} that there 
	is a third solution \( z \), then say that if \( z \) also solves this equation, then \( z \) is either
	\( x \) or \( y \).\footnote{Try convincing yourself as to why this is valid.} Therefore, the full 
	equation is:
	\[
		(\forall z \in \R)(\exists x, y \in \R)(x^2 - 1 = 0 \land y^2 - 1 = 0 \land x \neq y) \land 
		(z^2 - 1 = 0 \implies z = x \lor z = y)
	\] 
	%should z use the existence or universal quantifier?
\end{example}
And that's all for quantifiers! The final sentence we've made in this example box is a little complex, but spend 
some time with it, and see if you can identify the purpose of each piece in the sentence and how we joined 
them together. 

\section{De Morgan's Laws}
De Morgan's laws refer to how the negation operator \( \neg \) interacts with the other objects we've explored 
in this chapter. In terms of the conjunction \( \land \) and disjunction \( \lor \) symbols, De Morgan's laws 
says that:
\[
\neg(P \land Q) \equiv \neg P \lor \neg Q \quad \neg(P \lor Q) \equiv \neg P \land \neg Q
\] 
You can check using truth tables that these are in fact logically equivalent. The reason these laws are so powerful 
is because they allow us to potentially discover that two statements are in fact logically equivalent, without having 
to go through the pain of making a truth table.  

What about existential quantifiers? How does negation affect those symbols? Well, let's say you have a statement 
\( \forall x P(x) \), in other words that \( P(x) \) is true for all values of \( x \). 
What would the negation of this statement be? One way to say the opposite of this statement is to say that 
there \textit{exists} a value of \( x \) such that \( P(x) \) is true, or equivalently that \( \neg P(x) \) is true. 
And with that, we've discovered how negation works with quanitifers:
\[
\neg(\forall x P(x)) \equiv \exists x \neg P(x)
\] 
 

