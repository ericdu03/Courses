\chapter{Proofs}
In this chapter, we'll talk about the proof techniques you'll have to be familiar with in this class. Before we get
to that though, I want to emphasize that this chapter alone might be the single most important chapter in this 
entire book, especially if you intend to pursue any theory-heavy course in the future. 

To really explain why proofs are so important, we have to look at the structure upon which mathematical logic 
is built. At the very base, we have a set of axioms, which are a set of statements that are taken 
to be true without proof. These axioms serve as the foundation of all mathematical logic; without them, we 
wouldn't have anything to build off of.  

The way we build off these axioms is through proofs -- a process in which we take true statements (either 
axioms or previously proven statements), and massage them to arrive at a novel conclusion. At its core, 
what we are really doing is asking ``given the set of true statements we have, what else can we 
\textit{provably show} is also true?'' This is the fundamental question that pushes the boundaries of mathematics.

\section{Methods of Proof}
Now, how do we actually go about proving things? Well, we will generally be asked to prove a statement of the 
form \( P(n) \implies Q(n) \), so we need to show that \textit{if} \( P(n) \) is true, then \( Q(n) \) is also true. 
There are many different ways we can do that, which we will go over now. 

Also, I should mention that while I will try to motivate the thought process behind choosing any particular 
proof method for a problem, ultimately the process does take a lot of practice and there's no amount of 
explaining I can do to remedy that.  
\subsection{Direct Proof}
Perhaps the most obvious method of proof is called a \textit{direct proof}. Basically, this method is to start 
wtih the assumption that \( P(n) \) is true, and work our way towards the statement \( Q(n) \). Let's look at the 
example below:
\begin{example}{}{something}
	Let's prove the statement that if \( a \mid b \),\footnote{The $\mid$ symbol here stands for ``divides'', basically 
	meaning that \( \frac{b}{a} \) is an integer.} then \( a \mid kb \) for any integer \( k \). 

	To prove this using the direct proof method, all we have to do is start with the statement \( a \mid b \), and show 
	that for any integer \( k \), that \( a \mid kb \) as well. Starting with \( a \mid b \), another way we can 
	write that is that \( b \) is represented as some integer multiple of \( a \), so \( b = na \) for some 
	integer \( n \). With this in mind, the integer \( kb \) is now written as \( kb = kna \), and since \( k \) is
	also an integer, this implies that \( kb \) is also an integer multiple of \( a \). Therefore, 
	\( a \mid kb \)! And that concludes the proof. 
\end{example}

The hallmark of a direct proof is that we start with the assumption given to us in the ``if'' statement: in the 
example above, this corresponds to \( a \mid b \). Then, we take this assumption in hand, and use the information 
provided by the assumption to show that the ``then'' statement, that \( a \mid kb \) in our example, is also true.  

Usually, one of the things to look out for when trying to directly prove something is to see whether the ``if'' 
clause gives you information that looks like it can be applied to prove the statement. Looking back at the example, 
we see that since both the ``if'' and ``then'' statements involved looking at divisibility, then it isn't a stretch 
to imagine that we could take the information of \( a \mid b \) and directly massage it into the conclusion that 
\( a \mid kb \) as well. 

 
\subsection{Contrapositive}
Along with a direct proof, the method by contrapositive also involves proving a direct implication. The principle 
of the contrapositive is as follows: suppose you're asked to prove the statement \( P \implies Q \). You could go 
about this directly, but remember in section \ref{Logical Equivalence} we introduced the idea of a contrapositive, 
\( \neg Q \implies \neg P \), and because they have the same truth tables (see exercise \ref{exe:contrapositive}), 
then proving \( \neg Q \implies \neg P \) is the same as proving \( P \implies Q \) ! The reason this is 
convenient is that sometimes proving \( \neg Q \implies \neg P \) is far easier than \( P \implies Q \). 

We'll illustrate this with an example:

\begin{example}{}{}
	Prove that if  \( a \) is an irrational number, then for all integers
	 \( k \), \( ka \) is also an irrational number for all nonzero integers \( k \). 


	 Proving this via a direct proof would be relatively difficult -- even though we know that \( a  \) is irrational, 
	 there is no tangible information that allows us to use this information and prove that \( ka \) is irrational.  
	 On the other hand, consider the contrapositive statement: if \( ka \) is a rational number, then
	 \( a  \) must be a rational number for all integers \( k \). This is a far easier statement to prove. 

	 Recall that if \( ka \) is a rational number, then \( ka = \frac{p}{q} \) where \( p \) and \( q \) are 
	 integers. We can then express \( a = \frac{ka}{k} = \frac{p}{kq} \), and since \( k  \) is an integer, 
	 \( kq \) is also an integer, and hence the fraction \( \frac{p}{kq} \) is rational, as desired. 
\end{example}

This problem illustrates perfectly why the contrapositive may be a useful tactic for some proofs. Take some time to 
think about how difficult the forward direction would be: we know that \( a  \) is irrational, but how do we even 
begin representing what an irrational number is? Even if we could figure out how to do this, how do we go about 
showing that multiplication by any integer \( k \) is still irrational? 

On the other hand, consider the simplicity of the contrapositive stastement. We know very concretely how to represent 
\( ka \) if it were rational, and from there division by \( k \) is also well defined so we can conclude 
that \( \frac{p}{kq} \) is also rational, which completes the proof.
\subsection{Contradiction}
While direct and contrapositive proofs do have their place, probably by far the most popular proof technique 
is a proof by contradiction. The essence of this proof is exactly what the name suggests: instead of proving that 
something is true, you show that it can't be untrue instead. The basic structure of a proof by contradiction 
is as follows: suppose you wanted to prove that \( A \implies B \). Then, to prove by contradiction, we first 
assume that \( A \implies \neg B \). Then, we prove that this is impossible by showing that \( A \implies B
\) and  \( A \implies \neg B  \) must both be true at the same time -- by the law of the excluded middle
(theorem \ref{th:law of excluded middle}), both of these cannot be true at the same time. Therefore, 
\( A \implies \neg B \) must be false, and hence we've proven our original statement. 
 

\subsection{Induction}
 

   
 
   
 
  
 

