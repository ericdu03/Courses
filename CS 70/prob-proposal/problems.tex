\documentclass[10pt]{article}
\usepackage{../local}
\title{Problem Proposals}
\author{Eric Du}
\date{\today}
\begin{document}
	\maketitle
	
	\section*{Alternate Berlekamp-Welch}
	Recall that Berlekamp-Welch requires that we write the Error locator polynomial $E(x)$ as: 
	\[
	E(x) = (x - e_1)(x - e_2)\cdots(x - e_k)
	\] 
	What if we redefine $E(x)$ to instead return 0 at a \textit{correct point} instead? In this scheme, the 
	degree of $E(x)$ would be $n - k -1$, for a length $n$ message with $k$ corruptions. Note also that in this
	formulation of Berlekamp-Welch, we cannot use the normal equation $P(i)E(i) = r_i E(i)$, since $E(i) \neq 0$
	at an error. How many packets would be required to successfully recover the message in this scheme?

	\section*{RSA Correctness}
Firstly, this does not hold for all $m$. For instance, take $m = 4$ and $e = 3$, a quick computation shows that $d = 3$ as well, so we want to show that 

\[ x^9 \equiv x \pmod 4\]
But this isn't even true for all $x$: take $x = 2$, which gives $x^9 \equiv 0 \pmod 4$, which violates the expression.

But okay, we prove something slightly weaker: we prove instead that if $m$ is prime, 


But okay, let's prove something slightly weaker: let $m = p-1$ for some prime $p$. We prove that under this scheme, $x^{ed} \equiv x \pmod m$ does hold.


Firstly, $ed \equiv 1 \pmod m$. From here onwards, it will be useful to write $p-1$ instead of $m$, so I will do that. From $ed \equiv 1 \pmod{p-1}$, we can then rewrite this as $ed = k(p-1) + 1$. Therefore, we are asked to prove: 
\[ x^{k(p-1) + 1} \equiv x \pmod m\]	
	


\end{document}
