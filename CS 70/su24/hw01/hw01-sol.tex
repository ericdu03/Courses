\documentclass[10pt]{article}
\usepackage{../../../local}
\urlstyle{same}

\newcommand{\classcode}{CS 70}
\newcommand{\classname}{Discrete Mathematics and Probability}
\renewcommand{\maketitle}{%
\hrule height4pt
\large{Eric Du \hfill \classcode}
\newline
\large{HW 01} \Large{\hfill \classname \hfill} \large{\today}
\hrule height4pt \vskip .7em
\small{Header styling inspired by CS 70: \url{https://www.eecs70.org/}}
\normalsize
}
\linespread{1.1}
\begin{document}
	\maketitle
	
	\section*{Problem 1}

	\section*{Problem 2}
	Let \( R \) be a binary predicate such that the following are true:
	\begin{enumerate}[label=(\arabic*)]
		\item \( \forall x \forall y \left( R(x, y) \implies R(y, x) \right)  \)
		\item \( \exists x \forall y R(x, y) \)
	\end{enumerate}
	\begin{enumerate}[label=\alph*)]
		\item Prove or disprove whether the following are logically implied by the 
			conditions (1) and (2).
			\begin{enumerate}[label=\roman*)]
				\item \( \forall x \exists y R(x, y) \)

					\begin{solution}
						This is logically implied. From (2), we know that there 
						exists some \( a \) such that \( R(a, y) \) for all \( y \), 
						and (1) requires that \( R(a, y) \implies R(y, a) \), so 
						we know there exists some \( a \) such that \( R(y, a) \) 
						for all \( y \). This is precisely the statement 
						\( \forall x \exists y R(x, y) \), up to variable names. 
					\end{solution}
				\item \( \forall x R(x, x) \) 

					\begin{solution}
						This is not logically implied. Part (i) only implies the 
						\textit{existence} of a \( y \) such that 
						\( \forall x R(x, y)\), but it does not make the claim that 
						\( y = x \) always, a necessary condition to guarantee
						 that \( \forall x R(x, x) \) is implied. 
					\end{solution}
				\item \( \exists y \forall x R(x, y) \)

					\begin{solution}
						This is logically implied, just swap variable 
						names. 
					\end{solution}
				\item \( \forall x \forall y \left( R(x, y) \lor R(y, x) \right)  \)

					\begin{solution}
						From statement (i), we know that \( \forall x \), 
						there \textit{exists} a \( y \) such that \( R(x, y) \) 
						and condition (1) means that \( R(y, x) \) is true as well. 
						However, this does not imply that such an \( x \) works 
						such that \textit{all} values of \( y \) satisfy 
						\( R(y, x) \), hence this is not logically implied. 
					\end{solution}
			\end{enumerate}
		\item Consider the natural numbers with the binary predicate 
			\( R(x, y) \) as ``\( x \cdot y = 0 \)''. 
			\begin{enumerate}[label=\roman*)]
				\item Check that the conditions (1) and (2) are true of \( R \) 
					in this setting. 

					\begin{solution}
						Condition (1) is true becuase multiplication is commutative,
						and select \( x = 0 \) to satisfy condition (2). 
					\end{solution}
				\item Translate conditions (1) and (2), when applied to this 
					setting, into simple English sentences. 

					\begin{solution}
						See below
						\begin{enumerate}[label=\arabic*.]
							\item For all \( x, y \), if the product 
								\( x \cdot y = 0 \), then \( y \cdot x = 0 \)
							\item There exists a value of \( x \) such that 
								for all values of \( y \), \( x \cdot y = 0 \). 
						\end{enumerate}
					\end{solution}
			\end{enumerate}
	\end{enumerate}
	\pagebreak
	\section*{Problem 3}
	In this problem, we will prove the fundamental theorem of arithmetic: any 
	integer \( n \ge 2 \) can be factorized as a prodcut of powers of its 
	prime factors. That is, for any integer \( n \ge 2 \), we can write
	\[
	 n = p_1^{q_1} p_2^{q_2} \cdot \cdots \cdot p_m^{q_m}
	\] 
	where \( p_1, \dots, p_m \) are prime numbers and \( q_1, \dots, q_m  \) are 
	positive integers. 
	\begin{enumerate}[label=\alph*)]
		\item We first consider the case where \( n \) is prime. Show that the 
			fundamental theorem of arithmetic holds when \( n \) itself is a prime 
			number. 

			\begin{solution}
				Since \( n \) is prime, then \( n = p \), which is clearly 
				satisfied. 
			\end{solution}
		\item Now we consider the case when \( n \) is not prime: that is, 
			\( n \) is composite. By the definition of a compositte number, there 
			exists a positive integer \( d \) such that \( d \mid n \) and
			\( 1 < d < n \). We call \( d \) a \textit{nontrivial divisor} of \( n \). 

			Prove that if \( d \) and  \( n / d \) can be factorized as a product 
			of powers of its prime factors, then \( n \) can also be factorized 
			as a product of powers of its prime factors. 

			\begin{solution}
				Here we leverage the fact that \( d \cdot \frac{n}{d} = n \), 
				so let \( d = p_1^{q_1} \cdots p_m^{q_m} \), and 
				\( n / d = p_1^{r_1} \cdots p_m^{r_m} \). To make this 
				definition work, we will choose 
				\( p_m \) large enough such that the largest prime factor between 
				both numbers is equal to \( p_m \). Consequently, we must 
				allow for the possibility that \( q_i = 0 \) and \( r_i = 0 \). 
				Then, the product can be expessed as follows:
				\[
				d \cdot \frac{n}{d} = n = p_1^{q_1 + r_1} \cdots p_m^{q_m + r_m}
				\] 
				which is of the desired form. 
			\end{solution}
		\item Using induction and the two parts above, prove the fundamental 
			theorem of arithmetic. 

			\begin{solution}
				Base case: \( n = 2 \)\footnote{We skip \( n = 1 \) since 
				1 is neither prime nor composite.}, which is prime, so we are done. 

				Inductive hypothesis: For all values less than  \( k \), 
				the proposition holds true. 

				Inductive step: for \( k + 1 \), we know that it is either prime 
				or composite. If \( k + 1 \) is prime, then we are immediately done, 
				and if \( k + 1 \) is composite, then we know it can be written 
				in the desired form based on part (b). 

				The inductive step holds, and we are done. 
			\end{solution}
	\end{enumerate}
\end{document}
