\documentclass[11pt]{article}
\usepackage{header}
\def\title{Disc 01}


\begin{document}
\maketitle
\fontsize{12}{15}\selectfont


\Question{Contraposition}

Prove the statement ``if $a+b < c+d$ then $a < c$ or $b < d$"

\begin{solution}
    We prove this by showing the contrapositive is true. Essentially, we prove that if $a > c$ and $b >d$, then $a+b > c+d$. 

    This is easily shown to be true, since 

    \[ a +b > c + b > c+d\]

    We know that this is true because $a > c$ and since $b > d$ the second part of that inequality is true. Thus, $a + b > c+d$, and thus the original statement is also proven.
\end{solution}

\Question{Numbers of Friends}

Prove that if there are $n \ge 2$ people at a party, then at least 2 of them have the same number of friends at the party. Assume that friendships are always reciprocated: that is, if Alice is friends with Bob, then Bob is also friends with Alice. 

(Hint: The Pigeonhole Principle states that if $n$ items are placed in $m$ contianers, wher $n > m$, at least one contianer must contain more than one item. You may use this without proof.)

\begin{solution}
    As the hint suggests, we aim to prove this via the pigeonhole principle. Essentially, we prove that it is impossible to find a construction where each person has a unique number of friends. 

    Since there are $n$ people, this means that we shouldd have $n$ different numbers of friends. However, there are only $n - 1$ possible assignments for the number of friends a person can have: this is because nobody can have $n$ friends, and nobody can have $0$ friends.\footnote{Here we assume that you can't be friends with yourself, which is rather sad to think about but let's not entertain that for any longer than we need.}. You cannot have $0$ friends either, since that would require somebody to have $n-1$ friends in a group of $n - 2$ people (since one person now has $0$ friends), which is also impossible. Therefore, since there are $n$ differnet people and we only have $n-1$ different possible assignments, it is guaranteed that at least 2 of them have the same number of friends at a party. 
\end{solution}

\Question{Pebbles}

Suppose you have a rectangular array of pebbles, where each pebble is either red or blue. Suppose that for every way of choosing one pebble from each column, there exists a red pebble among the chosen ones. Prove that there must exist an all-red column.

\begin{solution}
    The statement of the problem can essentially be rephrased as: regardless of the way we choose one pebble from each column, it is guaranteed that at least one red pebble must be chosen. If we didn't have an all-red column, then there is at least one blue pebble in each column, and that would be a way to select one pebble from each column such that no red pebbles were ever chosen. Thus, there must be an all-red column.
\end{solution}

\Question{Preserving SEt Operations}



\end{document}