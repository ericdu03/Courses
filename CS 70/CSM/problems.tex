\documentclass[10pt]{article}
\usepackage{../../local}

\newcommand{\classcode}{CS 70}
\newcommand{\classname}{Problems}
\renewcommand{\maketitle}{%
\hrule height4pt
\large{Eric Du \hfill \classcode}
\newline
\large{HW 1} \Large{\hfill \classname \hfill} \large{\today}
\hrule height4pt \vskip .7em
\normalsize
}
\linespread{1.1}
\begin{document}
%	\maketitle
%	Consider throwing $n$ balls into $n$ bins uniformly at random. Let $X$ be the number of balls in the 
%	first bin.
%	\begin{enumerate}[label=\alph*)]
%		\item What is the expected value of $X$?
%		\item What is the variance of $X$?
%	\end{enumerate}


%	What is the covariance of $X$ and $X^3$ where $X$ is a uniformly distributed variable on the interval $[0, 
%	1]$? (i.e. $X \sim U[0, 1]$)

	
%	\section{Counting}
\section*{Problem}
	Suppose two integers \( a \) and \( b \) are drawn uniformly from \( [-n, \dots, n] \), 
	that is, \( a, b \in \Z \) and \( -n \le a,b, \le n \). Find the probability that \( |a-b| \le k \). 
	You may assume that \( k < \frac{n}{2} \). 

	\begin{solution}
		At first glance, this problem doesn't really give us a point of entry to even begin counting, because even for 
		any particular \( k \) (say, \( k = 10 \)), there are so many values of \( a \) and \( b \) that satisfy this 
		that it's hard to count. The strategy then is to \textit{give ourselves somewhere to start}, then sum over 
		all possible ways to start to account for all the possibilities.

		With that in mind, how do we employ this strategy? What I am really asking here is, what point of 
		attack do we want to take? We could fix \( k \) and count all possible \( a, b \), which works and isn't too 
		much more complicated than the solution I'm about to present,\footnote{You should try doing it like this, 
		it's good practice!} but what I'm going to do instead is fix a value of \( a \), analyze 
		the number of values \( b \) can take such that \( |a - b| \le k \), then sum over all the values of 
		\( b \) given for each \( a \) to get our total count.

		So with our strategy in place, let's start counting, and the first thing we immediately recognize is that 
		depending on the value of \( a \), there's a different number of \( b \) that work. Specifically there's 
		three cases:
		\begin{itemize}
			\item \( a < -n + k \)
			\item \( -n + k \le  a \le  n - k \) 
			\item \( a > n-k \)
		\end{itemize}
		To see why this is, consider \( n = 10 \) and \( k = 3 \). Then, consider a number that lives in the first 
		case, say \( a = -8 \). Then, what values of \( b \) exist such that \( |a - b| \le 5 \)? Well, we can 
		split these numbers into three categories:
		\begin{itemize}
			\item Numbers to the left of \( a \): \( b = -10, -9 \) 
			\item \( a \) itself: \( b = -8 \) 
			\item Numbers to the right of  \( a \): \( b = -7, -6, -5 \). 
		\end{itemize}
		What I want you to notice is the number of numbers in each category. The first category has less than \( k \)
		numbers, the second category has one number, and the third category has \( k \) numbers. For every number 
		in \( a < -n + k \), we will always have less than \( k \) numbers to the left, since we are limited 
		by \( -n \) to the left. We will alwyas have \( b = a\) as an option, and to the right we will always have 
		\( k \) since we're not limited by anything. 

		In general you can visualize this first case as the following 
		diagram, where \( a = -n + i \) and \( i \) is some number that is less than \( k \):
		\begin{center}
			\begin{tikzpicture}[scale=0.5]
				\draw[stealth-stealth]  (-10, 0) node[left] {\( -n \) } -- (10, 0) node[right] {\( n \) };
				\draw[-stealth, blue] (-8, -1) node[below] {\( a = -n + i\) } -- (-8, 0); 	
				\draw[dashed, red] (-8, 0) -- (-8, 2);
				\draw[dashed, red] (-10, 0) -- (-10, 2);
				\draw[stealth-stealth, red] (-10, 1) -- node[midway, above] {\( i \) } (-8, 1); 
				\draw[dashed, red] (-5, 0) -- (-5, 2);
				\draw[stealth-stealth, red] (-8, 1) -- node[midway, above] {\( k \) } (-5, 1);
			\end{tikzpicture}
		\end{center}
		Here, the range indicated in red is the values that \( b \) can take such that \( |a - b| \le k \). So, 
		there are \( i \) values to the left, \( k \) values to the right, and including \( a \) itself, 
		then we have \( i + k + 1 \) values in total for values in this category. 

		Now let's look at the second case, returning back to our example of \( n = 10 \) and \( k = 3 \). Here, we can 
		pick \( a = -1 \) as an example. Then, we can split the values of \( b  \) into the same three 
		categories:
		\begin{itemize}
			\item Numbers to the left of \( a \): \( b = -4, -3, -2 \) 
			\item \( a \) itself: \( b = -1 \) 
			\item Numbers to the right of  \( a \): \( b = 0, 1, 2 \)
		\end{itemize}
		Again, counting the number of numbers in each category, we see that we have \( k \) numbers both to the right 
		and left of \( a \), and this is true of every number in this category. This is becuase every number 
		is at least \( k \) away from \( n \), so we will always have \( k \) available values for \( b \) on either 
		side of \( a \). Visually, it looks something like this:
		\begin{center}
			\begin{tikzpicture}[scale=0.5]
				\draw[stealth-stealth]  (-10, 0) node[left] {\( -n \) }-- (10, 0) node[right] { \( n \) };
				\draw[-stealth, blue] (-1, -1) node[below] {\( a \) } -- (-1, 0); 	
				\draw[dashed, red] (-4, 0) -- (-4, 2);
				\draw[dashed, red] (-1, 0) -- (-1, 2);
				\draw[stealth-stealth, red] (-4, 1) -- node[midway, above] {\( k \) } (-1, 1); 
				\draw[dashed, red] (2, 0) -- (2, 2);
				\draw[stealth-stealth, red] (-1, 1) -- node[midway, above] {\( k \) } (2, 1);
			\end{tikzpicture}
		\end{center}
		So, there are \( k \) numbers to the left and right, and including \( a \) itself, this makes \( 2k + 1 \) 
		numbers for every number in this section.
		Now for the last category, let's choose \( a = 8 \). Then:
		\begin{itemize}
			\item Numbers to the left of \( a \): \( b = 5, 6, 7 \) 
			\item \( a \) itself: \( b = 8 \) 
			\item Numbers to the right of  \( a \): \( b = 9, 10 \)
		\end{itemize}
		Notice the symmetry in this case to the first case (except the minus sign): 
		the numbers to the left of \( a \) are the same 
		as those to the right of \( a \) in the first case, and the numbers to the right of \( a \) are the same as
		those to the left of \( a \) in the first case. So, due to this symmetry, we conclude that the counting for the 
		numbers in this category are the same as those in the first, so we can just double the first case to account 
		for the numbers here. You can also see this symmetry visually:
		\begin{center}
			\begin{tikzpicture}[scale=0.5]
				\draw[stealth-stealth]  (-10, 0) node[left] {\( -n \) } -- (10, 0) node[right] {\( n \) };
				\draw[-stealth, blue] (8, -1) node[below] {\( a = n - i\) } -- (8, 0); 	
				\draw[dashed, red] (8, 0) -- (8, 2);
				\draw[dashed, red] (10, 0) -- (10, 2);
				\draw[stealth-stealth, red] (10, 1) -- node[midway, above] {\( i \) } (8, 1); 
				\draw[dashed, red] (5, 0) -- (5, 2);
				\draw[stealth-stealth, red] (8, 1) -- node[midway, above] {\( k \) } (5, 1);
			\end{tikzpicture}
		\end{center}
		Notice that this diagram is exactly the mirror image of what we had in case 1, so indeed there is symmetry here.
		And just like from the first case, there are \( k + i + 1\) total values for \( b \) here. 

		Now, it's time to sum everything up, and let's consider how many values of \( a \) exist in each category. 
		For the first category, since we go up to \(a <  -n + k \), this implies that the range of numbers is from 
		\( -n \) to \( -n + k - 1 \) or equivalently \( i \in \{0, 1, \dots, k - 1\}  \). Remember that 
		for each one of these values, we have \( k + i + 1 \) values for \( b \). 

		For the second category, 
		we range from  \( -n + k \) to \( n - k \), so to calculate the number of numbers here we take the difference 
		then add 1 to account for \( a = 0 \), giving us \( n - k - (-n + k) + 1 \) total values of \( a \), 
		each one having \( 2k + 1 \) values for \( b \). We then said 
		that we'd double case 1 to account for the third category so we're done counting. In total, we have:
		\[
		N(A) = 2\left( \sum_{i=0}^{k-1} k + i + 1 \right) + \sum_{i=1}^{(n - k) - (-n + k) + 1} 2k + 1
		\] 
		The rest is an algebra exercise that I won't really go into too much, but eventually we get
		\[
		N(A) = -k^2 + k + 4nk + 2n + 1
		\] 
		so in total, since the probability is calculated as \( P(A) = \frac{N(A)}{N(\text{all})} \), then:
		\[
		P(A) = \frac{-k^2 + k + 4nk + 2n + 1}{(2n + 1)^2}
		\] 
		Really, the takeaway from this problem is what I mentioned in section: take note of how we broke down the 
		problem by considering individual values of \( a \), asking how many values of \( b \) there are, then 
		finally summing over all these possible values to get the number of pairs \( (a, b) \) that satisfy 
		our desired property. 

		As practice, I encourage you to try and reproduce this solution without looking at it, and once you can do that,
		try counting the number of \( a, b \) in a different way: instead of fixing \( a \) like we did here, try 
		fixing \( k' \), figure out all the possible values of  \( a, b \) for that particular \( k' \), and sum 
		over all possible values of \( k' \) that satisfy our original constraint. This method of counting 
		requires you to be a \textit{little} more careful than this method, but I don't believe it to be much harder
		and I think it's good practice. 
	\end{solution}
\end{document}
