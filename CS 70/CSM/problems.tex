\documentclass[10pt]{article}
\usepackage{../../local}


\newcommand{\classcode}{CS 70}
\newcommand{\classname}{Problems}
\renewcommand{\maketitle}{%
\hrule height4pt
\large{Eric Du \hfill \classcode}
\newline
\large{HW 1} \Large{\hfill \classname \hfill} \large{\today}
\hrule height4pt \vskip .7em
\normalsize
}
\linespread{1.1}
\begin{document}
	\maketitle
	Consider throwing $n$ balls into $n$ bins uniformly at random. Let $X$ be the number of balls in the 
	first bin.
	\begin{enumerate}[label=\alph*)]
		\item What is the expected value of $X$?
		\item What is the variance of $X$?
	\end{enumerate}


	What is the covariance of $X$ and $X^3$ where $X$ is a uniformly distributed variable on the interval $[0, 
	1]$? (i.e. $X \sim U[0, 1]$)

	\section{Halting Problem}
		Basically problem comes down to reducing any given problem to the halting problem. We can 
			do this by describing a program that halts if we reach the desired input, and not otherwise. 
			Generally, things where we need to determine something about \textit{how} a program executes (for 
			instance, executing a specific line) is uncomputable, and things that can be tracked (like 
			how much memory a program uses) can be computable.
\end{document}
