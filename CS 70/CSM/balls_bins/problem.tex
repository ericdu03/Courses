\documentclass[10pt]{article}
\usepackage{../../../local}
\urlstyle{same}

\linespread{1.1}
\begin{document}
	I know that this is overkill but the problem basically lived in my head rent free since our section today 
	and I just 
	\textit{had} to write about it: 

	For the case with \( n \) balls and \( n \) indistinguishable bins, what you can basically think of this problem 
	as is asking how many ways are there to divide \( n \) into a set of integers that sum to \( n \). Coincidentally, 
	there's this function in number theory called the partition function \( p(n) \) (see 
	\url{https://en.wikipedia.org/wiki/Integer_partition}) which does exactly that.  

	In the wikipedia article, it explains that there is no known closed form expression for \( n \), so 
	there's no way to write down \( p(n) \) without using infinite sums and other complicated mathematical 
	functions. 


	If you're satisfied with this as an answer, that's perfectly fine. Me, however, I was interested now in 
	what would the number of ways be if you had \( n \) balls into \( n-1 \) bins now, 
	and I reasoned that this is basically the same as finding the number of ways of partitioning \( n-1 \), 
	then multiplying by \( n \). You can argue this is the case by first taking away one ball, then 
	the remaining \( n-1  \) 
	balls into \( n-1  \) bins gives us \( p(n-1) \) ways, and for every partition there's \( n \) ways we can 
	throw this last ball into one of the bins giving us \( np(n-1) \) in total. 
	 
	Ok, what about \( n-k \) bins now? Well, this would be the same as the previous case, except now we take 
	away \( k \) balls, so the partition is now \( p(n-k) \). Then, we have \( k \) balls left, to distribute into 
	\( n-k \) bins, so this is the balls and bins formula we had. Putting these two together, we have:
	\[
		N = {k + (n-k) - 1 \choose k}p(n-k) = {n -1 \choose k}p(n-k)
	\] 
	I think it's safe to say that you won't have to be worried about a problem like this appearing on an exam. 
\end{document}
