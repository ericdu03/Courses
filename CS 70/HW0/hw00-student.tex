\documentclass[11pt]{article}
\usepackage{header}
\usepackage{dirtytalk}
\def\title{HW 00}

\begin{document}
\maketitle
\fontsize{12}{15}\selectfont
\linespread{1.3}
\allowdisplaybreaks


\begin{center}
    Due: Saturday, 8/27, 4:00 PM \\
    Grace period until Saturday, 8/27, 6:00 PM \\
\end{center}

\section*{Sundry}
Before you start writing your final homework submission, state briefly how you worked on it.  Who else did you work with?  List names and email addresses.  (In case of homework party, you can just describe the group.)


\begin{solution}
  I did not work with anyone to complete this homework.
\end{solution}


\vspace{15pt}

\Question{Administrivia}

\begin{Parts}

\Part Make sure you are on the course Ed (for Q\&A) and Gradescope (for submitting homeworks, including this one). Find and familiarize yourself with the course website. What is its homepage's URL?

\Part Read the policies page on the course website.
	\begin{enumerate}[(i)]  
		\item What is the breakdown of how your grade is calculated? \\
		\begin{solution}
        There are two ways the grades are broken down, depending on whether the \say{no homework} option was chosen or not. If it is not chosen, the grade breakdowns are:

        \begin{itemize}
          \item \textbf{Discussion attendance:} 5\%
          \item \textbf{Vitamins: 5\%} (lowest 2 vitamins before and lowest 2 after the midterm are dropped)
          \item \textbf{Homework:} 20\% (lowest 2 homeworks before and after the midterm are dropped)
          \item \textbf{Midterm:} 25\%
          \item \textbf{Final:} 45\% 
        \end{itemize}

        Otherwise, if the \say{no-homework} option was chosen, then the following is the breakdown:

        \begin{itemize}
          \item \textbf{Discussion attendance:} 6.25\%
          \item \textbf{Vitamins:} 6.25\% (lowest 2 vitamins before and lowest 2 after are dropped)
          \item \textbf{Midterm:} 31.25\%
          \item \textbf{Final:} 56.25\%
        \end{itemize}

        There is also a partial clobber policy that has yet to be released on the website.

    \end{solution}
		\item What is the attendance policy for discussions?
		
    \begin{solution}
      You need to attend at least 13 discussions to receive full credit on the discussions; the only way to earn these points is to attend the discussion section you are assigned to.
    \end{solution}
		\item When are homeworks released and when are they due?

		\begin{solution}
      Homeworks are released on Sunday, and is due the following Saturday at 4:00 PM, with a grace period until 6:00 PM. 
    \end{solution}
		\item How many "drops" do you get for vitamins? For homework?\\
		\begin{solution}
      You get two drops for vitamins before the midterm and also two more drops for vitamins after the midterm. The same policy goes for homework: the lowest two homeworks before and after the midterm are dropped.
    \end{solution}
		\item When is the midterm? When is the final?\\
		\begin{solution}
      The midterm date is still TBD on the website, but the final date is Tuesday December 13, from 3:00 to 6:00 pm.  
    \end{solution}
	\end{enumerate}

\end{Parts}

\Question{Course Policies}

Go to the course website and read the course policies carefully. Leave a followup on Ed if you have any questions. Are the following situations violations of course policy? Write "Yes" or "No", and a short explanation for each.

\begin{Parts}
  \Part Alice and Bob work on a problem in a study group. They write up a solution together and submit it, noting on their submissions that they wrote up their homework answers together.
  
  \begin{solution}
    Yes, this is a violation of course policy because they wrote up the solution together, instead of writing them up separately on their own.
  \end{solution}
  \Part Carol goes to a homework party and listens to Dan describe his approach to a problem on the board, taking notes in the process. She writes up her homework submission from her notes, crediting Dan.

  \begin{solution}
    No, this is not a violation of course policy. Carol's did write her solution to the problem on her own, and discussion approaches with one another is allowed. She also correctly acknowledged Dan in her homework, who clearly gave her significant ideas about how to solve the problem.
  \end{solution}
  
  \Part Erin comes across a proof that is part of a homework problem while studying course material. She reads it and then, after she has understood it, writes her own solution using the same approach. She submits the homework with a citation to the website.

  \begin{solution}
    No, this is not a violation of course policy. Erin did not copy the website source verbatim, but instead spent the effort to read and understand the approcah, then applies it by writing her own unique solution. She also properly credits the website as the inspiration for her solution, as per the course policy.
  \end{solution}
  
  
  \Part Frank is having trouble with his homework and asks Grace for help. Grace lets Frank look at her written solution. Frank copies it onto his notebook and uses the copy to write and submit his homework, crediting Grace.
  
  \begin{solution}
    Yes, this is a violation of course policy. Despite having credited Grace in his solution, Frank has still violated course policy by copying Grace's solution verbatim, and submitting it as if it were his own.
  \end{solution}
  
  \Part Heidi has completed her homework using \LaTeX. Her friend Irene has been working on a homework problem for hours, and asks Heidi for help. Heidi sends Irene her PDF solution, and Irene uses it to write her own solution with a citation to Heidi.
  
  \begin{solution}
    Yes, this is a violation of course policy, as she clearly used Irene's solution, copied it verbatim, and submittied it as if it were her own.
  \end{solution}
  
  \Part
  Joe found homework solutions before they were officially released, and every time he got stuck, he looked at the solutions for a hint. He then cited the solutions as part of his submission.

  \begin{solution}
    Yes, this is \textit{definitely} a violation of academic policy. Not only should he not have access to the homework solutions, but he also used the solution extensively to complete the homework. 
  \end{solution}
\end{Parts}

\Question{Use of Ed}

Ed is incredibly useful for Q\&A in such a large-scale class. We will use Ed for all important announcements. You should check it frequently. We also highly encourage you to use Ed to ask questions and answer questions from your fellow students.

\begin{Parts}
  
    \Part Read the Ed Etiquette section of the course policies and explain what is wrong with the following hypothetical student question: "Can someone explain the proof of Theorem XYZ to me?" (Assume Theorem XYZ is a complicated concept.)

    \begin{solution}
      Since theorem XYZ is assumed to be a complicated concept, it is likely that it will take longer tn 5 minutes to fully explain the proof. As such, it is not a question that should be posted to Ed, but rather be asked during office hours or homework parties, as per the 5 minute test.
    \end{solution}

    \Part When are the weekly posts released? Are they required reading?

    \begin{solution}
      Weekly posts are released on Sundays and are required readings.
    \end{solution}

    \Part If you have a question or concern not directly related to the course content, where should you direct it?

    \begin{solution}
      The general email (fa22@eecs70.org) would be a good place to start for any general concerns about the course. If this concern is specifically related to DSP accomodations, it should be directed to (dsp@berkeley.edu) instead.
    \end{solution}

\end{Parts}

\Question{Academic Integrity}

Please write or type out the following pledge in print, and sign it.

\begin{quote}
I pledge to uphold the university's honor code: to act with honesty, integrity, and respect for others, including their work. By signing, I ensure that all written homework I submit will be in my own words, that I will acknowledge any collaboration or help received, and that I will neither give nor receive help on any examinations. 
\end{quote}

\begin{solution}
  I pledge to uphold the university's honor code: to act with honesty, integrity, and respect for others, including their work. By signing, I ensure that all written homework I submit will be in my own words, that I will acknowledge any collaboration or help received, and that I will neither give nor receive help on any examinations. 

  By submitting this homework I affirm that I've signed the pledge.
\end{solution}

\Question{Propositional Practice}
In parts (a)-(c), convert the English sentences into propositional logic. In parts (d)-(f), convert the propositions into English. In part (f), let $P(a)$ represent the proposition that $a$ is prime.
\begin{Parts}

\Part There is one and only one real solution to the equation $x^2 = 0$.

\begin{solution}
\[(\exists x \in \mathbb R) (x^2 = 0) \land (\exists y \in \mathbb R)(y^2 = 0 \implies x = y)\]

\end{solution}
\Part Between any two distinct rational numbers, there is another rational number.

\begin{solution}
  \[ (\forall x, y \in \mathbb Q) [x > y \implies (\exists z \in \mathbb Q)(x > z > y)]\]
\end{solution}
\Part If the square of an integer is greater than 4, that integer is greater than 2 or it is less than -2.
\begin{solution}
\[(\forall x \in \mathbb R) (x^2 > 4 \implies x > 2 \lor x < -2)\]
\end{solution}
\Part $(\forall x \in \mathbb{R})\ (x \in \mathbb{C})$

\begin{solution}
  All real numbers are also complex numbers.
\end{solution}
\Part $(\forall x,y \in \mathbb{Z}) (x^2-y^2 \not=10)$


\begin{solution}
  The difference between any two squares is not 10.
\end{solution}
\Part $(\forall x \in \mathbb{N})\ \left[ \ (x > 1) \implies \ (\exists a, b \in \mathbb{N})  \ \left( (a + b = 2x)\land P(a) \land P(b) \right) \ \right]$

\begin{solution}
Every even integer greater than 2 is the sum of two primes.
\end{solution}
\end{Parts}

\end{document}
