%\section{Introduction}
\section{Lecture 1}
\subsection{Motivations} 
\begin{itemize}
	\item Why study this class?
		\begin{itemize}
			\item Given a "black box" circuit, with input and output leads, we can determine what's within 
				the "black 
				box".
			\item In this particular case, if our black box contains a voltage divider, and the output voltage is given 
				by the equation:
				\[
				v_{\text{out}}(t) = \frac{R_2}{R_1+ R_2}v_{\text{in}}(t)
				\] 
				In principle though, the signal can be anything that we want: for facial recognition software, the input
				signal could be the configuration of the intensity the camera picks up. There's many more we went over, 
				don't really want to write it all down. 
		\end{itemize}
	\item In essence, there's a lot of systems that can be modeled by a system that takes in a signal \( x(t) \), and 
		outputs a signal \( y(t) = f(x(t))\).
		\begin{itemize}
			\item The signals are usually functions of time, location, in any number of dimensions.
			\item The systems does some sort of transformation on an input signal. In particular, we will study 
				linear systems, shift-invariant systems, etc.
				
				We'll talk about mathematical operations that we use to perform these transformations: Fourier, 
				Laplace, Z-transformations, convolutions, correlation, etc.\item 
	\end{itemize}
\end{itemize}
\subsection{Types of Signals}
\begin{itemize}
	\item \textbf{Continuous-time:} signals defined over continuous variables (e.g. position, time). For 
		instance, a signal \( x(t) \) is continuous for our purposes, since time is a continuous variable. 

		Further, because \( t \) is continuous, then \( x \) must also be continuous.  If the signal is differentiable, 
		then the derivative \( \dv{x(t)}{t} \) also exists. 

		\question{\( t \) being continuous does not imply that \( x(t) \) is continuous (e.g. Thomae function),
		but is it true for this class?} 
	\item \textbf{Discrete-Time:} These are signals defined over discrete variables. For instance, if we had 
		\( x[n] \) as a signal, where \( n \) is an integer. 

		We don't have a concept of differentiability, but we can compute the difference: \( x[n] - x[n- 1] \), and 
		talk about that quantity. 
	\item \textbf{Real-Valued:} A signal \( x(t) \) is real-valued if \( x(t) \in \mathbb R \), where \( \mathbb R \) 
		denotes the set of all real numbers. 
	\item \textbf{Complex-Valued:} A signal \( x(t)  \) is complex-valued if \( x(t) \in \mathbb C \), where 
		\( \mathbb C \) denotes the set of complex numbers. 
	\item Note that while we're using the continuous-time notation here, the same concepts apply with discrete-time 
		signals. 

		\begin{itemize}
			\item Quick recap on complex numbers: denoted by \( a + bj \) or \( a + bi \), where \( i \) and \( j \) 
				denote the imaginary unit. 
			\item They are defined as \( i^2 = -1 \) or \( j^2 = -1 \).
			\item \( a \) is the real part, and \( b \) is the imaginary part. 
			\item We can plot these values in the complex plane, using the real and imaginary representation:
				\begin{center}
					\begin{tikzpicture}
						\draw[-stealth, thick] (-3, 0) -- (3, 0) node[above] {real};
						\draw[-stealth, thick] (0, -3) -- (0, 3) node[above right] {imaginary};
						\draw[red] (0, 0) -- (1.8, 1.4) node[above right, black] 
							{\( z = a + bj \) }; 
						\filldraw[red] (1.8, 1.4) circle (2pt); 
						\draw[dashed] (1.8, 1.4) -- (1.8, 0) node[below] {\( a \) };
						\draw[dashed] (1.8, 1.4) -- (0, 1.4) node[left] {\( b \) };
					\end{tikzpicture}
				\end{center}
				Or using the magnitude-phase representation:
				\begin{center}
					\begin{tikzpicture}
						\draw[-stealth, thick] (-3, 0) -- (3, 0) node[above] {real};
						\draw[-stealth, thick] (0, -3) -- (0, 3) node[above right] {imaginary};
						\draw[red] (0, 0) -- node[midway, above left] {\( m \) } (1.8, 1.4) node[above right, black] 
							{\( z = m\cdot e^{j \theta} \) }; 
						\filldraw[red] (1.8, 1.4) circle (2pt); 
						\draw[red] (1, 0) arc (0:37.87:1) node[midway, right]{\( \theta \) };
						\draw[dashed] (1.8, 1.4) -- (1.8, 0) node[below] {\( m \cdot \cos (\theta) \) };
						\draw[dashed] (1.8, 1.4) -- (0, 1.4) node[left] {\( m \cdot \sin (\theta) \) };
					\end{tikzpicture}
				\end{center}
				We represent the magnitude as \( m = |z| \), and the phase angle \( \theta \) is the angle made 
				with the real axis. 
		\end{itemize}
	\item \textbf{Periodic Signal:} Two quantities we'll introduce here: the period \( T \) is the time it takes 
		for the signal to repeat itself. \( T \) is measured in units of time, generally seconds.
		
		The frequency \( f \) is the "inverse" of period, defined by \( f = \frac{1}{T} \). We will also use 
		the angular frequency \( \omega \), defined by \( \omega = \frac{2\pi}{T} = 2 \pi f \). Angular frequency
		is mainly going to be used when we involve complex numbers. We will see:
		\[
		e^{j \omega t} = e^{j (2 \pi f t)} = \cos(2 \pi ft) + i \sin (2 \pi ft)
		\] 
	\item \textbf{Dimensionality:} We will deal with multi-dimensional signals: an example of a 2D signal are images, 
		which determine the color of a pixel based on a row and column. The spaces that we'll be working with are 
		either \( \mathbb R^{n} \) or \( \mathbb C^{n} \). 
\end{itemize}
\subsection{Signal Transformations}
\begin{itemize}
	\item \textbf{Shifts:} Essentially just shifts the signal along one dimension: \( x(t) \to x(t - T) \). \( T \) 
		is some constant. If  \( T > 0 \), then the shift is to the \textit{right}, and if \( T < 0 \) then 
		the shift is to the \textit{left}.
	\item \textbf{Scaling:} We can multiply a signal \( x(t)  \) by some constant \( a \) : \( x(t) \to a\cdot x(t) \).
		If \( a < 1 \), then we shrink \( x(t) \), and if \(  a > 1 \) then we amplify the signal. 
	\item \textbf{Reversal:} Given \( x(t) \), we can "reverse time" by adding a negative to the argument: \( x(t)
		\to x(-t)\). Visually, all we do is flip the signal around the \( y \)-axis.  
\end{itemize}
\subsection{Signal Properties}
\begin{itemize}
	\item \textbf{Even:} Functions which satisfy \( x(t) = x(-t) \). In other words, if we perform a reversal, the 
		signal stays the same.
	\item \textbf{Odd:} Functions which satisfy \( x(t) = -x(-t) \). If we perform a reversal, the signal becomes the 
		negative of itself.
	\item \textbf{Periodic:} If \( T \) is the period, then \( nT \) is also a period for any \( n \in \mathbb Z \).
		However, we will call \( T \) the fundamental period; the smallest \( T \) for which the function 
		repeats. 

		For the function \( \sin(2 \pi ft) \), the fundamental period is \( 1 / f \).  
\end{itemize}
\subsection{Model Functions}
\begin{itemize}
	\item These are called model functions because they're idealized models to analyze. 
	\item \textbf{Heaviside Step function:} For the continuous-time case it's usually modeled by:
		\[
		u(t) = \begin{cases}
			0 & \text{for \( t < 0 \)}\\
			1 & \text{for \( t \ge 0 \)}
		\end{cases}
		\] 
		In the discrete-time case, it's written as:
		\[
			u[n] = \begin{cases}
				0 & \text{for \( n < 0 \)}\\
				1 & \text{for \(n \ge  0\)}
			\end{cases}
		\] 
\end{itemize}



