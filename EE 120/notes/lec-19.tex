\section{Transfer Function of LTI Systems}
\begin{itemize}
	\item For an LTI system with the standard input and output pairs, we know that since \( y(t) = x(t) * h(t) \), 
		then this means that 
		\[
		Y(s) = X(s) H(s)
		\] 
		where \( H(s) \) denotes the Laplace transform of the impulse response, 
		\[
		H(s) = \int_{-\infty}^{\infty} h(t) e^{-st}\diff t 
		\] 
	\item For an LCCDE of the form:
		\[
			\sum_{k=0}^{N} a_k \dv[k]{y(t)}{t} = \sum_{k=0}^{M} b_k \dv[k]{x(t)}{t}
		\] 
		then we can take the Laplace transform of both sides, 
		\[
		\sum_{k=0}^{N} a_k s^{k}Y(s) = \sum_{k=0}^{M} b_k s^{k}X(s)
		\] 
		So, this means that:
		\[
		H(s) = \frac{Y(S)}{X(s)} = \frac{\sum_{k=0}^{\infty} b_k s^{k}}{\sum_{k=0}^{N} a_k s^{k}}
		\] 
\end{itemize}
\subsection{RLC Circuit}
\begin{itemize}
	\item Consider the following circuit:
		\begin{center}
			\begin{circuitikz}[american]
				\draw (0, 3) to [voltage source, l_ = \( x(t) \)] (0, 0);
				\draw (0, 3) to [R] (2, 3) to [R] (4, 3);
			\end{circuitikz}
		\end{center}
\end{itemize}
\subsection{Effect of Poles}
\begin{itemize}
	\item For non-repeated poles (as in, the poles aren't degenerate), then the transfer function can 
		be written generically as:
		\[
		H(s) = \sum_{i=1}^{N} \frac{A_i}{s - a_i}
		\] 
		If the system is causal, then recall that for an LTI system this means that \( h(t) = 0 \) for \( t < 0 \)
		\[
		h(t) = \sum_{i=1}^{N} A_i e^{-a_i t}u(t)
		\] 
	\item If \( \alpha_i \) is repeated \( m \) times, then the system response will include these terms:
		\[
			t^{m-1}e^{\alpha_i t}, \cdots,te^{\alpha_1}t, e^{\alpha_i t}
		\] 
\end{itemize}
\subsection{Stability of Causal System}
\begin{itemize}
	\item Given a causal system described by  \( H(s) \), we saw earlier that \( Y(s) = H(s) X(s) \). There's also 
		a theorem that says that the system is stable if and only if all poles of \( H(s) \)  have strictly negative 
		real parts. 
		This is because of two reasons:
		
		\begin{enumerate}[label=\arabic*.]
			\item This is because if \( H(s) \) is rational, 
				then the causality is equivalent to the region of convergence to 
				the right of the rightmost pole (on the real line)
			\item The absolute integrability condition \( \int_{-\infty}^{\infty} |h(t)|\diff t < \infty \) means 
				that the imaginary axis is within the ROC. 
		\end{enumerate}
\end{itemize}
\subsection{Connected LTI Sytems}
\begin{itemize}
	\item For systems connected in series, then \( H(s) = H_1(s)H_2(s) \), and in parallel then
		\( H(s) = H_1(s) + H_2(s) \). This is the exact same as the Fourier transform \( H(\omega) \) properties. 
	\item For feedback control systems (e.g. a compensator), then we define an error function 
		\( E(s) = X(s) - H_2(s) Y(s) \). The subtraction comes from the fact that the feedback is fed 
		as a minus sign. Then, \( Y(s) = H_1(s) E(s) \), and solving both equations 
		gives:
		\[
		Y(s) = H_1(s) X(s) - H_1(s) H_2(s) Y(s) \implies H(s) = \frac{Y(s)}{X(s)} = \frac{H_1(s)}{1 + H_1(s)H_2(s)}
		\] 
\end{itemize}
\subsection{Z-transform}
\begin{itemize}
	\item Very much the same as the Laplace transform, but for discrete-time signals. 
	\item This is very similar to the DTFT formula:
		\[
			X(e^{j \omega}) = \sum_{n=-\infty}^{\infty} x[n] e^{-j \omega n}
		\] 
		This is periodic because the signal is discrete in time. The z-transform is the generalized version of this 
		formula, written as:
		\[
			X(z) = \sum_{n=-\infty}^{\infty} x[n] z^{-n}, z \in \C
		\] 
		so really, the only difference is that we swapped \( e^{-j \omega} \) for a general complex 
		number \( z \). Because \( z \) also has a magnitude term, there's also a corresponding region of 
		convergence for \( z \). 
\end{itemize}
\subsection{Pairs}
\begin{itemize}
	\item Given \( x[n] = \delta[n] \), then:
		\[
			X(z) = \sum_{n=-\infty}^{\infty} \delta[n] z^{-n} = 1
		\] 
		as long as \( z \neq 0 \). With that caveat in mind, the ROC is the entire complex plane. 
	\item Given \( x[n] = a^{n}u[n] \):
		\[
			X(z) = \sum_{n=-\infty}^{\infty} a^{n}u[n] z^{-n} = \sum_{n=0}^{\infty} a^{n}z^{-n}
			= \sum_{n=0}^{\infty} (az^{-1})^{n} = \frac{1}{1 - (az^{-1})}
		\] 
		the last step simplifies due to geometric series. The condition for this to converge is 
		that \( |az^{-1}| < 1 \), since that's when the geometric series converges. So this simplifies 
		to \( |z| > |a| \), or \( |z| > a \) since \( a \) is real. 

		Visually, this corresponds to a boundary at infinity and one that's a circle at radius \( a \). Consequently, 
		if the unit circle exists within the region of convergence, then we know that the DTFT of the signal 
		exists. 
	\item Given \( x[n] = -a^{n}u[-n - 1] \):
		\[
			X(z) = \sum_{n=-\infty}^{\infty} -a^{n}u[-n - 1] z^{-n} = \sum_{n=-\infty}^{-1} -a^{n}z^{-n}
			= 1 - \sum_{n=0}^{\infty} (-a^{-1}z)^{n}  = 1 - \frac{1}{1 - a^{-1}z}
		\] 
		The region of convergence is defined similarly: \( |za^{-1}| < 1 \), so \( |z| < |a| \). This is just a
		filled circle up to a radius of \( |a| \). Of course, the DTFT exists only when the radius of this circle 
		is larger than 1. 
	\item Given  \( x[n] = -\left( \frac{1}{2} \right)^{n}u[-n - 1] + \left( -\frac{1}{3} \right)^{n}u[n] \), 
		since the z-transform is linear, then:
		\[
		X(z) = \frac{1}{1 - 2z } + \frac{1}{1 + z^{-1} / 3} = \cdots = \frac{6z^2 - 6z - 1}{(3z + 1)(2z - 1)}
		\] 
\end{itemize}
\subsection{Z-transform properties}
\begin{itemize}
	\item The ROC for z-transforms are circles or rings instead of planes, and that's just becuase we're 
		converting \( e^{j \omega} \) to \( z \). 
	\item See the lecture slides for a table on the properties. 
\end{itemize}
\subsection{Inverse Z-transform}
\begin{itemize}
	\item We can compute the inverse Z-transform using partial fraction expansion. The reason we keep going back 
		to this is because many Z-transforms are characterized by rational functions, so if we can find a way to 
		split the fractions up then we can find the inverse from linearity.  
\end{itemize}




