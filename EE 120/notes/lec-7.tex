\section{Lecture 7}
\subsection{DTFT and Convergence}
\begin{itemize}
	\item Not all functions have a Fourier transform, and the problem of whether a function has a Fourier 
		integral is an incredibly complex problem with no simple statement. 
	\item However, we know that there are several sufficient (but not necessary) conditions. Firstly, 
		we know that \( x(t) \) must be absolutely integrable. That is, 
		\[
		\int_{-\infty}^{\infty} |x(t)| \diff t < \infty
		\] 
\end{itemize}
\subsection{Fourier Transform Pairs}
\begin{itemize}
	\item There are several pairs of Fourier transforms that are useful to memorize.
	\item The Delta function:
		\[
			x(t) = \delta(t - t_0) \leftrightarrow X(f) = e^{-j 2\pi ft_0}
		\]
		This actually has strong implications about the nature of the Fourier transform -- there is an 
		"uncertainty principle" that manifests itself here. A signal cannot be both localized in time and frequency at
		the same time.
	\item Complex exponentials:
		\[
		x(t) = e^{j \omega_0 t} = e^{j 2 \pi f_0 t} \leftrightarrow X(f) = \delta(f - f_0)
		\] 
		This is the same as the previous point, except now we're going backwards.   
	\item Cosine functions:
		\[
		x(t) = \cos(2 \pi f_0 t) \leftrightarrow X(f) = \frac{1}{2}\delta(f - f_0) + \delta(f + f_0))
		\] 
		This makes sense: a plane wave is a composition of a left and right travelling wave.  
	\item Sine functions:
		\[
		x(t) = \sin(2 \pi f_0 t) \leftrightarrow X(f) = \frac{1}{2j}(\delta(f - f_0) - \delta(f + f_0))
		\] 
		Note that the only difference here is the minus sign, as a result of the conversion of sine into 
		complex exponentials.
	\item Shah function:
		\[
		x(t) = III(t) \leftrightarrow X(f) = III(f) \text{ or } X(\omega) = \frac{1}{2\pi}III(\omega)
		\] 
	\item Rect function:
		\[
		x(t) = \sqcap(t) \leftrightarrow \sinc(f)
		\] 
\end{itemize}
