\section{2D Image Processing}
\begin{itemize}
	\item To start, we first characterize an image as a grid with 
		\( M  \) rows and \( N \) columns, which we denote as 
		living in \( \R^{M \times N} \). 
	\item So, a digital image can be represented as a matrix, generally 
		written as \( f[m, n] \in \R^{M \times N} \). The value of the 
		entry can symbolize an intensity (in grayscale).
	\item With color images, you can define them in RGB colors, then this 
		is a \( M \times N \times 3 \) dimensional matrix. Here, 
		each pixel \( f[m, n] \) is represented by a vector 
		\( (r, g, b) \in [0, 1] \), 
	\item Images can also be complex-valued, where each entry \( f[m, n] \) now has a magnitude and phase. 
		 \[
			 f[m, n] = \text{mag}[m, n] e^{i \text{phase}[m, n]}
		\] 
		We can now
		plot two different maps: one that just shows magnitude, and another that shows phase. 
\end{itemize}
\subsection{Sampling}
\begin{itemize}
	\item We can downsample in two different directions:
		\[
			x[m, n] \to x[2m, 3n]
		\] 
		so this means that we down sampled by a factor of 2 in rows and a factor of 3 in columns. So, the number of 
		rows reduces by a factor of 2, and the columns by a factor of 3. This is one way that we can reduce 
		the amount of information contained in an image, so that it may be sent over channels that can't handle 
		that much capacity. 
	\item We can also upsample, in the same way we defined it for one-dimensional signals.
		\[
			x[m, n] = \begin{cases}
				x\left[ \frac{m}{2}, \frac{n}{3} \right] & \text{\( m \) is a multiple of 2 and \( n \) is a multiple of 3}\\
				0 & \text{else}
			\end{cases}
		\] 
		so this is the way we increase the length of the signal, by basically sandwiching zeros between our known 
		values. The strategy to do this is to do the rows first, then do the columns.    
	\item The issue with upsampling is that because we fill in the matrix with zeros, it introduces undesirable 
		lines into the final image, which we can fix using interpolation.   
\end{itemize}
\subsection{Fourier space} 
\begin{itemize}
	\item Recall with discrete 1D sampling that in Fourier space, because upon upsampling we increase the 
		width in time, we decrease the width in frequency. What this means is that over the range 
		\( [-\pi, \pi] \), which is what we originally had, we now will have multiple copies of the 
		original image. 
	\item Then, the way to fix this we pass image through an ideal 2D low-pass filter, which will give us back 
		a single copy of the image. 
	\item We can also realize the same thing in ``time'', by interpolating with a sinc function. \question{Is this 
		convoluiton here?}
\end{itemize}
\subsection{2D Convolution} 
\begin{itemize}
	\item Here, we have a 2D continuous-time signal \( f(x, y) \) and \( g(x, y) \), and convolving them together 
		\( f * * g \):
		\[
		f * * g = \int_{-\infty}^{\infty} \int_{-\infty}^{\infty} f(x', y') g(x - x', y - y') \diff x', \diff y'  
		\] 
		This is exactly the formula that we derived a while back.   
	\item In discrete-time, it's basically the same:
		\[
			f * * g = \sum_{k=-\infty}^{\infty} \sum_{l=-\infty}^{\infty} f[k, l] g[m - k, n - l] 
		\] 
		Here, just like the other case, \( m, n \) is the coordinate, and the summation is over \( k \) and \( l \). 
	\item Visually, there's a couple ways to see what this actually is doing:
		\begin{enumerate}[label=\arabic*.]
			\item Flip the second matrix in two dimensions, shift in 2D, then multiply them element-by-element 
				and sum them together. So, we basically overlay everything on top of each other.  
		\end{enumerate}
\end{itemize}
