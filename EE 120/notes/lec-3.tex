%\section{Characterization Continued}
\section{Lecture 3}
\subsection{Step Response}
\begin{itemize}
	\item The step response function is the function \( y_{\text{step}}(t) \) when a step function 
		\( u(t) \) is fed into the system. In discrete-time: we feed \( u[n] \) into the system, and get 
		\( y_{\text{step}}[n] \) as an output.
	\item For instance, for the moving average filter defined earlier, we have the following result:
		\begin{center}
			\begin{tabular}{c|c}
				\( n \) &  \( y_{\text{step}}[n] \)\\
				\hline 
				-2 & 0\\
				-1 & 1/3\\
				0 & 2/3\\
				1 & 1\\
				2 & 1
			\end{tabular}
		\end{center}
		Note that this resembles a ramp function, and is called a ramp-step function.
	\item \textbf{Harmonic Response:} The harmonic response is the response by the system when presented with 
		a harmonic function, of the form \( Ae^{i \omega t}\). 

		In discrete time, we feed in \( Ae^{i \omega n} \) where \( n \) is an integer. 
	\item For the moving average filter, let's write out \( y[n] \) :
		\begin{align*}
			y[n] &= \frac{1}{3}\left(Ae^{i \omega (n - 1)} + Ae^{i \omega n} + Ae^{i \omega (n + 1)}\right)\\
			&= \frac{1}{3}\left( e^{-i \omega} + 1 + e^{i \omega} \right)  \\
			&= \frac{1}{3}(2 \cos \omega + 1) Ae^{i \omega n} \\
		\end{align*}
		The interesting thing here is that when given a harmonic function, the system response just scales the signal 
		by a constant amount!
\end{itemize}
\subsection{LCCDE} 
\begin{itemize}
	\item In this class, we will deal with lots of differential equations, so it's going to be very useful to 
		look at their form, and how to solve them.   
	\item There are two solutions to any differential equation: 
		\begin{itemize}
			\item \textbf{Particular Solution:} \( y_p(t) \) is called a particular solution if it satisfies:
				\[
					\sum_{k = 0}^{N}a_k \dv[k]{y_p(t)}{t} = \sum_{k = 0}^{N}b_k \dv[k]{x(t)}{t}
				\] 
			\item \textbf{Homogeneous Solution:} \( y_h(t) \) is called a homogeneous solution if it satisfies:  
				\[
						\sum_{k = 0}^{N}a_k \dv[k]{y_p(t)}{t} =	0
				\] 
		\end{itemize}
	\item In general, the solution will be a linear combination of the two: 
		\[
		y(t) = y_p(T) + ay_h(t)
		\] 
		the value of \( a \) is generally going to be given by some initial condition. 
	\item For the homogeneous solution, an ansats of the form \( Ae^{st} \) where \( s \) is an undetermined constant 
		will solve the differental equation. We can then determine the value of \( s \) by solving the resulting 
		polynomial.

		To determine the value of \( A \), these are determined by the initial conditions, and depending on the 
		number of initial conditions given, that would correspond directly to the number of distinct values of \( A \). 
\end{itemize}
\subsubsection{Example}
\begin{itemize}
	\item Given a first order LCCDE, with a step function input (this means that the right hand side is a 
		step function). This means we're solving for a solution 
		\( y(t) \) to:
		\[
			\dv{y(t)}{t} + ay(t) = bx(t) = bu(t)
		\] 
	\item First we look for the homogeneous solution, which will give us:
		\[
			\dv{y_h(t)}{t} + ay_h(t) = 0
		\] 
		This is a separable DE, so we move \( ay_h(t)  \) to the right hand side and integrate, 
		which gives us a solution of the form:
		\[
		y_h(t) = Ae^{-at}
		\] 
	\item The particular solution is the function \( y_p(t) \) that satisfies:
		\[
			\dv{y_p(t)}{t} + ay_p(t) = bu(t)
		\] 
		Now, we break this up into what values \( u(t)  \) takes along the real line. For \( t < 0 \), we have 
		\( u(t) =0\), which gives us a solution of \( y_p(t) = 0 \). Note that even though the right hand side 
		is zero doesn't mean this is homogeneous, since the homogeneous solution has 0 on the right hand side 
		\textit{for all \( t \)}.


	\item For \( t > 0 \), the differential equation becomes:
		\[
			\dv{y_p(t)}{t} + ay_p(t) = b
		\] 
		We can rearrange this slightly:
		\[
			\dv{\left( y_p(t) - \frac{b}{a} \right)}{t} + a\left( y_p(t) - \frac{b}{a} \right) = 0
		\] 
		You can verify that this is actually the same differential equation, since the derivative of a constant 
		is zero. But now, we can define \( z(t) = y_p(t) - \frac{b}{a} \), and solve instead a homogeneous differential
		equation for \( z(t) \). Going through the same steps as before, we have \( y_p(t) = \frac{b}{a} + Be^{-at} \). 
	\item We can combine both \( t < 0 \) and \( t > 0 \) with a nice analytical form:
		\[
		y_p(t) = \left( \frac{b}{a} + Be^{-at} \right) u(t)
		\] 
		\question{Is this necessary when possible?}
	\item The general solution is then:
		\[
		y(t) = y_p(t) + y_h(t) = \frac{b}{a} + Be^{-at} + Ae^{-at}
		\] 
		We need an initial condition to solve for \( A \) and \( B \) :  
		\begin{itemize}
			\item Initial rest condition: at \( t = 0 \), no input, so the output should also be 0. This 
				gives the equation:
				\[
				A + B = -\frac{b}{a}
				\]
		\end{itemize}				
\end{itemize}

