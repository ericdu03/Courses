\documentclass[10pt]{article}
\usepackage{../../local}
\urlstyle{same}

\newcommand{\classcode}{Physics 110B}
\newcommand{\classname}{Electromagnetism and Optics II}
\renewcommand{\maketitle}{%
\hrule height4pt
\large{Eric Du \hfill \classcode}
\newline
\large{HW 04} \Large{\hfill \classname \hfill} \large{\today}
\hrule height4pt \vskip .7em
\small{Header styling inspired by Berkeley EECS Department: \url{https://eecs.berkeley.edu/}}
\normalsize
}
\linespread{1.2}
\begin{document}
	\maketitle
	\section*{Problem 1}
	In this problem we will consider the reflection and transmission of EM waves in linear medium with the
	oblique incidence of a monochromatic sinusoidal plane wave and the polarization of the incidence wave to
	be \textit{perpendicular} to the plane of incidence. The coordinates are set up in a way as shown in the
	right figure. (You might need to zoom in with the pdf file to see the angle \( \phi_R \) and \( \phi_T
	\)). In this problem, you can assume that Snell's law holds, and we define the parameters \( \beta \equiv
	\frac{\mu_1v_1}{\mu_2v_2} \) and \( \alpha \equiv \frac{\cos \theta_2}{\cos \theta_1} \). 
	\begin{enumerate}[label=(\alph*)]
		\item Using boundary conditions, show that the reflected and transmitted waves have the same
			polarization as the incident wave. That is, show that 
			\[
				\phi_R = \phi_T = \frac{\pi}{2}
			\]

			\begin{solution}
				First, we break each field down via vector decomposition:
				\begin{align*}
					\mathbf{E}_I &= E_I \mathbf{\hat{y}}\\
					\mathbf{E}_R &= E_R \left[\sin \phi_R \mathbf{\hat{y}} + \cos \phi_R \sin \theta_1
					\mathbf{\hat{z}} + \cos \phi_R \cos \theta_1 \mathbf{\hat{x}} \right]\\
					\mathbf{E}_T &= E_T \left[ \sin \phi_T \mathbf{\hat{y}} + \cos \phi_T \sin \theta_2
						\mathbf{\hat{z}} - \cos
					 \phi_T \cos \theta_2 \mathbf{\hat{x}}\right]
				\end{align*}
				The \( \mathbf{B} \) field can also be found, using \( \mathbf{B} = \frac{1}{c}\mathbf{k}
				\times \mathbf{E} \) to get the direction:
				\begin{align*}
					\mathbf{B}_I &=  B_I (\sin \theta_1 \mathbf{\hat{z}} - \cos \theta_1 \mathbf{\hat{x}}) \\ 
					\mathbf{B}_R &= B_R (\cos \phi_R \mathbf{\hat{y}} + \sin \phi_R \sin \theta_1
					\mathbf{\hat{z}} + \sin \phi_R \cos \theta_1 \mathbf{\hat{x}}) \\ 
					\mathbf{B}_T &= B_T(\cos \phi_T \mathbf{\hat{y}} + \sin \phi_T \sin \theta_2
					\mathbf{\hat{z}} - \sin \phi_T \cos \theta_2 \mathbf{\hat{x}}) 
				\end{align*}
				The boundary conditions are imposed just the same as in lecture:
				\begin{align}
					\label{cond1}\epsilon_1(\mathbf{E}_I + \mathbf{E}_R)_z &= \epsilon_2 (\mathbf{E}_T)_z\\
					\label{cond2}(\mathbf{E}_I + \mathbf{E}_R)_{x, y} &= (\mathbf{E}_T)_{x, y} \\ 
					\label{cond3}(\mathbf{B}_I + \mathbf{B}_R)_z &= (\mathbf{B}_T)_z\\
					\label{cond4}\frac{1}{\mu_1}(\mathbf{B}_I + \mathbf{B}_r)_{x, y} 
								&=  \frac{1}{\mu_2}(\mathbf{B}_T)_{x, y} 
				\end{align}
				So, imposing all the boundary conditions, we have:
				\begin{align*}
					\text{\ref{cond1}:}& \quad \epsilon_1 E_R \cos \phi_R \sin \theta_1 = \epsilon_2 E_T \cos
					\phi_T \sin \theta_2\\
					\text{\ref{cond2}:}& \quad \begin{cases}
						E_R \cos \phi_R \cos \theta_1 = - E_T \cos \phi_T \cos \theta_2 & \text{(x
						direction)}\\
						E_I + E_R \sin \phi_R = E_T \sin \phi_T & \text{(y direction)}
					\end{cases}\\
					\ref{cond3}:& \quad \frac{1}{v_1}(E_I \sin \theta_1 + E_R \sin \phi_R \sin \theta_1) =
					\frac{1}{v_2}E_T \sin \phi_T \sin \theta_2\\
					\ref{cond4}:& \quad \begin{cases}
						\frac{1}{\mu_1v_1}(-E_I \cos \theta_1 + 
						E_R \sin \phi_R \cos \theta_1) =
						-\frac{1}{\mu_2v_2}E_T \sin \phi_T \cos \theta_2 & \text{(x direction)}\\
						\frac{1}{\mu_1v_1}E_R \cos \phi_R = -\frac{1}{\mu_2v_2}E_T \cos \phi_T & \text{(y
						direction)}
					\end{cases}
				\end{align*}
				Now, combining the first part of \ref{cond2} and second part of \ref{cond4}, we get:
				\[
					-\beta E_T \cos \phi_T \cos \theta_1 = - E_T \cos \phi_T \cos \theta_2 \implies E_T \cos
					\phi_T(\cos \theta_2 - \beta \cos \theta_1) = 0
				\]
				We know that \( \cos \theta_2 - \beta \cos \theta_1 \) is not always zero, so this leaves \(
				\cos \phi_T = 0 \), which requires \( \phi_T = \frac{\pi}{2} \). Plugging this back into the
				second part of \ref{cond4}, this also implies \( \phi_R = \frac{\pi}{2} \). 
			\end{solution}
		\item Show that the complex amplitudes \( \tilde E_{0I}, \tilde E_{0R} \) and \( \tilde E_{0T} \) are
			related to each other by 
			\[
				\tilde E_{0R} = \left( \frac{1 - \alpha \beta}{1 + \alpha \beta} \right)\tilde E_{0I} \quad
				\text{and} \quad \tilde E_{0T} = \left( \frac{2}{1 + \alpha \beta} \right)\tilde E_{0I}
			\]
			\begin{solution}
				With the condition on \( \phi_R \) and \( \phi_T \), it helps to simplify the boundary
				conditions, so \( \sin \phi_R = 1 \) and \( \cos \phi_R = 0 \):
				\begin{align*}
					\text{\ref{cond2}:}& \quad E_I + E_R = E_T\\
					\text{\ref{cond3}:}&\quad \frac{1}{v_1}(E_I \sin \theta_1 + E_R \sin \theta_1) =
					\frac{1}{v_2}E_T \sin \theta_2\\
					\text{\ref{cond4}:}& \quad \frac{1}{\mu_1v_1}(-E_I \cos \theta_1 + E_R \cos \theta_1) =
					-\frac{1}{\mu_2v_2}E_T \cos \theta_2
				\end{align*}
				The third equation can be written in terms of \( \alpha \) and \( \beta \):
				\[
					-E_I + E_R = - \alpha \beta E_T
				\]
				Now substitute in the first equation:
				\[
					-E_I + E_R = - \alpha \beta (E_I + E_R) \implies E_R = \left( \frac{1-\alpha \beta}{1 +
					\alpha \beta} \right)E_I
				\]
				as desired. Substituting this back into the first equation, we have:
				\[
					E_I + \left( \frac{1 - \alpha \beta}{1 + \alpha \beta} \right)E_I = E_T \implies E_T =
					\left(\frac{2}{1 + \alpha \beta}\right) E_I
				\]
				as desired. 
			\end{solution}
		\item In the case where \( \mu_1 \simeq \mu_2 \simeq \mu_0 \), show that the reflected wave has a \(
			\pi\)-phase shift with respect to the incident wave then \( v_2 < v_1 \). 

			\begin{solution}
				In this case, then \( \beta \approx \frac{v_1}{v_2} > 1 \), so all we need to show is that \(
				1 - \alpha \beta < 1\), or in other words \( \alpha > 1 \). We show this by rewriting \( \alpha
				\) using Snell's law
				\[
					\alpha = \frac{\cos \theta_2}{\cos \theta_1} = \frac{\sqrt{1 - \sin^2 \theta_2}}{\cos
					\theta_1} = \left( \frac{1 - \left( \frac{v_2}{v_1} \right)^2 \sin^2 \theta_1}{1 - \sin^2
					\theta_1} \right)^{1 / 2}
				\]
				Now, take a look at this quantity. We know that \( \frac{v_2}{v_1} < 1 \), so the quantity in
				the numerator is larger than that in the denominator. Therefore, the fraction is larger than
				one, so \( \alpha > 1 \). Therefore \( \alpha \beta > 1 \), and as such \( 1 - \alpha \beta <
				1 \), so that introduces a minus sign:
				\[
					E_R = - \frac{|1 - \alpha \beta|}{|1 + \alpha \beta|}E_I 
				\]
				which is equivalent to a \( \pi \) phase shift. 
			\end{solution}
		\item Is there \textit{Brewster's angle} when the incident wave has a polarization perpendicular to
			the plane of incidence? That is, is there an incident angle such that \( \tilde E_{0R} = 0 \)?  

			\begin{solution}
				Continuing the math from the previous section, we know that we can write:
				\[
					\alpha = \frac{1}{\beta}\frac{\sqrt{\beta^2 - \sin^2 \theta_1}}{\cos \theta_1}
				\]
				For \( \tilde E_{0R} = 0 \), then we require \( 1 -\alpha \beta = 0 \), so we are looking for 
				an angle \( \theta_1 \) such that:
				\[
					1 - \frac{\sqrt{\beta^2 - \sin^2 \theta_1}}{\cos \theta_1} = 0 \implies \cos \theta_1
					= \sqrt{\beta^2 - \sin^2 \theta_1}
				\]
				This eventually simplifies to \( \beta = 1 \), which is the case where the two media are
				identical (i.e. no boundary exists). Therefore, there is no Brewster's angle. 
			\end{solution}
		\item Consider the \( z \)-component of the Poynting vector.
			\[
				\mean{|\mathbf{S}_I \cdot \hat{\mathbf{z}}|} = \frac{|\tilde E_{0I}|^2}{2\mu_1v_1}\cos
				\theta_1, \quad \mean{|\mathbf{S}_R \cdot \hat{\mathbf{z}}|} = \frac{|\tilde
				E_{0R}|^2}{2\mu_1v_1} \cos \theta_1, \quad \text{and} \quad \mean{|\mathbf{S}_T \cdot
			\hat{\mathbf{z}}|} = \frac{|\tilde E_{0T}|^2}{2 \mu_2v_2}\cos \theta_2
			\]
			where \( \mean{\dots} \) means time average. The reflection and transmission coefficients are
			defined as
			\[
				R\equiv \frac{\mean{|\mathbf{S}_R \cdot \hat{\mathbf{z}}|}}{\mean{|\mathbf{S}_I \cdot
				\hat{\mathbf{z}}|}} \quad \text{and} \quad T\equiv \frac{\mean{|\mathbf{S}_T \cdot
				\hat{\mathbf{z}}|}}{\mean{|\mathbf{S}_I \cdot \hat{\mathbf{z}}|}}
			\]
			Explicitly check that \( R +T = 1 \). 

			\begin{solution}
				We use the formulas that we have from part (b):
				\[
					R = \frac{\frac{|E_R|^2}{2 \mu_1v_1}\cos \theta_1}{\frac{|E_I|^2}{2 \mu_1v_1}\cos
					\theta_1} = \frac{|E_R|^2}{|E_I|^2} = \left( \frac{1 - \alpha \beta}{1 + \alpha \beta}
					\right)^2
					\quad
					T = \frac{\frac{|E_T|^2}{2 \mu_2v_2} \cos \theta_2}{\frac{|E_I|^2}{2 \mu_1v_1}\cos
					\theta_1} = \alpha \beta \frac{|E_T|^2}{|E_I|^2} = \alpha \left( \frac{2}{1 + \alpha
					\beta} \right)^2
				\]
				Adding these two up:
				\[
					 R + T = \left( \frac{1 - \alpha \beta}{1 + \alpha \beta} \right)^2 + \alpha \beta \left(
					 \frac{2}{1 + \alpha \beta}\right)^2 = \frac{(1 - \alpha \beta)^2 + 4 \alpha \beta}{(1 +
				 \alpha \beta)^2} = \frac{(1 + \alpha \beta)^2}{(1 + \alpha \beta)^2} = 1
				\]
				as desired. 
			\end{solution}
	\end{enumerate}

	\pagebreak
	\section*{Problem 2} 
	According to Snell's law, when light passes from an optically dense medium into a less dense one (\( n_1
	> n_2\)), the propagation vector \( \mathbf{k} \) bends \textit{away} from the normal (Fig. below). In
	particular, if the light is incident at the \textbf{critical angle}
	\[
		\theta_c \equiv \sin^{-1}(n_2 / n_1)
	\]
	then \( \theta_T = 90^{\circ} \), and the transmitted ray just grazes the surface. If \( \theta_I
	\)\textit{exceeds} \( \theta_c \), there is no refracted ray at all, only a reflected one (this is the
	phenomenon of \textbf{total internal reflection}, on which light pipes and fiber optics are based). But
	the \textit{fields} are not zero in medium 2; what we get is a so-called \textbf{evanescent wave}, which
	is rapidly attenuated and transports no energy into medium 2. 

	A quick way to construct the evanescent wave is simply to quote the results of Section 9.3.3, with \( k_T
	= \omega n_2 / c\) and
	\[
		\mathbf{k}_T = k_T (\sin \theta_T \hat{\mathbf{x}} + \cos \theta_T \hat{\mathbf{z}})
	\]
	the only change is that
	\[
		\sin \theta_T = \frac{n_1}{n_2}\sin \theta_I
	\]
	is now greater than 1, and 
	\[
		\cos \theta_T = \sqrt{1 - \sin^2 \theta_T} = i\sqrt{\sin^2 \theta_T - 1}
	\]
	is imaginary. (Obviously, \( \theta_T \) can no longer be interpreted as an angle!)
	\begin{enumerate}[label=(\alph*)]
		\item Show that 
			\[
				\mathbf{E}_T(\mathbf{r}, t) = \tilde{\mathbf{E}}_{0_T} e^{-\kappa z}e^{i(kx - \omega t)}
			\]
			where 
			\[
				\kappa \equiv \frac{\omega}{c} \sqrt{(n_1 \sin \theta_I)^2 - n_2^2} \quad \text{and} \quad
				k\equiv \frac{\omega n_1}{c}\sin \theta_I
			\]
			this is a wave propagating in the \( x \) direction (\textit{parallel} to the interface!), and
			attenuated in the \( z \) direction. 

			\begin{solution}
				We did this portion in lecture. Consider first that the transmitted wave can be written as:
				\[
					\mathbf{E}_T = \Re\left\{ E_{0T} e^{i(\mathbf{k}_T \cdot \mathbf{r} - \omega t)} \right\}
				\]
				Here, we write
				\[
					\mathbf{k}_T = k_T \cos \theta_T \mathbf{\hat{z}} + k_T \sin \theta_T \mathbf{\hat{x}}
				\]
				From the problem statement, we know that \( \cos \theta_T \) is purely imaginary, so we can
				write \( i \kappa \) as:
				\[
					i \kappa = k_T \cos \theta_T = k_T i\sqrt{\sin^2 \theta_T - 1} = i \frac{k_T}{n_2}
					\sqrt{(n_1 \sin \theta_I)^2 - n_2^2} = i \frac{\omega}{c}\sqrt{(n_1 \sin \theta_I)^2 -
					n_2^2}
				\]
				Now, we can do the same thing for the \( \mathbf{\hat{x}} \) direction, using Snell's law
				once again:
				\[
					k = k_T \sin \theta_T = k_T \frac{n_1}{n_2} \sin \theta_I = \frac{\omega}{c} n_1 \sin
					\theta_1
				\]
				Combining these two equations, we get the desired result:
				\[
					\mathbf{E}_T = \mathbf{E}_{0T}e^{i(i \kappa z+ k x - \omega t)} = \mathbf{E}_{0T}
					e^{-\kappa z}e^{i (kx - \omega t)}
				\]
				as desired. 
			\end{solution}
		\item Noting that \( \alpha \) is now imaginary, use Eq. 9.110 to calculate the reflection
			coefficient for polarization parallel to the plane of incidence. [Notice that you get 100 percent
			reflection, which is better than a conducting surface (see, for example, prob. 9.23).]

			\begin{solution}
				Equation 9.110 reads:
				\[
					\tilde E_{0R} = \left( \frac{\alpha - \beta}{\alpha + \beta} \right)\tilde E_{0I}
				\]
				Now that \( \alpha \) is purely imaginary, then we can write \( \alpha = ia \) for \( a \in
				\R \), so therefore:
				\[
					R = \left| \frac{\alpha - \beta}{\alpha + \beta} \right|^2 = \left| \frac{- \beta +
					ia}{\beta + ia} \right|^2 = \left| \frac{a^2 +
					\beta^2}{a^2 + \beta^2} \right| = 1
				\]
				as desired.   
			\end{solution}
		\item Do the same for polarization perpendicular to the plane of incidence (use the results of Prob.
			9.17). 

			\begin{solution}
				We will take the results from problem 1 of the homework. None of the math actually changes
				except for the fact that now \( \alpha \) is again imaginary, so we have:
				\[
					R = \left( \frac{1 - \alpha \beta}{1 + \alpha \beta} \right)E_I,  \quad T = \frac{2}{1 +
					\alpha \beta} E_I
				\]
				Again, let \( \alpha = ia \) where \( a \in \R \), so:
				\[
					R = \left| \frac{1 - ia \beta}{1 + ia \beta} \right|^2 = \frac{1 + a^2 \beta^2}{1
					+ a^2 \beta^2}  = 1	
				\]
				so this is the same. 
			\end{solution}
		\item In the case of polarization perpendicular to the plane of incidence, show that the (real)
			evanescent wave fields are
			\begin{align*}
				\mathbf{E}(\mathbf{r}, t) &= E_0e^{-\kappa z}\cos(kx - \omega t) \hat{\mathbf{y}}\\
				\mathbf{B}(\mathbf{r}, t) &=  \frac{E_0}{\omega}e^{-\kappa z}\left[ \kappa \sin (kx - \omega
				t) \mathbf{ \hat{x}} + k \cos(kx - \omega t) \hat{\mathbf{z}}\right] 
			\end{align*}

			\begin{solution}
				From part (a) we know that:
				\[
					\mathbf{E}_T = \mathbf{E}_{0T}e^{-\kappa z}e^{i(kx - \omega t)}
				\]
				From problem 1, we know that \( \mathbf{E}_{0T} \) points strictly in the \( \mathbf{\hat{y}}
				\) direction, so taking the real part:
				\[
					\mathbf{E}_T = E_{0T}e^{-\kappa z} \cos(kx - \omega t) \mathbf{\hat{y}}
				\]
				For the \( \mathbf{B} \) field, we know from problem 1 that:
				\[
					\mathbf{B}_T = B_T(\mathbf{r}, t) (\sin \theta_T \mathbf{\hat{z}} 
					- \cos \theta_T \mathbf{\hat{x}}) = \frac{E_T(\mathbf{r}, t)}{v} (\sin \theta_T
					\mathbf{\hat{z}} - \cos \theta_T \mathbf{\hat{x}})
				\]
				From here, it will be useful to define a new constant \( a \):
				\[
					a = \frac{1}{n_2}\sqrt{\sin^2 \theta_T - 1}
				\]
				so we may write \( \cos \theta_T = ia \). Likewise, we can substitute \( \sin \theta_T =
				\frac{n_1}{n_2}\sin \theta_I \). This leaves us with:
				\[
					\mathbf{B}_T(\mathbf{r}, t) = \frac{E_{0T}}{v} e^{-\kappa z}\Re\left\{ (\cos (kx - \omega
					t) + i \sin (kx - \omega t))\left( \frac{n_1}{n_2}\sin \theta_I \mathbf{\hat{x}} - ia
				\mathbf{ \hat{z}} \right)\right\}
				\]
				The real part is given by:
				\[
					\mathbf{B}_T(\mathbf{r}, t) = \frac{E_{0T}}{v} e^{-\kappa z}\left( \frac{n_1}{n_2}\sin
					\theta_I \cos (kx - \omega t) \mathbf{\hat{z}} + a \sin(kx - \omega t) \mathbf{\hat{x}} \right)
				\]
				Rewriting \( a \) in terms of \( \kappa \) and other constants, we find that \( a =
				\frac{\kappa c}{\omega n_2} \). So, we have:
				\begin{align*}
					\mathbf{B}_T(\mathbf{r}, t) &= \frac{E_{0T}}{v} e^{-\kappa z}\left( \frac{n_1}{n_2}\sin
					\theta_I \cos (kx - \omega t) \mathbf{\hat{z}} + \frac{\kappa c}{\omega n_2}
						\sin(kx - \omega t) \mathbf{\hat{x}} \right)\\
							&= \frac{E_{0T}}{v_2}e^{-\kappa z}\frac{1}{n_2}\frac{c}{\omega} 
							\left( n_1 \sin \theta_I \frac{\omega}{c} \cos (kx - \omega t)\mathbf{\hat{z}} 
							+ \kappa \sin (kx - \omega t)
							\mathbf{\hat{x}} \right) \\ 
							&= \frac{E_{0T}}{\omega}e^{-\kappa z}
							\left( k \cos (kx - \omega t)\mathbf{\hat{z}} + \kappa \sin
							(kx - \omega t) \mathbf{\hat{x}}\right) 
				\end{align*}
				as desired. Note that \( E_{0T} = E_0 \) defined in the problem statement. It took me while
				to realize that this isn't the incident magnitude, but instead just another way of denoting the
				magnitude of the transmitted wave.  
			\end{solution}
		\item Check that the fields in (d) satisfy all of Maxwell's equations (Eq. 9.68). 

			\begin{solution}
				We'll check these one by one. Starting with \( \nabla \cdot \mathbf{E} = 0 \):
				\[
					\nabla \cdot \mathbf{E} = \pdv{y} E_0e^{-\kappa z}\cos (kx - \omega t) = 0
				\]
				Now for \( \nabla \cdot \mathbf{B} = 0 \): 
				\begin{align*}
					\nabla \cdot \mathbf{B} &= \frac{E_{0T}}{\omega}\left( \pdv{x} e^{-\kappa z} \kappa \sin
					(kx - \omega t) + \pdv{z} e^{-\kappa z}\cos(kx - \omega t) \right)\\
					&= \frac{E_{0T}}{\omega}\left[ e^{-\kappa z}\kappa k \cos(kx - \omega t) - \kappa e^{-\kappa
					z} k \cos(kx - \omega t) \right] = 0 
				\end{align*}
				Now, \( \nabla \times \mathbf{E} = -\partial_t \mathbf{B} \), starting with \( \nabla \times
				\mathbf{E}\):
				\[
					\nabla \times \mathbf{E} = \epsilon^{ijk}\partial_j E_k = \pdv{x} E_y \mathbf{\hat{z}} -
					\pdv{z} E_y \mathbf{\hat{x}} = E_{0T} \kappa e^{-\kappa z}\cos (kx - \omega t)
					\mathbf{\hat{x}} - E_{0T} e^{-\kappa z} \kappa \sin(kx - \omega t) \mathbf{\hat{z}}
				\]
				Now for \( -\partial_t \mathbf{B} \):
				\begin{align*}
					-\partial_t \mathbf{B} &= -\frac{E_{0T}}{\omega} e^{-\kappa z}\left( -\kappa \omega
					\cos(kx - \omega t) \mathbf{\hat{x}} + k \omega \sin(kx - \omega t) \mathbf{\hat{z}}
				\right)\\
				&= E_0 e^{-\kappa z}(\kappa \cos (kx - \omega t) \mathbf{\hat{x}} - k \sin (kx - \omega t)
				\mathbf{\hat{z}})
				\end{align*}
				Indeed they are equal. Finally, for \( \nabla \times \mathbf{B} = \mu \mathbf{J} + \mu
				\epsilon \partial_t \mathbf{E} \), we first note that \( \mathbf{J} = 0 \) here, so we need
				to show \( \nabla \times \mathbf{B} = \mu \epsilon \partial_T \mathbf{E} \). Starting with \(
				\nabla \times \mathbf{B}\):
				\[
					\nabla \times \mathbf{B} = \epsilon^{ijk}\partial_j B_k = \pdv{z}B_x \mathbf{\hat{y}} -
					\pdv{y} B_x \mathbf{\hat{z}} + \pdv{y} B_z \mathbf{\hat{x}} - \pdv{x} B_z \mathbf{\hat{y}}
				\]
				Note that neither component of \( \mathbf{B} \) has \( y \)-dependence, so this simplifies
				to:
				\begin{multline*}
					\nabla \times \mathbf{B} = \pdv{z}B_x \mathbf{\hat{y}} - \pdv{x} B_z \mathbf{\hat{y}} =
					\frac{E_{0T}}{\omega}(-\kappa) e^{-\kappa z}\kappa \sin (kx - \omega t) -
					\frac{E_{0T}}{\omega} e^{-\kappa z}(-k^2 \sin (kx - \omega t) \\
					= \frac{E_{0T}}{\omega}
					e^{-\kappa z} \sin (kx - \omega t) (k^2 - \kappa^2)
				\end{multline*}
				Now, we compute \( \mu \epsilon \partial_t \mathbf{E} \):
				\[
					\mu \epsilon \partial_T \mathbf{E} = E_{0T}e^{-\kappa z}\omega \sin(kx - \omega t)
				\]
				To match these, we simplify \( k^2 - \kappa^2 \), using their definitions listed in the
				problem outline:
				\[
					k^2 - \kappa^2 = \left( \frac{\omega}{c}n_1 \right)\sin^2 \theta_I - \left[ \left(
					\frac{\omega}{c} \right)^2 (n_1 \sin \theta_I)^2 - \left( \frac{\omega n_2}{c} \right)^2
					\right]= \left( \frac{\omega n_2}{c} \right)^2
				\]
				Therefore:
				\[
					\nabla \times \mathbf{B} = \frac{E_{0T}}{\omega}e^{-\kappa z}\sin(kx - \omega t) \left(
					\frac{\omega n_2}{c} \right)^2 = E_{0T}e^{-\kappa z} \omega \sin (kx - \omega t) \left(
				\frac{n_2}{c} \right)^2
				\]
				Using the fact that \( \frac{n_2}{c} = \frac{1}{v_2} = \sqrt{\mu \epsilon} \), then we get
				the desired equality:
				\[
					\nabla \times \mathbf{B} = \mu \epsilon E_{0T}e^{-\kappa z} \omega \sin(kx - \omega t) =
					\mu \epsilon \partial_t \mathbf{E}
				\]
			\end{solution}
		\item For the fields in (d), construct the Poynting vector, and show that, on average, no energy is
			transmitted in the \( z \) direction. 

			\begin{solution}
				The Poynting vector is defined as:
				\[
					\mathbf{S} = \frac{1}{\mu}(\mathbf{E} \times \mathbf{B}) = \frac{1}{\mu}
					\epsilon^{ijk}E_j B_k = \frac{1}{\mu}(E_y B_z \mathbf{\hat{x}} - E_y B_x
					\mathbf{\hat{z}}) 
				\]
				Expanding this out, we get:
				\begin{align*}
					\mathbf{S} &= \frac{1}{\mu}\left( E_{0T} e^{-\kappa z} \cos(kx - \omega t)
					\frac{E_{0T}}{\omega} e^{-\kappa z} k \cos(kx -\omega t) \mathbf{\hat{x}} -
				E_{0T}e^{-\kappa z} \cos(kx - \omega t) \frac{E_{0T}}{\omega} e^{-\kappa z} \kappa \sin(kx -
			\omega t) \mathbf{\hat{z}}\right)\\
			&= \frac{1}{\mu \omega}\left( E_{0T}e^{-\kappa z} \right)^2 \left( k \cos^2(kx - \omega t)
			\mathbf{\hat{x}} - \kappa \cos(kx - \omega t) \sin(kx - \omega t) \mathbf{\hat{z}} \right)
				\end{align*}
				Averaged over time, we see that the Poynting vector in the \( \mathbf{\hat{z}} \) direction
				has a \( \sin \theta \cos \theta \) term, and we know that
				\[
					\mean{\sin \theta \cos \theta} = \mean{\frac{1}{2} \sin(2 \theta)} = 0
				\]
				Therefore, there is no energy transmitted in the \( z \) direction. 
			\end{solution}
	\end{enumerate}
\end{document}
