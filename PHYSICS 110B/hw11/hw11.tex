\documentclass[10pt]{article}
\usepackage{../../local}
\urlstyle{same}

\newcommand{\classcode}{Physics 110B}
\newcommand{\classname}{Electromagnetism and Optics II}
\renewcommand{\maketitle}{%
\hrule height4pt
\large{Eric Du \hfill \classcode}
\newline
\large{HW 11} \Large{\hfill \classname \hfill} \large{\today}
\hrule height4pt \vskip .7em
\small{Header styling inspired by the Berkeley EECS Department: \url{https://eecs.berkeley.edu/}}
\normalsize
}
\linespread{1.2}
\begin{document}
	\maketitle
	\section*{Problem 1}
	Consider the four-component object \( \tilde u^{\mu} \) where
	\[
		\tilde u^{\mu} \equiv \dv{x^{\mu}}{t} = \begin{bmatrix} c \\ \dv{x^{i}}{t} \end{bmatrix}	
	\]
	Note that if \( x^{\mu} \) is the spacetime coordinate of a particle, then the spatial components of \(
	\tilde u^{\mu} \) is the 3-velocity.
	\begin{enumerate}[label=(\alph*)]
		\item Consider the Lorentz boost transformation in the \( x \)-direction, show that \( \tilde u^{\mu}
			\) is transformed to 
			\[
				\tilde u^{\mu} \to \tilde u'^{\mu} = \dv{x'^{\mu}}{t'} =
				\begin{bmatrix} c \dv{t'}{t'} \\ \dv{\mathbf{x}}{t'} \end{bmatrix}
				= 
				\begin{bmatrix} c \\ 
					\\
					\dfrac{u_x - v}{1 - \dfrac{v u_x}{c^2}} \\ 
					\\
					\dfrac{u_y}{\gamma\left( 1 - \dfrac{v u_x}{c^2} \right)}\\ 
					\\
					\dfrac{u_z}{\gamma\left( 1 - \dfrac{v u_x}{c^2} \right)}\end{bmatrix}
			\]

			\begin{solution}
				We do this by considering a Lorentz boost in the \( x \)-direction:
				\[
					\tilde u'^{\mu} = \begin{bmatrix} \dv{(ct')}{t}\\ \dv{x'}{t'} \\ \dv{y'}{t'} \\
					\dv{z'}{t'} \end{bmatrix} = \begin{bmatrix} c \\ \frac{\gamma(dx - v \diff t)}{\gamma
				(\diff t - \frac{v}{c^2} \diff x)}\\ \frac{dy}{\gamma(dt - \frac{v}{c^2} \diff x)} \\
			\frac{dz}{\gamma(dt - \frac{v}{c^2} \diff x)} \end{bmatrix} = 
				\begin{bmatrix} c \\ 
					\\
					\dfrac{u_x - v}{1 - \dfrac{v u_x}{c^2}} \\ 
					\\
					\dfrac{u_y}{\gamma\left( 1 - \dfrac{v u_x}{c^2} \right)}\\ 
					\\
					\dfrac{u_z}{\gamma\left( 1 - \dfrac{v u_x}{c^2} \right)}\end{bmatrix}
				\]
			\end{solution}
		\item Suppose we have another similar 4-component object \( \tilde w^{\mu} \) for another particle,
			which under the Lorentz boost transformed to \( \tilde w^{\mu} \). Write down \( \tilde u^{\mu} +
			\tilde w^{\mu}\). 

			\begin{solution}
				We take two vectors of the form expressed in (a) and add them together. We'll do this by
				component because it looks nicer that way:
				\[
					\frac{u_x - v}{1 - \frac{v u_x}{c^2}} + \frac{w_x - v}{1 - \frac{v w_x}{c^2}} =
					\frac{(u_x - v) \left( 1 - \frac{v w_x}{c^2} \right) + (w_x - v) \left( 1 - \frac{v
					u_x}{c^2} \right)}{\left( 1 - \frac{v u_x}{c^2} \right)\left( 1 - \frac{v w_x}{c^2}
			\right)} = \frac{(u_x - v)(c^2 - v w_x) + (w_x - v)(c^2 - v u_x)}{(c^2 - v u_x)(c^2 - vw_x)}
				\]
				For the component that isn't boosted:
				\[
					\frac{u_y}{\gamma\left( 1 - \frac{v u_x}{c^2} \right)} + \frac{w_y}{\gamma\left( 1 -
					\frac{v w_x}{c^2} \right)} = \frac{1}{\gamma}\left[ \frac{u_y(c^2 - vw_x) + w_y (c^2 -
			vu_x)}{(c^2 - vu_x)(c^2 - vw_x)}\right]
				\]
				so the full vector is:
				\[
					\tilde u^{\mu} + \tilde w^{\mu} = \begin{bmatrix} 2c\\ 
						\\
					\dfrac{(u_x - v)(c^2 - v w_x) + (w_x - v)(c^2 - v u_x)}{(c^2 - v u_x)(c^2 - vw_x)}\\
					\\
					\dfrac{1}{\gamma}\left[ \dfrac{u_y(c^2 - vw_x) + w_y (c^2 -
					vu_x)}{(c^2 - vu_x)(c^2 - vw_x)}\right]\\
					\\
					\dfrac{1}{\gamma}\left[ \dfrac{u_z(c^2 - vw_x) + w_z (c^2 -
					vu_x)}{(c^2 - vu_x)(c^2 - vw_x)}\right]
					\end{bmatrix}
				\]
			\end{solution}
		\item If instead we first add the two 4-component objects together to get \( \tilde b^{\mu} \equiv
			\tilde u ^{\mu} + \tilde w^{\mu} \). Calculate \( \tilde b'^{\mu} \), i.e. the transformation of
			\( \tilde b^{\mu} \) under the Lorentz boost in the \( x \)-direction. Is \( \tilde b'^{\mu} \)
			equal to \( \tilde u^{\mu} + \tilde w^{\mu} \). 

			\begin{solution}
				If we add them up beforehand, we can just take the answer from part (a) but treat each \( u
				\) as \( \tilde u + \tilde w \), so the vector becomes:
				\[
				\begin{bmatrix} c \\ \dfrac{u_x + w_x - v}{1 - \dfrac{v (u_x + w_x)}{c^2}} \\
					\\
					\dfrac{u_y + w_y}{\gamma\left( 1 - \dfrac{v (u_x + w_x)}{c^2} \right)}\\ 
					\\
					\dfrac{u_z + w_z}{\gamma\left( 1 - \dfrac{v (u_x + w_x)}{c^2} \right)}\end{bmatrix}
				\]
				Obviously, this is not equal to the result in part (b); this is also what we discussed in
				class, that the velocity 4-vector does not transform linearly under Lorentz transformation. 
			\end{solution}
	\end{enumerate}
	\textit{Note:} The purpose of this problem is twofold. Firstly, just because an object has four
	components, it does not mean that it transforms as 4-vectors, which are required to transform in the same
	way as the 4-position vector \( x^{\mu} \) under Lorentz transformation. Secondly, when objects do not
	transform like vectors, oftentimes the linearity of addition is not preserved under Lorentz
	transformation. But preserving the linearity is important if we want 4-momentum conservation to hold true
	in different inertial reference frames. 
	\begin{enumerate}[label=(\alph*), resume]
		\item Consider a \( 2\to 2 \) particle scattering, where particle \( A \) and \( B \) collide and
			come out with particle \( C \) and \( D \). Suppose that in the \( \mathcal{S} \)-frame the
			4-momentum is conserved, i.e.
			\[
				\mathbf{P}_1 + \mathbf{P}_2 = \mathbf{P}_3 + \mathbf{P}_4
			\]
			Show that if the 4-momentums \( \mathbf{P}_i \) transforms linearly under Lorentz transformation,
			that is 
			\[
				\Lambda(a \mathbf{P}_i + b \mathbf{P}_j) = a\Lambda(\mathbf{P}_i) + b \Lambda(\mathbf{P}_j)
			\]
			for any real numbers \( a \) and \( b \) and 4-momentums \( \mathbf{P}_i \) and \( \mathbf{P}_j
			\), then the 4-momentum conservation also holds true in other inertial reference frames, i.e. 
			\[
				\mathbf{P}'_1 + \mathbf{P}'_2 = \mathbf{P}'_3 + \mathbf{P}'_4
			\]

			\begin{solution}
				Because the Lorentz transformation acts linearly:
				\[
					\mathbf{P}_1' + \mathbf{P}_2' = \Lambda(\mathbf{P}_1) + \Lambda(\mathbf{P}_2) =
					\Lambda(\mathbf{P}_1 + \mathbf{P}_2) = \Lambda(\mathbf{P}_3 + \mathbf{P}_4) =
					\mathbf{P}_3' + \mathbf{P}_4'
				\]
				so the momentum is also conserved in any \( \mathcal{S}' \) frame. 
			\end{solution}
	\end{enumerate}
	\pagebreak
	\section*{Problem 2}
	For a particle of mass \( m \) moving in the \( \mathcal{S} \)-frame whose spacetime coordinate is
	described by \( x^{\mu} = (ct, \mathbf{x}(t)) \), its 4-momentum is defined as \( p^{\mu} \equiv m
	U^{\mu} \), where \( U^{\mu} \) is the particle's 4-velocity, 
	\[
		U^{\mu} \equiv \dv{x^{\mu}}{\tau}
	\]
	and \( \tau \) is the proper time measured by the clock moving with the particle. (It's the time measured
	in the particle's comoving frame). The 4-momentum of the particle is defined as \( p^{\mu} = m U^{\mu}
	\), while the relativistic energy \( E \) and the relativistic 3-momentum \( \mathbf{p} \) of the
	particle is defined via the temporal and spatial components of the 4-momentum where 
	\[
		E \equiv p^{0}c \quad \text{ and } \quad \mathbf{p} = p^{i}
	\]
	\begin{enumerate}[label=(\alph*)]
		\item Show explicitly that the relativistic energy and 3-momentum can be expressed as
			\[
				E = \gamma mc^2, \quad p^{i} = \gamma mv^{i}
			\]
			where \( v \equiv |\mathbf{v}| = |dx^{i} / dt| \) and \( \gamma \equiv 1 / \sqrt{1 - v^2 / c^2} \). 

			\begin{solution}
				We know that \( p^{\mu} = mU^{\mu} \), and since \( U \) is expressed as:
				\[
					\mathbf{U} = \begin{pmatrix} \gamma c \\ \gamma \mathbf{v} \end{pmatrix}
				\]
				The explicit reason for this is because we can write:
				\[
					U^{\mu} = \dv{x^{\mu}}{\tau} = \dv{x^{\mu}}{t}\dv{t}{\tau}  = \gamma v^{\mu}
				\]
				the zeroth index is just \( c \), and the other indices are the velocity in the \( x, y \)
				and
				\( z \) direction. Therefore, we know that:
				\[
					p = \begin{pmatrix} \gamma mc \\ \gamma m \mathbf{v} \end{pmatrix}
				\]
				and from there, we see that \( E = p^{0}c = (\gamma mc) c = \gamma mc^2 \). Likewise, \(
				p^{i} = \gamma m v^{i} \) drops out immediately. 
			\end{solution}
		\item The dot products of 4-vectors are Lorentz invariant. Express the dot product \( p^{\mu}p_\mu \)
			in terms of the particle's mass \( m \) and \( c \). 

			\begin{solution}
				We know that \( p = mU^{\mu} \), so we can write:
				\[
					p^{\mu}p_{\mu} = m^2 U^{\mu}U_{\mu}
				\]
				Now, to get an explicit relation, we compute:
				\[
					U^{\mu}U_{\mu} = -\gamma^2 c^2 + \gamma^2 (v_x^2 + v_y^2 + v_z^2) = \gamma^2(-c^2 + v^2)
				\]
				And since \( \gamma^2 = c^2 /(c^2 - v^2) \) (the alternate form of \( \gamma \)), then:
				\[
					\gamma^2 (-c^2 + v^2) = -c^2
				\]
				And hence, we have:
				\[
					p^{\mu}p_{\mu} = -m^2 c^2
				\]
				this is consistent with what we get in the notes. 
			\end{solution}
		\item Express \( p^{\mu}p_\mu \) in terms of the relativistic energy \( E \) and the relativistic
			3-momentum \( \mathbf{p} \). Combine this with part (b), write down the Lorentz invariant
			identity between \( E \), \( |\mathbf{p}| \) and \( m \). 

			\begin{solution}
				We can write \( \frac{E^2}{c^2} = \gamma m^2 c^2 \), and \( p^{i} = \gamma^2 m^2 (v^{i})^2 \),
				so can combine them in the following way to get rid of \( \gamma \) and get our identity:
				\[
					-\frac{E^2}{c^2} + (p^{i})^2 = -\gamma^2 m^2 c^2 + \gamma^2 m^2 (v^{i})^2 =
					\frac{c^2}{c^2 - v^2} m^2(-c^2 + v^2) = -m^2c^2
				\]
				as desired. 
			\end{solution}
		\item Express the particle's speed \( v = |\mathbf{v}| \) in terms of \( |\mathbf{p}|, E \) and \( c \). 

			\begin{solution}
				We do the same thing as in the previous subpart. We can write:
				\[
					\frac{|\mathbf{p}|^2}{E^2} = \frac{\gamma m (v^{i)^2}}{\gamma^2 m^2 c^{4}} \implies v^2 =
					v_x^2 + v_y^2 + v_z^2 = \frac{|\mathbf{p}|^2}{E^2} \cdot c^{4}
				\]
				Therefore:
				\[
					|\mathbf{v}| = \sqrt{v_x^2 + v_y^2 + v_z^2} = \frac{|\mathbf{p}|}{E} c^2
				\]
			\end{solution}
		\item In the form given in (d), and consider the identity you found in (c), what should the
			particle's speed be if the particle is massless? 

			\begin{solution}
				From part (c), if the particles are massless, then the right hand side \( -m^2 c^2 = 0 \), so
				therefore:
				\[
					(p^{i})^2 = \frac{E^2}{c^2} \implies p^{i} = \frac{E}{c}
				\]
				Combining that with part (d), if we substitute this relation in we get:
				\[
					|\mathbf{v}| = \frac{E / c}{E} c^2 = c
				\]
				so massless particles travel at the speed of light. 
			\end{solution}
	\end{enumerate}
	\pagebreak
	\section*{Problem 3}
	A pion at rest decays into a muon plus a neutrino. Assuming the neutrino is massless \( m_\nu \simeq 0
	\), find the speed of the muon in terms of the pion mass \( m_{\pi} \), muon mass \( m_\mu \) and the
	speed of light \( c \). 

	\begin{solution}
		There's a similar problem in Griffiths, problem 12.8. We follow much of the same derivation to solve
		this problem. The relativistic energy and momentum is conserved, so we set up those equations now.
		First, on the energy side:
		\[
			E_\text{before} = m_{\pi}c^2 \quad E_\text{after} = E_{\nu} + E_{\pi}
		\]
		Likewise, the momentum
		\[
			\mathbf{p}_\text{before} = 0 \quad \mathbf{p}_\text{after} = \mathbf{p}_{\nu} + \mathbf{p}_{\pi}
		\]
		By conservation of energy, we know that:
		\[
			E_{\nu} + E_{\pi} = m_{\pi}c^2
		\]
		and \( \mathbf{p}_{\nu} = \mathbf{-p}_{\pi} \). Since the neutrino is massless, then \( E_{\nu} =
		|\mathbf{p}_{\nu}|c \) and \( |\mathbf{p}_{\mu}| = \sqrt{E_{\mu}^2 - m_{\mu}^2 c^{4}} / c \). This
		then gives:
		\[
			E_{\mu} + \sqrt{E_{\mu}^2 - m_{\mu}^2 c^{4}} = m_{\pi}c^2 \implies E_{\mu} = \frac{(m_{\pi}^2 +
			m_{\mu}^2)c^2}{2 m_{\pi}}
		\]
		Now with the muon energy determined, we can now calculate its velocity. We know that from the
		previous problem:
		\[
			|v_{\mu}| = \frac{|\mathbf{p}_{\mu}|}{E_{\mu}}c^2
		\]
		so substituting what we have, we get:
		\[
			|v_{\mu}| = \frac{m_\pi}{c (m_\pi^2 + m_\mu^2)} \sqrt{\frac{c^{4}(m_\pi^2 - m_\mu^2)^2}{m_\pi^2}}
			= \frac{c(m_\pi^2 - m_\mu^2)}{m_\pi^2 + m_\mu^2}
		\]
		It's also nice to see that the final velocity derived here is less than \( c \), which is expected
		since the muon has nonzero mass. 
	\end{solution}
\end{document}
