\documentclass[10pt]{article}
\usepackage{../../local}
\urlstyle{same}

\newcommand{\classcode}{Physics 110B}
\newcommand{\classname}{Electromagnetism and Optics II}
\renewcommand{\maketitle}{%
\hrule height4pt
\large{Eric Du \hfill \classcode}
\newline
\large{HW 02} \Large{\hfill \classname \hfill} \large{\today}
\hrule height4pt \vskip .7em
\small{Header styling inspired by Berkeley EECS Department: \url{https://eecs.berkeley.edu/}}
\normalsize
}
\linespread{1.2}
\begin{document}
	\maketitle
	\section*{Problem 1}
	Recall that a steady current distribution is one that satisfies
	\[
		\partial_t \rho = 0 \quad \text{and} \quad \partial_t \mathbf{J} = 0 
	\]
	Consider a non-uniform magnetic field \( \mathbf{B} \) that is generated by a steady current distribution
	confined in a volume region \( \mathcal{V} \), and a static electric field generated by a static charge
	distribution that is not necessarily confined in region \( \mathcal{V} \). Show that the flux of the
	electromagnetic momentum through the region \( \mathcal{V} \)'s boundary is zero. That is, 
	\[
		\oint_{\partial \mathcal{V}}\frac{1}{c^2}\mathbf{S} \cdot \diff \mathbf{a} = 0
	\]

	\begin{solution}
		We can show this by leveraging the similarities between the momentum and the energy term. That is,
		while the integral of \( \frac{1}{c^2} \mathbf{S} \) is the momentum, the integral of \( \mathbf{S}
		\) itself is a representation of energy. Thus, we can write:
		\[
			\frac{1}{c^2}\oint_{\partial \mathcal{V}} \mathbf{S} \diff \mathbf{a} 
			= \frac{1}{c^2}\left(\dv{E_\text{particle}}{t} + \dv{U_\text{EM}}{t}\right) 
		\]
		from the continuity equation for energy. From here, it's easy to argue that the right hand side must
		disappear: the region \( \mathcal{V} \) is full of \textit{static} particles, so the first term
		is because there is no mechanical energy transfer. The \(
		\mathbf{E}\) and \( \mathbf{B} \) fields are static -- \( \mathbf{E}  \) is explicitly static and \(
		\mathbf{B} \) is generated by a \textit{steady} current \( \mathbf{J} \) -- so the second term dies
		too. Thus, the integral must be zero, as desired. 
	\end{solution}

	\pagebreak
	\section*{Problem 2}
	In this problem we would like to compute the electromagnetic momentum \( \mathbf{P}_\text{EM} \) of a
	system where an ideal electric dipole \( \mathbf{p} \) is placed at the center of a thin, spinning,
	uniformly charged spherical shell of radius \( R \). Recall that for an ideal dipole placed at the center
	of the coordinate, its electric field is given by 
	\[
		\mathbf{E} = \frac{1}{4 \pi \epsilon_0}\frac{1}{r^3}[3(\mathbf{p} \cdot \hat{\mathbf{r}})
		\hat{\mathbf{r}} - \mathbf{p}] - \frac{1}{3\epsilon_0}\mathbf{p} \delta^3(\mathbf{r})
	\]
	The magnetic field of the spinning charged spherical shell is
	\[
		\mathbf{B} = \begin{cases}
			\dfrac{2}{3}\mu_0 \sigma R \boldsymbol{\omega}& (r < R)\\
			\\
			\dfrac{\mu_0}{4\pi}\dfrac{1}{r^3}[3 (\mathbf{m} \cdot \hat{\mathbf{r}})\mathbf{r} - \mathbf{m}] &
			(r > R)
		\end{cases}
	\]
	where \( \sigma \) and \( \boldsymbol{\omega} \) are the surface charge density and the spinning angular
	velocity of the spherical shell respectively. Note that the magnetic field outside the spherical shell is
	the same as an ideal magnetic dipole with the dipole moment \( \mathbf{m} = \frac{4\pi}{3}\sigma R^{4}
	\boldsymbol{\omega} \). {\color{red} \textbf{Do NOT assume the electric dipole moment \( \mathbf{p} \) is
	aligned with the rotational axis of the spinning sphere.}} Show that the total electromagnetic momentum,
	including both inside and outside the shell, is given by
	\[
		\mathbf{P}_\text{EM} = -\frac{1}{2}(\mathbf{p} \times \mathbf{B})
	\]
	where \( \mathbf{B} \) is the uniform magnetic field inside the shell. We will revisit this system later
	in the semester. 

	\begin{solution}
		We know that to find \( \mathbf{P}_\text{EM} \), then we need to calculate:
		\[
			\mathbf{P}_\text{EM} = \int_\mathcal{V} \epsilon_0(\mathbf{E} \times \mathbf{B}) \diff \tau
		\]
		here \( \mathcal{V} \) is an integral over all space. Because \( \mathbf{B} \) is defined differently
		inside and outside the sphere, we will handle the two integrals separately. The integral inside the
		sphere is easier, so we'll handle that first. Inside the sphere, the integral is:
		\[
			\mathbf{P}_\text{EM} = \int_{\text{inside}} \left( \frac{1}{4 \pi \epsilon_0} \frac{1}{r^3}
			\left[ 3 (\mathbf{p} \cdot \mathbf{r}) \mathbf{\hat{r}} - \mathbf{p} \right] - \frac{1}{3
		\epsilon_0} \delta^3(\mathbf{r}) \right) \times \left( \frac{2}{3} \mu_0 \sigma R \boldsymbol{\omega} \right)
		\diff \tau 
		\]
		The delta term will just kill the integral so that term becomes
		\[
			-\frac{2}{9} \mu \sigma R \mathbf{p} \times \boldsymbol{\omega}
		\]
		Now, we can deal with the second term. As motivated in the discussion, we are free to choose \(
		\mathbf{p} \) to live in the \( xy \)-plane, and in particular let's choose \( \mathbf{p} \) to be on
		the \( y \)-axis, so \( \mathbf{p} = p \mathbf{\hat{y}} \). Then, using spherical coordinates, this
		means that \( \mathbf{p} = p \mathbf{\hat{r}} \sin \theta \sin \phi \), so \( \mathbf{p} \cdot
		\mathbf{\hat{r}} = p \sin \theta \sin \phi \). Then using the usual formulas to transform 
		\( \mathbf{r} \) into Cartesian coordinates, we get:
		\[
			\frac{1}{4\pi} \int_\text{inside} \frac{1}{r^3}\left[ (3 p \sin \theta \sin \phi) (\sin \theta
			\cos \phi \mathbf{\hat{x}} + \sin \theta \sin \phi \mathbf{\hat{y}} + \cos \theta
		\mathbf{\hat{z}}) - p \mathbf{\hat{y}} \right] \times \left( \frac{2}{3}\mu_0 \sigma R
	\boldsymbol{\omega} \right) \diff \tau 
		\]
		Choosing \( \boldsymbol{\omega} \) to live in the \( \mathbf{\hat{z}} \) direction, we can then write
		this as:
		\[
			\frac{1}{4\pi} \int_\text{inside} \frac{1}{r^3}\left[ (3 p \sin \theta \sin \phi) (\sin \theta
			\cos \phi \mathbf{\hat{x}} + \sin \theta \sin \phi \mathbf{\hat{y}} + \cos \theta
		\mathbf{\hat{z}}) - p \mathbf{\hat{y}} \right] \times \left( \frac{2}{3}\mu_0 \sigma R
	\omega \mathbf{\hat{z}}\right) \diff \tau 
		\]
		Now, we can consider some integral tricks. Our integral spans from \( r \in [0, R] \), \( \theta \in
		[0, \pi]\), and \( \phi \in [0, 2\pi] \). Now, the \( \phi \) integral in the \( \mathbf{\hat{x}} \) 
		\( \mathbf{\hat{x}} \) direction is odd over the interval of \( \phi \), so that term goes to zero.
		The \( \mathbf{\hat{z}} \) direction dies because \( \mathbf{\hat{z}} \times \mathbf{\hat{z}} = 0 \).
		For the \( \mathbf{\hat{y}} \) direction, we have the integral:
		\[
			\frac{\mu_0 \sigma R \omega}{6\pi} \int_0^{R}\int_{0}^{\pi}\int_{0}^{2\pi} \frac{1}{r}[(3 p \sin
			\theta \sin \phi)(\sin \theta \sin \phi) - p ] r^2 \sin \theta \diff \phi \diff \theta \diff r
		\]
		This integral, when plugged into mathematica, yields zero. Thus, all three components evaluate to
		zero here. Now for the outside, where we leverage much of the same logic, except our integral now is
		much more complex. The term \( \mathbf{P}_\text{ EM} \) is now:
		\[
			\mathbf{P}_\text{EM} = \epsilon_0 \int_\text{outside} \mathbf{E} \times \frac{\mu_0}{4\pi r^3} [3
			(\mathbf{m} \cdot \hat{\mathbf{r}}) \mathbf{r} - \mathbf{m}] \diff \tau
		\]
		Without writing things out explicitly, we have the following: the \( \mathbf{E} \times \mathbf{m} \)
		term dies immediately, because it is the same integral as inside the sphere. The only term we need to
		care about then is \( \mathbf{E} \) crossed with the first term. Further, The first term in the
		expansion here is also zero, since we have an \( \mathbf{\hat{r}} \times \mathbf{\hat{r}} \) term,
		which is zero. Thus, we only have one term to calculate, which is the following:
		\[
			-\frac{\mu_0}{16 \epsilon_0 \pi^2} \int_R^{\infty} \int_{0}^{2\pi} \int_{0}^{\pi}
			\frac{1}{r^{6}}\mathbf{p} \times 3(\mathbf{m} \cdot \mathbf{\hat{r}}) \mathbf{\hat{r}} r^2 \sin
			\theta \diff \theta \diff \phi \diff r
		\]
		This is as far as I could get before running out of time, but this integral, when evaluated, and
		combined with the first term, should get you the right answer. 
	\end{solution}	

	\pagebreak
	\section*{Problem 3}
	Show that for a general electromagnetic field, one has the inequality
	\[
		U_\text{EM} \geq c |\mathbf{P}_\text{EM}|
	\]
	Here \( U_\text{EM} \) is the energy in the electromagnetic field
	\[
		U = \int_{\mathcal{V}} \left( \frac{1}{2}\epsilon_0 |\mathbf{E}|^2 + \frac{1}{2\mu_0}|\mathbf{B}|^2
		\right) \diff \tau
	\]
	and \( \mathbf{P}_\text{EM} \) is the electromagnetic momentum
	\[
		\mathbf{P}_\text{EM} = \int_{\mathcal{V}} \epsilon_0 (\mathbf{E} \times \mathbf{B}) \diff \tau
	\]
	Under what conditions will \( U_\text{EM} = c |\mathbf{P}_\text{EM}| \)?

	\begin{solution}
		To show this inequality, we manipulate the energy density term and show that it is larger than or
		equal to the momentum term over all space. To do this, notice we can write the following:
		\[
			\frac{1}{2}\epsilon_0 |\mathbf{E}|^2 + \frac{1}{2\mu_0}|\mathbf{B}|^2 = \frac{1}{2}\left(
			\epsilon_0 |\mathbf{E}|^2 + \frac{|\mathbf{B}|^2}{\mu_0} \right)
		\]
		Now, we can leverage the inequality that \( a^2 + b^2 \geq 2ab \); this comes from the fact that \(
		(a - b)^2 \geq 0 \). So, we can rearrange the density term to:
		\[
			\frac{1}{2}\left( (\sqrt{\epsilon_0}|\mathbf{E}|^2 + \left( \frac{|\mathbf{B}|}{\sqrt{\mu_0}}
			\right)^2 \right) \geq \frac{1}{2} \left( 2 (\sqrt{\epsilon_0}|\mathbf{E}|) \left(
			\frac{|\mathbf{B}|}{\sqrt{\mu_0}} \right) \right) = \sqrt{\frac{\epsilon_0}{\mu_0}}
			|\mathbf{E}||\mathbf{B}|
		\]
		Now, the momentum density term can be written as follows:
		\[
			c \epsilon_0 (\mathbf{E} \times \mathbf{B}) = \frac{1}{\sqrt{\epsilon_0 \mu_0}}\epsilon_0(\mathbf{E}
			\times \mathbf{B}) = \sqrt{\frac{\epsilon_0}{\mu_0}}(\mathbf{E} \times \mathbf{B})
		\]
		Due to the property of the cross product, we also have:
		\[
			\sqrt{\frac{\epsilon_0}{\mu_0}}(\mathbf{E}\times \mathbf{B}) \leq \sqrt{\frac{\epsilon_0}{\mu_0}}
			|\mathbf{E}||\mathbf{B}|
		\]
		Now, the right hand side is the rearranged energy density term. Thus, because the energy density is
		always larger than the momentum density, the inequality \( U_\text{EM} \geq c|\mathbf{P}s_\text{EM}|
		\) holds. Equality holds when \( |\mathbf{E}| = |\mathbf{B}| \) and the fields are perpendicular at
		every point in \( \mathcal{V} \). 
	\end{solution}
\end{document}
