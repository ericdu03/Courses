\documentclass[10pt]{article}
\usepackage{../../local}
\urlstyle{same}

\newcommand{\classcode}{Physics 110B}
\newcommand{\classname}{Eletromagnetism and Optics II}
\renewcommand{\maketitle}{%
\hrule height4pt
\large{Eric Du \hfill \classcode}
\newline
\large{HW 07} \Large{\hfill \classname \hfill} \large{\today}
\hrule height4pt \vskip .7em
\small{Header styling inspired by the Berkeley EECS Department: \url{https://eecs.berkeley.edu/}}
\normalsize
}
\linespread{1.2}
\begin{document}
	\maketitle
	\section*{Problem 1}
	In introductory physics courses, Faraday's law is often introduced in the form of
	\begin{equation}
		\label{eq1}
		\oint_\mathcal{C} \mathbf{E} \cdot \diff \boldsymbol{\ell} = -\dv{t} \int_{\mathcal{S}} \mathbf{B}
		\cdot \diff \mathbf{a}
	\end{equation}
	However, one should keep in mind that in this form, the contour \( \mathcal{C} \) and surface \(
	\mathcal{S} \) is stationary. Thus, for example, one cannot use this to explain how a rotating coil in a
	constant magnetic field can have an induced current generated in the coil. In fact, for a rotating coil
	in a constant magnetic field, there is no induced electric field being generated. After all, physical
	laws, as far as we know, should be local and the most appropriate form of Faraday's law should be the
	differential form
	\begin{equation}
		\nabla \times \mathbf{E} = -\partial_t \mathbf{B}
	\end{equation}
	And because the magnetic field is constant, we have \( \nabla \times \mathbf{E} = 0 \) and there must be
	no induced electric field generated. What force is responsible for the induced current in the rotating
	coil then? The origin of the induced current is the \textit{motional emf} along the loop due to the
	magnetic force caused by the rotational motion. The motional emf is given by
	\begin{equation}
		\mathcal{E} = \oint_{\mathcal{C}} (\mathbf{v} \times \mathbf{B}) \cdot \diff \boldsymbol{\ell}
	\end{equation}
	and in Chap. 7 of Griffiths, he showed that for time-independent magnetic fields, the motional emf due to
	the motion of the loop is given by
	\begin{equation}
		\label{eq4}
		\oint_{\mathcal{C}}(\mathbf{v} \times \mathbf{B}) \diff \boldsymbol{\ell} = -\dv{t}
		\int_{\mathcal{S}}\mathbf{B} \cdot \diff \mathbf{a}
	\end{equation}
	In his proof, the magnetic field is assumed to be time-independent, which is why the induced electric
	field does not show up in the left hand side of the equation. 

	Apparently, neither Eq. \ref{eq1} nor Eq. \ref{eq4} are really satisfying, as they either assume the
	magnetic field is constant over time, or the loop is stationary. But ultimately these assumptions are
	unnecessary, and both equations can be combined together to form the complete \textit{flux rule}
	\[
		\oint_{\mathcal{C}} (\mathbf{E} + \mathbf{v} \times \mathbf{B}) \cdot \diff \boldsymbol{\ell} =
		-\dv{t} \int_{\mathcal{S}}\mathbf{B} \cdot \diff \mathbf{a}
	\]
	In this problem, we will derive this equation by explicitly showing that the magnetic flux changing rate
	on the right is indeed equal to the line integral on the left. 

	To this end, we first rewrite the magnetic flux \( \int_{\mathcal{S}}\mathbf{B} \cdot \diff \mathbf{a} \)
	as the line integral of the vector potential \( \mathbf{A} \),
	\[
		\int_{\mathcal{S}}\mathbf{B} \cdot \diff \mathbf{a} = \oint_{\mathcal{C}}\mathbf{A} \cdot \diff
		\boldsymbol{\ell} = \oint_{\mathcal{C}} \mathbf{A} (\mathbf{x}, t) \cdot \diff \mathbf{x}
	\]
	Here the position \( \mathbf{x} = (x, y, z) \) of the loop elements can change over time resulting the
	loop \( \mathcal{C} \) to deform. Not only can the loop move in space, it can also shrink, contract or
	bend. We can think that we mark the points on the loop with a parameter \( \lambda \), where \( \lambda
	\in [0, 2\pi) \) and \( \lambda = 0 \) and \( \lambda = 2\pi \) mark the same point. \( x(\lambda, t) \)
	can then be thought of as a function that maps \( [0, 2\pi) \) to \( \R^3 \) at time \( t \). The line
	integral can then be written as
	\[
		\oint_{\mathcal{C}}\mathbf{A}(\mathbf{x}, t) \cdot \diff \mathbf{x} = \int_{0}^{2\pi}
		\mathbf{A}(\mathbf{x}(\lambda, t), t) \cdot \pdv{\mathbf{x}}{\lambda} \diff \lambda
	\]
	When we now take the time derivative of the integral we need to consider the time changing rate of \(
	\mathbf{x}(\lambda, t) \) and 
	\[
		\mathbf{v} \equiv \pdv{\mathbf{x}}{t}
	\]
	is the velocity of the loop element. From here, show that indeed
	\[
		-\dv{t}\left( \int_{0}^{2\pi} \mathbf{A}(\mathbf{x}(\lambda, t), t) \cdot \pdv{\mathbf{x}}{\lambda}
		\diff \lambda \right) = \oint_{\mathcal{C}}(\mathbf{E} + \mathbf{v} \times \mathbf{B}) \cdot \diff
		\boldsymbol{\ell}
	\]

	\begin{solution}
		 First, bring the time derivative into the integral, turning it into a partial derivative, and also
		 use product rule here:
		 \[
			 -\dv{t} \int_{0}^{2\pi} \mathbf{A}(\mathbf{x}(\lambda, t), t) \cdot \pdv{\mathbf{x}}{\lambda}
			 \diff \lambda = -\int_{0}^{2\pi} \pdv{t} \left( \mathbf{A}(\mathbf{x}(\lambda, t), t) \cdot
			 \pdv{\mathbf{x}}{\lambda} \right) d \lambda = -\int_{0}^{2\pi} \left(\pdv{\mathbf{A}}{t} \cdot
		 \pdv{\mathbf{x}}{\lambda} + \mathbf{A} \pdv{\mathbf{x}}{t}{\lambda} \right)d\lambda
		 \]
		 From here we focus on the integrand since that's where all the math is going to take place. Here,
		 we'll use index notation because it's easier than trying to write this in Leibniz notation.
		 Replacing the above with index notation, we basically have:
		 \[
			 (\partial_t A)^{k} \pdv{x_k}{\lambda} + A^{k} \pdv{x_k}{t}{\lambda}
		 \]
		 Further, we can simplify the first term:
		 \[
			 (\partial_t A)^{k} = (\partial_t x^{k}) \partial_i A^{k} + \partial_t A^{k} = v^{k}\partial_i
			 A^{k} + \partial_t A^{k}	
		 \]
		 The \( \partial_t A^{k} \) term is the explicit time derivative of \( \mathbf{A} \). Now, the \(
		 \partial_t A^{k} \) is just \( \partial_t \mathbf{A} = -\mathbf{E} - \nabla V \). The 
		 \( \nabla V \) part is a total derivative, so under the integral it will go to zero. So, for the
		 second term, we're left with:
		 \[
			 -\int_{0}^{2\pi}
			 - \mathbf{E}\cdot  \pdv{x_k}{\lambda} \diff \lambda = \oint_{\mathcal{C}}\mathbf{E} \cdot \diff
			 \boldsymbol{\ell}
		 \]
		 Then, we have the \( v^{k}\partial_i A^{k} \) term to deal with. This term is supposed to give the
		 \( \mathbf{v} \times \mathbf{B} \) term, but I couldn't work it out fully. The furthest I got was as
		 follows: we can rewrite \( v^{k}\partial_i A^{k} \) as:
		 \[
		 	v^{k}\epsilon^{kin} \partial_i A_n + v^{k}\delta_n^{m}\delta_m^{i} \partial_n A_m
		 \]
		 essentially, now this allows us to write the first term as \( \mathbf{v} \times (\nabla \times \mathbf{A})
		 \), so that gets us the \( \mathbf{v} \times \mathbf{B} \) term. The second term that we generate is
		 now supposed to cancel with the \( A^{k}\pdv{x_k}{t}{\lambda} \) term we have from before.  
	 \end{solution}
	\pagebreak
	\section*{Problem 2}
	The Helmholtz theorem implies that the vector potential \( \mathbf{A} \) can be decomposed into the
	divergenceless part \( \mathbf{A}_{\perp} \) and the curl-less part \( \mathbf{A}_{\parallel} \). That
	is, \( \mathbf{A} = \mathbf{A}_{\perp} + \mathbf{A}_{\parallel} \) where \( \nabla \cdot
	\mathbf{A}_{\perp} = 0 \) and \( \nabla \times \mathbf{A}_{\parallel} = 0 \). Under the gauge
	transformation \( \mathbf{A} \to \mathbf{A} + \nabla \lambda \), which part of the vector potential
	transforms? Recall that physical observables should be invariant under gauge transformation. Thus, should
	physical observables be expressed in terms of \( \mathbf{A}_{\perp} \) or \( \mathbf{A}_{\parallel} \)? 

	\begin{solution}
		After the transformation \( \mathbf{A} \to \mathbf{A} + \nabla \lambda \), it's still possible to
		decompose it into its parallel and perpendicular components:
		\[
			\mathbf{A}' = (\mathbf{A}_{\perp} + (\nabla \lambda)_{\perp}) + (\mathbf{A}_{\parallel} + (\nabla
			\lambda)_{\parallel})
		\]
		Checking the divergence of the parallel component:
		\[
			\nabla \cdot (\mathbf{A}_{\perp} + (\nabla \lambda)_{\perp}) = \nabla \cdot \mathbf{A}_{\perp} +
			\nabla \cdot (\nabla \lambda)_{\perp} = \nabla \cdot (\nabla \lambda)_{\perp} \neq 0
		\]
		This quantity is not necessarily zero depending on the choice of \( \lambda \). However, for the
		parallel component:
		\[
			\nabla \times (\mathbf{A}_{\parallel} + (\nabla \lambda)_{\parallel}) = \nabla \times
			\mathbf{A}_{\parallel} + \nabla \times (\nabla \lambda) _{\parallel} = 0
		\]
		This is equal to zero since \( \nabla \times \mathbf{A}_{\parallel} = 0 \) and \( \nabla \times
		(\nabla \lambda) = 0 \) because it's a vector identity. Therefore, it's the parallel component that
		is invariant under the gauge \( \nabla \lambda \). So, physical observables should be expressed in
		terms of \( \mathbf{A}_{\parallel} \), since that quantity is invariant under the gauge
		transformation.  
	\end{solution}

	\pagebreak
	\section*{Problem 3}
	\begin{enumerate}[label=(\alph*)]
		\item Find the fields, and the change in current distributions, corresponding to 
			\[
				V(\mathbf{r}, t) = 0, \quad \mathbf{A}(\mathbf{r}, t) = -\frac{1}{4\pi \epsilon_0}
				\frac{qt}{r^2} \mathbf{\hat{r}}
			\]

			\begin{solution}
				Using the required formulas:
				\[
					\mathbf{E} = -\nabla V - \partial_t \mathbf{A} = \frac{1}{4\pi \epsilon_0} \frac{q}{r^2}
					\mathbf{\hat{r}}
				\]
				For \( \mathbf{B} \), we have \( \mathbf{B} = \nabla \times \mathbf{A} \), but since \(
				\mathbf{A} \) only has a radial component, its curl is naturally zero therefore \( \mathbf{B}
				= 0\). 
			\end{solution}
		\item Use the gauge function \( \lambda = - (1 / 4 \pi \epsilon_0) (qt / r) \) to transform the
			potentials, and comment on the result.  

			\begin{solution}
				Under the gauge transformation, \( V \to V - \partial_t \lambda \), so therefore:
				\[
					V' = -\frac{1}{4\pi \epsilon_0}\frac{q}{r}
				\]
				and \( \mathbf{A}' \to \mathbf{A} + \nabla \lambda \) so:
				\[
					\mathbf{A}' = -\frac{1}{4\pi \epsilon_0} \frac{qt}{r^2}\mathbf{\hat{r}} - \frac{1}{4\pi
					\epsilon_0}qt \nabla \left( \frac{1}{r} \right)
				\]
				Now, we use the relation \( \nabla\left( \frac{1}{r} \right) = -\frac{\mathbf{\hat{r}}}{r^2}
				\), so thus:
				\[
					\mathbf{A}' = -\frac{1}{4\pi \epsilon_0} \frac{qt}{r^2}\mathbf{\hat{r}} + \frac{1}{4\pi
					\epsilon_0} qt \frac{\hat{r}}{r^2} = \mathbf{0}
				\]
				So this gauge transformation turns the potential \( V \) into the potential of a stationary
				charge, and changes the magnetic potential \( \mathbf{A} \) to be the zero vector. In this
				sense, this gauge basically just swaps out what is nonzero -- initially it was \( V \), now
				it's \( \mathbf{A} \). 
			\end{solution}
	\end{enumerate}
\end{document}


