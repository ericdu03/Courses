\documentclass[10pt]{article}
\usepackage{../../local}
\urlstyle{same}

\newcommand{\classcode}{Physics 110B}
\newcommand{\classname}{Electromagnetism and Optics II}
\renewcommand{\maketitle}{%
\hrule height4pt
\large{Eric Du \hfill \classcode}
\newline
\large{HW 06} \Large{\hfill \classname \hfill} \large{\today}
\hrule height4pt \vskip .7em
\small{Header styling inspired by the Berkeley EECS Department: \url{https://eecs.berkeley.edu/}}
\normalsize
}
\linespread{1.2}
\begin{document}
	\maketitle
	\section*{Problem 1}
	Show that the mode \( \text{TE}_{00} \) cannot occur in a rectangular wave guide. [\textit{Hint:} In
	this case \( \omega / c = k \), so Eqs. 9.182 are indeterminate, and you must go back to Eq. 9.181. show
	that \( B_z \) is constant, and hence -- applying Faraday's law in integral frorm to a cross section --
	that \( B_z = 0 \), so this would be a TEM mode.] 

	\begin{solution}
		Using the hint, we go back to Eq. 9.181, which are the differential equations obtained via Maxwell's
		equations. TE waves have \( E_z = 0 \), so the equations we are interested in are:
		\begin{align}
			\pdv{B_y}{x} - \pdv{B_x}{y} &= 0\\
			\label{eq1}\pdv{B_z}{y} - ik B_y &= -\frac{i \omega}{c^2} E_x\\
			\label{eq2}ik B_x - \pdv{B_z}{x} &= -\frac{i \omega}{c^2} E_y
		\end{align}
		Further, from equations (i), (ii) and (iii) we also have the relations:
		\begin{align*}
			-ik E_y &= i \omega B_x\\
			ik E_x &= i \omega B_y\\
			\pdv{E_y}{x} - \pdv{E_x}{y} &= i \omega B_z
		\end{align*}
		Finally, using the relation that \( \omega / c = k \), we get:
		\begin{align*}
			i \frac{\omega}{c^2}E_x &= i \frac{\omega}{c} B_y\\
			-i \frac{\omega}{c^2}E_y &=  i \frac{\omega}{c}B_x
		\end{align*}
		As such, equations \ref{eq1} and \ref{eq2} become:
		\begin{align*}
			\pdv{B_z}{y} &= 0 \\ 
			\pdv{B_z}{x} &= 0 
		\end{align*}
		So along the \( xy \) direction, we have \( B_z = \text{const.} \), so using Faraday's law we indeed
		get \( B_z = 0 \), and hence this is a TEM mode. Therefore, the \( \text{TE}_{00} \) mode doesn't
		exist.  
	\end{solution}

	\pagebreak
	\section*{Problem 2} 
	Confirm that the energy in the \( \mathrm{TE}_{mn} \) mode travels at the group velocity. [\textit{Hint:}
	Find the time averaged Poynting vector \( \mean{\mathbf{S}} \) and the energy density \( \mean u \) (use
	Prob. 9.12 if you wish). Integrate over the cross section of the wave guide to get the energy per unit
	time and per unit length carried by the wave, and take their ratio.]

	\begin{solution}
		Problem 9.12 gives us formulas formulas for computing \( \mean{u} \) and \( \mean{\mathbf{S}} \):
		\[
			\mean{u} = \frac{1}{4}\Re\left( \epsilon_0 \tilde{\mathbf{E}} \cdot \tilde{\mathbf{E}}^{*} +
			\frac{1}{\mu_0} \tilde{\mathbf{B}} \cdot \tilde{\mathbf{B}}^{*} \right) \quad \mean{\mathbf{S}} =
			\frac{1}{2\mu_0} \Re \left( \tilde{\mathbf{E}} \times \tilde{\mathbf{B}}^{*} \right)
		\]
		From the book, we know that the equation for \( B_z \) is given as:
		\[
			B_z = B_0 \cos\left( \frac{m \pi x}{a} \right) \cos\left( \frac{n \pi y}{b} \right)
		\]
		So, for the \( x \) and \( y \) directions of the waves, we know that \( E_z = 0\), so using equation
		9.180 we have:
		\begin{align*}
			E_x &= -\frac{i \omega}{\left( \frac{\omega}{c} \right)^2 - k^2} B_0 \frac{n\pi}{b} \cos \left(
			\frac{m \pi x}{a} \right) \sin\left( \frac{n \pi y}{b} \right)\\
			E_y &= \frac{i \omega}{\left( \frac{\omega}{c} \right)^2 - k^2} B_0 \frac{m \pi}{a} \sin\left(
			\frac{m \pi x}{a} \right) \cos\left( \frac{n \pi y}{b} \right) \\ 
			B_x &= -\frac{ik}{\left( \frac{\omega}{c} \right)^2 - k^2} B_0 \frac{m \pi}{a} \sin \left(
			\frac{m \pi x}{a} \right) \cos(\frac{n \pi y}{b})  \\ 
			B_y &= -\frac{ik}{\left( \frac{\omega}{c} \right)^2 - k^2} B_0 \frac{n \pi}{b} \cos(\frac{m \pi x}{a})
			\sin(\frac{n \pi y}{b}) 
		\end{align*}
		And of course \( E_z = 0 \) because we're dealing with a TE wave. Now, we can compute \( \mean{u} \):
		\begin{multline*}
			\mean{u} = -\frac{\epsilon_0 \omega^2 B_0^2}{4\left[ (\omega / c)^2 - k^2 \right]^2}\left( \left( \frac{n
			\pi}{b} \right)^2 \sin^2\left( \frac{n \pi y}{b} \right)\cos^2\left( \frac{m \pi x}{a} \right) -
		\left( \frac{m \pi}{a} \right)^2 \sin^2\left( \frac{m \pi x}{a} \right) \cos^2 \left( \frac{n \pi
	y}{b} \right)\right) \\ 
	- \frac{k^2 B_0^2}{4\mu_0[(\omega / c)^2 - k^2]^2}\left( \left( \frac{m \pi}{a} \right)^2 \sin^2\left(
	\frac{m \pi x}{a} \right) \cos^2\left( \frac{n \pi y}{b} \right) + \left( \frac{n \pi}{b} \right)^2
\cos^2\left( \frac{m \pi x}{a}\right) \sin^2 \left( \frac{n \pi y}{b} \right) \right) + \frac{B_0^2}{\mu_0}
\cos^2\left( \frac{m \pi x}{a} \right) \sin^2\left( \frac{n \pi y}{b} \right)
		\end{multline*}
		Similarly, we can compute \( \mean{\mathbf{S}} \) using these formulas. Before we do that, one
		simplification we can make: because \( E_z = 0 \), then only the \( z \) component of \( S \)
		survives, and hence we have:
		\[
			\mean{\mathbf{S}} = \frac{k \omega B_0^2}{2\mu_0[(\omega / c)^2 - k^2]^2}\left( \left( \frac{n
			\pi}{b} \right)^2 \cos^2\left( \frac{m \pi x}{a} \right)\sin^2\left( \frac{n \pi y}{b} \right) +
		\left(\frac{m \pi}{a} \right)^2 \sin^2\left( \frac{m \pi x}{a} \right) \cos^2\left( \frac{n \pi y}{b} \right)
	\right) \mathbf{\hat{z}}
		\]
		Now, we are ready to compute the integrals. Integrating \( \mean{u} \) first, which I did using a
		calculator (thank you wolfram), I get:
		\[
			\int \mean{u} \diff a = -\frac{\epsilon \omega^2 B_0^2}{4\left[ (\omega / c)^2 - k^2 \right]^2}
			\frac{(a^2n^2 - b^2m^2)\pi^2}{4ab} - \frac{k^2 B_0^2}{4\mu_0[(\omega / c)^2 - k^2]^2}
			\frac{(b^2m^2 - a^2n^2)\pi^2}{4ab} + \frac{B_0^2}{4 \mu_0} \frac{ab}{4}
		\] 
		For \( \mean{\mathbf{S}} \) (again, wolfram), I get:
		\[
			\int \mean{\mathbf{S}} \diff \mathbf{a} = \frac{k \omega B_0^2}{2 \mu_0[(\omega / c)^2 - k^2]^2}
			\frac{(b^2 m^2 + a^2 n^2) \pi}{4ab}
		\]
		For the \( \mean{u} \) integral, we can simplify it using \( c = \frac{\omega}{k} \), which
		after some algebra gets us:
		\[
			\int \mean{u} \diff a = \frac{B_0^2}{4[(\omega / c)^2 - k^2]^2}\frac{-2k^2}{\mu_0} 
			\frac{(a^2n^2 - b^2m^2)\pi^2}{4ab} + \frac{B_0^2}{4 \mu_0}\left( \frac{ab}{4} \right)
		\]
		From here, we're supposed to take the ratio of the energy per unit time versus the energy per unit
		length, which means to take the integral of \( \mean{\mathbf{S}} \) over the integral of \( \mean{u}
		\). I couldn't finish the writeup in time though sadly.  
	\end{solution}

	\pagebreak
	\section*{Problem 3} 
	Work out the theory of TM modes for a rectangular wave guide. In particular, find the longitudinal
	electric field, the cutoff frequencies, and the wave and group velocities. Find the ratio of the lowest
	TM cutoff frequency to the lowest TE cutoff frequency, for a given wave guide. [\textit{Caution:} What is
	the lowest TM mode?]

	\begin{solution}
		For TM modes, we have \( B_z = 0 \), but we follow much of the logic from the derivation of the TE
		waves. We let \( E_z(x, y) = X(x) Y(y) \), and since \( B_z \) satisfies the same wave equation as \(
		E_z\) in the TE case, then we also get solutions:
		\[
			X(x) = A \sin(k_x x) + B \cos(k_x x)
		\]
		The boundary condition for conductors requires that \( E_z = 0 \) inside conductors, so first \( B =
		0 \), so we're left with \( X(x) = A \sin(k_x x) \). The same boundary condition then implies:
		\[
			k_x = \frac{m \pi}{a}
		\]
		just like the TE case. Here, \( m = 1, 2, 3, \dots \), and the same goes for the \( y \) direction, so we
		get:
		\[
			E_z = E_0 \sin\left( \frac{m\pi x}{a} \right)\sin(\frac{n \pi y}{b})
		\]
		with \( m, n = 1, 2, 3, \dots \). Note that we cannot allow either \( m, n = 0 \) because that would
		mean zero wave, which is a trivial solution. The formula for the wave number is the same as that in the TE
		wave case:
		\[
			k = \sqrt{\left( \frac{\omega}{c} \right)^2 - \pi^2 \left[ \left( \frac{m}{a} \right)^2 + \left(
			\frac{n}{b}\right)^2 \right]}
		\]
		The cutoff frequency then follows the same equation:
		\[
			\omega_c = c \pi \sqrt{\left( \frac{m}{a} \right)^2 + \left( \frac{n}{b} \right)^2}
		\]
		But here, the lowest cutoff frequency is the lowest value of \( m = n = 1 \), so:
		\[
			\omega_c = c \pi \sqrt{\frac{1}{a^2} + \frac{1}{b^2}}
		\]
		So the ratio of cutoff frequencies is just the ratio of this to \( \omega_{10} \) in the TE wave
		case:
		\[
			\frac{\omega_c}{\omega_{10}} = a\sqrt{\frac{1}{a^2} + \frac{1}{b^2}}
		\]
		and that concludes the problem. 
	\end{solution}
\end{document}
