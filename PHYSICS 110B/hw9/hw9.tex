\documentclass[10pt]{article}
\usepackage{../../local}
\urlstyle{same}

\newcommand{\classcode}{Physics 110B}
\newcommand{\classname}{Electromagnetism and Optics II}
\renewcommand{\maketitle}{%
\hrule height4pt
\large{Eric Du \hfill \classcode}
\newline
\large{HW 09} \Large{\hfill \classname \hfill} \large{\today}
\hrule height4pt \vskip .7em
\small{Header styling inspired by the Berkeley EECS Department: \url{https://eecs.berkeley.edu/}}
\normalsize
}
\linespread{1.2}
\begin{document}
	\maketitle
	\section*{Collaborators}
	I worked with \textbf{Teja Nivarthi} on this assignment. He was particularly helpful in solving problem
	3b and onwards.  

	\section*{Problem 1}
	A current \( I(t) \) flows around the circular ring in Fig. 11.8. Derive the general formula for the
	power radiated (analogous to Eq. 11.60), expressing your answer in terms of the magnetic dipole moment \(
	m(t)\), of the loop. [Answer \( P = \mu_0 \ddot m^2 / 6\pi c^3 \).]

	\begin{solution}
		The first steps of my solution follow what's been done in lecture. The vector potential is given by:
		\[
			\mathbf{A} = \frac{\mu_0}{4\pi}\int \frac{\mathbf{J}(t_r, \mathbf{r}')}{\rcurs} \diff \tau
		\]
		Here, we use the same approximations as we derived in lecture:
		\[
			\rcurs = \sqrt{r^2 + b^2 - 2rb \sin \theta \cos \phi} \approx r\left( 1 - \frac{b}{r}\sin \theta
			\cos \phi\right)
		\]
		so:
		\[
			\frac{1}{\rcurs} = \frac{1}{r}\left(1 + \frac{b}{r} \sin \theta \cos \phi\right) 
		\]
		So now the vector potential reads:
		\[
			\mathbf{A} = \frac{\mu_0}{4\pi} \oint_{\mathcal{C}} I\left( t - \frac{\rcurs}{c} \right)
			\frac{1}{r}\left( 1 + \frac{b}{r}\sin \theta \cos \phi \right) \diff \mathbf{l}
		\]
		Firstly, we can integrate over \( \phi \), while setting \( \diff \mathbf{l} = \cos \phi
		\hat{{\boldsymbol{\phi}}} \) (we also ignore the \( \mathbf{\hat{x}} \) terms since they cancel, just
		like in lecture.) Since \( \frac{\rcurs}{c} \) is small, we can Taylor expand the current
		term around \( t - \frac{\rcurs}{c} \), which gives us:
		\[
			I(t) = I\left( t - \frac{\rcurs}{c} \right) + \dot I\left( t -
			\frac{\rcurs}{c} \right)\left( \frac{b}{c}\sin \theta \cos \phi \right)
		\]
		For notational purposes I will denote \( t' = t - \frac{\rcurs}{c} \). So, the integral is:
		\[
			I(t) = \frac{\mu_0b}{4\pi r} \hat{\boldsymbol{\phi}} \int_{0}^{2\pi}\left[ I(t') + \dot I(t')
			\left( \frac{b}{c} \sin \theta \cos \phi \right)\right] \left( 1 + \frac{b}{r}\sin \theta \cos
		\phi \right) \cos \phi \diff \phi
		\]
		Now, the \( \int_{0}^{2\pi}\cos \phi \diff \phi \) and the \( \int_{0}^{2\pi} \cos^3\phi \diff \phi
		\) terms all go to zero, so there are only two terms that survive. Those are:
		\[
			\frac{\mu_0b \boldsymbol{\hat{\phi}}}{4\pi r} \int_{0}^{2\pi}I(t') \frac{b}{r}\sin \theta \cos^2
			\phi + \dot I(t') \frac{b}{c}\sin \theta \cos^2 \phi \diff \phi
		\]
		The integral evaluates to \( \pi \), so the vector potential is written as (from this point onward,
		\( \dot I = \dot I(t') \) and likewise, \( \ddot I = \ddot I(t') \)):
		\[
			\mathbf{A} = \frac{\mu_0b^2}{4r}\left( \frac{I}{r} + \frac{\dot I}{c} \right) \sin \theta
			\boldsymbol{\hat{\phi}}
		\]
		Now, with \( \mathbf{A} \) determined, we can calculate \( \mathbf{E} \) and \( \mathbf{B} \):
		\begin{align*}
			\mathbf{E} &= -\pdv{A}{t} = -\frac{\mu_0b^2}{4r}\left( \frac{\dot I}{r} + \frac{\ddot I}{c}
			\right)\sin \theta \boldsymbol{\hat{\phi}}\\
				\mathbf{B} &= \nabla \times \mathbf{A} = 
				\frac{1}{r\sin \theta} \pdv{\theta}
				(\mathbf{A}_{\phi} \sin \theta) - \pdv{r} (r \mathbf{A}_{\phi})\\  
				&= \frac{1}{r\sin \theta} \frac{\mu_0b^2}{4r} 
				\left( \frac{\dot I}{r} + \frac{\ddot I}{c} \right)(2 \sin \theta \cos
				\phi) \mathbf{\hat{r}} - \frac{1}{r}\left( \frac{\mu_0b^2}{4}\left( \frac{\dot I}{r}\sin
				\theta \right) - \frac{\mu_0b^2}{4}\left( \frac{\ddot I}{c^2} \right)\sin \theta
			\right)\boldsymbol{\hat{\theta}}
		\end{align*}
		The extra factor of \( -\frac{1}{c} \) comes from the chain rule, since \( t' \) has \( r \)
		dependence. Now, we get rid of all the terms with greater than \( \frac{1}{r^2} \) dependence, since
		we are interested in the radiation zone, so therefore \( \mathbf{B} \) simplifies to only the last
		term:
		\[
			\mathbf{B} = \frac{\mu_0b^2}{4rc^2} \ddot I \sin \theta \boldsymbol{\hat{\theta}}
		\]
		Similarly, we drop the first term in \( \mathbf{E} \):
		\[
			\mathbf{E} = -\frac{\mu_0b^2}{4r} \frac{\ddot I}{c} \sin \theta \boldsymbol{\hat{\phi}}
		\]
		The Poynting vector is then calculated as:
		\[
			\mathbf{S} = \frac{1}{\mu_0}(\mathbf{E} \times \mathbf{B}) = \frac{\mu_0^2 b^{4}}{16r^2 c^3}
			\ddot I^2 \sin^2 \theta \mathbf{\hat{r}}
		\]
		The power is then:
		\[
			P = \oint \mathbf{S} \diff \mathbf{a} = \frac{\mu_0^2 b^{4}}{16c^3} \ddot I^2
			\int_{0}^{\pi}\int_{0}^{2\pi} \frac{\sin^2 \theta}{r^2} r^2 \diff \theta \diff \phi
		\]
		The \( \theta \) integral integrates to \( 4 / 3 \)
	and the \( \phi \) integral integrates to \( 2\pi \). So, we have:
	\[
		P = \frac{\mu_0b^2\pi}{6c^3}\ddot I^2
	\]
	Now, \( m = \pi b^2 I \), so \( \ddot m = \pi b^2 \ddot I \), and hence:
	\[
		P = \frac{\mu_0\ddot m^2}{6\pi c^3}
	\]
	\end{solution}
	\pagebreak
	\section*{Problem 2}
	An ideal electric dipole is situated at the origin; its dipole moment points in the \( \mathbf{\hat{z}}
	\) direction, and is quadratic in time:
	\[
		\mathbf{p}(t) = \frac{1}{2}\ddot p_0 t^2 \mathbf{\hat{z}} \quad (-\infty < t < \infty)
	\]
	where \( \ddot p_0 \) is a constant. 

	\begin{enumerate}[label=(\alph*)]
		\item Use the method of Section 11.1.2 to determine the (exact) electric and magnetic fields, for all
			\( r > 0 \) (there's also a delta-function term at the origin, but we're not concerned with
			that). [\textit{Partial answer: \( V = (\mu_0 \ddot p_0 / 8\pi) \cos \theta [(ct / r)^2 - 1] \),
			\( \mathbf{A} = (\mu_0 \ddot p / 4 \pi c) [(ct / r) - 1] \mathbf{\hat{z}} \).}]

			\begin{solution}
				In section 11.1.2, we have \( \mathbf{p}(t) = p_0 \cos(\omega t) \mathbf{\hat{z}} \). The
				retarded potential is then:
				\[
					V(\mathbf{r}, t) = \frac{1}{4\pi \epsilon_0d} \left[ \frac{\ddot p_0 \left( t - \rcurs_+ /
					c \right)^2}{2 \rcurs_+} - \frac{\ddot p_0 \left( t - \rcurs_- / c \right)^2}{2\rcurs_-} \right]
				\]
				Applying the same approximations as we did in lecture (I don't want to write them all out),
				the vector potential is given by:
				\[
					V = \frac{1}{4\pi \epsilon_0 d} \left[ \frac{\ddot p_0}{2}\left( t - \frac{\rcurs_+}{c}
					\right)^2 \frac{1}{r}\left( 1 + \frac{d}{2r}\cos \theta \right) - \frac{\ddot p_0}{2}
				\left( t - \frac{\rcurs_-}{c} \right)^2 \frac{1}{r}\left( 1 - \frac{d}{2r}\cos \theta \right) \right]
				\]
				Now, we can expand the \( \rcurs \) term:
				\[
					t - \frac{\rcurs_{\pm}}{c} = t - \frac{r}{c} \pm \frac{d}{2c}\cos \theta
				\]	
				Since we've already approximated \( \frac{d}{r} \ll 1 \), it is also reasonable to assume
				that \( \frac{d}{c} \ll 1 \), and so we drop these \( \frac{d}{c} \) terms. Therefore, the
				potential is:
				\[
					\frac{1}{4\pi \epsilon_0 d} \frac{\ddot p_0}{2}\left( t - \frac{r}{c} \right)^2 \frac{1}{r}
					\left( \frac{d}{r}\cos \theta \right) = \frac{\ddot p_0}{8 \pi \epsilon_0} \left(
					\frac{r}{c} \right)^2 \left( \frac{ct}{r} - 1 \right)^2 \frac{\cos \theta}{r^2}	
				\]
				Finally, we make the following simplification:
				\[
					\left( \frac{ct}{r} - 1 \right)^2 \approx \left( \frac{ct}{r} \right)^2 - 1
				\]
				And after this we arrive at the partial answer provided:
				\[
					V = \frac{\ddot p_0 \mu_0}{8\pi}\left[ \left( \frac{ct}{r} \right)^2 - 1 \right] \cos
					\theta
				\]
				The current is given by:
				\[
					\mathbf{I}(t) = \dv{q}{t} \mathbf{\hat{z}} = \ddot p_0 t \mathbf{\hat{z}}
				\]
				So the vector potential is:
				\[
					\mathbf{A} = \frac{\mu_0}{4\pi} \int_{-d / 2}^{d / 2} \frac{\ddot p_0 \left( t -
					\frac{\rcurs}{c} \right)}{\rcurs}\mathbf{\hat{z}} \diff z
				\]
				Using the same method in the textbook, we just replace this by the value at the center:
				\[
					\mathbf{A} = \frac{\mu_0}{4\pi}\frac{\ddot p_0(t - r / c)}{r}\mathbf{\hat{z}} =
					\frac{\mu_0\ddot p_0}{4\pi c} \left( \frac{ct}{r} - 1 \right) \mathbf{\hat{z}}
				\] 
				Now, to find the fields we use \( \mathbf{E} = -\nabla V - \partial_t \mathbf{A} \) and \(
				\mathbf{B} = \nabla \times \mathbf{A} \). So computing the gradient for spherical
				coordinates, we have:
				\begin{equation*}
					\mathbf{E} = -\frac{\ddot p_0 \mu_0}{8\pi} \cos \theta \left[ - \frac{2(ct)^2}{r^3}
					\right]\mathbf{\hat{r}} + \frac{1}{r}\frac{\mu_0 \ddot p_0}{8\pi} \sin \theta \left[
					\left( \frac{ct}{r} \right)^2 - 1 \right] \boldsymbol{\hat{ \theta}} - \frac{\mu_0 \ddot
				p_0}{4 \pi c}\left( \frac{c}{r} \right) \mathbf{\hat{z}}
				\end{equation*}
				We can convert \( \mathbf{\hat{z}} = \cos \theta \mathbf{\hat{r}} - \sin \theta
				\boldsymbol{\hat{\theta}} \), and doing some simplification we get:
				\[
					\mathbf{E} = \frac{\ddot p_0}{4\pi} \cos \theta \left[ \frac{(ct)^2}{r^3} - \frac{1}{r}
					\right]\mathbf{\hat{r}} + \frac{\mu_0 \ddot p_0}{8 \pi r} \left[ \left( \frac{ct}{r}
					\right)^2 + 1 \right]\boldsymbol{\hat{\theta}}
				\]
				As for the magnetic field, we get:
				\begin{align*}
					\nabla \times \mathbf{A} &= \nabla \times \left[ \frac{\mu_0 \ddot p_0}{4 \pi c}\left(
					\frac{ct}{r} - 1 \right)(\cos \theta \mathbf{\hat{r}} - \sin \theta
				\boldsymbol{\hat{\theta}} \right]\\
				&= \frac{1}{r} \left[ \frac{\mu_0 \ddot p_0}{4\pi c} \sin \theta \boldsymbol{\hat{\theta}} -
				\frac{\mu_0 \ddot p_0}{4 \pi c} \left( \frac{ct}{r} - 1 \right)(- \sin \theta)
			\right]\boldsymbol{\hat{\phi}} \\ 
			&= \frac{\mu_0 \ddot p_0}{4 \pi} \frac{t}{r^2} \sin \theta \boldsymbol{\hat{\phi}} 
				\end{align*}
			\end{solution}
		\item Calculate the power \( P(r, t) \), passing through a sphere of radius \( r \). [\textit{Answer:
			\( (\ddot p_0^2 / 12 \pi \epsilon_0 r^3) t [t^2 + (r / c)^2] \).}]

			\begin{solution}
				The power is calculated as 
				\[
					P = \int \mathbf{S} \cdot d\mathbf{a}
				\]
				so our first goal is to find \( \mathbf{S} \):
				\[
					\mathbf{S} = \frac{1}{\mu_0}(\mathbf{E} \times \mathbf{B}) = \left( \frac{\mu_0^2 \ddot
					p_0^2}{16 \pi^2} \right)^2 \frac{t}{r^2} \sin \theta \cos \theta \left[ \frac{(ct)^2}{r^3} -
					\frac{1}{r} \right] (-\boldsymbol{\hat{\theta}}) + \left( \frac{\mu_0^2 \ddot p_0^2t}{16
					\pi^2 r^3}
					\sin \theta \cdot \frac{1}{2}\left[ \left( \frac{ct}{r} \right)^2 + 1 \right] \right)\mathbf{\hat{r}}
				\]
				The unit vector \( \mathbf{a} \) through a sphere of radius \( r \) is in the \(
				\mathbf{\hat{r}} \) direction, so only the second term survives. Therefore, we have:
				\[
					P = \frac{\mu_0^2 \ddot p_0^2 t}{16 \pi^2 r^3} \frac{1}{2}\left[ \left( \frac{ct}{r} \right)^2 -
					1\right] r^2 \int_{0}^{\pi}\int_{0}^{2\pi} \sin^3 \theta \diff \theta \diff \phi =
					\frac{\mu_0^2 \ddot p_0^2 t}{16 \pi^2 r^3} \cdot \frac{1}{2} \left[ \left( \frac{ct}{r} \right)^2 + 1
					\right]r^2 \int \sin^3 \diff \theta \diff \phi = \frac{\mu_0^2 \ddot p_0^2 t}{12 \pi r}
					\left[ \left( \frac{ct}{r} \right)^2 + 1 \right]
				\]
				This is equivalent to the answer in the prompt. 
			\end{solution}
		\item Find the total power radiated (Eq. 11.2), and check that your answer is consistent with Eq.
			11.60.

			\begin{solution}
				The total power radiated is given by:
				\[
					P_\text{rad}(t_0) = \lim_{r \to \infty} P\left( r, t_0 + \frac{r}{c} \right)
				\]
				We just take the limit as \( r \to \infty \) of the previous answer, I just plugged this into
				mathematica and I got:
				\[
					P_\text{rad}(t_0) = \frac{ \mu_0 \ddot p_0^2}{4 c \pi}
				\]
				which is exactly consistent with Eq. 11.60.
			\end{solution}
	\end{enumerate}

	\pagebreak
	\section*{Problem 3}
	Suppose the (electrically neutral) \( yz \) plane carries a time-dependent but uniform surface current \(
	K(t) \mathbf{\hat{z}}\). 
	\begin{enumerate}[label=(\alph*)]
		\item Find the electric and magnetic fields at a height \( x \) above the plane if
			\begin{enumerate}[label=(\roman*)]
				\item a constant current is turned on at \( t = 0 \):
					\[
						K(t) = \begin{cases}
							0 & t \leq 0\\
							K_0 & t > 0
						\end{cases}
					\]
					\begin{solution}
						The plane is electrically neutral, so the scalar potential is zero, and we only have
						the vector potential. Because we have a surface current, the integral for the vector
						potential becomes:
						\[
							\mathbf{A} = \frac{\mu_0}{4\pi}\int \frac{K(t_r)}{\rcurs} \diff a
						\]
						To compute this integral, we use polar coordinates. Let \( r \) define the radial
						distance from the origin on the plane, then we can calculate this as :
						\[
							\mathbf{A} = \frac{\mu_0}{4\pi}\int \frac{K(t_r)}{\rcurs} r \diff r \diff \theta
						\]
						Due to \( K \) being a constant, we only need to integrate over a finite \( r \),
						just like we did in the previous homework:
						\[
							r_c = \sqrt{(ct)^2 - x^2}
						\]
						Therefore, we have:
						\[
							\mathbf{A} = \frac{\mu_0 K_0}{4\pi} \int_{0}^{2\pi} \int_{0}^{r_c}
							\frac{r}{\sqrt{r^2 + x^2}} \diff r \diff \theta
						\]
						Using Mathematica:
						\[
							\mathbf{A} = \frac{\mu_0K_0}{4\pi} (2\pi) (-x + \sqrt{x^2 + r_c^2}) =
							\frac{\mu_0K_0}{2} (ct - x) \mathbf{\hat{z}}
						\]
						Then, \( \mathbf{E} \) and \( \mathbf{B} \) are easy:
						\[
							\mathbf{E} = -\partial_t \mathbf{A} = \frac{\mu_0 K_0c}{2} \mathbf{\hat{z}} \quad
							\mathbf{B} = \nabla \times \mathbf{A} = \frac{\mu_0 K_0}{2} \mathbf{\hat{y}}
						\]
					\end{solution}
				\item a linearly increasing current is turned on at \( t = 0 \):
					\[
						K(t) = \begin{cases}
							0 & t \leq 0\\
							\alpha t & t > 0
						\end{cases}
					\]

					\begin{solution}
						Again, we follow the same thing as the previous problem, except now we have \( t
						\) dependence (or implicitly, we get \( r \) dependence). Now, the integral becomes:
						\[
							\mathbf{A} = \frac{\mu_0 \alpha \mathbf{\hat{z}}}{4\pi} \int_{0}^{2\pi} \int_{0}^{r_c} \frac{t -
							\rcurs / c}{\rcurs} \diff a = \frac{\mu_0 \alpha \mathbf{\hat{z}}}{2} \left[\int_{0}^{r_c}
							\frac{rt}{\rcurs} \diff r - \int_{0}^{r_c} \frac{r}{c} \diff r \right]
						\]
						so this is just equal to:
						\[
							\mathbf{A} = \frac{\mu_0 \alpha \mathbf{\hat{z}}}{2} \left[ t (ct - x) - \frac{r_c^2}{2c} \right]
						\]
						I don't want to expand \( r_c \) out so I will leave it in this form. From here, \(
						\mathbf{E} \) follows pretty quickly (I am skipping the algebra
						in the interest of clarity):
						\[
							\mathbf{E} = -\partial_t \mathbf{A} =  - \frac{\mu_0 \alpha}{2} \left[ (ct - x) +
							ct - \frac{1}{2c}(2c^2 t)\right] = - \frac{\mu_0 \alpha}{2}\left[ ct - x \right]
						\]
						The magnetic field also follows:
						\[
							\mathbf{B} = \nabla \times \mathbf{A} = -\pdv{A_z}{x} \mathbf{\hat{y}} 
							= - \frac{\mu_0 \alpha}{2} \left[ -t - \frac{1}{2c}(-2x) \right] = 
							-\frac{\mu_0\alpha}{2} \left[ \frac{x}{c} - t \right]
						\]
					\end{solution}
			\end{enumerate}
		\item Show that the retarded vector potential can be written in the form:
			\[
				\mathbf{A}(x, t) = \frac{\mu_0c}{2} \mathbf{\hat{z}} \int_{0}^{\infty} K\left( t -
				\frac{x}{c} - u \right) \diff u
			\]
			and from this determine \( \mathbf{E} \) and \( \mathbf{B} \).

			\begin{solution}
				To begin, the integral is:
				\[
					\mathbf{A}(x, t) = \frac{\mu_0}{4\pi} \int \frac{K(t_r)}{\rcurs} \diff a
				\]
				We've seen the radial symmetry here, so the \( 4\pi \) in the denominator will become just a
				2. Then, what we are integrating is:
				\[
					\mathbf{A}(x, t) = \frac{\mu_0}{2} \mathbf{\hat{z}} \int \frac{K(t - \rcurs / c)}{\rcurs} \diff a
				\]
				Writing this out in terms of \( \rcurs = \sqrt{r^2 + x^2} \):
				\[
					\mathbf{A}(x, t) = \frac{\mu_0}{2} \mathbf{\hat{z}}\int_{0}^{\infty} \frac{K\left( t - \sqrt{r^2 + x^2} /
					c \right)}{\sqrt{r^2 + x^2}} r \diff r 
				\]
				Now, we make the \( u \)-substitution such that:
				\[
					t - \frac{\sqrt{r^2 + x^2}}{c} = t - \frac{x}{c} - u
				\]
				so from here, we calculate:
				\[
					u = \frac{1}{c}\left[ \sqrt{r^2 + x^2} - x \right] \implies 
					du = \frac{1}{c}\left[\frac{1}{2} \frac{2r}{\sqrt{r^2 + x^2}} \diff r\right]
				\]
				So now, the integral becomes:
				\[
					\mathbf{A}(x, t) = \frac{\mu_0}{2}\mathbf{\hat{z}} \int_{0}^{\infty} K\left(t - \frac{x}{c} -
					u\right) \cdot \frac{r}{\sqrt{r^2 + x^2}} \cdot \frac{c \sqrt{r^2 + x^2}}{r} \diff u =
					\frac{\mu_0c}{2} \mathbf{\hat{z}} \int_{0}^{\infty}K\left( t - \frac{x}{c} - u \right)
					\diff u
				\]
				From here, we can determine \( \mathbf{E} \) and \( \mathbf{B} \):
				\[
					\mathbf{E} = -\partial_t \mathbf{A} = -\frac{\mu_0c}{2} \mathbf{\hat{z}}
					\int_{0}^{\infty}\pdv{t} K\left( t - \frac{x}{c} - u \right) \diff u
				\]
				And we can make a change of variables \( u \to t \) which costs us a negative sign, so:
				\[
					\mathbf{E} = -\frac{\mu_0c}{2}\mathbf{\hat{z}} \int_{0}^{\infty}-\pdv{u} K\left( t -
					\frac{x}{c} - u \right) \diff u = \frac{\mu_0c}{2} \left[ K(-\infty) - K\left( t -
				\frac{x}{c} \right) \right] \mathbf{\hat{z}}
				\]
				Likewise we can do the same with \( \mathbf{B} \):
				\[
					\mathbf{B} = \nabla \times \mathbf{A} = -\pdv{A_z}{x} \mathbf{\hat{y}} =
					-\frac{\mu_0c}{2} \mathbf{\hat{y}} 
					\int_{0}^{\infty}\pdv{x} K \left( t - \frac{x}{c} - u \right) \diff u
				\]
				We do the same trick as with \( \mathbf{E} \), except now it costs us a factor of \(
				\frac{1}{c} \), so we have:
				\[
					\mathbf{B} = -\frac{\mu_0c}{2}\int_{0}^{\infty}\frac{1}{c}\pdv{u} K\left( t - \frac{x}{c}
					- u \right)\diff u = -\frac{\mu_0}{2}\left[ K(-\infty) - K\left( t - \frac{x}{c} \right)
					\right] \mathbf{\hat{y}}
				\]
			\end{solution}	
		\item Show that the total power radiated per unit area of surface is 
			\[
				\frac{\mu_0c}{2}[K(t)]^2
			\]
			Explain what you mean by "radiation", in this case, given that the source is not localized. 

			\begin{solution}
				The Poynting vector is given by 
				\[
					\mathbf{S} = \frac{1}{\mu_0}(\mathbf{E} \times \mathbf{B}) = -\frac{1}{\mu_0} \left(
					\frac{\mu_0}{2} \right)^2 c \left[ K(-\infty) - K\left( t - \frac{x}{c} \right) \right]^2
					(- \mathbf{\hat{x}})
				\]
				If we allow \( K(-\infty) \to 0 \), then this equation simplifies to:
				\[
					\mathbf{S} = \frac{\mu_0c}{2}\left[ K\left( t - \frac{x}{c} \right) \right]^2
					\mathbf{\hat{x}}
				\]
				Therefore, at any moment in time, the total power radiated should not be calculated relative
				to point \( x \) but instead calculated absolutely, in which case we replace the argument to
				\( K \) by \( K(t) \). Therefore, we have the total power per unit area:
				\[
					\mathbf{S} = \frac{\mu_0c}{2}\left[ K(t) \right]^2 \mathbf{\hat{x}}
				\]
				Here, radiation still means the same thing -- basically the portion of
				the EM field that has a \( \frac{1}{r} \) dependence, which decays slowly enough that it is
				nonzero even at infinity. We can see this in the fact that the vector potential we integrated 
				always has an \( \frac{1}{\rcurs} \) dependence. 
			\end{solution}
	\end{enumerate}
\end{document}
