\documentclass[10pt]{article}
\usepackage{../../local}
\urlstyle{same}

\newcommand{\classcode}{Physics 110B}
\newcommand{\classname}{Electromagnetism and Optics II}
\renewcommand{\maketitle}{%
\hrule height4pt
\large{Eric Du \hfill \classcode}
\newline
\large{HW 03} \Large{\hfill \classname \hfill} \large{\today}
\hrule height4pt \vskip .7em
\small{Header styling inspired by Berkeley EECS Department: \url{https://eecs.berkeley.edu/}}
\normalsize
}
\linespread{1.2}
\begin{document}
	\maketitle
	\section*{Problem 1}
	\begin{enumerate}[label=(\alph*)]
		\item Show explicitly that a standing wave alone in the \( z \)-direction, \( f(z, t) = A \sin(kz)
			\cos(\omega t) \), satisfies the 1D wave equation \( (\partial_z^2 - \frac{1}{v^2}\partial_t^2) f
			= 0\). What is the relation between \( \omega \) and \( k \) in order for \( f(z, t) \) to
			satisfy the wave equation?

			\begin{solution}
				We can compute the two terms separately:
				\[
					\partial_z^2 f(z, t) = -Ak^2 \sin(kz)\cos(\omega t) \quad \frac{1}{v^2}\partial_t^2 f =
					-\frac{1}{v^2} A \omega^2 \sin(kz) \cos(\omega t)
				\]
				Putting these two together, we get the equation:
				\[
					0 = -Ak^2 \sin(kz) \cos(\omega t) + \frac{1}{v^2}A \omega^2 \sin(kz) \cos(\omega t) =
					\frac{\omega^2}{v^2} - k^2 
				\]
				So we require that \( \frac{\omega}{v} = k \) or \( \omega = k v \). 
			\end{solution}
		\item As discussed in class, it is sometimes mathematically convenient to express a sinusoidal wave
		\( A \cos(kz - \omega t + \delta) \) as \( \Re\left[\tilde A e^{i(kz - \omega t)}\right] \) where we
		absorb the phase constant into the amplitude and define the \textit{complex amplitude} \( \tilde A
		\equiv Ae^{i \delta } \). Using this, show that if we have two sinusoidal waves
		\[
			f_1 = A_1\cos(kx - \omega t + \delta_1) \quad \text{and} \quad f_2 = A_2\cos(kx - \omega t +
			\delta_2)
		\]
		the superposition of the two waves still yield a wave of the same frequency and wavelength where
		\[
			f_3 = f_1 +f_2 = A_3 \cos(kz - \omega t + \delta_3)
		\]
		and find \( A_3 \) and \( \delta_3 \). 

		\begin{solution}
			Using Euler's formula:
			\[
				f_1 + f_2 = \Re\left[ A_1 e^{i(kz - \omega t + \delta_1)} \right]  + \Re\left[ A_2e^{i(kz -
				\omega t + \delta_2)} \right] = \Re\left[ A_1 e^{i (kz - \omega t + \delta_1)} + A_2e^{i (kz
				- \omega t + \delta_2)} \right]
			\]
			We can then factor out a common factor of \( Ae^{i(kz - \omega t)} \):
			\[
				f_1 + f_2 = \Re\left[ e^{i(kz - \omega t)}\left( A_1 e^{i \delta_1} + A_2e^{i\delta_2}
				\right) \right]
			\]
			We will focus on the term in parentheses now, since that will be our \( A_3e^{i \delta_3} \), the
			new amplitude and phase. Using \( e^{i \theta} = \cos \theta + i \sin \theta
			\), then we get:
			\[
				A_3 e^{i \delta_3} = A_1 e^{i \delta_1} + A_2 e^{i \delta_2} 
				= A_1 \cos \delta_1 + A_2 \cos \delta_2 + i(A_1 \sin
				\delta_1 + A_2 \sin \delta_2)
			\]
			So then, the amplitude is given by:
			\[
				A_3 = \sqrt{(A_1 \sin \delta_1 + A_2 \sin \delta_2)^2 + (A_1 \cos \delta_1 + A_2 \cos
				\delta_2)^2}
			\]
			and the phase is:
			\[
				\delta_3 = \tan^{-1}\left( \frac{A_1 \sin \delta_1 + A_2 \cos \delta_2}{A_1 \cos \delta_1 
				+ A_2 \cos \delta_2} \right) 
			\]
			And thus, we can conclude:
			\[
				f_1 + f_2 = \Re\left[ A_3e^{i \delta_3} e^{i(kz - \omega t)} \right] = A_3 \cos(kz - \omega t
				+ \delta_3)
			\]
			as required. It's not a very clean solution, but I don't see anything better unfortunately. 
		\end{solution}
	\end{enumerate}

	\pagebreak
	\section*{Problem 2}
	In Griffiths Chap. 9.1, he reviewed the topic of waves propagating in 1D strings. Consider a string of
	mass density \( \mu_1 \) connected with a string of mass density \( \mu_2 \). The first string lies in the
	region \( z < 0 \), while the second string is in region \( z > 0 \). The two strings are connected at \(
	z = 0\). Since they are tied together, they have the same tension \( T \), resulting in wave speeds of \(
	v_1 = \sqrt{T / \mu_1}\) and \( v_2 = \sqrt{T / \mu_2} \) in the two strings respectively. Suppose we
	have a wave \( f(z, t) \) propagating in the strings.
	\begin{enumerate}[label=(\alph*)]
		\item What is the physical reason to impose the boundary condition \( f(0^{-}, t) = f(0^{+}, t) \)
			where the wavefunction is continuous at \( z = 0 \)?

			\begin{solution}
				The physical reason is that we require the string to be continuous, so the value of \( f \)
				should approach the same value from both sides at every time \( t \).  
			\end{solution}
		\item Suppose the connecting point has negligible mass, what is the physical reason to impose the
			boundary condition
			\[
				\pdv{f}{z}\eval_{0^{-}} = \pdv{f}{z}\eval_{0^{+}}
			\]
			\begin{solution}
				Despite this being technically two strings attached together at \( z = 0 \), we want the
				entire system to behave like one string with two different mass densities. As such, we
				require there to be no corners or other discontinuities in the string, and as such we impose
				that the string be differentiable at \( z = 0 \) from both sides, and we impose that the
				derivatives are equal so that the connection point does not become a corner.    
			\end{solution}
		\item Suppose you send an incident wave from the left \( (z < 0) \) with a specific shape \( g_I(z -
			v_1t) \). It gives rise to a reflective wave \( h_R(z + v_1t) \) in \( z < 0 \) and a transmitted
			wave \( g_T(z - v_2t) \) in \( z > 0 \). By imposing the boundary condition in (a) and (b), find
			\( h_R \) and \( g_T \). 

			\begin{solution}
				We do the same thing as we did in lecture. The full equation for the wave is:
				\[
					f(z, t) = \begin{cases}
						g_I(z - v_1t) + h_R(z + v_1t) & z < 0\\
						g_T(z - v_2t) & z > 0
					\end{cases}
				\]
				Now, we impose the first boundary condition, which gives:
				\[
					g_I(0^{-} - v_1t) + h_R(0^{-} + v_1t) = g_T(0^{+} - v_2t)
				\]
				Given that the wave is continuous, this basically just translates to:
				\begin{equation}
					\label{cond1}
					g_{I}(-v_1t) + h_R(v_1t) = g_T(-v_2t)
				\end{equation}
				Now for the second condition. Here, we use the results right after equation 9.4 (I use the
				fourth edition), we have:
				\[
					\pdv{f}{z} = \dv{g}{u} \pdv{u}{z} = \dv{g}{u} \quad \pdv{f}{t} = -v\dv{g}{u}
				\]
				So putting these two equations together to eliminate \( \dv{g}{u} \), since we don't have the
				explicit form of \( g \):
				\[
					\pdv{f}{z} = -\frac{1}{v}\pdv{f}{t}
				\]
				And as such, we have the relations:
				\begin{align*}
					\pdv{g_I}{z} &= -\frac{1}{v_1}\pdv{g_I}{t}\\
					\pdv{h_R}{z} &= \frac{1}{v_1}\pdv{h_R}{t} \\ 
					\pdv{h_T}{z} &= -\frac{1}{v_2}\pdv{g_T}{t} \\ 
				\end{align*}
				Now, putting the two ends together and evaluating at \( z = 0 \), we have:
				\[
					-\frac{1}{v_1}\pdv{g_I(-v_1t)}{t} + \frac{1}{v_1} \pdv{h_R(v_1t)}{t} 
					= -\frac{1}{v_2}\pdv{g_T(-v_2t)}{t} 
				\]
				Now, taking the partial derivative out, we see that this boundary condition basically
				translates to:
				\begin{equation}
					\label{cond2}
					-\frac{1}{v_1}g_I(-v_1t) + \frac{1}{v_1}h_R(v_1t) = -\frac{1}{v_2} g_T(-v_2t)
				\end{equation}
				Multiplying this equation on both sides by \( v_2 \) and adding both gives:
				\[
					g_I(-v_1t) \left( 1 - \frac{v_1}{v_2} \right) + 2h_R(v_2t) = 0
				\]
				Finally, we get:
				\[
					h_R = \frac{g_I}{2}\left( \frac{v_2}{v_1} - 1 \right)
				\]
				Notice that I've dropped the evaluations in this final equation. This is allowed, since the
				form of \( h_R \) is determined by \( g_I \), and it should be determined over all space and
				time to be consistent. Multiplying \ref{cond2} by \( v_2 \) and subtracting instead gives \(
				g_T\):
				\[
					g_T = \frac{g_I}{2}\left( 1 + \frac{v_1}{v_2} \right)
				\]
				and we are done. 
			\end{solution}
	\end{enumerate}

	\pagebreak
	\section*{Problem 3}
	Suppose string 2 is embedded in a viscous medium (such as molasses), which imposes a drag force that is
	proportional to its (transverse) speed:
	\[
		\Delta F_\text{drag}= -\gamma \pdv{f}{t} \Delta z
	\]
	\begin{enumerate}[label=(\alph*)]
		\item Derive the modified wave equation describing the motion of the drag. 

			\begin{solution}
				With the drag force, our net force now becomes:
				\[
					\Delta F = T \pdv[2]{f}{z} \Delta z - \gamma \pdv{f}{t} \Delta z
				\]
				Combining with Newton's second law, then we have:
				\[
					\mu (\Delta z) \pdv[2]{f}{t} = (\Delta z) \left( T\pdv[2]{f}{z} - \gamma \pdv{f}{t} \right)
				\]
				So we come to the equation :
				\[
					\mu \pdv[2]{f}{t} + \gamma \pdv{f}{t} = T \pdv[2]{f}{z}
				\]
			\end{solution}
		\item Solve this equation, assuming the string vibrates at the incident frequency \( \omega \). That
			is, look for solutions of the form \( \tilde f(z, t) = e^{-i \omega t}\tilde F(z) \).  

			\begin{solution}
				Evaluating the partial derivatives to the suggested form of \( f \), we get:
				\begin{align*}
					\pdv[2]{f}{t} &= -\omega^2 e^{-i \omega t}\tilde F(z)\\
					\pdv{f}{t} &= - i \omega e^{-i \omega t}\tilde F(z) \\ 
					\pdv[2]{f}{z} &= e^{-i \omega t}\pdv[2]{\tilde F}{z} 
				\end{align*}
				So the differential equation evaluates as:
				\[
					\mu (- \omega^2) e^{-i \omega t}\tilde F(z) - \gamma i \omega e^{-i \omega t}\tilde F(z)
					= T e^{-i \omega t}\pdv{\tilde F}{z}
				\]
				so now we're left with:
				\[
					T \pdv[2]{\tilde F}{z} = \tilde F(z) \left( -\mu \omega^2 - \gamma i \omega \right)
					\implies \pdv[2]{\tilde F}{z} + \frac{\gamma i \omega + \mu \omega^2}{T} \tilde F(z) = 0
				\]
				I plugged this differential equation into Mathematica (I just didn't want to through all the
				algebra), we get solutions of the form:
				\[
					F(z) = c_1\exp\left(-\frac{(-1)^{3 / 4}z \sqrt{ \omega} \sqrt{\gamma - i
						\omega}}{\sqrt{T}}\right) + 
					c_2 \exp\left(\frac{(-1)^{3 / 4}z \sqrt{ \omega} \sqrt{\gamma - i
						\omega}}{\sqrt{T}}\right)
				\]
				Yeah. It's not pretty but those are the solutions. 
			\end{solution}
		\item Show that the waves are \textbf{attenuated} (that is, their amplitude decreases with increasing
			\( z \)). Find the characteristic penetration distance, at which the amplitude is \( 1 / e \) of
			its original value, in terms of \( \gamma, T, \mu \) and \( \omega \). 

			\begin{solution}
				From the previous part, we see that \( F(z) \) is comprised of two exponentials, one positive
				and one negative. We reject the positive one, because this corresponds to a wave which is
				gaining energy, and is thus nonphysical. As such, we only have one physical solution:
				\[
					F(z) = c_1 e^{- \kappa z}
				\]
				I've eaten all the constants inside \( \kappa \in \mathbb C\) because I don't want to write
				it all out. With this, we get the full form of \( \tilde f(z, t) \) inside medium 2:
				\[
					\tilde f(z, t) = A_T e^{-i (\omega t - kz)}e^{-\kappa z}
				\]
				I've replaced \( c_1 \) with \( A_T \) to match the convention used in the book. Here, we see
				that the amplitude is does indeed decrease with increasing \( z \), because of the \(
				e^{-\kappa z} \) term.   
				The characteristic penetration distance is given by \( \frac{1}{\kappa} \), so 
				this is equal to:
				\[
					\frac{1}{\kappa} = \sqrt{\frac{T}{\omega}} \frac{1}{\sqrt{ \gamma - i \omega}}
					\frac{1}{(-1)^{\frac{3}{4}}}
				\]
			\end{solution}	
		\item If a wave of amplitude \( A_I \), phase \( \delta_i = 0 \), and frequency \( \omega \) is
			incident from the left (string 1), find the reflected wave's amplitude and phase.

			\begin{solution}
				The full form of \( \tilde f(z, t) \) over all space and time is given then as:
				\[
					\tilde f(z, t) = \begin{cases}
						\tilde A_I e^{-i (\omega t - k_1z)} + A_R e^{-i (\omega t + k_1z + \delta_R)} & z < 0\\
						\tilde A_T e^{-i (\omega t - k_2z - \kappa z)} & z > 0
					\end{cases}
				\]
				We can define a new \( k_2' = k_2 + i\kappa \), so now \( \tilde f(z, t) \) is:
				\[
					\tilde f(z, t) = \begin{cases}
						\tilde A_I e^{-i (\omega t - k_1z)} + \tilde A_R e^{-i (\omega t + k_1z)} & z < 0\\
						\tilde A_T e^{-i (\omega t - k_2' z)}
					\end{cases}
				\]
				So this is just the same as the standard case, except now \( k_2' \in \mathbb C \), but that
				doesn't change the form of the answer. We can use Equation 9.29 in the textbook (again,
				fourth edition numbering), which gives us:
				\[
					\tilde A_R = \left( \frac{k_1 - k_2'}{k_1 + k_2'} \right) \tilde A_I
				\]
				Next we can use Mathematica to separate the magnitude and phase. To do the magnitude, we
				can write \( |\tilde A_R| = \sqrt{\tilde A_R \tilde A_R^{*}} \) or basically we square root
				\( A_R \) times its conjugate. This gives:
				\[
					A_R = \sqrt{\frac{k_1 - 2k_1k_2 + k_2^2 + \kappa^2}{k_1 + 2 k_1 k_2 + k_2^2 + \kappa}}
					A_I
				\]
				to find the phase, I ran \texttt{ComplexExpand} on \( A_R \) to get the real and imaginary
				parts:
				\[
					A_R = \frac{k_1^2 - k_2^2 - \kappa^2}{(k_1 + k_2)^2 + \kappa^2} A_I  - \frac{2 i k_1
					\kappa}{(k_1 + k_2)^2 + \kappa^2} A_I
				\]
				So the phase \( \delta_R \) is given by:
				\[
					\delta_R = \tan^{-1}\left( \frac{-2 k_1 \kappa}{(k_1 + k_2)^2 + \kappa^2} \right)
				\]
			\end{solution}
	\end{enumerate}
\end{document}
