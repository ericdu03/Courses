\section{January 31}
We'll pick up where we left off from last time, talking about the momentum flux:
\[
	\dv{\mathbf{p}_\text{particle}^{i}}{t} + \dv{t} \int_{\mathcal{V}} \frac{\mathbf{S}}{c^2} \diff \tau =
	\oint_{\partial \mathcal{V}}\sigma^{ik} \diff \mathbf{a}_k
\]
We mentioned how you can interpret this in two main ways: first, you can interpret this as an equation like
\( \mathbf{F} = \dv{\mathbf{p}}{t} \), and think of the term on the right hand side as a generalized force \(
\mathbf{F}_\text{net}^{i}\). The other way to think about it, and the way we will focus on, is thinking of it
as a continuity equation. So, in the same way charge is conserved, it is valid to think of this equation as
\[
	\dv{\mathbf{p}_\text{net}}{t} = -\oint_{\partial \mathcal{V}} T^{ik} \diff \mathbf{a}_k
\]
Here, we define \( T^{ik} = - \sigma^{ik} \), so \( T \) is also a rank (2, 0) tensor. Intuitively, we think
of this quantity as the \textit{momentum flux}, or the "flow of momentum" in or out of \( \mathcal{V} \). One
thing to note is that \( T^{ik} \) is defined so that outgoing flow is positive, which is the opposite
convention of \( \sigma^{ik} \). To complete the equation, it is actually nice to separate out the particle
momentum from the field momentum, so we write:
\[
	\dv{\mathbf{p}_\text{particle}^{i}}{t} + \dv{t} \int_\mathcal{V} \frac{\mathbf{S}^2}{\diff \tau} =
	-\oint_{\mathcal{V}}\left(T_\text{EM}^{ik} + T_\text{particle}^{ik}\right) \diff \mathbf{a}_k
\]
And here we've essentially defined a new tensor \( T_\text{particle}^{ik} = mnv^{i}v^{k} \). 

\begin{example}[Radiation Pressure]
	In this example, imagine we have a "fluid of radiation", with a radiation pressure \( P \). It is well
	known that \( P = \frac{1}{2}U_\text{EM} = \frac{1}{3}\left( \frac{\epsilon_0}{2}|\mathbf{E}|^2 +
	\frac{1}{2\mu_0}|\mathbf{B}|^2 \right) \). We will prove this using \( T^{ik} \).   

	\begin{solution}
		Recall that the equation for \( \sigma^{ik} \) is
		\[
			\sigma^{ik} = \epsilon_0 \left[ E^{i}E^{k} - \frac{1}{2}\delta^{ik} |\mathbf{E}|^2 \right] +
			\frac{1}{\mu_0}\left( B^{i}B^{k} - \frac{1}{2}\delta^{ik}|\mathbf{B}|^2 \right)
		\]
		The pressure here is the diagonal term, as we mentioned from last time. Next, we make the observation
		that for a stationary EM fluid (really just think about this as EM waves propagating out in all
		directions), then the system should be isotropic. That is, there should not be a preference of one
		axis over another. Thus, we can write
		\[
			\mean{E_x^2} = \mean{E_y^2} = \mean{E_z^2} \implies \mean{|\mathbf{E}|^2} = \mean{E_x^2} +
			\mean{E_y^2} + \mean{E_z^2} = 3\mean{E_x^2}
		\]
		The \( \mathbf{B} \) field follows suit in the same way. 
		And thus, the pressure \( P = \mean{\sigma^{11}} \) can be calculated as:
		\begin{align*}
			P = \mean{\sigma^{11}} &= \epsilon_0 \left( \mean{E_x^2} - \frac{1}{2}\mean{|\mathbf{E}|^2}
			\right) + \frac{1}{\mu_0}\left( \mean{B_x^2} - \frac{1}{2} \mean{|\mathbf{B}|^2}\right) \\
								   &= \epsilon_0 \left( \frac{1}{3}\mean{|\mathbf{E}|^2} -
								   \frac{1}{2}\mean{|\mathbf{E}|^2} \right) + 
								   \frac{1}{\mu_0} \left( \frac{1}{3}\mean{|\mathbf{B}|^2} -
								   \frac{1}{2}\mean{|\mathbf{B}|^2} \right)\\
								   &= -\frac{1}{6}\left( \epsilon_0 \mean{|\mathbf{E}|^2} +
								   \frac{1}{\mu_0}\mean{|\mathbf{B}|^2} \right) \\ 
								   &= -\frac{1}{3}U_\text{EM} 
		\end{align*}
		Note the reason this is negative here is because we use \( T^{ik} \) instead of \( \sigma^{ik} \),
		which are opposite to each other in sign. That is, because radiation flows \textit{away} from an
		object, this is seen as "negative pressure".  
	\end{solution}
\end{example}

One thing to note about \( \sigma^{ik} \) is that it is a symmetric tensor, \( \sigma^{ik} = \sigma^{ki} \).
To see why this is necessary, consider a cube in space (see diagram), 
and we have a nonzero \( \sigma^{21} \) acting on it. Then,
there is a shear acting in the \( \hat{\mathbf{x}} \) direction, and an equivalent one acting on the back of
the cube in the \( -\mathbf{\hat{x}} \) direction. In order for this to not generate an angular momentum on
the cube, it is necessary that we have \( \sigma^{12} \) on the other two sides in order for the torque to
cancel out, and for the cube to remain stationary. In other words, the takeaway is this:
\begin{center}
	\fbox{
	\begin{minipage}{0.72\textwidth}
		In order for angular momentum to be conserved locally, the stress tensor must be symmetric.    
	\end{minipage}
	}
\end{center}

\subsection{Energy and Momentum for the EM Field}
Recall that \( \mathbf{S} \) represents the energy flux, and we mentioned earlier that the second term in
\ref{cont-momentum} is an integral of the momentum in the EM field, so naturally we can think of \(
\frac{1}{c^2}\mathbf{S} \) as the momentum density in the fields. Now, consider a small cylinder with energy
\( U_\text{EM} \) and cross sectional area \( A \). Then, the energy flow through \( A \) can be written as:
\[
	\mathbf{S} = \text{energy flow through \( A \)} = \frac{\text{energy passing through}}{\text{area}} =
	\frac{U_\text{EM}(c \Delta t) A}{A \delta t} = U_\text{EM} c
\] 
The length is \( c \Delta t \) because EM waves travel at the speed of light. Now, since the momentum density
\( \vec{\mathscr{P}} \) is written as \( \frac{1}{c^2}\mathbf{S} \), then we have the equation:
\[
	|\vec{\mathscr{P}}| c^2 = U_\text{EM} c \implies |\vec{\mathscr P}| c = U_\text{EM}
\]
Finally, recall that in special relativity, we have the relation \( E^2 = p^2 c^2 + m^2 c^{4} \), and since
photons are thought of as electromagnetic waves, comparing these two conclusions forces us to conclude that
photons are massless! 

And this discussion concludes the content for chapter 8. Next, we will enter chapter 9, where we talk about
electromagnetic waves.   

\subsection{EM Waves} 
Recall from your introductory classes that a sinusoidal plane wave propagating in the \( \hat{\mathbf{x}} \) direction
can be described by 
\[
	\phi(\mathbf{r}, t) = A \cos(kx - \omega t + \delta)
\]
We can also write this as a complex exponential, leveraging Euler's identity \( e^{i \theta} = \cos \theta +
i \sin \theta\):
\[
	\phi(\mathbf{r}, t) = \Re\left[ Ae^{i(kx - \omega t)}e^{i \delta} \right] =
	\Re\left[\tilde A e^{i(kx - \omega t)}\right]
\]
Here, we let \( \tilde A = Ae^{i \delta} \), so \( A \in \R \) while \( \tilde A \in \C \). We don't have
time to continue this discussion, but a primer for what we will be doing next lecture, we will explore the
formalism behind the equation for a plane wave that travels in multiple dimensions. That is, we will explain
why when dealing with more than one dimension, we write:
\[
\phi'(\mathbf{r}, t) = \Re\left[\tilde A e^{i(\mathbf{k} \cdot \mathbf{r} - \omega t)}\right]
\]
with \( |k| = \frac{2\pi}{\lambda} \). In particular, why do we have \( \mathbf{k}\cdot \mathbf{r} \)?
 



