\section{April 21}
\subsection{Self-Force}
Last lecture, we left off with the equation:
\begin{equation}
	\label{abraham-lorentz-partial}
	\frac{q^2 \dot a}{12 \pi \epsilon_0 c^2} = \left( m_0 + \frac{1}{4\pi \epsilon_0} \frac{(q / 2)^2}{d}
	\right)a
\end{equation}
where we now have an added term on the right as an "added mass" term due to the velocity of the particle.
Based on the way we've structured this equation, you can regard the left hand side as the \( \mathbf{F}_\text{rad} \)
that we're looking for. However, you may notice that this formula is off by a factor of \(1 / 2\),
and the reason for this is so far we've only considered the contribution from one \( q / 2 \) to the other \(
q / 2\) charge, but didn't consider the self force on the \( q / 2 \) charges \textit{themselves}.  

To reconcile this, suppose we have:
\[
	\mathbf{F}_\text{rad}(q) = \mathbf{F}_\text{rad, int}\left( \frac{q}{2} \right) 
	+ \mathbf{F}_\text{rad, self}\left( \frac{q}{2}\right)
\]
Now we break each of the \( q / 2 \) charges into dumbbells with charge \(  q / 4 \), so now:
\[
	\mathbf{F}_\text{rad} \left( \frac{q}{2} \right) = \mathbf{F}_\text{rad, int}\left( \frac{q}{4} \right) +
	2 \mathbf{F}_\text{rad, self}\left( \frac{q}{4} \right)
\]
Here, we can drop the "self" subscript in \( \mathbf{F}_\text{rad, self} \) since the force itself is a
self-force, so it's functionally the same as \( \mathbf{F}_\text{rad} \). Using this fact, we move it to the
left hand side. Then, if you keep iterating this process, making smaller and smaller dumbbells, you
eventually get the system:
\begin{align*}
	\mathbf{F}_\text{rad}(q) - 2 \mathbf{F}_\text{rad}\left( \frac{q}{2} \right) &= \mathbf{F}_\text{rad,
	int}\left( \frac{q}{2} \right)\\
		2 \mathbf{F}_\text{rad}\left( \frac{q}{2} \right) - (2\times 2) \mathbf{F}_\text{rad}\left( \frac{q}{4} \right) &= 2
	\mathbf{F}_\text{rad, int} \left( \frac{q}{4} \right)\\
	4 \mathbf{F}_\text{rad}\left( \frac{q}{4} \right) - (4 \times 2)\mathbf{F}_\text{rad}\left( \frac{q}{8}
	\right) &= \mathbf{F}_\text{rad, int}\left( \frac{q}{4} \right)\\
			&\vdots
\end{align*}
We now add all these equations up together, and we see that the left hand telescopes to \(
\mathbf{F}_\text{rad}(q) \). The right hand side, using our partial result (\cref{abraham-lorentz-partial}),
we get:
\[
	\mathbf{F}_\text{rad}(q) = \frac{\mu_0 \dot a q^2}{3 \pi c} \cdot \frac{1}{4} \left( 1 + \frac{1}{2} +
	\frac{1}{4} + \frac{1}{16} + \dots \right) = \frac{\mu_0 \dot a q^2}{12 \pi c} \frac{1}{1 - 1 / 2} =
	\frac{\mu_0 \dot a q^2}{6 \pi c}
\]
which is exactly the Abraham-Lorentz formula. This also concludes our discussion of chapter 11, and we now
move on to special relativity. 

A small foreword on the special relativity section: it does \textbf{not} follow chapter 12 of Griffiths, but
instead Chien-I's own provided notes. When I wrote up these notes, the content between the chapters is
relatively similar, but it should be noted that there will be differences at times (especially in the last
two lectures).    
  

\subsection{Special Relativity}
To begin the topic of special relativity, a good place to start are the fundamental postulates of special
relativity. These are:
\begin{enumerate}[label=\arabic*.]
	\item Physical laws should be the same in all \textbf{inertial} reference frames. 
	\item Motivated by Maxwell's equations \( c = \frac{1}{\sqrt{\mu_0 \epsilon_0}} \). Based on postulate 1,
		this implies that the speed of light is constant in all inertial reference frames. 
\end{enumerate}
Before we begin our discussion of special relativity is it interesting to note that while the second
postulate seems rather arbitrary, it is in fact quite a natural thing to suppose: the constants \( \mu_0 \)
and \( \epsilon_0 \) govern the strength of electric and magnetic fields, so if we are to believe the first
postulate, then we cannot believe \( \mu_0 \) or \( \epsilon_0 \) to change between reference frames -- and
from here the constancy of \( c \) across reference frames drops out. 

With these postulates established, simultaneity is now broken! Consider a situation where Alice is at rest
(\( v = 0 \)) and Bob is in a rocket ship moving at constant velocity \( v_0 \). As the rocket ship passes by
Alice, Bob emits two EM signals, which travel in opposite directions, as shown in the diagram below:
\begin{center}
	\begin{tikzpicture}[scale=1.5]
    \foreach \x/\y/\color/\opacity in {0/0/black/1,0/-1/black/0.5,1.5/-1/blue!70!cyan!40/0.6,3/-1/blue!40!cyan!60/0.7} {
        \draw[thick,color=\color,opacity=\opacity] (\x,\y) rectangle (\x+2.25,\y+0.5);
    }
    \foreach \pos in {(0.15,0.225), (0.15,-0.775)} {
        \node at \pos (a) {\Strichmaxerl[1.5]};
        }
    \foreach \x/\y/\color/\opacity in {0.4/0.1/black/1,0.4/0.4/black/1,0.4/-0.9/black/0.5,1.9/-0.6/blue!70!cyan!40/0.6,3.4/-0.9/blue!40!cyan!60/0.7} {
        \draw[thick,color=\color,opacity=\opacity] (\x,\y) -- (\x+0.25,\y);
    }
    \foreach \pos/\opacity in {(0,0.25)/1,(0,-0.75)/0.4} {
        \node[scale=0.5,thick,draw,isosceles triangle,minimum size=2pt,isosceles triangle apex angle=90,xshift=-0.3cm,opacity=\opacity] () at \pos {};
    }
    \draw[orange!70!white,-{Stealth[length=2pt]}] (0.425,0.15) -- (0.425,0.35);
    \draw[orange!70!white,-{Stealth[length=2pt]}] (0.625,0.35) -- (0.625,0.15);
    \draw[orange!70!white,-{Stealth[length=2pt]}] (0.525,-0.85) -- (2.015,-0.65) node[midway,above,scale=0.5] {$D'$};
    \draw[orange!70!white,-{Stealth[length=2pt]}] (2.035,-0.65) -- (3.525,-0.85) node[midway,above,scale=0.5] {$D'$};
    \foreach \x in {0.625,2.15} {
        \draw[green!70!blue!50] (\x,-1.1) -- (\x,-1.2);
    }
    \draw[green!70!blue!80] (0.625,-1.15) -- (2.15,-1.15) node[midway,below,scale=0.75] {$\frac12v\Delta t'$};
\end{tikzpicture}
\end{center}
From Bob's reference frame, because he is moving with the rocket, the two signals reach either end of the
ship simultaneously. However, in Alice's frame, the red signal reaches the left end before the right, because
the motion of the ship meant that the red signal had less distance to cover.
So, while Bob's reference frame preserves the simultaneity of both events, that simultaneity is broken
in Alice's reference frame, but Bob's frame is no more "correct" than Alice's. 

\subsection{Time Dilation}

Now suppose we have the following diagram, where Bob has a "clock" which sends a light signal up and down:
\begin{center}
	\begin{tikzpicture}[scale=1.5]
    \draw[thick] (0,0) rectangle (2.25,0.5);
    \node[scale=0.5,thick,draw,isosceles triangle,minimum size=2pt,isosceles triangle apex angle=90,xshift=-0.3cm] (a) at (0,0.25) {};
    \node at (1.125,0.225) (b) {\Strichmaxerl[1.5]};
    \node[scale=0.5] at (1.125,-0.1) (ll) {Bob};
    \node[scale=0.5] at (0,-0.2) (aa) {Alice};
    \node at (0,-0.5) (c) {\Strichmaxerl[1.5]};
    \draw[thick,-{Stealth[length=2pt]}] (1.125,0.6) -- (1.5,0.6) node[midway,above,scale=0.5] {$v$};
\end{tikzpicture}
\end{center}
If the distance between the two ends is \( D \), then the time measured by Bob would be \( \Delta t =
\frac{2D}{c} \). However, for Alice, due to the fact that the rocket ship is moving relative to her, would
see the time interval as:
\[
	\Delta t' = \frac{2D'}{c} = 2 \frac{\sqrt{\frac{1}{4} v^2 (\Delta t')^2 + D^2}}{c}
\]
Rearranging this to solve for \( \Delta t' \), we get:
\[
	\Delta t' = \frac{1}{\sqrt{1 - v^2 / c^2}}(\Delta t) = \gamma \Delta t
\]
This is the formula we are all familiar with to calculate time dilation. The factor \( \gamma \equiv
\frac{1}{\sqrt{1 - v^2 / c^2}} \) is a common one in relativity that you should be familiar with. We will
continue our discussion of this next time. 
