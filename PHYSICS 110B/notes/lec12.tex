\section{February 19}

Today, we will continue our discussion from last time about total internal reflection. Last time, we left off
with acknowledging that in the case where we move an angle \( \theta_I \) past the critical angle for total
internal reflection \( \theta_c \), that the \( \sin\theta_T \) term becomes imaginary. To further explore
this concept, consider now the
equations for the electric field on both sides of the medium:
\begin{align*}
	\mathbf{E} = \begin{cases}
		\Re\left\{ \tilde{\mathbf{E}}_I e^{i (\mathbf{k}_i \cdot \mathbf{r} - \omega t)} + \mathbf{E}_R
		\right\}e^{i(\mathbf{k}_R \cdot \mathbf{r} - \omega t)} & z < 0\\
		\Re\left\{ \tilde{\mathbf{E}}_{T} e^{i(\mathbf{k}_T \cdot \mathbf{r} - \omega t)} \right\}
	\end{cases}
\end{align*}
In the case where we don't have total internal reflection, it was natural to write \( \mathbf{k}_T = k_T \cos
\theta_T \mathbf{\hat{z}} + k_T \sin \theta_T \mathbf{\hat{x}} \). This was natural specifically because you
could imagine decomposing the vector \( \mathbf{k}_T \) into its \( x \) and \( z \) components. Now, the
trick when \( \sin \theta_T > 1 \) is to stop treating this as a geometric picture, but instead just
interpret \( \cos \theta_T \) and \( \sin \theta_T \) as a way to parametrize \( k_T \) in terms of \(
\theta_T \). This way, this decomposition is still allowed. Thus, we can still use Snell's law:
\[
	\sin \theta_T = \frac{v_2}{v_1} \sin \theta_I
\]
Similarly, we can expand \( \cos \theta_T \):
\begin{equation}
	\label{12:cos-imaginary}
	\cos \theta_T = \sqrt{-(\sin^2 \theta_T - 1)} = i \sqrt{\left( \frac{n_1}{n_2} \sin \theta_I \right)^2 -
	1} = i \frac{1}{n_2}\sqrt{(n_1 \sin \theta_I)^2 - n_2^2}
\end{equation}
So with this parametrization, the solution in the transmitted region is now:
\begin{align}
	E_T &= \Re\left\{ \tilde{\mathbf{E}}_T e^{i ( k_T \cos \theta_T z + k_T \sin \theta_T x - \omega t)} \right\}\\
	\label{12:ET}&= \Re\left\{ \tilde{\mathbf{E}}_T e^{-\kappa z} e^{i (k_T \sin \theta_T - \omega t)} \right\} 
\end{align}
We define \( \kappa \) as: 
\[
	\kappa = \frac{k_T}{n_2}\sqrt{(n_1 \sin \theta_I) - n_2^2} = \frac{\omega}{c} \sqrt{(n_1 \sin
\theta_I)^2 - n_2^2} 
\]
the key thing to note is that in equation \ref{12:ET}, the transmitted wave is not zero, but is a wave which
decays every quickly with a factor of \( e^{-\kappa t} \). Because of this quick decay, it's sometimes called
the \textbf{evanescent wave}. This wave transfers no energy, which can be seen through the computation of \(
R \). Because \( \cos \theta_T \) is purely imaginary by equation \ref{12:cos-imaginary}, then \( \alpha \)
is also imaginary, and hence we can write \( \alpha = ia \) with \( a \in \R \). Using the formula for \( R \)
derived at the end of last lecture, 
\[
	R = \left| \frac{\alpha - \beta}{\alpha + \beta} \right|^2 = \left| \frac{-\beta + ia}{\beta + ia}
	\right|^2 = \left| \frac{\alpha^2 + \beta^2}{\alpha^2 + \beta^2} \right| = 1
\]
Because \( R = 1 \), we conclude that there is no energy transmitted. So then if there is no energy transfer,
how is the evanescent wave allowed to exist? The answer turns out to be that \textit{on average}, there is no
net energy transfer in the \( z \) direction, which matches this result.

\subsubsection{Frustrated TIR}
The evanescent wave actually allows for a phenomenon called the \textit{frustrated} total internal
reflection. This occurs when you have two prisms separated by a very small gap, and you shine a ray through:
\begin{center}
	\begin{tikzpicture}[decoration = {markings, mark=at position 0.5 with {\arrow{>}}}]
		\draw [fill=cyan!40!white] (0, 0) -- (3, 0) -- (0, 3) -- cycle;
		\draw [fill=cyan!40!white] (3.1, 0) -- (0.1, 3) -- (3, 3) -- cycle;
		\draw[red!80!white, postaction = decorate] (2.4, -0.2) -- (1.5, 1.5);
		\draw[red!80!white, postaction = decorate] (1.5, 1.5) -- (-0.2, 2.4);
	\end{tikzpicture}
\end{center}
If the gap between the two prisms is small enough (i.e. smaller than \( \kappa ^{-1} \)), then the evanescent
wave doesn't fully die off, and we will thus get a nonzero propagating wave in the second prism. This is the
same phenomenon as tunneling in QM, except this is purely classical! 

\subsection{Wave Propagating through a Conductor}
Throughout our discussion of waves, we've considered media where there are no free charges and currents, so
\( \rho_f \) and \( \mathbf{J}_f = 0 \). This is not true in conductors, where we cannot control the current
that is generated by electromagnetic fields. To begin this discussion, again recall Maxwell's equations:
\begin{align*}
	\nabla \cdot \mathbf{D} &= \rho_f\\
	\nabla \cdot \mathbf{B} &= 0\\
	\nabla \times \mathbf{E} &= -\partial_T \mathbf{B} \\ 
	\nabla \times \mathbf{H} &= \mathbf{J}_f + \partial_t \mathbf{D} 
\end{align*}
If we then allow for the assumption that we are still in a linear medium, then this implies the equations:
\begin{align*}
	\nabla \cdot \mathbf{E} &= \frac{\rho_f}{\epsilon} \\ 
	\nabla \cdot \mathbf{B} &=  0 \\ 
	\nabla \times \mathbf{E} &= -\partial_t \mathbf{B}\\
	\nabla \times \mathbf{B} &= \mu \mathbf{J}_f + \mu \epsilon \partial_t \mathbf{E}
\end{align*}
If we are in an ohmic material, then we can use Ohm's law, \( \mathbf{J} = \sigma \mathbf{E} \), so then the
continuity equation for \( \rho_f \) gives:
\[
	\pdv{\rho_f}{t} = - \nabla \cdot \mathbf{J}_f = -\nabla \cdot (\sigma \mathbf{E}) 
	= - \frac{\sigma}{\epsilon} \rho_f
\]
This differential admits real solutions, specifically of the form: 
\[
	\rho_f(\mathbf{r}, t) = \rho(\mathbf{r}, 0) e^{- \sigma / \epsilon t}
\]
So this means that the free charges will eventually die out given long enough time. For our purposes, we will
work under this assumption. In particular, we will assume that the period of the EM waves in our systems are
much longer than the characteristic, essentially giving the free charges enough time to die out. This
translates to the conditions \( T \gg \frac{\epsilon}{\sigma} \) or \( \omega \ll \frac{\sigma}{\epsilon} \). 

 





