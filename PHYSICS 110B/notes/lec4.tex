\section{January 29}
Last time, we partially derived the continuity equation, leaving the index jumble on the right side unsolved.
We will continue now by simplifying the right hand side. To begin, let's write out what we had in the [stuff]
term from last time:
\[
	[\text{stuff}] = \epsilon_0 \partial_m (E^{m}E^{i}) - {\color{cyan!80!black}{\epsilon_0 E^{m}(\partial_m
	E^{i})}} -
	{\color{green!70!black}{\frac{1}{\mu_0}\partial^{i}(B^{k}B_k) + \frac{1}{\mu_0}\partial^{k}(B^{i}B_k)}} 
	+ {\color{red!95!black}{\frac{1}{\mu_0} B^{k}(\partial^{i} B_k) - \epsilon_0 E_n(\partial^{i} E^{n})}} 
	+ {\color{cyan!80!black}{\epsilon_0 E_m (\partial^{m}E^{i})}}
\]
I've color-coded the terms here for clarity. First, we note the following: when two indices are being
contracted, as is the case with \( \epsilon_0 E_m(\partial^{m} E^{i}) \), switching the upper and lower
indices on the \( m \) in the \( E \) and \( \partial \) terms doesn't change the value of the summation, so
this is equivalent to \( \epsilon_0 E^{m}(\partial^{m}E^{i}) \)\footnote{In reality, the two quantities are
	related by a so-called "metric", \( \delta_{ij} \), that allows transfer between upper and lower indices.
	In Cartesian coordinates, the metric allows for the cancellation of the two blue terms, but in relativity
with the Minkowski metric for example, this is not the case.} . In other words, because the \( m \) is being
contracted here, the two terms colored blue are equal and cancel each other. 

Next, let's again go back to our goal with this equation. Ultimately, we want to get rid of the non-total
derivative terms in this expression, since we want some continuity equation. The terms in green are already
total derivative terms, so we are happy with those. The problematic terms are the ones in red, so we will
deal with those. 

Starting with \( \epsilon_0 E_n (\partial^{i}E^{n}) \), we use product rule:
\[
	\partial^{i}(E_n E^{n}) = (\partial^{i}E_n)E^{n} + E_n(\partial^{i} E^{n}) = (\partial^{i} E_n)E^{n} +
	(\partial^{i}E_n) E^{n} = 2(\partial^{i}E_n)E^{n}
\]
Note that we can do this because when an index is summed over, we can switch the upper and lower indices
freely. So now, overall the expression becomes:
\begin{align*}
	[\text{stuff}] &= \epsilon_0 \partial_m(E^{m}E^{i}) - \frac{1}{\mu_0}\partial^{i}(B_k B^{k}) +
	\frac{1}{\mu_0}\partial^{i}(B^{i}B_k) + \frac{1}{2\mu_0}\partial^{i}(B^{k}B_k) -
	\frac{\epsilon_0}{2}\partial^{i}(E^{n}E_n)\\
	&= \epsilon_0 \left[ \partial_m (E^{m}E^{i}) - \frac{1}{2}\partial^{i}(E^{n}E_n) \right] + \frac{1}{\mu_0}
	\left[ \partial_m (B^{m}B^{i}) - \frac{1}{2}\partial^{i}(B^{k}B_k) \right]
\end{align*}
We will now perform one final trick: notice we have some \( \partial_m \) and \( \partial_i \) terms, but
ideally we want these two terms to be the same. So, in order to enforce this, we change the \( \partial_i \)
terms to \( \partial_m \), but insert a \( \delta^{im} \) term to compensate. Doing so, our equation becomes:
\[
	\epsilon_0 \left[ \partial_m (E^{m}E^{i}) - \frac{1}{2}\partial_m(\delta^{mi} E^{n}E_n) \right] +
	\frac{1}{\mu_0}\left[ \partial_m(B^{m}B^{i}) - \frac{1}{2}\partial_m(\delta^{mi} B^{k}B_k) \right]
\]
Now all the terms are a total derivative, so we are happy. Finally, we bring back the integral in front:
\[
	\int_{\mathcal{V}}\diff \tau \left[ \underbrace{\epsilon_0 \left(E^{m}E^{i} -
		\frac{1}{2}\delta^{mi}E^{n}E_n\right) + \frac{1}{\mu_0}\left(B^{m}B^{i} -
	\frac{1}{2}\delta^{mi}B^{n}B_n\right)}_{\sigma^{mi}} \right] = \int_{\mathcal{V}} \partial_m \sigma^{mi} 
	\diff \tau 
\]
We will now call the term in the square brackets \( \sigma^{mi} \), which is also known as the
\textbf{Maxwell Stress Tensor}. Using divergence theorem, we can transform this into a surface integral:
\[
	\int_{\mathcal{V}}\partial_m \sigma^{mi} \diff \tau = 
	\oint_{\partial \mathcal{V}} n_m \sigma^{mi} \diff a 
\]
\( n_m \) represents the normal vector coming out of the surface of \( \mathcal{V} \). Finally, we can
write the full equation:
\begin{equation}
	\dv{\mathbf{p}_\text{particle}^{i}}{t} + \dv{t} \int_\mathcal{V} \frac{S^{i}}{c^2} \diff \tau =
	\oint_{\partial \mathcal{V}} n_m \sigma^{mi} \diff a 
	\label{cont-momentum}
\end{equation}
How do we interpret this equation? The first term is the change in momentum of the particles, the second term
represents the momentum in the EM field, and the term on the right side can be thought of as a "generalized
force" acting on the boundary of \( \mathcal{V} \). 

\subsection{Stress Tensors}
Now would be a good time to talk a bit about how the Maxwell stress tensor behaves. Like the standard stress
tensor for materials, \( \sigma^{mi} \) represents the force in the \( i \)-th direction, on a surface whose
normal vector points in the \( m \)-th direction. That is, \( \sigma^{ii} \) represents pressure terms, while
\( \sigma^{ij} \) represents shear terms. That's all for now, we will talk more about this equation and
conservation of momentum next lecture.    

