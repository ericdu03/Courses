\section{February 10}
Today, we will talk about the transmission and reflection of waves in a linear medium. We will find that all
the properties we know about reflection and transmission: Snell's law, the law of reflection, they all
follow directly from enforcing the boundary conditions given by Maxwell's equations. 

Recall Maxwell's equations in a linear medium:
\begin{align*}
	\div \mathbf{D} &=  0 \\ 
	\div \mathbf{B} &= 0 \\ 
	\curl \mathbf{E} &= -\partial \mathbf{B} \\ 
	\curl \mathbf{H} &= \partial \mathbf{D} 
\end{align*}

These equations then imply the following boundary conditions:
\begin{align}
	\epsilon_1 E_1^{\perp} &= \epsilon_2 E_2^{\perp}\label{a} \\ 
	E_1^{\parallel} &= E_2^{\parallel}\label{b} \\ 
	B_1^{\perp} &= B_2^{\perp} \label{c} \\ 
	\mu_1 B_1^{\parallel} &= \mu_2 B_2^{\parallel} \label{d}
\end{align}

This is a result of using the relations \( \mathbf{D} = \epsilon \mathbf{E} \) and \( \mathbf{B} = \mu \mathbf{H} \)
in linear media. Now, with these boundary conditions set, we are ready to consider the transmission and
reflection of waves in a linear medium.  

\subsection{Normal Incidence}
First, we will consider the case where the incident wave is perpendicular to the interface. This makes the
situation easier to analyze. Let the system be described as follows: we have an incident wave from the left,
and a plane of material at \( z = 0 \). Then, the \( \mathbf{E} \) field over all space is given by:
\[
	\mathbf{E} = \begin{cases}
		\mathbf{E}_I e^{i (k_1 z - \omega t )} + \mathbf{E}_R e^{i(-k_2 z - \omega t)} & z < 0\\
		\mathbf{E}_T e^{i (k_2 z - \omega t)} & z > 0
	\end{cases}
\]
Technically, we need to take the real part of these equations, but we're going to omit that detail for
now.\footnote{What this really means is that each bold-faced vector \( \mathbf{E} \) in this equation is a
\textit{complex-valued} vector, of which you have to take the real part to get the amplitude.}  
Before we go further, it is important to note that \( \omega \) is the same for all three terms, which we can
argue to be the case in two different ways. Firstly, if you think of the incident wave as an electric field,
then it makes sense that the response from the dipoles should also follow the same frequency. Mathematically,
we also know that \( \omega \) must be the same because when we end up matching the boundary conditions, we
end up getting an equation of the form:
\[
	\mathcal{A}e^{- i \omega_I t} + \mathcal{B} e^{-i \omega_R t} = \mathcal{C}e^{- i \omega_T t}
\]
Here, \( \mathcal{A}, \mathcal{B} \) and \( \mathcal{C} \) are arbitrary constants. 
If we want this equation to hold up for all time \( t \), then the only way is if \( \omega_I = \omega_R =
\omega_T \). Now, we begin to impose the boundary conditions. The waves here are only in the \( \hat{z} \)
direction, since \( \mathbf{E} \) and \( \mathbf{B} \) are transverse waves, so \( \mathbf{E}_I, \mathbf{E}_R
\) and \( \mathbf{E}_T  \) are all in the \( xy \) plane. This immediately means that boundary conditions
\ref{a} and \ref{c} are trivially satisfied, and hence we only care about \ref{b} and \ref{d}. Starting with
\ref{b}:
\[
	\mathbf{E}_I e^{-i \omega t} + \mathbf{E}_R e^{-i \omega t} = \mathbf{E}_T e^{- i \omega t}
\]
so this gives \( \mathbf{E}_I = \mathbf{E}_R = \mathbf{E}_T \). Similarly, we can write down equations for the
magnetic field:
\[
	\mathbf{B} = \begin{cases}
		\mathbf{B}_I e^{i (k_1z - \omega t)} + \mathbf{B}_R e^{i(k_1 z - \omega t)} & z < 0 \\
		\mathbf{B}_T = e^{i(k_2 z - \omega t)} & z > 0
	\end{cases}
\]
Boundary condition \ref{d} then says:
\[
	\frac{1}{\mu_1} (\mathbf{B}_I + \mathbf{B}_R) = \frac{1}{\mu_2} \mathbf{B}_T
\]
We know that the \( \mathbf{B} \) field travels in the \( z \) direction, so using equation \ref{B-wave}, we
can write:
\[
	\mathbf{\hat{z}} \times \mathbf{E}_1 - \mathbf{\hat{z}} \times \mathbf{E}_R = \beta
	\mathbf{\hat{z}} \times \mathbf{E}_T
\]
We absorb all the constants into \( \beta = \frac{\mu_1v_1}{\mu_2v_2} \). Without loss of generality, we can
also suppose \( \mathbf{E} \) aligns in the \( \mathbf{\hat{x}} \) direction, so then we can write:
\begin{align*}
	\mathbf{E}_R &= E_R \mathbf{\hat{n}}_R = E_R (\cos \theta_R \mathbf{\hat{x}} + \sin \theta_r
	\mathbf{\hat{y}}) \\
	\mathbf{E}_T &= E_T \mathbf{\hat{n}}_T = E_T\left( \cos \theta_T \mathbf{\hat{x}} + \sin \theta_T
	\hat{\mathbf{y}} \right)
\end{align*}

Then, the \( y \)-component of boundary condition \ref{b} gives \( E_R \sin \theta_R = E_T \sin \theta_T \),
whereas the \( x \)-component of boundary condition \ref{d} gives \( E_R \sin \theta_R = -\beta E_T \sin
\theta_T \). These two equations must simultaneously be true, and since \( \beta >0 \), this forces \(
\theta_R = \theta_T = 0 \). So, we find that the reflected and transmitted waves have the same polarization
as the incident wave! 

Then, comparing the \( x \)-component of \ref{b} we get \( E_I + E_R = E_T \), while the \( y \)-component of
\ref{d} gives \( E_I - E_R = \beta E_T \). Solving both equations gives:
\[
	E_T = \frac{2}{1 + \beta}E_I , \quad E_R = \frac{1-\beta}{2}E_T = \left( \frac{1 - \beta}{1 + \beta}
	\right)E_I
\]
And this solves the boundary conditions for this specific situation! To conclude this lecture, notice that
for most materials \( \mu_1 \approx \mu_2 \approx \mu_0 \), so in these cases \( \beta \approx
\frac{v_1}{v_2} \). Further, if we have a medium where \( v_1 > v_2 \) (for example, from air to water), then
\( \beta > 1 \), so the reflected wave is written as:
\[
	\mathbf{E}_R = - \left| \frac{1 - \beta}{1 + \beta} \right|\quad \mathbf{E}_I = \left| \frac{1 - \beta}{1 +
	\beta} \right|e^{ i \pi} \left(E_1 e^{i \delta_I}\right) 
	= \left| \frac{1- \beta}{1 + \beta} \right|E_I e^{i(\delta_I
	+ \pi)}
\]
 This result actually is the motivation for why we say that the reflected wave has a \( \pi \) phase shift
 relative to the incident wave, and all we needed to do was solve some boundary conditions to get it!  
 




