\section{April 14}
\subsection{Radiation of a Moving Point Charge}
Recall that at the end of last lecture, we derived the Larmor formula (\cref{Larmor}), which relates the
radiation of a point charge to its acceleration. While the previous derivation does indeed work, it's not
exactly very satisfying (at least in my view), since you still make the approximation that a point charge is
a charge enclosed in a very small volume. We will correct this by deriving the Larmor formula through a
completely different means, which doesn't involve such an assumption.  

To begin this analysis, we start off with a diagram:
\begin{center}
	\begin{tikzpicture}[>=Stealth, scale=3.5]

	  \draw[dotted] (0, 0) circle (0.5);

	  % Red dot representing \vec{r}(t)
	  \filldraw[red] (0,0) circle (0.02);
	  \node[below left=-0.1, red] at (0.07,-0.07) {$\vec{w}(t)$};

	  % Blue vector from center to boundary
	  \coordinate (C) at (0,0);
	  \coordinate (T) at (30:0.5);
	  \draw[->, thick, blue] (C) -- (T) node[above right] {\( t_s \)};

	  \draw[dashed] (160:1) .. controls (-0.5, 0.3) and (-0.2, 0.1) .. (0, 0) 
							.. controls (0.2, -0.1) and (0.5, -0.3) .. (345:1);

	\end{tikzpicture}
\end{center}
We regard the particle as having position given by \( \mathbf{w}(t) \), and its signal emitted at time \( t
\) reaches our sphere at time \( t_s \). In this context, \( t \) is considered the retarded time. Now,
recall that for a moving particle, we derived earlier that the electric field follows:
\[
	\mathbf{E}(r, t_s) = \frac{q}{4\pi \epsilon_0} \frac{\rcurs}{(\brcurs \cdot \mathbf{u})^2} \left[ (c^2 -
	v^2) \mathbf{u} + \brcurs \times (\mathbf{u} \times \mathbf{a}) \right]
\]
Recall that the first term represents the velocity field, and the second represents the acceleration field.
The \( \mathbf{B} \) field follows \( \mathbf{B} = \frac{1}{c}\mathbf{\hat{r}} \times \mathbf{E} \) as usual.
Our goal is to derive the Larmor formula, so we want to find an expression for the power \( P(t) \) coming
out of the particle, which is the same power that arrives to the sphere at time \( t_s \). As usual, the
power is a surface integral of the Poynting vector \( \mathbf{S} \), so:
\[
	\mathbf{S} = \frac{1}{\mu_0}(\mathbf{E} \times \mathbf{B}) = \frac{1}{\mu_0c}(\mathbf{E} \times
	\hat{\brcurs} \times \mathbf{E}) = \frac{1}{\mu_0c}\left[ |\mathbf{E}|^2 \hat{\brcurs} -
	\mathbf{E}(\mathbf{E} \cdot \hat{\brcurs}) \right]
\]
Because we are only considering the radiation field, we can drop the velocity field since it doesn't go as \(
\frac{1}{\rcurs}\). Given this assumption, this means that \( \mathbf{E} \) is approximately proportional to
\( \brcurs \times (\mathbf{u} \times \mathbf{a}) \), and hence the second term which has an \( \mathbf{E}
\cdot \brcurs \) term will die. Therefore, the final equation for \( \mathbf{S} \) is:
\[
	\mathbf{S} = \frac{1}{\mu_0c}|\mathbf{E}|^2 \hat{\brcurs}
\]
Now we look to simplify the \( \mathbf{E} \) term from this equation. To do this, we will first assume tat \(
v = 0\) but \( a \neq 0 \); this may look like a strange assumption at first, but bear with me as we derive
this and eventually generalize to the \( v \neq 0 \) case. Under these assumptions, then \( \mathbf{u} = c
\hat{\brcurs} - \mathbf{v} = c \hat{\brcurs}\):
\begin{align*}
	\mathbf{E} &= \frac{q^2}{16 \pi^2 \epsilon_0^2} \frac{\rcurs^2}{ (\rcurs^3 c^3)^2} \left| c^3
	\hat{\brcurs} + c \brcurs \times (\hat{\brcurs} \times \mathbf{a}) \right|^2\\
			   &= \frac{q^2}{16 \pi^2 \epsilon_0^2} \frac{1}{\rcurs^{4} c^{6}}\left| c^3 \hat{\brcurs} + c
			   \hat{\brcurs} (\mathbf{a} \cdot \brcurs) - \mathbf{a} \brcurs \right|^2 
\end{align*}
From here, we only take terms that are proportional to \( \rcurs^2 \) -- this is so that we get only the \(
\sim \frac{1}{\rcurs^2} \) terms for \( \mathbf{S} \), and from there when integrating we get out only the \(
\frac{1}{\rcurs}\) terms. Now, expanding the square:
\begin{align*}
	|\mathbf{E}|^2 &= \frac{q^2}{16 \pi^2 \epsilon_0^2} \frac{1}{r^{4}c^{6}} \left( c^2 a^2 \rcurs^2 +
	c^2(\mathbf{a} \cdot \brcurs)^2 - 2c^2 (\mathbf{a} \cdot \brcurs)^2 \right)\\
	&= \frac{\mu_0^2 q^2}{16 \pi^2} \frac{1}{\rcurs^{4}} \left( a^2 \rcurs^2 - (\mathbf{a} \cdot \brcurs)^2 \right) 
\end{align*}
Now, \( \mathbf{a} \cdot \rcurs \) is the same as \( a \rcurs \cos \theta \) by definition of the dot
product, so this now simplifies to:
\[
	|\mathbf{E}|^2 = \frac{\mu_0^2 q^2}{16 \pi^2} \frac{a^2}{\rcurs^2} \sin^2 \theta
\]
So now we have our final expression for \( \mathbf{S} \):
\[
	\mathbf{S} = \frac{\mu_0 q^2 a^2}{16 \pi^2 c}\left( \frac{\sin^2 \theta}{\rcurs^2} \right) \hat{\brcurs}
\]
Finally, the power:
\[
	P = \oint \mathbf{S} \diff \mathbf{a} = \frac{\mu_0 q^2 a^2}{16 \pi^2 c} \int_{0}^{\pi} \sin^2 \theta
	\sin \theta \diff \theta \int_{0}^{2\pi} \diff \phi = \frac{\mu_0 q^2 a^2}{6 \pi c}
\]
which is exactly the Larmor formula, this time derived without the assumption of volume. It should also be
clear after this derivation that the Larmor formula only works when \( v = 0 \), as that was one of the
assumptions we made to simplify \( \mathbf{E} \). Generally, this is a pretty good approximation anyways for
\( v \ll c \), but to be completely formal, when \( v \neq 0 \) we have to consider not only the fact that \(
\mathbf{E}\) changes but \( \mathbf{S} \) changes too. To see this, consider a moving particle, and think
about the number of wavefronts emitted over a given time interval, versus the number of wavefronts received
at a distance away:
\begin{center}
	\begin{tikzpicture}
		\filldraw[red] (0,0) circle (0.1) node[below=0.2] {particle};
		\draw[dashed] (60:1) arc [start angle = 60, end angle = -60, radius = 1];
		\draw[dashed] (60:1.5) arc [start angle = 60, end angle = -60, radius = 1.5];
		\draw[dashed] (60:1.9) arc [start angle = 60, end angle = -60, radius = 1.9];
		\draw (4, 1) -- (4, -1) node[below] {screen};
	\end{tikzpicture}
\end{center}
Because of this velocity, you can check that the effective wavelength \( \lambda_\text{eff} = (c - v) T \),
where \( T \) is the period between the wavefronts. 
The number of wavefronts received \( N_\text{receive} \) is given by:
\[
	N_\text{receive} = \frac{(c \Delta t) / \lambda_\text{eff}}{\Delta t} = \frac{c}{(c - v)T} = \left(
	\frac{c}{c - v} \right)N_\text{emit}
\]
Now in the general case only the velocity in the direction of emission matters, so we replace \( v \) with \(
\mathbf{v} \cdot \brcurs \), giving us the formula:
\[
	N_\text{emit} = \left( 1 - \frac{\mathbf{v} \cdot \brcurs}{c} \right)N_\text{receive}
\]
We will continue this discussion next time and see how this subtlety affects the power emitted.  
 
 
 
  
