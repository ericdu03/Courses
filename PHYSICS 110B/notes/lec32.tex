\section{April 18}
\subsection{Self-Force/Radiation Reaction}
Recall the Larmor formula for nonrealtivistic particles:
\[
	P_\text{rad} = \frac{\mu_0 q_0^2 a^2}{6 \pi c}
\]
From this formula, it may be natural to assume that the power, which is also written as \( \mathbf{F} \cdot
\mathbf{v} = - P_\text{rad} \), but this is \textit{not} true! The reason for this is because the Larmor
formula only considers radiation that extends to infinity, so the power contributed by the velocity field is
missing. 

However, because energy continuously flows in and out of the velocity field, it is possible for us to still
use the above formula, just only when the net velocity of the particle is zero. For periodic motion, this
would mean that we can consider the motion over a time interval \( \tau \) where the particle returns to the
same place, at which point the contribution by the velocity field should be zero. Thus:
\begin{align*}
	\int_{0}^{T} \mathbf{F} \cdot \mathbf{v} \diff t &= - \int_{0}^{T} \frac{\mu_0 q^2 a^2}{6 \pi c} \diff
	t\\
	 &= - \int_{0}^{T} \frac{\mu_0 q^2}{6 \pi c}
	 \dv{\mathbf{v}}{t} \dv{\mathbf{v}}{t} \diff t \\ 
	 &= \int_{0}^{T} \frac{\mu_0 q^2}{6 \pi c} \dv[2]{\mathbf{v}}{t} \cdot \mathbf{v} \diff t - 
	 \int_{0}^{T} \dv{t} \left( \frac{\mu_0 q^2}{6 \pi c} \dv{\mathbf{v}}{t} \mathbf{v} \right) \diff t	
\end{align*}
The second term equals zero if the particle returns to the same spot. This leaves us with the equality:
\[
	\int_{0}^{T} \mathbf{F} \cdot \mathbf{v} \diff t = \int_{0}^{T} \frac{\mu_0 q^2}{6 \pi c}
	\dv[2]{\mathbf{v}}{t} \mathbf{v} \diff t
\]
comparing the integrands, we get the result:
\[
	\mathbf{F}_\text{rad} = \frac{\mu_0 q^2}{6 \pi c} \dv[2]{v}{t}
\]
This is known as the \textbf{Abraham-Lorentz} formula. However, this is not a very rigorous way to obtain this
equation. Firstly, not all motion is periodic, so why does this formula work for nonperiodic motion too?
Secondly, this only tells you about the self-force for the component \( \mathbf{F}_\text{rad} \) that is
along the direction of \( \mathbf{v} \), since the integrand is really \( \mathbf{F} \cdot \mathbf{v} \) on
the left hand side. So we need an alternative method to derive this formula.    

Before we do though, there is something interesting about this formula that we can point out right away. When
forces have \( \dot{\mathbf{a}} \) dependence, invoking Newton's second law:
\[
	\frac{\mu_0^2 q^2}{6 \pi c} \dot{\mathbf{a}} = m \mathbf{a}
\]
this is a differential equation for the acceleration, with solution \( a(t) = a_0 e^{t / \tau} \), and \(
\tau \) is the prefactor on \( \dot{\mathbf{a}} \). The interesting thing is that \( a(t) \) is now increasing
with increasing time, meaning the acceleration gets faster and faster as you go on. Conversely, if you insist
that \( a = 0 \), then you will find that if you do try to apply an external force, the particle starts
responding to that force before you even act on it. Obviously both of these solutions are non-physical, but so
far there is no mathematical reason why we should reject them. 

\subsection{The Dumbbell Model}
Now, we move on to finding a better way to derive the Abraham-Lorentz formula. To do so, we use a so-called
\textit{dumbbell model}, in which we split a charge into a tiny dumbbell with half the charge each:  
\begin{center}
	\begin{tikzpicture}
		\coordinate (positive1) at (1, 2.5);
		\coordinate (negative1) at (1, 0.8);

		\coordinate[-stealth] (positive2) at (3, 2.5);
		\coordinate[-stealth] (negative2) at (3, 0.8);

		\draw (-1, 0) -- (4, 0) node[right] {\( x \)};
		\draw (0, -1) -- (0, 3) node[left] {\( y \)};
		\filldraw[red] (positive1) circle (0.03) node[left] {\( +q / 2 \)};
		\filldraw[blue] (negative1) circle (0.03) node[left] {\( -q / 2 \)};

		\draw (positive1) -- node[midway, left] {\( d \)} (negative1);

		\filldraw[red] (positive2) circle (0.03) node[right] {\#1};
		\filldraw[blue] (negative2) circle (0.03) node[right] {\#2};

		\draw[dashed] (positive1) -- (positive2);
		\draw[dashed] (negative1) -- node[midway, below] {\( \ell \)} (negative2);
	\end{tikzpicture}
\end{center}
Recall that the equation for the field of moving charges is given by:
\[
	\mathbf{E}(\mathbf{r}, t) = \frac{q}{4\pi \epsilon_0} \frac{\rcurs}{\brcurs \cdot \mathbf{u}} \left[ (c^2
	- v^2) \mathbf{u} + \brcurs \times (\mathbf{u} \times \mathbf{a}) \right]
\]
For simplicity of the model, we will assume that \( v(t_r) = 0  \) so that \( \mathbf{u} = c \hat{\brcurs}
\), and \( \mathbf{a} = a \mathbf{\hat{x}} \). 
Now, let's consider the force on charge \#1 due to charge \#2. In this case, we have:
\[
	\brcurs = \ell \mathbf{\hat{x}} + d \mathbf{\hat{y}}, \quad \hat{\brcurs} = \frac{\ell \mathbf{\hat{x}} +
	d \mathbf{\hat{y}}}{\sqrt{\ell^2 + d^2}}
\]
This makes the triple cross product:
\[
	\brcurs \times (\mathbf{u} \times \mathbf{a}) = (\brcurs \cdot \mathbf{a}) \mathbf{u} - (\brcurs \cdot
	\mathbf{u}) \mathbf{a}
\]
Notice that we also only need to consider the \( x \)-component, since the \( y \)-components will cancel
each other when you add the forces on both charges (due to symmetry). Therefore, the force on 1 may be
written as:
\[
	\mathbf{E}_{1x} = \frac{q / 2}{4\pi \epsilon_0} \frac{\sqrt{\ell^2 + d^2}}{c \sqrt{\ell^2 + d^2}} \left[ c^3
	\frac{\ell}{\sqrt{\ell^2 + d^2}} + \ell a \frac{c\ell}{\sqrt{\ell^2 + d^2}} - ca\sqrt{\ell^2 + d^2} \right] =
	\frac{q}{8\pi \epsilon_0} \frac{1}{c^2(\ell^2 + d^2)^{3 /2}} \left( lc^2 - d^2 \right)\mathbf{\hat{x}}
\]
By symmetry \( E_{2x} \) is the exact same. Therefore, the self force may be written as:
\[
	\mathbf{F}_\text{self} = \left( \frac{q}{2} \right) \mathbf{E}_{1x} + \left( \frac{q}{2} \right)
	\mathbf{E}_{2x} = \frac{q^2}{8 \pi \epsilon_0 c^2} \frac{\ell c^2 - d^2}{(\ell^2 + d^2) ^{3 /
	2}}\mathbf{\hat{x}}
\]
so far this treatment is exact, but we ultimately want to model the situation in the \( d \to 0 \) limit
so we need to Taylor expand the above equation in orders of \( d \). We can't immediately Taylor expand this
just yet, since \( \ell \) also has \( d \) dependence that we need to expand. To begin, we note that \( \ell
= x(t) - x(t_r) \), so we first Taylor expand \( x(t) \) around \( t = t_r \):
\[
	x(t) = x(t_r) + \dot x(t_r) (t - t_r) + \frac{\ddot x(t_r)}{2}(t - t_r)^2
\]
Noting that \( t - t_r = T \) and \( \dot x(t_r) = 0 \) so the second term dies, we can write \( \ell \) as:
\begin{equation}
	\label{ell}
	\ell = \frac{1}{2} \ddot x(t_r) T^2 + \frac{1}{6} \dddot x(t_r) T^3 + \dots 
\end{equation}
simultaneously, we have a relation for \( \ell \) based on pure geometry:
\[
	d = \sqrt{(cT)^2 - \ell^2} = \sqrt{1 - \frac{1}{c^2 T^2} \left( \frac{1}{2} aT^2 + \frac{1}{6} \dot a T^3 +
	\dots \right)} \approx cT - \frac{a^2 T^3}{8c}
\]
reversing this equation and solving for \( T \):
\[
	T = \frac{d}{c} + \frac{a^2}{8c^2}\left( \frac{d}{c} \right)^3 + \dots
\]
Putting this back into \cref{ell} we finally have \( \ell \) in terms of \( d \):
\[
	\ell = \frac{1}{2}a \left( \frac{d}{c} \right)^2 + \frac{1}{6}\dot a \left( \frac{d}{c} \right)^3 +
	O(d^{4}) 
\]
Thus, we now have:
\[
	\mathbf{F}_\text{self} = \frac{q^2}{8 \pi \epsilon_0 c^2} \frac{\ell c^2 - ad^2}{(\ell^2 + d^2)^{3 / 2}}
	\simeq \frac{q^2}{8 \pi \epsilon_0 c^2} \left[ -\frac{a}{2d} + \frac{\dot a}{6c} + O(d^2) \right]
\]
Notice that we have a term proportional to \( a \). In some sense, this term can be regarded as an effective
mass term. To see what this means, consider Newton's second law with \( \mathbf{F}_\text{self} \):
\[
	\left( -\frac{q^2}{16 \pi \epsilon_0d}\frac{1}{c^2} \right) a + \frac{q^2 \dot a}{4 \pi \epsilon_0}
	\frac{1}{3c^2} = m_0a
\]
but we can rearrange this equation into:
\[
	\frac{q^2 \dot a}{4 \pi \epsilon_0} \frac{1}{3c^2} = \left( m_0 + \frac{q^2}{16 \pi \epsilon_0 d}
	\frac{1}{c^2} \right)a
\]
Notice how the factor of \( \frac{q^2}{16 \pi \epsilon_0 d} \) acts as "extra mass" in addition to \( m_0 \),
which is why it is sometimes called the effective mass term. Looking at this term in close detail, you can
see why we can regard it this way. This term denotes the potential energy between the two charges, and as a
form of energy based on Einstein's equation \( E = mc^2 \) we know that energy and mass are closely
connected. So, in light of this it does make sense that you can regard this energy as a form of extra mass. 
