\section{February 12}
Last time, we derived the intensity of \( E_R \) and \( E_T \) in terms of \( E_I \), where we have:
\[
	E_R = \left( \frac{1 - \beta}{ 1 + \beta} \right)E_I \quad E_T = \frac{2}{1 + \beta}E_I
\]
Now, recall that since \( \mathbf{S} = \frac{1}{\mu}\mathbf{E} \times \mathbf{B} \), then in a linear medium,
because of equation \ref{B-wave}, we can rewrite this as \( \mathbf{S} = \frac{1}{v}\mathbf{E} \times
(\frac{1}{v} \mathbf{k}\times \mathbf{E}) \), so the time average of \( \mathbf{S} \) is written as:
\[
	\mean{\mathbf{S}} = \frac{E^2}{\mu v} \mean{\cos^2(kz - \omega t)} = \frac{E^2}{2 \mu v} \propto E^2
\]
So, the average reflection, which is denoted as:
\[
	R \equiv \frac{\mean{\mathbf{S}_R}}{\mean{\mathbf{S}_I}} = \frac{\frac{1}{2 \mu_1 v_1}
	|E_R|^2}{\frac{1}{2\mu_1v_1}|E_I|^2} = \left( \frac{1 - \beta}{1 + \beta} \right)^2
\]
Since \( \mu_1 \approx \mu_2 \) for most materials, then \( \beta \approx \frac{v_1}{v_2} \), so most of the
time, 
\[
	R = \left(\frac{v_2 - v_1}{v_2 + v_1}\right)
\]
Similarly, we have a transmission coefficient
\[
	T \equiv \frac{\frac{1}{2\mu_2v_2}|E_T|^2}{\frac{1}{2\mu_1v_1}|E_I|^2} \approx \frac{v_1}{v_2}\left(
	\frac{2}{1 + \beta} \right)^2 = \frac{4v_1v_2}{(v_1 + v_2)^2}
\]
And as a sanity check, we should get \( R + T = 1 \) because of energy conservation, which is exactly what
we have:
\[
	R + T = \frac{(v_2 - v_1)^2}{(v_2 + v_1)^2} + \frac{4v_2v_1}{(v_1 + v_2)} = 1
\]
This, above all else, gives us confidence that the math carried out properly.  

\subsection{Oblique Incidence}
Now, we will consider the more general case, where the incident wave is no longer perpendicular to the
boundary. Here, we will see that a simple application of boundary conditions leads to fundamental laws of
reflection and refraction. Consider the following case of oblique incidence:
\begin{center}
	\begin{tikzpicture}[decoration = {markings, mark=at position 0.5 with {\arrow{>}}}]
		\draw[dashed, -stealth] (-3, 0) -- (3, 0) node[right] {\( z \)};
		\draw[-stealth] (0, -2) -- (0, 2) node[above] {\( x \)};
		\draw[cyan, postaction=decorate] (-2, -1) -- node[midway, below right] {\( \mathbf{k}_I \)} (0, 0); 
		\draw[red, postaction=decorate] (0, 0) -- node[midway, above right] {\( \mathbf{k}_R \)} (-2, 1.5);
		\draw[orange, postaction=decorate] (0, 0) --node[midway, above left] {\( \mathbf{k}_T \)} (3, 0.5);
	\end{tikzpicture}
\end{center}
the laws of optics don't change when we have oblique incidence, so we will still impose the boundary
condition that the transition is continuous. Therefore, we should expect the equations on both sides of \( z
= 0\) to match. Just like the perpendicular incidence, when we match the boundary condition at \( z = 0 \) we
will get equations of the form:
\[
	\left( \phantom{aa} \right) e^{i \mathbf{k}_I \cdot \mathbf{r}}\eval_{z =0} + 
	\left( \phantom{aa} \right) e^{i \mathbf{k}_R \cdot
	\mathbf{r}}\eval_{z = 0} = \left( \phantom{aa} \right)e^{i \mathbf{k}_T \cdot \mathbf{r} }\eval_{z = 0}
\]
For this equation to be true for all \( t \), then we require that we have the same dependence in the \( xy
\) plane for all three terms. This implies the conditions:
\begin{equation}
	\label{11:k-condition}
	(\mathbf{k}_I)_x x + (\mathbf{k}_I)_y y = (\mathbf{k}_R)_x x + (\mathbf{k}_R)_y y = (\mathbf{k}_T)_x x +
	(\mathbf{k}_T)_y y
\end{equation}
Now, without loss of generality, we can choose \( \mathbf{k}_I \) to lie in the \( xz \) plane only, so \(
(\mathbf{k}_I)_y = 0 \). But this must mean that the \( y \) component for the other two terms is also zero!
Therefore, all the rays lie in the same plane, and therefore our original diagram is accurate.

Further, this also means that from \ref{11:k-condition} we get that \( (\mathbf{k}_I)_x = (\mathbf{k}_I)_y =
(\mathbf{k}_I)_z \), and hence:
\[
	\mathbf{k}_I \sin \theta_I = \mathbf{k}_R \sin \theta_R = \mathbf{k}_T \sin \theta_T
\]
But since \( |\mathbf{k}_I| = |\mathbf{k}_R| \) as they travel through the same medium, then the only
conclusion is that \( \theta_I = \theta_R \), which is the reflection rule. The other condition we then
becomes 
\[
	\frac{\omega_1}{v_1}\sin \theta_I = \frac{\omega}{v_2}\sin \theta_T \implies \frac{1}{v_1} \sin \theta_I
	= \frac{1}{v_2} \sin \theta_T
\]
Given that \( n = \frac{c}{v} \), then we can manipulate this further to get:
\[
	n_1 \sin \theta_I = n_2 \sin \theta_T
\]
which is Snell's law! 

\subsection{Polarized Light}
Now, we will consider the case where the light is polarized. First, we will consider light that is polarized
in the plane of incidence (the case perpendicular will be left as homework):

\begin{center}
	\begin{tikzpicture}[decoration = {markings, mark=at position 0.5 with {\arrow{>}}}]
		\draw[dashed, -stealth] (-4, 0) -- (4, 0) node[right] {\( z \)};
		\draw[-stealth] (0, -2) -- (0, 2) node[above] {\( x \)};
		\draw[cyan, postaction=decorate] (-2.5, -1.25) -- node[midway, below right] {\( \mathbf{k}_I \)} (0, 0); 
		\draw[red, postaction=decorate] (0, 0) -- node[midway, above right] {\( \mathbf{k}_R \)} (-2.5, 1.25);
		\draw[orange, postaction=decorate] (0, 0) --node[midway, above left] {\( \mathbf{k}_T \)} (3, 0.5);
		\draw (2.3, 0) arc [start angle = 0, end angle = 9, radius = 2.3] 
			node[midway, right] {\( \theta_\text{out} \)};
		\draw (-1, 0) arc [start angle = 180, end angle = 205, radius = 1] node[midway, left] {\(
			\theta_\text{in} \)};
		\draw (-1, 0) arc [start angle = 180, end angle = 155, radius = 1] node[midway, left] {\(
			\theta_\text{in} \)};
		\draw[-stealth]  (-2, -1) -- (-2.15, -0.75) node[above left] {\( \mathbf{E}_I \)};
		\draw[red!40!white, -stealth]  (-2, 1) -- (-1.9, 2) node[above] {\( \mathbf{E}_R \)};
		\draw[red!40!white, dashed] (-2, 1) -- (-1.5, 2);
		\draw[red!40!white] (-1.75, 1.5) arc [start angle = 70, end angle = 93, radius = 0.5];
		\draw[red!40!white] (-1.8, 1.7) node {\scalebox{.4}{\( \phi_R \)}};
		\draw[blue!40!white, -stealth] (2, 0.33) -- (2.2, 1) node[above] {\( E_T \)};
		\draw[blue!40!white, dashed] (2, 0.33) -- (1.85, 1.23);
		\draw[blue!40!white] (1.95, 0.65) arc [start angle = 100, end angle = 82, radius = 0.5];
		\draw[blue!40!white] (2.03, 0.85) node {\scalebox{.4}{\( \phi_T \)}};
	\end{tikzpicture}
\end{center}
To begin, we will consider the most general case, so we do not assume here that \( \mathbf{E}_T \) and \(
\mathbf{E}_R \) lie in the plane of incidence. This is why we have the \( \phi_T \) and \( \phi_R \) angles.
The boundary conditions don't change despite this though, and applying them here gives us:
\begin{align}
	\label{11:cond1}\epsilon_1(\mathbf{E}_I + \mathbf{E}_R)_z &= \epsilon_2 (\mathbf{E}_T)_z\\
	\label{11:cond2}(\mathbf{E}_I + \mathbf{E}_R)_{x, y} &= (\mathbf{E}_T)_{x, y} \\ 
	\label{11:cond3}(\mathbf{B}_I + \mathbf{B}_R)_z &= (\mathbf{B}_T)_z\\
	\label{11:cond4}\frac{1}{\mu_1}(\mathbf{B}_I + \mathbf{B}_r)_{x, y} &=  \frac{1}{\mu_2}(\mathbf{B}_T)_z 
\end{align}
From the vector decomposition, we can also get these relations (check these yourself on your own time): 
\begin{align*}
	\mathbf{E}_I &= - \mathbf{E}_I \sin \theta_\text{in} \mathbf{\hat{z}} + \mathbf{E}_I \cos
	\theta_\text{in}\mathbf{\hat{x}}\\
	\mathbf{E}_R &= \mathbf{E}_R \left[ \cos \phi_R \sin \theta_\text{in}\mathbf{\hat{z}} + \cos \phi_R \cos
	\theta_\text{in} \mathbf{\hat{y}} + \sin \phi_R \mathbf{\hat{y}}\right]\\
	\mathbf{E}_T &= \mathbf{E}_T \left[ -\cos \phi_T \sin \theta_\text{out} \mathbf{\hat{z}} + \cos \phi_T
	\cos \theta_\text{out} \mathbf{\hat{x}} + \sin \phi_R \mathbf{\hat{y}} \right] 
\end{align*}
Combining the vector decomposition and the boundary conditions, we get the following set of equations:
\begin{align}
	\text{(\ref{11:cond1}):} \quad & \epsilon_1 \sin \theta_\text{in}(E_I + E_R \cos \phi_R) = \epsilon_2 \sin
	\theta_\text{out}(-E_T \cos \phi_T)\\
	\label{11:cond5} \text{(\ref{11:cond2}):} \quad & \begin{cases}
		\cos \theta_\text{in}(E_I + E_R \cos \phi_R) = \cos \theta_\text{out} (E_T \cos \phi_T)\\
		E_R \sin \phi_R = E_T \sin \phi_T \\ 
	\end{cases}\\
	\text{(\ref{11:cond3}):} \quad & \frac{E_R}{v_1} \sin \phi_k \sin \theta_\text{in} = \frac{E_T}{v_2}\sin \theta_T
	\sin \theta_\text{out}\\
	\label{11:cond6} \text{(\ref{11:cond4}):} \quad & \begin{cases}
		\frac{E_R}{v_1}\cos \theta_\text{in} \sin \phi_R = - \frac{E_T}{\mu_2v_2} \cos \theta_\text{out} \sin
		\theta_T\\
		\frac{1}{\mu_1v_1}(E_I - E_R \cos \phi_R) = \frac{1}{\mu_2v_2} E_T \cos \phi_T
	\end{cases}
\end{align}

Combining the second part of \ref{11:cond5} with the first part of \ref{11:cond6}, we can get:
\[
	E_R \sin \phi_T \cos \theta_\text{in} = -\beta E_T \sin\phi_T \cos \theta_\text{out} \implies
	\sin \phi_T(\cos \theta_\text{in} + \beta \cos \theta_\text{out}) = 0
\]
The only way this equation is true for all \( \theta_\text{in} \) is if \( \phi_T = 0 \), which then implies
that \( \phi_R = 0 \) by the second half of \ref{11:cond5}. So, what we find is that indeed \( E_R \) and \(
E_T\) do lie in the plane of incidence.  

