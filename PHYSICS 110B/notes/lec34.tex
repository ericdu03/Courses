\section{April 23}
\subsection{Time Dilation/Proper time}
Last lecture, we covered the idea of time dilation, and derived the relation \( \Delta t = \gamma \Delta t'
\). In that example, we had Bob moving with the rocket ship, where the two events, the emission of the photon
and its absorption back occur at the same location in the \( S' \) frame. Because these two events occur at
the same location in the \( S' \) frame, the time difference measured in such a reference frame is called the
\textit{proper time}, and is denoted \( \Delta \tau \). This quantity will be important in our later
discussions. 

\subsection{Length Contraction}
Here, we touch on the topic of length contraction. Consider a situation below, where Bob is on a rocket ship
and Alice is standing by on a ledge as Bob goes past. Bob is holding a long rod, and at \( t = 0 \), the
front end of that rod passes by Alice. The length of the rod, as measured by Bob, is called the
\textit{proper length}, denoted by \( L_0 \). The proper length is the length measured by an observer that is
moving with no relative motion with the object in question (i.e. Bob). 

At some time \( \Delta t \) later, the back of the rod passes Alice, so for Alice she will measure the length
as \( L = v \Delta t \). Because Alice measures the rod at the same location, her length is denoted by the
proper time, \( L = v \Delta \tau \). For Bob, Alice moves over a distance \( L_0 \) in a time \( \Delta t'
\), with a speed \( v \). Therefore, Bob writes \( L_0 = v \Delta t' \). Because we know that \( \Delta t'
 = \gamma \Delta t\), then we get \( L_0 = v \gamma \Delta t = \gamma L \), and hence we get \( L =
 \frac{L_0}{\gamma} \), which is the standard formula for length contraction. Do note, however, that the
 lengths in the \textit{perpendicular} direction, are not contracted! You can see this by the fact that \(
 \gamma = 1 \) in that case.   

\subsection{Lorentz Boost/Transformation}
We now tackle the above phenomena from a theoretical perspective, by building up to Lorentz transformations.
Consider two frames \( \mathcal{S} \) and \( \mathcal{S}' \), where \( \mathcal{S}' \) travels with speed \(
v\). And, consider an event at time \( (t, x) \) in \( \mathcal{S} \). We now ask the question: what is the
relation between \( (t, x) \) and \( (t', x') \)?   

\begin{center}
	\begin{tikzpicture}
		\draw[-stealth] (0, 0) -- (3, 0);
		\draw[-stealth] (0, 0) -- (0, 3) node[above] {\( \mathcal{S} \)};
		\draw[-stealth, red] (1, 0) -- (4, 0);
		\draw[-stealth, red] (1, 0) -- (1, 3) node[above] {\( \mathcal{S}'\)} ;
		\draw[-stealth, red] (1.2, 2.5) -- (1.5, 2.5) node[right] { \( v \) };
		\filldraw[blue] (3.2, 0.5) circle (0.02) node[right] {\( (t, x) \)};
	\end{tikzpicture}
\end{center}
In Galilean relativity, the transformation would be:
\[
	t' = t \quad x' = x - vt
\]
because time is assumed to pass the same in different reference frames. However, as we shall see, this is not
the case in relativity. To see this more clearly, let's consider a specific situation: at \( t = 0 \), frame
\( \mathcal{S}' \) starts moving, and simultaneously a light pulse at \( (t, x) = (t', x') = (0, 0) \) is
sent out. At some time later, it travels a distance \( L_0 \) as measured in \( \mathcal{S}' \). 
 
\begin{center}
	\begin{tikzpicture}[decoration=snake]
		\draw[-stealth] (0, 0) -- (3, 0);
		\draw[-stealth] (0, 0) -- (0, 3) node[above left] {\( \mathcal{S} \)};
		\draw[-stealth, red] (0.1, 0) -- (3.1, 0);
		\draw[-stealth, red] (0.1, 0) -- (0.1, 3) node[above right] {\( \mathcal{S}'\)};
		\draw[-stealth, red] (0.3, 2.5) -- (0.7, 2.5) node[right] { \( v \) };
		\draw[red] (0.5, 0.5) rectangle node[midway, above=0.2] {\( L_0 \)} ++(3, 0.4);
		\draw[orange, -stealth, decorate] (0.5, 0.7) -- (1.2, 0.7);
		\node at (1.5, -0.5) {Event 1};
	\end{tikzpicture}
	\hspace{3cm}
	\begin{tikzpicture}[decoration=snake]
		\draw[-stealth] (0, 0) -- (3, 0);
		\draw[-stealth] (0, 0) -- (0, 3) node[above] {\( \mathcal{S} \)};
		\draw[-stealth, red] (1, 0) -- (4, 0);
		\draw[-stealth, red] (1, 0) -- (1, 3) node[above] {\( \mathcal{S}'\)} ;
		\draw[-stealth, red] (1.2, 2.5) -- (1.5, 2.5) node[right] { \( v \) };
		\draw[red] (1.4, 0.5) rectangle node[midway, above=0.2] {\( L_0 \)} ++(3, 0.4);
		\draw[orange, -stealth, decorate] (3.7, 0.7) -- (4.4, 0.7);
		\node at (1.5, -0.5) {Event 2};
	\end{tikzpicture}
\end{center}
In the \( \mathcal{S} \) frame, event 2 happens at a coordinate \( (\Delta t, \Delta x) \). Note that this is
both improper time and length, since the events are not occurring in the same location, and \( \mathcal{S} \)
is not moving with the length \( L_0 \). In the \( \mathcal{S}' \) frame though, event 2 happens at the
coordinates \( (\Delta t', L_0) \). Note the proper time here because \( \mathcal{S}' \) moves with the
length \( L_0 \). Given the formula for length contraction though, we can go ahead and calculate what \(
\Delta x \) should be:
\[
	\Delta x = v \Delta t + L = v \Delta t + \frac{L_0}{\gamma} = v \Delta t + \frac{\Delta x'}{\gamma}
	\implies \Delta x' = \gamma(\Delta x - v \Delta t)
\]
And we arrive at the familiar formula for a Lorentz boost in the \( x \)-direction.
For the time portion, we have \( \Delta t' = \frac{L_0}{c} \) for Bob, and for Alice we have:
\[
	\Delta t = \frac{\Delta x}{c} = \frac{1}{c}\left( v \Delta t + \frac{L_0}{\gamma} \right) =
	\frac{v}{c}\Delta t + \frac{\Delta t'}{\gamma} \implies \Delta t' = \left( 1 - \frac{v}{c} \right)
	\Delta t \cdot \gamma
\]
Hence, we have:
\[
	\Delta t = \frac{\Delta x'}{c} = \frac{\gamma}{c}\left( \Delta x - v \Delta t \right) = \gamma\left(
	\frac{\Delta x}{c} - \frac{v}{c}\Delta t \right) = \gamma\left( \Delta t - \frac{v}{c}\frac{\Delta x}{c} \right) 
	= \gamma\left(\Delta t - \frac{v}{c^2}\Delta x\right)
\]
Here, we arrive at the formula for the Lorentz boost for time. Both of these formulas can be summarized into
one transformation, called the \textit{Lorentz Boost} or Lorentz Transformation:
\[
	\begin{cases}
		\Delta t' = \gamma\left( \Delta t - \frac{v}{c^2}\Delta x \right)\\
		\Delta x' = \gamma(\Delta x - v \Delta t)\\
		\Delta y' = \Delta y \\
		\Delta z' = \Delta z
	\end{cases},
	\quad \gamma = \frac{1}{\sqrt{1 - v^2 / c^2}}
\]
Note that this is only in the \( x \)-direction, but the other directions are the exact same derivation. One
interesting thing about this transformation is that the following holds true:   
\[
	-(c \Delta t')^2 + (\Delta x')^2 = -(c \Delta t)^2 + (\Delta x)^2
\]
This equality really tells us something about what the Lorentz transformation does to our coordinate system.
Just like how under rotation, the Euclidean distance: \( (\Delta x')^2 + (\Delta y')^2 + (\Delta z')^2 \)
remains invariant, the above quantity is invariant under Lorentz transformation. In this sense, we can think
of \( (c \Delta t)^2 + (\Delta x)^2 \) as a notion of "length", which remains invariant under the Lorentz
transformation.
 
There also happens to be another way to write this equality, which is to use matrix notation:
\[
	\begin{pmatrix} c \Delta t & \Delta x & \Delta y & \Delta z \end{pmatrix} 
	\begin{pmatrix} -1 & & &\\ & 1 & & \\ & & 1 & \\ & & & 1\end{pmatrix}
	\begin{pmatrix} c \Delta t\\ \Delta x \\ \Delta y\\ \Delta z \end{pmatrix} = \text{const.}
\]
which can also be written more compactly as \( \eta_{\mu \nu}(\Delta x)^{\mu} (\Delta x)^{\nu} =
\text{const.} \). Our requirement that this remains invariant can then be written as:
\[
	\eta_{\mu \nu}(\Delta x)^{\mu}(\Delta x)^{\nu} = \eta_{\rho \sigma}(\Delta x)^{\rho}(\Delta x)^{\sigma}
\]
Notice that \( \Delta x \) is the same on both sides, but the \textit{metric} \( \eta_{\rho \sigma} \) does
not use the same indices. So how does this equation reflect invariance? As we'll see in the next lecture,
we've encoded the transformation in changing \( \eta_{\mu \nu} \to \eta_{\rho \sigma} \). 


