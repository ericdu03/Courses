\section{January 27}
Last lecture, we derived the Poynting theorem, which gave us a continuity equation for energy. Today, our
objective will be to derive a similar continuity equation for momentum. Before we do that however, there are
a couple of remarks we should make about the Poynting vector. From last lecture, we have:
\[
	\dv{E_\text{particle}}{t} + \dv{U_\text{EM}}{t} = -\oint_{\partial \mathcal{V}} \left( \frac{\mathbf{E}
	\times \mathbf{B}}{\mu_0} \right) \diff \mathbf{a} 
\]
where we established that the left hand side represents the change in energy density over time within the
volume \( \mathcal{V} \). The right hand side is the flux integral of the Poynting vector \( \mathbf{S} =
\frac{1}{\mu_0}(\mathbf{E} \times \mathbf{B}) \), which has units of energy per unit time per area. 

The first remark we should make is how we should intuitively interpret \( \mathbf{S} \). Consider a simple
circuit, like the one shown below:
\begin{center}
	\begin{circuitikz}
		\draw(0, -1) to[battery1] (0, 1) to (4, 1) to[bulb] (4, -1) to (0, -1); 
	\end{circuitikz}
\end{center}
Locally on the bulb, it's not hard to derive that \( \mathbf{S} \) points radially inward toward the load (a
lightbulb). Because energy is being expended by the bulb, the energy must come from \textit{somewhere}, but
where is it coming from? Initially it might seem like \( \mathbf{S} \) tells us that energy is coming from
thin air, but what it's really saying is that the energy is being taken away from the \( \mathbf{E}  \) and
\( \mathbf{B} \) fields, and going into the bulb. In particular, the energy flow is as follows:
\[
	\mathbf{S} \rightarrow \text{E-field} \rightarrow \text{particles} \rightarrow \text{bulb}
\]
Further, if you work out the math you will find that the \( \mathbf{S} \) field points radially outwards  
\subsection{Continuity Equation for Momentum}
To derive the continuity equation for momentum, we will invoke the same kind of logic we used to arrive at
the conservation of energy equations. First, we will begin with an equation that describes the change in
momentum over time, which is incidentally the equation for force:
\[
	\dv{\mathbf{p}_\text{particle}}{t} = \mathbf{F} = \int (dq) (\mathbf{E} + \mathbf{v} \times \mathbf{B})
\]
exchanging \( q \) in favor of \( \rho \), this integral becomes:
\[
	\dv{\mathbf{p}_\text{particle}}{t} = \int_{\mathcal{V}} \rho \mathbf{E} + \mathbf{J} \times \mathbf{B}
	\diff \tau
\]
Now, our goal here will be the same as last lecture: we want to massage this equation into one which has a
boundary term and also a volume term. The former will represent the "flow" of momentum through the surface of
\( \mathcal{V} \), and the volume term will represent the momentum stored inside the volume \( \mathcal{V}
\). To begin, we first invoke Gauss's law and the Ampere-Maxwell law to rewrite \( \rho \) and \( \mathbf{J}
\) in terms of \( \mathbf{E} \) and \( \mathbf{B} \):
\begin{align*}
	F^{i} &= \int [\epsilon_0(\partial_m E^{m})E^{i} + \frac{1}{\mu_0}(\curl B)_j B_k - \epsilon_0
	\epsilon^{ijk}(\partial_t E_j) B_k] \diff \tau\\
		  &= \int [\underbrace{\epsilon_0 (\partial_m E^{m}) E^{i}}_1 + \underbrace{\frac{1}{\mu_0}\epsilon^{ijk}
	\epsilon_{jmn}(\partial^{m}B^{n})B_k}_2 - 
	\underbrace{\epsilon_0 \epsilon^{ijk} (\partial_t E_j)B_k}_3] \diff \tau 
\end{align*}
We will deal with these terms separately, starting with the third term. We first rewrite this using the
product rule as the difference of two terms:
\[
	\int \epsilon_0 \epsilon^{ijk}(\partial_t E_j) B_k \diff \tau = -\int \epsilon_0
	\epsilon^{ijk}\partial_t(E_j B_k) - \epsilon_0 \epsilon^{ijk}E_j (\partial_t B_k) \diff \tau
\]
Then by Faraday's law, we have \( \curl E = -\partial_t B_k \), so this allows us to write the second term
using another Levi-Civita symbol:
\[
	-\dv{t} \int (\epsilon_i \epsilon^{ijk} E_j B_k) \diff \tau - \int \epsilon_0 \epsilon^{ijk}
	E_j(\epsilon_{kmn}\partial^{m} E^{n}) \diff \tau 
\]
Now, we have the following identity when we have two Levi-Civita symbols (again, review the index notation if
you need to): 
\[
	\epsilon^{ijk}\epsilon_{mnk} = \delta^{i}_m \delta^{j}_n - \delta^{i}_n \delta^{j}_m
\]
We then use \( \epsilon_{kmn} = \epsilon_{mnk} \), and invoke the above rule:
\[
	-\dv{t} \int (\epsilon_0 \epsilon^{ijk} E_j B_k) \diff \tau - \int \epsilon_0 (\delta^{i}_m \delta^{j}_n
	- \delta^{i} _n \delta^{j}_m) E_j \partial^{m} E^{n} \diff \tau 
\]
Finally, this term becomes:
\[
	-\dv{t} \int \epsilon_0 \epsilon^{ijk} E_j B_k \diff \tau - \int \epsilon_0 [E_n(\partial^{i} E^{n}) -
	E_m(\partial^{m} E^{i})] \diff \tau 
\]
and that's all we can do with the third term. We'll deal with the first term next, which is pretty easy.
Using the product rule, we get:
\[
	\int \epsilon_0 (\partial_m E^{m}) E^{i} \diff \tau = \int[\epsilon_0 \partial_m (E^{m} E^{i}) -
	\epsilon_0 E^{m} \partial_m E^{i}] \diff \tau
\]
Finally, we deal with the second term. We will begin in the same fashion as the previous term, by
writing this using product rule:
\begin{align*}
	-\frac{1}{\mu_0}\int \epsilon^{jik} \epsilon_{jmn}(\partial^{m}B^{n})B_k \diff \tau &=
	-\frac{1}{\mu_0} \int \epsilon^{jik} \epsilon_{jmn} \left[ \partial^{m}(B^{n}B_k) -
	B^{n}(\partial^{m}B_k)\right] \diff \tau \\ 
	&= -\frac{1}{\mu_0} \int \left( \delta^{i}_m \delta^{k}_n - \delta^{i}_n \delta^{k}_m \right)
	[\partial^{m} (B^{n}B_k) - B^{n}(\partial^{m}B_k)] \diff \tau  \\ 
	&= -\frac{1}{\mu_0}\int \partial^{i}\partial^{k}B_k - \partial^{k}(B^{i}B_k) - B^{k}(\partial^{i} B_k) +
	B^{i}(\partial^{k}B_k) \diff \tau
\end{align*}
The last term in this integral is \( \div \mathbf{B} \), which is always zero. 
Now, combining all the terms together, we get:
\begin{multline*}
	\dv{\mathbf{p}_\text{particle}^{i}}{t} = \int \diff \tau \left[ \epsilon_0 \partial_m (E^{m}E^{i}) -
	\epsilon_0 E^{m}(\partial_m E^{i}) - \frac{1}{\mu_0} \partial^{i}(B^{k}B_k) +
\frac{1}{\mu_0}\partial^{k}(B^{i}B_k) + B_k (\partial^{i} B_k) - \epsilon_0 E_n (\partial^{i} E_n) +
\epsilon_0 E_m (\partial^{m} E^{i}) \right] \\- \dv{t} \int \epsilon_0 \epsilon^{ijk}E_j B_k \diff \tau
\end{multline*}
Moving the time derivative term to the other side, we get:
\[
	\dv{\mathbf{p}_\text{particle}^{i}}{t} + \dv{t} \int \epsilon_0 \epsilon^{ijk}E_j B_k \diff \tau = \int
	\diff \tau \left[ \text{stuff} \right]
\]
The "stuff" here is everything in the square brackets, and we will simplify this next time. But, notice that
there are some things that are already starting to come out of this equation. Namely, the second term in this
equation is \( \epsilon_0 (\mathbf{E} \times \mathbf{B}) = \epsilon_0 \mu_0 \mathbf{S} \), so this term
represents the momentum carried by the electromagnetic fields. Next time, we will simplify the right hand
side into a nicer form.    
