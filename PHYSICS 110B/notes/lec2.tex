\section{January 24}
In this lecture, we will first begin by discussing the classical continuity equation for charge, then use
this equation to develop equivalent equations for energy and momentum. From a high level standpoint, it is
clear that the latter two quantities must also have an associated continuity equation, since they are also
conserved quantities. To begin, let's start with the equation for conservation of charge:
\begin{equation}
	\label{cons-charge}
	\dv{Q}{t} = - \oint_{\partial \mathcal{V}} \mathbf{J} \cdot \diff \mathbf{a} 
\end{equation}
Because we can write \( Q \) as a volume integral of the charge density: \( Q = \int_{\mathcal{ V}}\rho \diff
\tau\), so we can write:
\[
	\dv{t} \int_{\mathcal{ V}} \rho \diff \tau = -\oint_{\partial \mathcal{ V}} \mathbf{J} \cdot \diff \mathbf{a}
\]
We can invoke the divergence theorem to transform the right hand side into a volume integral, and also move
the total derivative inside the integral turning it into a partial derivative:
\[
	\int_{\mathcal{V}} \pdv{\rho}{t}\diff \tau = -\int_{\mathcal{V}}\mathbf{J} \cdot \diff \mathbf{a}
\]
and since these two quantities must be equal at all times, then we arrive at the (local) 
continuity equation for charge:
\begin{equation}
	\label{charge continuity}
	\pdv{\rho(\mathbf{r}, t)}{t} = - \div \mathbf{J}
\end{equation}

\subsection{Poynting's Theorem}
Recall from 110A that we have the following definition for the Poynting vector:
\begin{equation}
	\label{poynting}
	\mathbf{S} = \frac{1}{\mu_0}(\mathbf{E} \times \mathbf{B})
\end{equation}
with units of energy per area per time. Similarly, recall the equations for the energy density stored in the
electromagnetic field:
\begin{align*}
	U_E &= \frac{1}{2}\epsilon_0 |\mathbf{E}|^2\\
	U_B &= \frac{1}{2\mu_0}|\mathbf{B}|^2 
\end{align*}
These two equations will become relevant later in the lecture. First, let's establish our goal: because
energy is a conserved quantity, we want to find an equation of the same form as the one above, but for
energy. To do this, we begin by considering the force on a charge \( dq \):
\[
	\diff \mathbf{F} = \diff q (\mathbf{E} + \mathbf{v} \times \mathbf{B})
\]
Then, the power done by the electromagnetic field (work over time) over some volume \( \mathcal{V} \) is:
\begin{align*}
	\dv{W_\text{EM}}{t} &= \int_{\mathcal{V}}(\rho \diff \tau) (\mathbf{E} + \mathbf{v} \times
	\mathbf{B})\cdot \mathbf{v} \diff \tau\\
	&= \int_{\mathcal{V}}\rho \mathbf{E} \cdot \mathbf{v} \diff \tau \\ 
	&= \int_{\mathcal{V}}\mathbf{J} \cdot \mathbf{E} \diff \tau 
\end{align*}
Now, we use the Ampere-Maxwell equation (eq. \ref{ampere-maxwell}) to rewrite \( \mathbf{J} \) purely in
terms of \( \mathbf{B} \) and \( \mathbf{E} \):
\begin{align*}
	\dv{W_\text{EM}}{t} &= \int_{\mathcal{V}}\left( \frac{1}{\mu_0}(\curl \mathbf{B}) - \epsilon_0 \partial_t
	\mathbf{E}\right) \cdot \mathbf{E} \diff \tau \\ 
	&= \int_{\mathcal{V}}\frac{1}{\mu_0}(\curl \mathbf{B}) \cdot \mathbf{E} \diff \tau -
	\int_{\mathcal{V}}\epsilon_0 (\partial_t \mathbf{E}) \cdot \mathbf{E} \diff \tau 
\end{align*}
Here, we will do the computation of each term separately, starting with the second term. Notice that it's
actually part of a product rule, namely \( \partial_t(\mathbf{E} \cdot \mathbf{E}) \), and when we expand the
product rule we get two identical terms. Therefore, we can actually rewrite the second term as:
\[
	\int_{\mathcal{V}} \epsilon_0(\partial \mathbf{E}) \cdot \mathbf{E} \diff \tau =
	\frac{1}{2}\int_{\mathcal{V}} \epsilon_0 \partial_t (\mathbf{E} \cdot \mathbf{E}) \diff \tau =
	\dv{t} \int_{\mathcal{V}} \frac{\epsilon_0}{2}  |\mathbf{E}|^2 \diff \tau 
\]
Now we deal with the first term. To rewrite this term, it's useful to use index notation to simplify the
math. Review the index notation from 110A if you need to, but the integrand becomes:
\begin{align*}
	(\curl \mathbf{B}) \cdot \mathbf{E} &= \epsilon^{ijk}(\partial_j B_k) E_i = \epsilon^{ijk} \partial_j (B_k
	E_i) - \epsilon^{ijk}B_k (\partial_j E_i)\\
	&= -\epsilon^{ijk}\partial_j (B_k E_i) - \epsilon^{kij}B_k(\partial_j E_i) \\ 
	&= -\epsilon^{ijk}\partial_j (B_k E_i) + \epsilon^{kji} B_k (\partial_j E_i) 
\end{align*}
Note that the Levi-Civita tensor \( \epsilon^{ijk} \) does not change under cyclic permutations of summation,
but changes sign when we perform a swap of two adjacent indices. Now, the first term gives \( \div(\mathbf{E}
\times \mathbf{B})\), and the second term gives \( \mathbf{B} \cdot (\curl \mathbf{E}) \). So, the first
integral becomes:
\[
	-\frac{1}{\mu_0} \int_{\mathcal{V}} \div(\mathbf{E} \times \mathbf{B}) \diff \tau +
	\frac{1}{\mu_0}\int_{\mathcal{V}} \mathbf{B} \cdot (\curl \mathbf{E}) \diff \tau 
\]
Now finally, we can use Faraday's law (eq. \ref{faraday-law}) to write \( \curl \mathbf{E} = -\partial_t
\mathbf{B} \), which in the second term allows you to write it as \( \frac{1}{2\mu_0}|\mathbf{B}|^2 \).
Simultaneously, we use divergence theorem on the first term to write it as a surface integral:  
\[
	-\frac{1}{\mu_0} \oint_{\partial \mathcal{V}}(\mathbf{E} \times \mathbf{B}) \diff \mathbf{a} +
	\frac{1}{\mu_0} \int_{\mathcal{V}} \mathbf{B} \cdot (-\partial_t \mathbf{B}) \diff \tau =
	-\frac{1}{\mu_0}\oint_{\partial \mathcal{ V}}(\mathbf{E} \times \mathbf{B}) \diff \mathbf{a} - \dv{t}
	\int_{\mathcal{V}}\left( \frac{1}{2\mu_0} |\mathbf{B}|^2\right) \diff \tau 
\]
Now, we can put this all together:
\[
	\dv{W_\text{EM}}{t} = -\frac{1}{\mu_0} \oint_{\partial \mathcal{V}} (\mathbf{E} \times \mathbf{B}) \diff
	\mathbf{a} - \dv{t} \int_{\mathcal{V}}\left( \frac{1}{2\mu_0}|\mathbf{B}|^2 +
	\frac{\epsilon_0}{2}|\mathbf{E}|^2 \right) \diff \tau 
\]
Moving the first term to the left hand side:
\begin{equation}
	\label{poynting-thm}
	\dv{W_\text{EM}}{t} + \dv{t} \int_{\mathcal{V}}\left( \frac{1}{2\mu_0}|\mathbf{B}|^2 +
	\frac{\epsilon_0}{2}|\mathbf{E}|^2 \right)\diff \tau = -\frac{1}{\mu_0}\oint_{\partial
\mathcal{V}}(\mathbf{E} \times \mathbf{B}) \diff \mathbf{a}
\end{equation}
This final equation known as Poynting's theorem. Essentially, you can read it as follows:
\[
	\dv{t}(E_\text{particle} + E_\text{electric} + E_\text{magnetic}) = -\frac{1}{\mu_0} \oint_{\partial
	\mathcal{V}} (\mathbf{E} \times \mathbf{B}) \diff \mathbf{a}
\]
So the left hand side represents the change of energy in the volume \( \mathcal{V} \), and the right hand
side represents the energy flow through the surface of the volume \( \mathcal{V} \). From this equation, it's
easy to see that \( \mathbf{S} \) represents the energy flux through the volume \( \mathcal{V} \), and
essentially shows us the direction of energy flow around a surface. 
   






