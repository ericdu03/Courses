\section{May 2}

\subsection{Natural Units}
In this lecture, we should mention that from here on out we will be using natural units, where \( c = \hbar = 1
\). This is so that we don't have to carry the constants everywhere we go, and it also has some beneifts in
the way of dimensional analysis. In particular, since \( c = 1 \), then the dimension for length is the same
as time:
\[
	[L] = [T]
\]
which is particularly natural especially in relativity since length and time can be interchanged with each
other. Setting \( \hbar = 1 \), which usually carries the units of \( [E][T] \), means that we now regard
energy and time as inverses:
\[
	[E] = [T]^{-1}
\]
From \( E = mc^2 \), because \( c = 1 \), then this implies that \( [E] = [M] \) and combining this with the
previous relations we get the big chain:
\[
	[L] = [T] = [E]^{-1} = [M]^{-1}
\]
\subsection{Action of a Free scalar field}
Consider a scalar field \( \phi = \phi(x) = \phi(t, \mathbf{x}) \). Our goal is to find an action that
satisfies the three criteria:
\begin{enumerate}[label=\arabic*.]
	\item Quadratic in \( \phi \). We want this because we want the equations of motion to be linear in \(
		\phi \), and hence we want the action to be quadratic in \( \phi \). 
	\item Lorentz invariant. We want this because the action gives us the equations of motion, through the
		stationary action principle, and obviously we want them to be the same in all inertial frames.   
	\item Involve \( \partial_\mu \phi \). We want this because we want a time evolution \( \dot \phi \)
		term. We don't want to just use \( \partial_t \phi \) since in relativity time and space are
		interchangeable, so we use the general derivative \( \partial_\mu \phi \) instead. 
\end{enumerate}

It turns out, the action that satisfies these three is:
\[
	S = -\frac{1}{2}\int d^{4}x \left[ \eta^{\mu \nu}\partial_\mu \phi \partial_\nu \phi + m^2 \phi^2 \right]
\]
In natural units, \( S \) should be dimensionless, and you can check that the right hand side has units of \(
[E][T] \) so the equation is correct. Now, just like any other action, we first vary the field by introducing
\( \phi(x) \to \phi(x) + \delta \phi(x) \): 
\[
	\delta S = -\frac{1}{2}\int d^{4}x \, \delta \left[ \eta^{\mu \nu}\partial_\mu \phi \partial_\nu \phi 
	+ m^2 \phi^2 \right] = -\frac{1}{2}\int d^{4}x  \, \left[ 2 \eta^{\mu \nu} \partial_\mu \phi
	\, \delta(\partial_\nu \phi) + 2m^2 \phi\,  \delta \phi \right]
\]
Now, in the first term the variation of the derivative, \( \delta(\partial_\nu \phi) \) can be written as:
\[
	\delta (\partial_\nu \phi) = \partial_\nu (\phi + \delta \phi) - \partial_\nu \phi = \partial_\nu (\delta
	\phi)
\]
So we can write the entire integral as:
\[
	- \int d^{4}x \, \left[ \partial_\nu \left( \eta^{\mu \nu}\partial_\mu \phi \partial_\nu \phi \right) -
	\partial_\nu \left( \eta^{\mu \nu} \partial_\nu \phi \right) \delta \phi + m^2 \phi \, \delta \phi \right]
\]
Now we will take integration by parts. We will assume that we only vary the field locally, so at the extremes
\( \delta \phi = 0 \), allowing us to drop the boundary term. So, this gives us:
\[
	\int d^{4} x \, \left[ \eta^{\mu \nu} \partial_\mu \partial_\nu \phi - m^2 \phi \right] \delta \phi
\]
If we require \( \delta S = 0 \) for any variation in the field, then the term in square brackets must be
zero. Removing the index notation, the equation reads:
\[
	(-\partial_t^2 + \nabla^2 - m^2) \phi = 0
\]
This is known as the \textbf{Klein-Gordon Equation}. Notice that it looks like a wave equation, except it
has a mass term. The other thing of note is that this equation is linear in \( \phi \), 
which explains why we wanted our action that is quadratic in \( \phi \) from earlier. 
If we had more higher order terms in the action, then they'd contribute
to the right hand side effectively acting as source terms just like how they appeared in the wave equations
for the fields \( V \) and \( \mathbf{A} \).   

For equations of motion without a source, then our solutions are plane waves:
\[
	\phi = A e^{i(Et - \mathbf{p} \cdot \mathbf{x})} = Ae^{-i p^{\mu}p_{\mu}}
\]
Substituting this back into the equation, we get:
\[
	(E^2 - |\mathbf{p}|^2 - m^2) A e^{i (Et - \mathbf{p} \cdot \mathbf{x})} = 0 \implies E^2 - |\mathbf{p}|^2
	- m^2 = 0
\]
This is exactly the relativistic equation for energy: \( E^2 = p^2 c^2 + m^2 c^{4} \). 

\subsection{Action of Massless 4-Vector Fields}
The above section takes care of scalar fields, but as we've studied in electromagnetism, the electric and
magnetic fields are vector fields, so this section will be dedicated to writing its action and the resulting
equations of motion. As we will see, Maxwell's equations will come right out at the end. 

We will consider a "free" field at first, then add source terms later. Because we are working in a
relativistic context, we should use the 4-vector field \( A^{\mu} \) instead of the standard 3-vector. This
allows us to impose the following conditions:
\begin{enumerate}[label=\arabic*.]
	\item Quadratic in \( A_\mu \) because we want an equation of motion that is linear in \( A_\mu \).    
	\item Lorentz invariance. We want Lorentz invariance here for the same reason as the scalar field.  
	\item Need to involve \( \partial_\nu A_\nu \), just like we involved \( \partial_\mu \phi \) from
		before. 
	\item The action should be massless. We want this in particular because we want to match
		electromagnetism, which has no mass terms. As such, we will drop the \( \frac{1}{2}m^2 A^{\mu}A_\mu
		\) term from the action.  
\end{enumerate}
So what kind of action can we write? If we want Lorentz invariance, that also satisfies the third condition,
then one thing we can do is write something like:
\[
	(\partial_\rho A_\sigma) (\partial_\mu A_\nu)
\]
but we can't just leave it as is, since the action must be dimensionless. So, we need to find a way to
contract these indices, of which there are two ways:
\begin{enumerate}[label=\arabic*.]
	\item \( (\partial^{\mu}A^{\nu})(\partial_\mu A_\nu) \)
	\item \( (\partial_\mu A^{\mu})(\partial_\nu A^{\nu}) \)
\end{enumerate}
Note there is a third way \( A^{\mu}(\partial^2 A_\mu) \), but if we expand this out:
\[
	A^{\mu}(\partial^2 A_\mu) = A^{\mu}(\partial^{\nu} \partial_\nu A_\mu) = \partial^{\nu}(A^{\mu}
	\partial_\nu A_\mu) - (\partial^{\nu} A^{\mu})(\partial_\nu A_\mu)
\]
We've now written this in the form of a total derivative term and the same equation as in method 1. Because
we eventually get rid of total derivative terms anyways, this way of contracting ends up being the same as
the first, so there are really only two unique ways to contract these indices. Our action can now be written
as a combination of the two ways:
\[
	S = -\frac{1}{2}\int d^{4}x \, \left[ a \left( \partial_\mu A_\nu \right)\left( \partial^{\mu}A^{\nu}
	\right) + b \left( \partial_\mu A^{\mu} \right) \left( \partial_\nu A^{\nu} \right) \right]
\]
Now, we vary the action:
\[
	\delta S = -\frac{1}{2}\int d^{4}x \, \left[ a (\partial_\mu A_\nu) \delta(\partial^{\mu}A^{\nu}) + b
	(\partial_\mu A^{\mu}) \delta(\partial_\nu A^{\nu}) \right] 
\]
Now we do integration by parts and remove the total derivative:
\begin{align*}
	\delta S &= \int d^{4}x \, \left[ a \partial^{\mu} (\partial_\mu A_\nu) \delta A^{\nu} + b
	\partial_\nu(\partial_\mu A^{\mu}) \delta A^{\nu} \right] \\ 
	&= \int d^{4}x \, \left[ a \partial^{\mu}\partial_\mu A_\nu + b \partial_\nu \left( \partial_\mu A^{\mu}
	\right) \right] \delta A^{\nu} 
\end{align*}
Now, requiring that \( \delta S = 0 \) for any \( \delta A^{\nu} \) means we get:
\[
	\partial^{\mu} \partial_\mu A_\nu + b \partial_\nu (\partial_\mu A^{\mu}) = 0
\]
The free equation of motion is then (we raised the free index \( \nu \) here, it does nothing except makes
the equation a bit nicer to look at):
\[
	a \partial^{\mu} \partial_\mu A^{\nu} + b \partial^{\nu}(\partial_\mu A^{\mu}) = 0
\]
So this is the free equation. Now, we want to add the effect of sources, which we will do by adding them to
the right side of this equation. Because the left hand side is a 4-vector, then the thing we add on the right
must also be a 4-vector:
\begin{equation}
	\label{vector-action}
	a \partial^{\mu} \partial_\mu A^{\nu} + b \partial^{\nu} (\partial_\mu A^{\mu}) = J^{\nu}
\end{equation}
Adding \( J^{\nu} \) here is analogous to what we had with the Lorentz gauge back in chapter 10: 
\begin{align*}
	(\partial_t^2 - \nabla^2) \phi &= \rho\\
	(\partial_t^2 - \nabla^2) \mathbf{A} &= \mathbf{J}
\end{align*}
where terms like the charge and current density can be regarded as "sources".  We will also require \( J^{\nu} \)
to be conserved, such that:
\[
	\partial_\nu J^{\nu} = 0
\]
One way to argue that we need this constraint is to think about charges and currents: we want these
quantities to be conserves, so its generalized version \( J^{\nu} \) should also be conserved. The above
equation for conservation also has a nice meaning if we allow \( J^{\nu} = (\rho, \mathbf{J}) \) when we
expand out the summation notation:
\[
	\partial_t \rho + \nabla \cdot \mathbf{J} = 0
\]
this is the standard continuity equation! Hopefully this small demonstration shows why we need this constraint, and
that it is indeed a well-motivated result. Now, for consistency, if we take \( \partial_\nu \) of both sides
of \cref{vector-action}:
\[
	\partial_\nu \left( a \partial^{\mu}\partial_\mu A^{\nu} + b \partial^{\nu}\left( \partial_\mu A^{\mu}
	\right) \right) = \partial_\nu J^{\nu} = 0
\]
So this gives us the relation:
\[
	a \partial^2 (\partial_\nu A^{\nu}) + b \partial^2 (\partial_\mu A^{\mu}) = 0
\]
both terms here \( \partial_\nu A^{\nu} \) and \( \partial_\mu A^{\mu} \) are of the same form, so the only
combination of \( a, b \) that makes this zero is \( a = -b \). By convention, we will let \( a = 1 \) and \(
b = -1\). So in summary, the action with the conserved current reads:
\[
	S = -\frac{1}{2}\int d^{4} \, \left( \partial_\mu A_\nu \partial^{\mu} A^{\nu} - (\partial_\mu A^{\mu})
	(\partial_\nu A^{\nu}) \right) + \int d^{4}x \, A_\mu J^{\mu}
\]
Now, we define \( F_{\mu \nu} \equiv \partial_\mu A_\nu - \partial_\nu A_\mu \). Then, \( F^{\mu \nu} F_{\mu
\nu} \) gives:
\begin{align*}
	F^{\mu \nu} F_{\mu \nu} &= \left( \partial_\mu A_\nu - \partial_\nu A_\mu \right)\left(
	\partial^{\mu}A^{\nu} - \partial^{\nu} A^{\mu} \right)\\
	&= 2(\partial_\mu A_\nu)(\partial^{\mu}A^{\nu}) - 2(\partial_\nu A_\mu)(\partial^{\nu}A^{\mu}) \\ 
	&= 2(\partial_\mu A_\nu) (\partial^{\mu}A^{\nu}) - 2(\partial^{\mu}A_\mu) (\partial_\nu A^{\nu}) 
\end{align*}
This is exactly twice the first integral, so we can rewrite the action as:
\[
	S = -\frac{1}{4}\int d^{4}x \, F_{\mu \nu} F^{\mu \nu} + \int d^{4}x \, A_\mu J^{\mu}
\]
Now, forcing \( \delta S = 0 \) eventually gets us:\footnote{we skipped the algebra in class in the interest
of time.}
\begin{equation}
	\label{maxwell-1}
	\partial_\mu F^{\mu \nu} = J^{\mu}
\end{equation}
In addition, because of the antisymmetry of \( F^{\mu \nu} \) (which you can see from its definition), we
have the relation:
\begin{equation}
	\label{maxwell-2}
	\partial_\lambda F_{\mu \nu} + \partial_\nu F_{\lambda \mu} + \partial_\mu F_{\nu \lambda} = 0
\end{equation}
As it turns out, \cref{maxwell-1,maxwell-2} are exactly Maxwell's equations in index notation! In particular,
\cref{maxwell-1} gives Gauss's and the Ampere-Maxwell law, since these two equations deal with source terms.
The other two are given by \cref{maxwell-2}. To see this worked out 
explicitly, let \( A^{\mu} = (V, \mathbf{A}) \) and \( A_\mu = (-V, \mathbf{A}) \), then from
\cref{maxwell-1} we have:
\begin{align}
	\label{f0i}F_{0i} &= \partial_0 A_i - \partial_i A_0 = \partial_t A_i - \partial_i V = \partial_t A_i - \nabla V =
	E_i\\
	\label{fij}F_{ij} &= \partial_i A_j - \partial_j A_i = \epsilon_{ijk}B^{k}
\end{align}
So, putting \cref{f0i} into \cref{maxwell-1} gives us:
\[
	\partial_0 F^{00} + \partial_i F^{0i} = \rho \implies \partial_i E_i = \nabla \cdot \mathbf{E} = \rho
\]
Working this out with the other indices gives you Ampere-Maxwell too. For \cref{maxwell-2}, if you let \(
\lambda \mu \nu = 0ij \) and iterate, then you get:
\[
	\partial_0 F_{ij} + \partial_j F_{0i} + \partial_{i}F_{i0}
\]
Then, using \cref{f0i,fij}, then this becomes:
\[
	-\partial_t \left( \epsilon_{ijk}B^{k} \right) + \partial_j E_i - \partial_i E_j = 0
\]
Contracting with \( \epsilon^{ijm} \epsilon_{ijk} \), then this becomes:
\[
	-2 \partial_t B^{m} - \epsilon^{ijm}\left( \partial_i E_j - \partial_j E_i \right)	
\]
Finally, \( \partial_i E_j - \partial_j E_i = 2 \partial_i E_j \), so we indeed get Faraday's law:
\[
	\nabla \times \mathbf{E} = -\partial_t \mathbf{B}
\]
likewise, \( \nabla \cdot \mathbf{B} = 0 \) follows as well from the other indices. And that concludes our
derivation of Maxwell's equations! It's nice that we've essentially come full circle from the beginning: we
started with Maxwell's equations, and finished by deriving them. What is truly remarkable is that these
equations naturally fall out as a result of our two constraints on the action and the current, where neither
of them directly reference the equations at all -- they are simply a product of these two constraints. If
that's not beautiful, I don't know what is. 
 
 
