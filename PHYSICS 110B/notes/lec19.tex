\section{March 7}

\question{There was an error in the lecture recording, so the notes for this lecture were pieced together
	using the provided note and lecture audio.} 

Last time, we talked about the Coulomb gauge, which gave us the equations of motion:
\begin{align*}
	\nabla^2 V &= -\frac{\rho}{\epsilon_0} \\ 
	\left( \nabla^2 - \frac{1}{c^2}\partial_t^2 \right)\mathbf{A} &= -\mu_0 \mathbf{J} + \nabla\left(
	\frac{1}{c^2}\partial_t V \right)
\end{align*}

As we've said before, the Coulomb gauge is not very useful in electrodynamics, since the time derivative
terms are quite a nightmare to deal with. What is worth commenting on is the equation for the static
potential: the fact that the equation doesn't change even when we introduce time dependence means that the
electrostatic potential can actually be measured acausally:
\[
	V(t, \mathbf{r}) = \frac{1}{4 \pi \epsilon_0} \int \frac{\rho(t', \mathbf{r}')}{\rcurs} \diff \tau'
\]
While this may be disturbing at first, it is important to note that \( V \) by itself is not measurable, and
it is instead the electric field \( \mathbf{E} \) that we measure. To that end, the electric field takes on
the form \( \mathbf{E} = -\nabla V - \partial_t \mathbf{A} \), which is not instantaneous so there is no real
causality breaking here. 

\subsection{The Lorentz Gauge}
Aside from the Coulomb gauge, we also work with the Lorentz gauge, which sets:
\begin{equation}
	\label{19:lorentz-gauge}
	\nabla \cdot \mathbf{A} + \frac{1}{c^2}\partial_t V = 0
\end{equation}
To see how this is enforced, suppose we perform the transformation \( (V, \mathbf{A}) \to (A' - \partial_t
\lambda, A' + \nabla \lambda) \). Then, in order for the above equation to hold:
\[
	\nabla(\mathbf{A}' + \nabla \lambda) + \frac{1}{c^2}\partial_t (V' - \partial_t \lambda) = \nabla \cdot
	\mathbf{A}' + \nabla^2 \lambda + \frac{1}{c^2}\partial_t V' - \frac{1}{c^2}\partial_t^2 \lambda
\]
so, enforcing \( \lambda \) to satisfy:
\[
	\left( \nabla^2 - \frac{1}{c^2}\partial_t^2 \right)\lambda = -\left( \nabla \cdot \mathbf{A}' +
	\frac{1}{c^2}\partial_t^2 V' \right)
\]
ensures that we get the desired condition. With this condition, the equation of motion now reads:
\begin{align}
	\nabla^2 V - \frac{1}{c^2}\partial_t^2 V &= -\frac{\rho}{\epsilon_0} \\ 
	\nabla^2 \mathbf{A} - \frac{1}{c^2}\partial_t^2 \mathbf{A} &= -\mu_0 \mathbf{J}
\end{align}
This is obtained by putting gauge fixing condition into equations \ref{17:potential1} and
\ref{17:potential2}. These are the wave equations for the potentials \( V \) and \( \mathbf{A} \). However,
this one gauge fixing condition doesn't completely fix the gauge. You can actually add another gauge \(
\lambda' \) while still keeping equation \ref{19:lorentz-gauge} intact, as seen below:
\[
	\nabla \cdot(\mathbf{A} + \nabla \lambda) + \frac{1}{c^2}\partial_t(V - \partial_t \lambda') = \left(
	\nabla \cdot \mathbf{A} + \frac{1}{c^2}\partial_t V\right) + \left( \nabla^2 \lambda' -
\frac{1}{c^2}\partial_t^2 \lambda'\right) = 0 
\]
so as long as we choose \( \lambda' \) such that the second term is zero, we are allowed to add this extra
gauge. The implications of this are as follows: because we modify \( V \) by adding a time derivative to \(
\partial_t \lambda' \) and we modify \( \mathbf{A} \) by adding a gradient \( \nabla \lambda \) term, it is
possible to find a \( \lambda' \) that makes both \( V - \partial_t \lambda' \) and \( A + \nabla \lambda' \)
zero. So, this means that we can always take some nonzero \( (V, \mathbf{A}) \) and use this
extra gauge symmetry to transform both fields into the zero field \( (0, \mathbf{0}) \). This reduces another
degree of freedom, and hence we are left with two, consistent with EM waves.    

\subsection{Retarded Potentials and Fields} 
The benefit of the Lorentz gauge is that it transforms the equations of motion for \( V \) and \( \mathbf{A}
\) into four copies of the same equation, named the \textbf{Klein-Gordon equation}:
\[
	\left( \nabla^2 - \frac{1}{c^2} \pdv[2]{t}\right)\phi(t, \mathbf{r}) = \rho(t, \mathbf{r})
\]
Here, \( \phi(t, \mathbf{r}) \) represents the potential and \( \rho(t, \mathbf{r}) \) is a general source
term. To solve this equation, we can use Green's functions, which first involve solving the equation using a
delta function as the source:
\[
	\left( \nabla^2 - \frac{1}{c^2}\pdv[2]{t} \right)G(t, \mathbf{r}) = \delta^{(4)}(t, \mathbf{r})
\]
Then, once the solution to \( G(t, \mathbf{r}) \) is found, we can find the general solution for \( \phi \)
by simply integrating:
\begin{equation}
	\label{19:potential}
	\phi(t, \mathbf{r}) = \int d^{4}x'\ \rho(t', \mathbf{r}') G(t - t', \mathbf{r} - \mathbf{r}')
\end{equation}
You can also check explicitly that this satisfies the Klein-Gordon equation, we'll leave that as an exercise.
The intuitive idea for why this works is that we first solve the equation for a point potential with the delta
function \( \delta^{(4)}(t, \mathbf{r}) \), which gives the contribution to the potential \( \phi \) by a
single point source at \( (t, \mathbf{r}) \). Then, we integrate over all the point sources with the
integral, which come together to give us the full equation for the potential of all sources in the system,
and that solves for the field. 

\subsection{Solving the Green's Function Equation}
One way we can solve for Green's function is to solve it in momentum space, by considering the solution to
the Fourier transform of the equation. To this end, we can write Green's function as a sum of momentum modes
\( G_k \):
\[
	G(t, \mathbf{r}) = \int d^{4}k \ G_k e^{i k \cdot x}
\]
and here we use the four-vector \( k = (k^{0}, \mathbf{k}) \) with \( k \cdot x = -k^{0}t + \mathbf{k} \cdot
\mathbf{r} \) (we use the \( (-, +, +, +) \) convention here). In momentum space, let's derive the equation
for Green's function:
\begin{align*}
	\left(\nabla^2 - \partial_t^2\right) \frac{1}{(2\pi)^{4}}\int d^{4}k \ G_k e^{i k \cdot x} 
	&= \frac{1}{(2\pi)^{4}}
	\int d^{4}k \ e^{ik \cdot x}\\
	\left(\nabla^2 - \partial_t^2\right) \int d^{4}k \ G_k e^{ik \cdot x} &= \int d^{4}k \ e^{i k \cdot x}\\
	\left(\nabla^2 - \partial_t^2\right) G_k e^{ik \cdot x} &= e^{ik \cdot x}
\end{align*}
Now let's expand the \( k \cdot x \) on the left hand side:
\[
	G_k \left( \nabla^2 - \partial_t^2 \right)e^{i(- k^{0}t + \mathbf{k} \cdot \mathbf{r}} = e^{ik \cdot x}
\]
Then, computing the derivatives, we have:
\begin{align*}
	G_k \left( (k^{0})^2 - \mathbf{k} \cdot \mathbf{k} \right)e^{i k\cdot x} &= e^{i k \cdot x}\\
	\left( (k^{0})^2 - \mathbf{k} \cdot \mathbf{k} \right)G_k &=  1 \\ 
	\therefore G_k &=  \frac{1}{\left( (k^{0})^2 - \mathbf{k} \cdot \mathbf{k} \right)} 
\end{align*}
With this, we can now write the full integral for \( G(t, \mathbf{r}) \):
\begin{equation}
	\label{19:greens-function}
	G(t, \mathbf{r}) = \frac{1}{(2\pi)^{4}} \int d^{4}k \ \frac{e^{i k \cdot x}}{\left( (k^{0})^2 -
	\mathbf{k} \cdot \mathbf{k} \right)}
\end{equation}
There are some problems with this integral though: firstly, it's not well-defined, especially when \( |k^{0}|
= \pm \sqrt{\mathbf{k}\cdot \mathbf{k}}\), which will cause the integral to blow up. Therefore, treating this
integral classically is not the right idea, and we will instead have to borrow some tools from complex
analysis.    
 
 
  




