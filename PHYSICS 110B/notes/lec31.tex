\section{April 16}
\subsection{Lienard Formulation}
Last lecture, we left off with our discussion of a general formula for the power radiated by a particle when
\( v \neq 0 \). Picking up where we left off, we concluded last that the power emitted and the power received
at a time \( t_s \) later may be written as:
\begin{align*}
	\dv{W}{t} &= \dv{W}{t_s} \left( 1 - \frac{\brcurs \cdot \mathbf{v}}{c} \right)\\
			  &= \dv{W}{t_s}\left( 1 - \left( c \rcurs - \brcurs \cdot \mathbf{v} \right) \right) \\ 
			  &= \dv{W}{t_s} \frac{1}{c\rcurs} (\mathbf{u} \cdot \rcurs) 
\end{align*}
Now our \( \mathbf{E} \) field is of the form:
\[
	\mathbf{E} = \frac{q}{4\pi \epsilon_0} \frac{\rcurs}{(\brcurs \cdot \mathbf{u})}\left[ \brcurs \times (c
	\mathbf{\hat{r}} \times \mathbf{a})\right]
\]
Now, the power emitted by the particle may be written as \( P = \int \mathbf{S} \diff \mathbf{a} \), and the
infinitesimal area may be decomposed as \( \diff \mathbf{a} = r^2 \diff \Omega \), where \( \Omega \)
represents the solid angle. Thus, we may write \( \dv{P}{\Omega} \) as:
\[
	\dv{P}{\Omega} = \underbrace{\frac{1}{\mu_0c} \left( \frac{q^2}{16 \pi^2 \epsilon_0^2}
			\frac{\rcurs^2}{(\brcurs \cdot \mathbf{u})^{6}}\left|\brcurs \times (\mathbf{u} \times \mathbf{a})\right|^2 
	\right)}_{\mathbf{S}} \cdot \rcurs^2
\]
Ordinarily this would be fine, but to account for the velocity of the particle, we have to scale this by the
factor introduced at the beginning of the lecture:
\[
	\dv{P}{\Omega} = \frac{1}{\mu_0c} \left( \frac{q^2}{16 \pi^2 \epsilon_0^2} \frac{\rcurs^2}{(\brcurs \cdot
	\mathbf{u})^{6}}\left|\brcurs \times (\mathbf{u} \times \mathbf{a})\right|^2 \right) \cdot \rcurs^2
	\left( \frac{\mathbf{u} \cdot \brcurs}{c\rcurs} \right) = \frac{q^2}{16 \pi^2 \epsilon_0} \frac{|\brcurs
	\times (\mathbf{u} \times \mathbf{a})|^2}{(\hat{\brcurs} \times \mathbf{u})^{5}}
\]
The power then, is this this integral over the solid angle:
\begin{equation}
	\label{lienard}
	P = \oint \dv{P}{\Omega} \sin \theta \diff \theta \diff \phi = \frac{\mu_0 q^2 \gamma^{6}}{6 \pi c}\left(
	a^2 - \left| \frac{\mathbf{v} \times \mathbf{a}}{c} \right|^2 \right)
\end{equation}
This is called the \textbf{Lienard generalization} of the Larmor formula. Notice the factor of \( \gamma \)
present in the numerator, meaning that the added contribution due to the velocity (from the first term) is
very negligible, until we get to speeds \( v \sim c \).   


\subsection{Bremsstrahlung (Braking Radiation)}
This occurs particularly in the case where \( \mathbf{v} \parallel \mathbf{a} \). So, first we calculate some
cross products:
\[
	\mathbf{u} \times \mathbf{a} = (c \hat{\brcurs} - \mathbf{v}) \times \mathbf{a} = c
	\mathbf{\hat{\brcurs}} \times \mathbf{a} \implies \hat{\brcurs} \times (\mathbf{u} \times \mathbf{a}) = c
	\hat{\brcurs} \times (\hat{\brcurs} \times \mathbf{a}) = c \hat{\brcurs} (\hat{\brcurs} \cdot \mathbf{a})
	- c \mathbf{a}
\]
Thus, the numerator \( |\brcurs \times (\mathbf{u} \times \mathbf{a})|^2 \) becomes:
\[
	 |\brcurs \times (\mathbf{u} \times \mathbf{a})|^2 = c^2(a^2 + (\mathbf{a} \cdot \hat{\brcurs})^2 -
	 2(\mathbf{a} \cdot \hat{\brcurs}^2) = a^2 c^2 (1 - \cos^2 \theta) = c^2 a^2 \sin^2 \theta
\]
The denominator \( \hat{\brcurs} \cdot \mathbf{u} \) becomes:
\[
	\hat{\brcurs} \cdot \mathbf{u} = \hat{\brcurs} \cdot (c \hat{\brcurs} - \mathbf{v}) = c - \mathbf{v}
	\cdot \hat{\brcurs} = c - v \cos \theta = c(1 - \beta \cos \theta)
\]
we define \( \beta = v / c \). Therefore, putting this all together:
\[
	\dv{P}{\Omega} = \frac{q^2}{16 \pi^2 \epsilon_0^2} \frac{\left| \hat{\brcurs} \times(\mathbf{u} \times
	\mathbf{a}) \right|^2}{(\hat{\brcurs} \cdot \mathbf{u})^{5}} = \frac{\mu_0 q^2 a^2}{16 \pi^2 c}
	\frac{\sin^2 \theta}{(1 - \beta \cos \theta)^{5}}
\]
There is also a maximum angle that the radiation is shot out of, which you will do for homework. 
\begin{example}[Stability of "Classical" Hydrogen Atom]
	Consider the classical model of a Hydrogen atom, with a proton with \( +e \) charge in the center and an
	electron orbiting it at a distance of \( R \sim 10^{-10} \) m. Classically, the energy of the particle is
	given by:
	\[
		U = \frac{1}{2}mv^2 - \frac{ke^2}{R}
	\]
	According to the formula for centripetal force:
	\[
		\frac{ke^2}{R} = m a_c = \frac{mv^2}{R} \implies \frac{1}{2} mv^2 = \frac{ke^2}{2R}
	\]
	combining this with the previous equation yields a total energy of:
	\[
		E = - \frac{ke^2}{2R}
	\]
	This numerically comes out to about -13.6 eV, the well known ground state energy of hydrogen. From this
	calculation, we can also deduce \( v \), which comes out to roughly \( v \sim 10^{-2}c \), so this motion
	is still considered non-relativistic. So, we can use the Larmor formula:
	\[
		P = \frac{\mu_0 q^2 a^2}{6 \pi c}
	\]
	Now that we know accelerating particles give off radiation, so given that the electron radiates energy
	how long does it take before it loses enough energy to crash onto the proton? Well, we can calculate that
	now:
	\[
		\dv{E}{t} = -P = -\frac{\mu_0 q^2 a^2}{6 \pi c} \quad \dv{t} \left( - \frac{ke^2}{2r} \right) =
		\frac{\mu_0 q^2}{6 \pi c} \left( \frac{v^2}{r} \right)^2 = - \frac{\mu_0 q^2}{6 \pi c}\left(
		\frac{ke^2}{mr^2} \right)^2
	\]
	equating the two, we now have a differential equation in \( r \):
	\[
		\frac{ke^2}{2r^2} \dv{r}{t} = - \frac{\mu_0 q^2}{6 \pi c} \left( \frac{ke^2}{mr^2} \right)^2
	\]
	doing separation of variables and integrating to the characteristic time \( \tau \), we get:
	\[
		\tau = \frac{4\pi^2 \epsilon_0^2}{e^{4}} m^2 c^3 R_0^3 \sim 10^{-11} \text{ seconds}
	\]
	This is \textit{really bad!} Obviously this is not true, and it's one of the many things that motivated
	quantum mechanics. As you know, quantum mechanics doesn't treat the electron as a physical object around
	the atom but rather models its position as a wavefunction, resolving this paradox.   
\end{example}







