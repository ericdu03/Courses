\section{April 30}
\subsection{4-Velocity}
Earlier, we saw that \( \tilde u = \dv{x^{\mu}}{t} \) is not a valid 4-vector, as it does not transform like
\( x^{\mu} \) under Lorentz transforms.
A better alternative is to use the proper time instead: \( u^{\mu} = \dv{x^{\mu}}{\tau} \). 
This is guaranteed to be a valid 4-vector, since the numerator is a
4-vector and the proper time is Lorentz invariant. Thus, \( U^{\mu} \) transforms as \( U^{\mu} \to
\Lambda^{\mu}_\nu V^{\nu} \).

In component form, \( u \) can be written as:
\[
	u = \begin{pmatrix} c \dv{t}{\tau} \\ \dv{x^{i}}{\tau} \end{pmatrix} = \begin{pmatrix} \gamma c \\
\dv{t}{\tau} \dv{x^{i}}{\tau} \end{pmatrix} = \begin{pmatrix} \gamma c \\ \gamma v \end{pmatrix}
\]
Now here's the trick with 4-velocity: if we move together with the particle, then our relative speed with it
will be zero, so the 4-velocity vector is:
\[
	u^{\mu} = \begin{pmatrix} c \\ \mathbf{0} \end{pmatrix}
\]
Now, if we take the dot product \( U^{\mu}U_{\mu} = \eta_{\mu \nu}U^{\mu}U^{\nu} = \eta_{00} U^{0}U^{0} =
-c^2 \). But using the property that the dot product is Lorentz invariant, it means that the inner product \(
U^{\mu}U_{\mu} = -c^2\)	in \textit{all} frames, even when the velocity vector is not zero! This gives us an
important rule to remember when calculating dot products: we want to always choose a frame in which it is
easiest to calculate dot products, and leverage Lorentz invariance. 

\subsection{4-Momentum}
With the 4-velocity defined, it is then natural to define also the 4-momentum: \( P^{\mu} = m U^{\mu} \).
This is particularly a natural form to choose since it is a natural extension of our classical momentum \( p
= mv \). Like 4-velocity, we need to ensure that momentum is conserved, so we want the 4-momentum to also
behave as a 4-vector -- this is easy to guarantee since we've already established \( U^{\mu} \) as a
4-vector. 

The fact that \( P^{\mu} \) transforms linearly under Lorentz transformations actually guarantees
conservation of momentum! To see this, consider a collision between two particles, that generates two other
ones:
\begin{center}
	\begin{tikzpicture}
		\draw[-stealth] (-1, 1) node[left] {\( 1 \)} -- (0, 0.5);
		\draw[-stealth] (-1, -1) node[left] {\( 2 \)} -- (0, -0.5);
		\draw[-stealth] (1, 0.5) -- (2, 1) node[right] {\( 3 \)};
		\draw[-stealth] (1, -0.5) -- (2, -1) node[right] {\( 4 \)};
	\end{tikzpicture}
\end{center}
Suppose in the \( \mathcal{S} \) frame, the 4-momentum \( P_1^{\mu} + P_2^{\mu} = P_3^{\mu} + P_4^{\mu} \).
Then, in the \( \mathcal{S}' \) frame, we have:
\[
	P_1' + P_2' = \Lambda(P_1) + \Lambda(P_2) = \Lambda(P_1 + P_2) = \Lambda(P_3 + P_4) = P_3' + P_4'
\]
so the momentum is automatically conserved! The explicit form of \( P^{\mu} \) is more or less the 
same as \( U^{\mu} \):
\[
	P^{\mu} = \begin{pmatrix} mv^{0} \\ m U^{i} \end{pmatrix} = \begin{pmatrix} \gamma mc \\ \gamma
	m \mathbf{v} \end{pmatrix} = \begin{pmatrix} E / c \\ \mathbf{p} \end{pmatrix}
\]
Here, we define \( E = \gamma mc^2 \) to be the relativistic energy and \( \mathbf{p} = \gamma mv \) to be
the relativistic momentum. Notice how naturally these formulas come out simply from our constraint that we
want our vectors to transform linearly under the Lorentz transform; hopefully this gives more insight into
how these formulas came to be, and that they're not as contrived as they appear to be in your introductory
classes. 

Now's also a good place to note that when \( v \ll c \), the classical formulas come out. When \(  v\ll c \),
then \( \gamma \approx 1 \), so \( \mathbf{p} \approx m \mathbf{v} \). Likewise, if we Taylor expand \( E \):
\[
	E = \left( 1 + \frac{1}{2} \frac{v^2}{c^2} \right) mc^2 = mc^2 + \frac{1}{2}mv^2
\]
the first term represents the rest mass energy and is the formula \( E = mc^2 \), and the second term 
is exactly the kinetic energy term we're all familiar with. 

We can also write the speed of a particle in terms of \( E \) and \( \mathbf{p} \):
\[
	\frac{pc}{E} = \frac{\gamma mv c}{\gamma mc^2} = \frac{v}{c} \implies v = \frac{pc^2}{E}
\]
Another thing: if we take \( p_{\mu}p^{\mu} \):
\[
	p_{\mu}p^{\mu} = \eta_{\mu \nu}p^{\mu}p^{\nu} = -\frac{E^2}{c^2} + |\mathbf{p}|^2
\]
On the other hand, we know that since \( p^{\mu} = m U^{\mu} \), then the inner product is also equal to:
\[
	p_{\mu}p^{\mu} = m U_{\mu}(m U^{\mu}) = m^2 U_{\mu}U^{\mu} = -m^2 c^2
\]	
So we can combine these two equations together:
\begin{equation}
	\label{relativistic-energy}
	-\frac{E^2}{c^2} + |\mathbf{p}|^2 = -m^2 c^2 \implies E^2 = |\mathbf{p}|^2 c^2 + m^2 c^{4}
\end{equation}
this should also be a familiar equation. One property about this equation is that because it is a result of
equating two dot products, this identity is \textit{Lorentz invariant}, and holds true for any object in any
frame. 

So far, the above equations for particles with mass, but without mass, what happens? Well,
\cref{relativistic-energy} tells us that when \( m = 0 \), then \( E = |\mathbf{p}|c \), so its velocity:
\[
	v = \frac{pc^2}{pc} = c
\]
so massless particles travel at the speed of light!   

\subsection{4-Forces}
With 4-momentum established, it is now natural for us to go even further, and generalize forces into a
4-vector. We call this \( f^{\mu} \), which we will define as:
\[
	f^{\mu} = \dv{p^{\mu}}{\tau}
\]
again, as a classical generalization of Newton's second law \( F = \dv{p}{t} \). In component form:
\[
	\dv{p^{\mu}}{\tau} = \begin{pmatrix} \frac{1}{c} \dv{E}{\tau} \\ \dv{\mathbf{p}}{\tau} \end{pmatrix} =
\begin{pmatrix} \frac{1}{c} \dv{t}{\tau} \dv{E}{t} \\ \dv{t}{\tau} \dv{\mathbf{p}}{\tau} \end{pmatrix} = 
\begin{pmatrix} \frac{\gamma}{c} \dv{E}{t} \\ \gamma \dv{\mathbf{p}}{t} \end{pmatrix}
\]
If there is no 4-force acting on our particle, then we expect that \( \dv{p^{\mu}}{\tau} = 0 \), which is the
equation of motion of a free particle.

\subsection{Action of Free Relativistic Particles}
According to the stationary action principle, the evolution of any physical system should be one such that
the action is stationary. That is, we require \( \delta S = 0 \). Note that we only require the derivative to
be zero, not that it is minimized or maximized.\footnote{The Lagrangian is always convex, so we can always
guarantee minima or maxima, there won't be any "saddle points".} According to special relativity, because
physics should behave the same in all inertial frames, then the action should also be Lorentz invariant. 

When a particle travels through space, the path it traces out is called its \textbf{worldline}. Because it
describes how the particle moves, a natural candidate for the action would be the spacetime interval of its
worldline:
\[
	S_\text{particle} = -mc \int dS
\]
here we have a prefactor of \( mc \) for historical reasons, we don't need to care about these prefactors
very much. If we now parametrize our action by \( \lambda \), then the action may be written as:
\[
	S_\text{particle} = -mc \int \sqrt{\eta_{\mu \nu} \dv{x^{\mu}}{\lambda} \dv{x^{\nu}}{\lambda}} \diff\lambda
\]
Then, when we vary the action using \( x^{\mu}(\lambda) \to x^{\mu} + \delta x^{\mu} \), then we have:
\begin{align*}
	\delta S &= -mc \int \delta\sqrt{-\eta_{\rho \sigma} \dv{x^{\rho}}{\lambda} \dv{x^{\sigma}}{\lambda}}
	\diff \lambda \\ 
			 &= -mc \int \frac{\delta\left( -\eta_{\mu \nu} \dv{x^{\mu}}{\lambda} \dv{x^{\nu}}{\lambda}
			 \right)}{2 \sqrt{-\eta_{\rho \sigma} \dv{x^{\rho}}{\lambda} \dv{x^{\sigma}}{\lambda}}} \\ 
			 &= -mc \int \frac{-2 \eta_{\mu \nu} \dv{x^{\mu}}{\lambda} \dv{(\delta x^{\nu})}{\lambda}}{2
			 \sqrt{-\eta_{\rho \sigma} \dv{x^{\rho}}{\lambda} \dv{x^{\sigma}}{\lambda}}} 
\end{align*}
Now we do integration by parts, which means we slap a differential around everything but the \( \dv{(\delta
x^{\nu}}{\lambda} \) term, giving us:
\[
	\delta S = mc \int \dv{\lambda} \left( \frac{\eta_{\mu \nu} \dv{x^{\mu}}{\lambda}}{\sqrt{-\eta_{\rho
	\sigma} \dv{x^{\rho}}{\lambda} \dv{x^{\sigma}}{\lambda}}} \right) \delta x^{\nu}
\]
If we then require that \( \delta S = 0 \) for any \( \delta x^{\nu} \), then the requirement is that
everything else must equal to zero:
\[
	-mc \left( \frac{\eta_{\mu \nu} \dv{x^{\mu}}{\lambda}}{\sqrt{-\eta_{\rho \sigma} 
	\dv{x^{\rho}}{\lambda} \dv{x^{\sigma}}{\lambda}}} \right) = 0
\]
This equation may look ugly at first, but it is only written as such because we haven't specified how we want
to parametrize the worldline. If we choose a simple parametrization like the proper time (i.e. \( \lambda =
\tau \)), then we get:
\[
	-mc \dv{\tau} \left( \frac{\eta_{\mu \nu} \dv{x^{\mu}}{\tau}}{\sqrt{-\eta_{\rho \sigma}
	U^{\rho}U^{\sigma}}} \right) = -m \dv{p_{\nu}}{\tau} = 0 \implies \dv{p_{\mu}}{\tau} = 0
\]
so what comes out is a very natural equation: the statement that the net force on a free particle is zero.  
 





 














