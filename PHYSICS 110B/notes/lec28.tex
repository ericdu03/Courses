\section{April 9}
Last lecture, we found \( V \) and \( \mathbf{A} \), and with these fields determined we can get the \(
\mathbf{E} \) and \( \mathbf{B} \) fields:
\begin{align*}
	\mathbf{E} &= \frac{\mu_0 p_0 \omega^2}{4\pi} \left( \frac{\sin \theta}{r} \right)\cos\left[ \omega\left(
		t - \frac{r}{c}\right) \right] \boldsymbol{\hat{\theta}}\\
	\mathbf{B} &= - \frac{\mu_0 p_0 \omega^2}{4\pi c} \left( \frac{\sin \theta}{r} \right) \cos \left[
		\omega\left( t - \frac{r}{c} \right) \right] \boldsymbol{\hat{\phi}}
\end{align*}
here, we denote \( p_0 = qd \) is the polarization and \( q = q_0 \cos (\omega t) \) is an oscillating
charge. There are a couple things to note about this equation. Firstly, the \( \mathbf{E} \) and \(
\mathbf{B} \) oscillate in perpendicular directions, which makes sense given what we learned in chapter 9.
Further, these results are also consistent with our conclusion that the \( \mathbf{E} \) and \( \mathbf{B} \)
fields are in phase with each other, which is also what we found in chapter 9. With the \( \mathbf{E} \) and
\( \mathbf{B} \) fields, we can now figure out the Poynting vector:
\[
	\mathbf{S} = \frac{1}{\mu_0}(\mathbf{E} \times \mathbf{B}) = \frac{\mu_0}{c}\left\{ \frac{p_0
	\omega^2}{4\pi}\left( \frac{\sin \theta}{r} \right)\cos\left[ \omega\left( t - \frac{r}{c} \right)
\right] \right\}^2 \mathbf{\hat{r}}
\]
As a time-averaged quantity:
\[
	\mean{\mathbf{S}} = \frac{\mu_0 p_0^2 \omega^2}{32 \pi^2 c} \frac{\sin^2 \theta}{r^2}\mathbf{\hat{r}}
\]
What's interesting to note about this is that the time averaged \( \mathbf{S} \) is pointing in the \(
\mathbf{\hat{r}} \) direction, which is exactly perpendicular to the oscillation direction. This is what we
found in chapter 9, albeit through an intuitive argument back then. Here, we see the explicit mathematical
derivation.  

\subsection{Rayleigh and Mie Scattering}
When considering the interaction between light and particles, there are two limits that we can consider. The
first of which is when the wavelength of light is much larger than the size of the particle. In this limit,
because the particles are so small, we can essentially view them as vibrating coherently with the incoming
electric field, and therefore they radiate dipole radiation coherently. 

In this situation, the primary contribution of the waves comes from the dipole radiation, and therefore the
radiated electromagnetic wave is frequency dependent (you can see this via the \( \omega^2 \) dependence
above). This dependence explains why the sky is blue -- when the light from the sun scatters off molecules 
in the atmosphere (\( \mathrm{O_2, N_2, H_2} \)), the light scatters off them via dipole radiation. Further,
since the power scales proportional to \( \omega^{4} \), this heavily favors large frequencies, which is why
we see the sky as primarily blue. This phenomenon is known as \textbf{Rayleigh Scattering}. 

The other limit is when we consider the size of the particle to be much larger than the wavelength. In this
limit, the wave nature of the EM waves is suppressed, and in this case the light bounces off these materials
just like particles off a mirror. In this case, there is no frequency dependence, and this phenomenon
explains why clouds are white -- water molecules are on the order of 1mm, whereas light waves have
wavelengths on the order of 500nm, so all the light bounces off equally, leaving us with white clouds. This
phenomenon is called \textbf{Mie Scattering}.  

\subsection{Magnetic Dipoles}
Now, we will consider discussing magnetic dipoles. Consider the following situation, where we have a loop
with a current \( I = I_0 \cos(\omega t) \) shown in the diagram:

\begin{center}
	\begin{tikzpicture}[scale=1.5]
		\draw[thick,-stealth] (0,0) -- (0,2) node[right] {$z$};
		\draw[thick,-stealth] (0,0) -- (2,0) node[right] {$y$};
		\draw[thick,-stealth,rotate=-45] (0,0) -- (0,-2) node[below] {$x$};
		\draw[densely dashed] (0,1.75) -- (-0.9,0.85) -- (-0.9,-0.9);
		\draw[thick,-{Stealth[length=3pt]},blue!70!cyan!60] (-0.9,0.85) -- (-0.5,0.85) node[below,scale=0.5] {$\hat{\phi}=\hat{y}$};
		\filldraw[black] (-0.9,0.85) circle (0.025cm) node[left,scale=0.75] {$P$};
		\draw[thick,orange!70!white] (0,0) ellipse (0.99cm and 0.33cm);
		\draw[thick,-stealth,orange!70!white] (0,0) -- (1,0) node[midway,above,scale=0.5] {$b$};
		\draw[densely dotted] (0,0) -- (0.32,-0.32);
		\draw[-{Stealth[length=2pt]}] (0.32,-0.32) -- (0.55,-0.28) node[midway,below,scale=0.4] {$d\mathbf{l}'$};
		\draw[rotate=-45] (0.15,0) arc (0:-90:0.15cm) node[midway,below,scale=0.35] {$\phi$};
		\node[scale=0.75] at (1,1) (a) {$I = I_0\cos(\omega t)$};
		\draw[-{Stealth[length=2pt]},red!40!white] (-0.98,-0.08) node[left,scale=0.35] {$d\vec{e}'$} -- (-0.85,-0.27);
\end{tikzpicture}
\end{center}

We want to find the electric and magnetic fields over all space.
As always, the vector potential can be calculated using:
\[
	\mathbf{A} = \frac{\mu_0}{4\pi}\int \frac{\mathbf{J}(\mathbf{r}, t - \rcurs / c)}{\rcurs} \diff \tau' =
	\frac{\mu_0}{4\pi} \int \frac{I(\mathbf{r}, t - \rcurs / c)}{\rcurs} \diff \mathbf{l}'
\]
The reason we can write it in this form is because of the following relation for current density:
\( J \diff \tau = J A \diff l = I \diff l \). Now although \( \diff \mathbf{l} \) has both an \( x \) and \(
y\)-component, the \( x \) component will eventually cancel due to the symmetry in the system. Thus, we're
only left with the \( y \)-component. Because we need only care about the \( y \) component, then we may
write:
\[
	\mathbf{A} = \frac{\mu_0}{4\pi} \mathbf{\hat{y}} \int \frac{I(\mathbf{r}, t - \rcurs / c)}{\rcurs} \cos
	\phi' \diff \mathbf{l}'
\]
We will now use the approximations we had from earlier: \( b \ll r \), \( \frac{c}{\omega} \sim \lambda \gg b
\) and \( r \gg \frac{c}{\omega} \). Making these approximations, we eventually get the formula:
\[
	\mathbf{A}(\mathbf{r}, t) \simeq -\frac{\mu_0 m_0}{3\pi}\frac{\omega}{c}\left( \frac{\sin \theta}{r}
	\right) \sin\left[ \omega\left( t - \frac{r}{c} \right) \right]\boldsymbol{\hat{\phi}}
\]
You can verify this by explicitly computing the integral; in the interest of time we won't compute it here.
Since \( \rho = 0 \), then \( V = 0 \), so the \( \mathbf{B} \) field comes directly out of \( \nabla \times
\mathbf{A}\):
\[
	\mathbf{B} = \nabla \times \mathbf{A} = -\frac{\mu_0 m_0 \omega^2}{4 \pi c^2}\left( \frac{\sin \theta}{r}
	\right)\cos\left[ \omega\left( t - \frac{r}{c} \right) \right]\boldsymbol{\hat{\theta}}
\]
The \( \mathbf{E} \) field can be found using \( -\partial_t \mathbf{A} \):
\[
	\mathbf{E} = -\partial_t \mathbf{A} = \frac{\mu_0 m \omega^2}{4\pi c}\left(\frac{\sin \theta}{r}
	\right)\cos\left[ \omega\left( t - \frac{r}{c} \right) \right]\boldsymbol{\hat{\phi}}
\]
With \( \mathbf{B} \) and \( \mathbf{E} \), we may now calculate the power:
\[
	P_m = \int \mean{\mathbf{S}} r^2 \sin \theta \diff \theta \diff \Omega = \frac{\mu_0 m_0^2 \omega^{4}}{12
	\pi c^{3}}
\]
Now, with the electric dipole calculated from last lecture, we can now compare the power radiated by both the
electric and magnetic dipole:
\[
	\frac{P_m}{P_e} = \frac{1}{c^2}\left(\frac{m_0^2}{P_0^2}\right) = \frac{1}{c^2}\left( \frac{\pi I_0
	b^2}{qd} \right)^2 \sim \frac{\omega^2 b^2}{c^2}
\]
To get the approximation, we use \( I_0 = q \omega \) and use \( b \sim d \), because we assume the electric
and magnetic sources are on the same scale. Now, because we've assumed earlier that \( b \ll \frac{c}{\omega}
\), this implies \( \omega b / c \ll 1 \), hence \( P_m \ll P_e \). This shows that the electric power
dominates the magnetic power under our assumption, and this explains why we only focused on \( \mathbf{E} \)
waves in chapter 9.



