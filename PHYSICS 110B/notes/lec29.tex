\section{April 11}
\subsection{Radiation for an Arbitrary Source Distribution}
Now that we've looked at radiation for specific source configurations, we will now look at how to calculate
the radiation emitted by an arbitrary source distribution. Consider an arbitrary ball of sources, for which
we want to find the field \( V(\mathbf{r}, t) \) everywhere. Well, the equations don't change, so:
\[
	V(\mathbf{r}, t) = \frac{1}{4\pi \epsilon_0} \int \frac{\rho(\mathbf{r}, t_r)}{\rcurs} \diff \tau'
\]
Calculating this for a general source is usually hard, so we will make some simplifying approximations.
First, because the radiation field dominates at large \( r \), we will make the approximation that \( r \gg
r' \), so that the \( \frac{1}{r^2} \) terms have already died out. With this assumption, we can approximate
\( \rcurs \):
\[
	\rcurs = \sqrt{r^2 + r'^2 + 2 \mathbf{r} \cdot \mathbf{r}'} \simeq r \left( 1 - \frac{\mathbf{r} \cdot
	\mathbf{r}'}{r^2} \right)
\]
Using the approximation that \( (1 + x)^{n} \approx 1 + nx \) when \( x \) is small (and indeed, \(
\mathbf{r}\cdot \mathbf{r}' / r^2 \) is small since the denominator is quadratic in \( r \)), then we can
write \( 1 / \rcurs \):
\[
	\frac{1}{\rcurs} = \frac{1}{r}\left( 1 + \frac{\mathbf{r} \cdot \mathbf{r}'}{r^2} \right)
\]
So, we can now Taylor expand \( \rho \):
\[
	\rho(\mathbf{r}, t - \rcurs / c) = \rho\left( \mathbf{r}', t - \frac{r}{c} + \frac{\mathbf{r} \cdot
	\mathbf{r}'}{c} \right) \approx \rho(\mathbf{r}, t_0) + \dot \rho(\mathbf{r}, t_0) \left(
	\frac{\mathbf{\hat{r}} \cdot \mathbf{r}'}{c} \right) + \frac{1}{2} \ddot \rho (\mathbf{r}', t_0) \left(
	\frac{\mathbf{\hat{r}} \cdot \mathbf{r}'}{c} \right)^2
\]
Now, we impose our second assumption, this one being on \( \dot \rho \) and the higher derivative terms. We
will assume that the time variation must be fast enough. This is enforced through the ratios:
\[
	\left| \frac{\ddot \rho}{\dot \rho} \right|, \left| \frac{\dddot \rho}{\ddot \rho} \right|, \left|
	\frac{\rho^{4}}{\dddot \rho} \right|, \dots \ll \frac{c}{r'}
\]
You can essentially think of this as requiring that the wavelength of the waves is much larger than the
structure size, or mathematically \( \mathbf{r}' \ll cT \simeq \lambda \). Intuitively this also makes sense,
since high frequency waves die out and don't make it very far, and what's left are the low frequency waves
with large \( \lambda \). Practically speaking, what this approximation does is allow us to keep only the
first order \( \mathbf{r}' \) terms. Therefore, our \( V(\mathbf{r}, t) \) becomes:
\begin{align*}
	V(\mathbf{r}, t) &= \frac{1}{4\pi \epsilon_0} \frac{1}{r}\left[ \int \rho(\mathbf{r}, t_0) \diff \tau' +
	\frac{\mathbf{\hat{r}}}{r} \int \mathbf{r}' \rho(\mathbf{r}', t_0) \diff \tau' +
\frac{\mathbf{\hat{r}}}{c} \dv{t} \int \mathbf{r}' \rho(\mathbf{r}', t_0) \diff \tau' \right]\\
&= \frac{1}{4\pi \epsilon_0}\frac{Q}{r} + \frac{1}{4\pi \epsilon_0} \frac{\hat{\mathbf{r}} \cdot \mathbf{p}(t_0)}{r^2} 
+ \frac{1}{4\pi \epsilon_0} \frac{\hat{\mathbf{r}} \dot{\mathbf{p}}(t_0)}{cr}	
\end{align*}
For the second term, we use the fact that the electric dipole moment is defined as
\( \mathbf{p} = \int \mathbf{r}' \rho(\mathbf{r}') \diff \tau \) to simplify it. The first two terms should
be familiar: these are the multipole expansion terms, and the third one is due to radiation. The third term
ends up being the only term we care about, since when we calculate the gradient of the potential the third
term is the only one that produces a \( \frac{1}{r} \) dependence term. 
 
For the vector potential, we have the equation:
\[
	\mathbf{A} = \frac{\mu_0}{4\pi} \int \frac{\mathbf{J}(\mathbf{r}, t_r)}{\rcurs} \diff \tau
\]
Since \( J = \rho v \), computing an integral like \( \int \mathbf{J} \diff \tau \) is essentially the same
as summing over each individual charge:
\[
	\int \mathbf{J} \diff \tau \sim \sum_i q_i v_i = \sum_i q_i \dv{\mathbf{r}'_i}{t} = \dv{t} \sum_i q_i
	\mathbf{r}'_i \sim \dot{\mathbf{p}}(t_0)
\]
so we can loosely approximate such an integral as the time derivative of the dipole moment. Then, because \(
\int \mathbf{J} \diff \tau \) is already on the order of \( \mathbf{r}' \), any other higher order terms will
be second order corrections, and hence we can just take \( \frac{1}{\rcurs} \simeq \frac{1}{r} \). All in
all, the relevant potentials are:
\[
	V \simeq \frac{1}{4\pi \epsilon_0} \frac{\mathbf{r} \cdot \mathbf{p}(t_0)}{cr} \quad \mathbf{A} \simeq
	\frac{\mu_0}{4\pi} \frac{\dot{\mathbf{p}}(t_0)}{r}
\]
To find the field, we now take \( \mathbf{E} = -\nabla V - \partial_t A \) and \( \mathbf{B} = \nabla
\times \mathbf{A} \) as usual. Starting with \( \nabla V \), recall that the only term we care about in \( V
\) is the third term, so:
\[
	(\nabla V)_i = \frac{1}{4\pi \epsilon_0} \frac{1}{cr} \nabla \left[ \mathbf{\hat{r}} \cdot
	\dot{\mathbf{p}}(t_0) \right]
\]
Now the gradient:
\[
	\partial_i p^{j}(t_0) = (\partial_i t_0) \ddot p^{j}(t_0) = -\frac{1}{c} \hat{r}_i \ddot p^{j}(t_0)
\]
where we use \( \partial_j t_0 = -\frac{1}{c} \nabla r \). Now, written in vector form:
\[
	(\nabla V)_i = -\frac{1}{4\pi \epsilon_0} \frac{\mathbf{\hat{r}} \cdot \ddot{\mathbf{p}}(t_0)}{rc^2}
	\mathbf{\hat{r}} 
\]
The \( \partial_t \mathbf{A} \) term is easy:
\[
	\partial_t \mathbf{A} = \frac{\mu_0}{4\pi} \frac{\ddot{\mathbf{p}}(t_0)}{r}
\]
So putting these two together to get \( \mathbf{E} \):
\[
	\mathbf{E} = -\frac{1}{4\pi \epsilon_0} \frac{\mathbf{\hat{r}} \cdot
	\ddot{\mathbf{p}}(t_0)}{rc^2}\mathbf{\hat{r}} - \frac{\mu_0}{4\pi} \frac{\ddot{\mathbf{p}}(t_0)}{r} =
	\frac{\mu_0}{4\pi} \left[ \mathbf{\hat{r}} \times (\mathbf{\hat{r}} \times \ddot{\mathbf{p}}(t_0) \right]
\]
where we've used the vector triple product identity \( \mathbf{a} \times (\mathbf{b} \times \mathbf{c}) =
\mathbf{b}(\mathbf{a} \cdot \mathbf{c}) - \mathbf{c}(\mathbf{a} \cdot \mathbf{b}) \) to write it in its final
form. To find \( \mathbf{B} \), we use \( \mathbf{B} = \nabla \times \mathbf{A} \):
\[
	\mathbf{B} = \nabla \times \mathbf{A} = \frac{\mu_0}{4\pi} \epsilon^{ijk}\partial_j \dot p_k(t_0) =
	\frac{\mu_0}{4\pi} \epsilon^{ijk} (\partial_j t_0) \ddot p_k
\]
Again using \( \partial_j t_0 = -\frac{1}{c} \nabla r \), we have:
\[
	\mathbf{B} = \frac{\mu_0}{4\pi r} \epsilon^{ijk} \left( -\frac{1}{c} \hat{r}_j \right) \ddot p_k =
	-\frac{\mu_0}{4\pi cr} \left[ \mathbf{\hat{r}} \times \ddot{\mathbf{p}}(t_0) \right]
\]
Notice that indeed we have \( \mathbf{B} = \frac{1}{c}\mathbf{\hat{r}} \times \mathbf{E} \), you can check
this for yourself also if you'd like. 

\begin{example}
	Under the special case \( \mathbf{p} = p \mathbf{\hat{z}} \), \( \dot{\mathbf{p}} = \dot{p}
	\mathbf{\hat{z}} \), and \( \ddot{\mathbf{p}} = \ddot{p} \mathbf{\hat{z}} \) (i.e. one-dimensional motion),
	then the \( \mathbf{E} \) and \( \mathbf{B} \) fields take the form:
	\begin{align*}
		\mathbf{E} &= \frac{\mu_0 \ddot p(t_0)}{4\pi}\left( \frac{\sin \theta}{r} \right)
		\boldsymbol{\hat{\theta}} \\ 
		\mathbf{B} &= \frac{\mu_0 \ddot p(t_0)}{4\pi c}\left( \frac{\sin \theta}{r} \right)\boldsymbol{\hat{\phi}} 
	\end{align*}
	Combined, we can calculate \( \mathbf{S} \):
	\[
		\mathbf{S} = \frac{1}{\mu_0}(\mathbf{E} \times \mathbf{B}) = \frac{\mu_0 [\ddot{p}(t_0)]^2}{16 \pi^2
		c}\left( \frac{\sin \theta}{r} \right)^2 \mathbf{\hat{r}}
	\]
	So the power is:
	\[
		P = \int \mathbf{S} \cdot \diff \mathbf{a} = \frac{\mu_0 [\ddot{p}(t_0)]^2}{16 \pi^2 c} \int \frac{\sin^2
		\theta}{r^2} (r^2 \sin \theta) \diff \theta \diff \phi = \frac{\mu_0 [\ddot p(t_0)]^2}{6 \pi c}
	\]
	In the case of a point charge, we have \( \mathbf{p} = q \mathbf{d} \), and hence \( \ddot{\mathbf{p}} =
	q \ddot{\mathbf{a}} = qa \mathbf{\hat{z}} \), so the power becomes:
	\begin{equation}
		\label{Larmor}
		P = \frac{\mu_0 q^2 a^2}{6 \pi c}
	\end{equation}
	This is the well-known \textbf{Larmor Formula}, which gives the power radiated by a point charge. Notice
	that it only radiates power when it is accelerated. We will revisit this formula and derive it from a
	different perspective next week.
\end{example}

 
 
 
