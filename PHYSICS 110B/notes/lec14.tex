\section{February 24}

\subsection{Normal Incidence on a Conductor}
Last time, we stopped by considering EM waves in a conductor, now we will consider reflection and
transmission on a conductor, in the same way we did this for linear dielectrics. Consider the following case
of an electromagnetic wave \( \mathbf{E}_I \) on the boundary between a linear dielectric and a linear
conductor.   
\begin{center}
	\begin{tikzpicture}[scale=0.5]
		\draw[-stealth] (-4, 0) -- (4, 0) node[right] {\( z \)};
		\draw[-stealth] (0, -4) -- (0, 4) node[above] {\( x \)};
		\draw[stealth-stealth, orange] (-3, -1) -- (-3, 1) node[above] {\( \mathbf{E}_I \)}; 
		\draw (-3, 4) node {linear dielectric};
		\draw (3, 4) node {linear conductor};
	\end{tikzpicture}
\end{center}
By fitting boundary conditions, we can find that the transmitted wave has the same polarization as the
incident wave, and we also have the following quantities:
\[
	\tilde \beta = \left( \frac{\mu_1v_1}{\mu_2 \omega} \right)\tilde k_2 \quad \tilde E_{0R} = \left(
	\frac{1 - \tilde \beta}{ 1+ \tilde \beta} \right)\tilde E_{0I} \quad E_{0T} = \left( \frac{2}{1 + \tilde
\beta} \right)\tilde E_{0I}
\]
Recall also our value for \( \tilde k \):
\[
	k = a + ib = \frac{\omega}{v} \left[\sqrt{\frac{\sqrt{1 + \left( \frac{\sigma}{\epsilon \omega}
		\right)^2} + 1}{2} }+ i \sqrt{\frac{\sqrt{1 + \left( \frac{\sigma}{\epsilon \omega} \right)^2} -
1}{2}}\right]
\]
Yes it's ugly, but that's what it is. Note also that for a very good conductor (that is, for \( \sigma \to
\infty \)), then \( |\tilde \beta| \to \infty \) as well. In this case, we see that the ratio 
\[
	\lim_{\tilde \beta \to \infty} \frac{1 - \tilde \beta }{1 + \tilde \beta} = -1
\]
and this means that we get \( \tilde E_{0R} = - \tilde E_{0I} \), which amounts to picking up a \( \pi \)
phase shift on reflection. Likewise, we find that \( \tilde E_{0T} \to 0 \).

\subsection{Anomalous Dispersion}
Previously, we've considered a model where we have an incident electric field \( \mathbf{E} \), which under a
roughly quadratic potential, oscillates sinusoidally. The equation of motion for such a particle is:
\[
	\ddot x = -m \omega_0^2 x + \frac{q E_0}{m}\cos(\omega t)
\]
Given this type of motion, our goal in this section is to show that the index of refraction is given by:
\[
	n = 1 + \frac{q^2 N}{2 \epsilon_0 m} \frac{1}{\omega_0^2 - \omega^2}
\]
Here \( N \) represents the number of charges per volume. This function \( n(\omega) \) explains exactly why
materials like glass bend blue light more than red light, and is the phenomenon we call \textbf{anomalous
dispersion}. This phenomenon actually has a fairly simple explanation, and it has to do with absorption and
damping. By assumption, let's add a damping term to our equation of motion:
\[
m \ddot x = - m \omega_0^2 x - m \gamma \dot x + q E_0 \cos(\omega t)
\]
And here we will solve this by considering \( x(t) = \Re(\tilde x(t)) \). We will let our ansatz be \( \tilde
x(t) = \tilde x_0 e^{-i \omega t}\), which yields the equation:
\[
	(- m \omega^2 + i m \gamma \omega + m \omega_0^2) \tilde x_0 = q E_0
\]
This leads to the equation:
\[
	\tilde x_0 = \frac{q E_0}{m(\omega_0^2 - \omega^2) - i m \gamma \omega}
\]
so we've solved for the equation of motion. Now, as the charge oscillates, we get a dipole moment:
\[
	\mathbf{p} = q \tilde x(t) = \frac{q^2}{m} \frac{1}{(\omega_0^2 - \omega^2) - i \gamma \omega} E_0 e^{-i
	\omega t}
\]
and the total polarization can be written as :
\[
	\tilde{\mathbf{P}} = \tilde{\mathbf{p}} N = \frac{q^2N}{m}\frac{1}{(\omega_0^2 0 - \omega^2) - i \gamma
	\omega} E_0 e^{-i \omega t}
\]
For a linear dielectric, we have \( \tilde{\mathbf{P}} = \epsilon \tilde \chi_e \mathbf{E} \), so using the
above equation, we can deduce the electric susceptibility:
\[
	\tilde \chi_e = \frac{q^2 N}{\epsilon_0 m}\frac{1}{(\omega_0^2 - \omega^2) - i \gamma \omega}
\]
Then, the permittivity is:
\[
	\tilde \epsilon = \epsilon_0\left( 1 + \frac{q^2 N}{\epsilon_0 m} \frac{1}{(\omega_0^2 - \omega^2) - i
	\gamma \omega} \right)
\]
Now with \( \tilde \epsilon \) solved, recall the wave equation we have when \( \mu \approx \mu_0 \) and \(
\epsilon \approx \epsilon_0 \):
\[
	(\nabla^2 - \mu \tilde \epsilon \partial_t^2) \mathbf{E} = 0
\]
Now, with the ansatz \( \tilde{\mathbf{E}} = \tilde E_0 e^{-i(\tilde k z - \omega t)} \), we get the equation
\( -\tilde k^2 + \mu \tilde \epsilon \omega^2 = 0 \) so this implies the solution:
\[
	\tilde k = \sqrt{\mu_0 \tilde \epsilon} \omega = \sqrt{\mu_0 \epsilon_0} ( 1 + \chi_e)^{1 / 2} \omega
\]
Usually, \( \tilde \chi_e \ll 1  \), so we Taylor expand here using \( (1 + x)^{n} \approx 1 + nx \).
Therefore, \( \tilde k \) becomes:
\[
	\tilde k = \sqrt{\mu_0 \epsilon_0}\left( 1 + \frac{q^2 N}{2 \epsilon_0 m}\frac{1}{(\omega_0^2 - \omega^2)
	- i \gamma \omega} \right)\omega = \sqrt{\mu_0 \epsilon_0}\left( 1 + \frac{q^2N}{2 \epsilon_0
	m}\frac{(\omega_0^2 - \omega^2) + i \gamma \omega}{\omega_0^2 - \omega^2 + \gamma^2 \omega^2} \right)
\]
This last step then allows you to think of \( \tilde k = k + i \kappa \), as you can split this into a real
and imaginary part:
\[
	\tilde k = \sqrt{\mu_0 \epsilon_0}\omega \left[ \left( 1 + \frac{q^2 N}{2 \epsilon_0 m }\frac{\omega_0^2
	- \omega^2}{(\omega_0^2 - \omega^2) + \gamma^2 \omega^2} \right) + i \left(\frac{q^2N}{2 \epsilon_0
m}\frac{\gamma \omega}{(\omega_0^2 - \omega^2) + \gamma^2 \omega^2}\right) \right]
\]
So this gives the equation: \( \tilde{\mathbf{E}} = \tilde{\mathbf{E}_0} e^{-\kappa z} e^{-i(kz - \omega t)}
\). Then, the refractive index \( n = \frac{c}{v} = \frac{c}{\omega} k \), so substituting \( k \):
\[
	n = 1 + \frac{q^2N}{2 m \epsilon_0}\frac{\omega_0^2 - \omega^2}{(\omega_0^2 + \omega^2) + \gamma ^2
	\omega^2}
\]

