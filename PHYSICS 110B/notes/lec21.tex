\section{March 17}
Recall that last time, we talked about analytic functions, and we said that if \( f = f(z) \) is analytic on
a contour \( \mathcal{C} \) and the region inside \( \mathcal{C} \), then \( \oint_{\mathcal{C}}f(z) \diff z
= 0 \). Then, if \( f \) has a singularity within \( \mathcal{C} \), then 
\[
	\oint_{\mathcal{C}}f\left( z \right) \diff z = 2\pi i a_{-1}
\]
where \( a_{-1} \) is the coefficient of the inverse power \( (z - z_0)^{-1} \). Now, even though in the
previous example we used the contour of a circle, we can generalize this to any contour as well. Consider the
following contour:
\begin{center}
	\begin{tikzpicture}[decoration={markings,mark=at position 0.5 with {\arrow{stealth}}}, scale = 2]
		\draw[thick,-stealth] (0,-0.25) -- (0,2) node[right] {$\mathbb{C}$};
		\draw[thick,-stealth] (-0.25,0) -- (2,0);
		\draw[thick,green!70!blue!50,fill=blue!70!cyan!60,fill opacity=0.25] (1,0.5) circle (0.75cm)
			node[below,right,yshift=-0.75cm,xshift=1.3cm,opacity=1,scale=0.75] {\( \mathcal{C}_2 \)};
		\draw[thick,green!70!blue!50,fill=white] (1,0.5) circle (0.25cm) node[above=0.1,scale=0.75]
			{\( \mathcal{C}_1 \)};
		\filldraw[white] (1.125,0.45) rectangle (2,0.55);
		\draw[thick,green!70!blue!50,postaction={decorate}] (1.228,0.55) -- (1.76175,0.55)
			node[midway,above,scale=0.5] {\( -\mathcal{P} \)};
		\draw[thick,green!70!blue!50,postaction={decorate}] (1.76175,0.45) -- (1.228,0.45)
			node[midway,below,scale=0.5] {\( \mathcal{P} \)};
		\filldraw[green!70!blue!50] (1,0.5) circle (0.025cm);
\end{tikzpicture}
\end{center}
The diagram may be hard to see, but essentially the inner circle is \( \mathcal{C}_1 \), the outer curve is
\( \mathcal{C}_2 \), and we have a path \( \mathcal{P} \) that enters and exits. Assume that we make the gap
between \( \mathcal{P} \) and \( -\mathcal{P} \) to be infinitesimally small, so that the entire integral
approximates the contour \( \mathcal{C} \) as best as possible.
Here, the enclosed region of \( \mathcal{C} \) does not enclose any singularities, so we know already that \(
\oint_{\mathcal{C}} f(z) \diff z = 0 \). Now, if we expand the left hand side:
\[
	\oint_{\mathcal{C}}f(z) \diff z = \int_{\mathcal{C}_1} f(z) \diff z + \int_{\mathcal{C}_2} f(z) \diff z +
	\int_{\mathcal{P}} f(z) \diff z + \int_{-\mathcal{P}}f(z) \diff z
\]
the integrals over \( \mathcal{P} \) and \( -\mathcal{P} \) of course cancel, so we have:
\[
	2\pi i \Res[f(z_0)] = \int_{\mathcal{C}_2}f(z) \diff z
\]
In some sense this is actually very similar to Ampere's law. Recall that Ampere's law states \( \oint
\mathbf{B} \diff \boldsymbol{\ell} = \mu_0 I_{\text{enc}} \), so if we imagine a current going in the 
\( \mathbf{\hat{z}} \) direction a loop enclosing it in the \( xy \)-plane:
\begin{center}
	\begin{tikzpicture}
		\draw(-1, 0) -- (4, 0);
		\draw(0, -1) -- (0, 4);
		\filldraw (2, 2) circle (0.02);
		\draw (2, 2) circle (0.1) node [above right] {\( I \)};
		\draw[green!80!black] (2, 2) circle (0.8);
		\node[green!80!black] at (3, 2.5) {\( \mathcal{C} \)};
	\end{tikzpicture}
\end{center}
We know from 110A that here the \( \mathbf{B} \) field from the wire can be expressed as
\[
	B = \frac{\mu_0 I}{2 \pi r}
\]
(ignoring the vector comopnent for now), so if we deifne \( \tilde I = \frac{\mu_0 I}{2\pi} \), then we write
Ampere's law in this case as
\[
	\int \mathbf{B} \diff \boldsymbol{\ell} = 2\pi \tilde I
\]
So in this sense, we can think of the current flowing perpendicular as a kind of singularity. Of course, you
could extend this argument to having multiple currents, and likewise there's nothing stopping us from doing
the same in complex analysis, so in general:
\[
	\oint_{\mathcal{C}}f(z) \diff z = \sum_n 2\pi i \Res[f(z_n)]
\]
Now, coming back to physics, remember that our goal is to solve for Green's function \( G(t, \mathbf{r}) \)
for the Klein-Gordon equation \( (\nabla^2 - \partial_t^2)G(t, \mathbf{r}) = \delta^{(4)}(t, \mathbf{r}) \).
Moreover, we showed that:
\[
	G(t, \mathbf{r}) = \int \frac{d^{4}k}{(2\pi)^{4}} \frac{e^{i (-k^{0}t + \mathbf{k} \cdot
	\mathbf{r}})}{(k^{0})^2 - |\mathbf{k}|^2}
\]
Now, we can write the denominator as a difference of squares:
\[
	G(t, \mathbf{r}) = \int \frac{d^{4}k}{(2\pi)^{4}} e^{i(-k^{0} t + \mathbf{k} \cdot \mathbf{r})} \left[
	\frac{1}{((k^{0} - |\mathbf{k}|) (k^{0} + |\mathbf{k}|)} \right]	
\]
Clearly, we have singularities at \( k^{0} = \pm |\mathbf{k}| \). Using what we've learned from complex
analysis, integrating over \( k^{0} \in \R \) is the same as integrating over \( k^{0} \in \C \), 
but integrating over the real line only. This means we take the integral:
\begin{center}
	\begin{tikzpicture}[decoration = {markings, mark=at position 0.5 with {\arrow{>}}}, scale=0.5]
		\draw (-3, 0) -- (3, 0);
		\draw (0, -3) -- (0, 3);
		\filldraw[cyan!80!black] (-1, 0) circle (0.02) node[below] {\( -|\mathbf{k}| \)};
		\filldraw[cyan!80!black] (1, 0) circle (0.02) node[below] {\( |\mathbf{k}| \)};
		\draw[green!80!black, postaction=decorate] (-3, 0) -- (-1, 0); 
		\draw[green!80!black, postaction=decorate] (1, 0) -- (3, 0);
		\draw[green!80!black, postaction=decorate] (-1, 0) -- (1, 0);
	\end{tikzpicture}
\end{center}
Clearly we can't do this immediately because of the singularities at \( \pm |\mathbf{k}| \). Further, because
the residue theorem requires a closed loop, we can't immediately apply that either because the line is not
closed. So, our strategy is to basically integrate over a loop still, but make the extra part we add
contribute nothing to the overall integral. We can do that as follows: for \( t > 0 \), we can use the loop:
\begin{center}
	\begin{tikzpicture}[decoration = {markings, mark=at position 0.5 with {\arrow{>}}}, scale = 0.5]
		\draw (-3, 0) -- (3, 0);
		\draw (0, -3) -- (0, 3);
		\draw[green!80!black, postaction=decorate] (-3, 0) -- (0, 0);
		\draw[green!80!black, postaction=decorate] (0, 0) -- (3, 0);
		\draw[green!80!black, postaction=decorate] (3, 0) arc [start angle = 0, end angle = -180, radius = 3];
	\end{tikzpicture}
\end{center}
The idea is basically to close the loop by using a very large semicircular arc, which can be parametrized as:
\[
	e^{-i(\Re(k^{0}) + i \Im(k^{0}))t} = e^{-i\Re(k^{0})t} e^{\Im(k^{0})t}
\]
when \( \Im(k^{0}) < 0 \) and \( t > 0 \), then the factor from this contour exponentially decays away, due
to the factor of \( e^{\Im(k^{0})t} \). So, we've successfully create a loop where the extra contribution
doesn't matter at all. Similarly, for \( t < 0 \), we can use the other half circle with the same purpose:  
\begin{center}
	\begin{tikzpicture}[decoration = {markings, mark=at position 0.5 with {\arrow{>}}}, scale = 0.5]
		\draw (-3, 0) -- (3, 0);
		\draw (0, -3) -- (0, 3);
		\draw[green!80!black, postaction=decorate] (-3, 0) -- (0, 0);
		\draw[green!80!black, postaction=decorate] (0, 0) -- (3, 0);
		\draw[green!80!black, postaction=decorate] (3, 0) arc [start angle = 0, end angle = 180, radius = 3];
	\end{tikzpicture}
\end{center}
Finally, we can deal with the singularities themselves. Because we aren't allowed to walk over them directly,
our strategy will basically be to "walk around" them:
\begin{center}
	\begin{tikzpicture}[decoration = {markings, mark=at position 0.7 with {\arrow{>}}}]
		\draw (-3, 0) -- (3, 0);
		\draw (0, -3) -- (0, 3);
		\filldraw[green!80!black] (-1, 0) circle (0.02) node[above] {\( -|\mathbf{k}| \)};
		\filldraw[green!80!black] (1, 0) circle (0.02) node[below] {\( |\mathbf{k}| \)};
		\draw[cyan!80!black, postaction=decorate] (-3, 0) -- (-1.2, 0);
		\draw[cyan!80!black, postaction=decorate] (-1.2, 0) arc [start angle = 180, end angle = 360, radius = 0.2];
		\draw[cyan!80!black, postaction=decorate] (-0.8, 0) -- (0.8, 0);
		\draw [cyan!80!black, postaction=decorate] (0.8, 0) arc [start angle = 180, end angle = 0, radius =
			0.2];
		\draw[cyan!80!black, postaction=decorate] (1.2, 0) -- (3, 0);
	\end{tikzpicture}
\end{center}
so now our integral can be decomposed into three parts: the straight parts, and the two loops, all of which
are well defined. Introducing these loops also begs another question: how should we go about choosing whether
we go above or below them? There are 4 ways in total that we can go over both singularities, so how do we
know the path we've chosen is the "correct" one? The answer to this question depends on the physics of the
system: in our case, since \( \delta^{(4)}(t, \mathbf{r}) \) has a spike at \( t = 0 \), then we expect that
for \( t < 0 \), our integral should evaluate to zero and for \( t > 0 \) we expect a nonzero contribution.
Therefore, for \( t < 0 \), we should choose a method that doesn't enclose the singularities at all.
By that same token, we should choose the contour that encloses both singularities in the \( t > 0 \) case.   


 


