\section{February 26}

With anomalous dispersion out the way, we now turn to the last topic of this section, wave guides. 

\subsection{Wave Guides}
First, consider the setup of an electromagnetic wave propagating through a conducting pipe (see figure
below). The idea of this setup is that waves are confined in the \( xy \)-plane, but propagation is allowed
in the \( \mathbf{\hat{z}} \) direction. With this setup, we naturally think of waves:
\begin{align*}
	\mathbf{E}(x, y, z, t) &= \tilde{\mathbf{E}}_0 (x, y) e^{i(kz - \omega t)}\\
	\mathbf{B}(x, y, z, t) &= \tilde{\mathbf{B}}_0 e^{i(kz - \omega t)} 
\end{align*}
The interior of the wave guide is vacuum, so Maxwell's equations in a vacuum (\( \rho = 0, \mathbf{J} = 0
\))	hold. Now for the boundary conditions. The first boundary condition is given by the properties of a
conductor. We know that \( \mathbf{E} = 0 \) inside a conductor, so by Faraday's law we have 
\( \partial_t \mathbf{B} = 0 \) inside, so \( \mathbf{B} \) is constant. For convenience, we will just choose
the \( \mathbf{B} = 0 \) inside the conductor, since a constant \( \mathbf{B} \) field can always just be
removed with no issues.\footnote{Technically, this follows from the fact that Maxwell's equations are
\textit{linear}, so derivatives of constants vanish.} So now, we have the same boundary conditions as we had
for a conductor:
\begin{align*}
	\epsilon_1 E_1^{\perp} - \epsilon_2 E_2^{\perp} &= \sigma_f\\
	E_1^{\parallel} &= E_2^{\parallel}\\
	B_1^{\perp} &= B_2^{\perp}\\
	\frac{1}{\mu_1}B_1^{\parallel} - \frac{1}{\mu_2}B_2^{\parallel} &= \mathbf{K}_f \times \mathbf{\hat{n}}
\end{align*}
Because of our boundary conditions of \( \mathbf{E} = 0 \) and \( \mathbf{B} = 0 \) inside the conductor, it
also makes sense for continuity's sake that \( E^{\parallel} = 0  \) and \( \mathbf{B}^{\perp} = 0 \) inside
the waveguide. These are actually the only two conditions we care about. The surface currents \( \mathbf{K}_f
\) and free charges \( \sigma_f \) will arrange themselves in a way such that these boundary conditions are
true. We can then start with the ansatz of a travelling wave:
\[
	\mathbf{E} = \tilde{\mathbf{E}}_0(x, y)e^{i(kz - \omega t)} \quad \mathbf{B} = \tilde{\mathbf{B}}_0(x, y) e^{i(kz
	- \omega t)}
\]
Now, with the ansatz in place, we invoke Faraday's law, using the \( y \) direction as an example:
\( (\nabla \times \mathbf{E})_y = -\partial_t B_y \), this gives the equations:
\[
	ik \tilde E_x - k \partial \tilde E_z = ik \omega \tilde B_y
\]
Similarly, Ampere-maxwell also gives us an equation relating the electric and magnetic fields, this time we
compute \( (\nabla \times \mathbf{B})_x = \frac{1}{c^2}\partial_t E_x \):
\[
	\omega \partial_y \tilde B_z - i k \omega \tilde B_y = -i \frac{\omega^2}{c^2}\tilde E_x
\]
Combining the two equations, we get:
\[
	\omega \partial_y \tilde B_z - (ik^2 \tilde E_x - \partial_x k \tilde E_z) = -i
	\frac{\omega^2}{c^2}\tilde E_x \implies \tilde E_x = \frac{i}{\left( \frac{\omega}{c} \right)^2 - k^2}(k
	\partial_x \tilde E_z - \omega \partial_y \tilde B_z)
\]
In a similar fashion, we can also extract the other components:
\begin{align*}
	\tilde E_y &= \frac{i}{\left( \frac{\omega}{c} \right)^2 - k^2} (k \partial_y \tilde E_z - \omega
	\partial_x \tilde B_z)\\
	\tilde B_x &= \frac{i}{\left( \frac{\omega}{c} \right)^2 - k^2}(k \partial_x B_z -
	\frac{\omega}{c^2}\partial_y \tilde E_z)\\
	\tilde B_y &= \frac{i}{\left( \frac{\omega}{c} \right)^2 - k^2} (k \partial_y \tilde B_z +
	\frac{\omega}{c^2}\partial_y \tilde E_z)
\end{align*}
The point of these equations is to show that in a wave guide, as long as \( \tilde E_z \) and \( \tilde B_z \) 
are determined, then the whole field is determined. Further, if you apply Gauss's law to \( \nabla \cdot
\mathbf{E} = 0 \), then you get \( \partial_x \tilde E_x + \partial_y \tilde E_y + ik \tilde E_z = 0 \).
Putting these equations together, we get the wave equation:
\[
	\left[ \partial_x^2 + \partial_y^2 + \left( \frac{\omega}{c} \right)^2 - k^2 \right]\tilde E_z = 0
\]
You can do the same thing for \( \mathbf{B} \) using \( \nabla \cdot \mathbf{B} = 0  \):
\[
	\left[ \partial_x^2 + \partial_y^2 + \left( \frac{\omega}{c} \right)^2 - k^2 \right]\tilde B_z = 0
\]
\subsection{Wave Modes}
Since \( E_z \) and \( \tilde B_z \) essentially determine the entire wave, we can classify waves into three
types:
\begin{enumerate}[label=\arabic*.]
	\item TE mode: Transverse \( \mathbf{E} \) wave, so \( \tilde E_z = 0 \)
	\item TM mode: Transverse \( \mathbf{B} \) wave, so \( \tilde B_z = 0 \)
	\item TEM mode: \( \tilde E_z = 0 \) and \( \tilde B_z = 0 \). 
\end{enumerate}
A comment about the TEM mode though: in a single wave guide like this, a TEM mode cannot exist. This is
because if \( \tilde E_z = 0 \), then \( \nabla \cdot \mathbf{E} = 0 \) so \( \mathbf{E} \) is
divergenceless, but by \( (\nabla \times \mathbf{E})_z = -\partial_t B_z = 0 \) we have that 
\( \mathbf{E} \) must also be curl-less. Combined with the fact that \( \mathbf{E} = - \nabla V \), then this
implies that the only valid solution to these equations is \( V = 0 \), or basically there is no wave inside
the wave guide.   


