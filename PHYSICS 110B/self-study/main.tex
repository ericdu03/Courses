\documentclass[10pt]{article}
\usepackage{physics}
\usepackage[tcbtheorems, narrowmargins]{../../preamble}
\urlstyle{same}

\newcommand{\classcode}{Physics 110B}
\newcommand{\classname}{Electrodynamics and Optics II}
\renewcommand{\maketitle}{%
\hrule height4pt
\large{Eric Du \hfill \classcode}
\newline
\large{Notes} \Large{\hfill \classname \hfill} \large{\today}
\hrule height4pt \vskip .7em
\small{Header styling inspired by CS 70: \url{https://www.eecs70.org/}}
\normalsize
}
\linespread{1.2}
\begin{document}
	\maketitle
	\setcounter{section}{7}
	\section{Conservation Laws}
	\subsection{Charge and Energy}
	\subsubsection{The Continuity Equation}
	Local conservation of charge is derived by starting from the equation for the total charge:
	\[
		Q(t) = \int_{\mathcal{V}} \rho(\mathbf{r}, t) \diff \tau
	\]
	and the total loss (or gain) of charge through the volume is given by the current density through
	the surface of \( \mathcal{V} \):
	\[
		\dv{Q}{t} = -\oint_{\mathcal{S}} \mathbf{J} \cdot \diff \mathbf{a}
	\]
	We can then take the derivative of the first equation:
	\[
		\dv{Q}{t} = \dv{t} \int_{\mathcal{V}} \rho(\mathbf{r}, t) \diff \tau = \int_{\rho}
		\pdv{\rho}{t} \diff \tau = -\int_{\mathcal{V}}\nabla \cdot \mathbf{J} \diff \mathbf{a}
	\]
	This allows us to come to the continuity equation:
	\[
		\pdv{\rho}{t} = -\nabla \cdot \mathbf{J}
	\]
	\subsubsection{Poynting's Theorem}
	The total energy stored in electromagnetic fields is given by the sum of the integrands to
	calculate the work required to arrange charges and generate currents:
	\[
		W_e = \frac{\epsilon_0}{2} \int E^2 \diff \tau \quad W_m = \frac{1}{2 \mu_0} \int B^2 \diff
		\tau
	\]
	So, the total energy is expressed as:
	\[
		U = \frac{1}{2}\left( \epsilon_0 E^2 + \frac{1}{\mu_0}B^2 \right)
	\]
	Suppose you have a charge with some fields \( \mathbf{E} \) and \( \mathbf{B} \). How much work
	are these forces doing on the charges over an interval \( dt \)? The force on a charge \( q \) is
	given by:
	\[
		\mathbf{F} \cdot \mathbf{dl} = q \mathbf{E} \cdot \mathbf{v} \diff t
	\]
	so, in terms of densities, we have \( q \to \rho \diff\tau \) and then we combine this with \(
	\mathbf{v} \) to get \( \rho \mathbf{v} \to \mathbf{J} \), so within a volume \( \mathcal{V} \),
	we have:
	\[
		\dv{W}{t} = \int_{\mathcal{V}} (\mathbf{E} \cdot \mathbf{J}) \diff\tau
	\]
	so, the work done per unit time is \( \mathbf{E} \cdot \mathbf{J} \), which we can use Maxwell's
	equations to eliminate \( \mathbf{J} \) in favor of \( \mathbf{E} \) and \( \mathbf{B} \):
	\[
		\mathbf{E} \cdot \mathbf{J} = \frac{1}{\mu_0} \mathbf{E} \cdot (\nabla \times \mathbf{B}) -
		\epsilon_0 \mathbf{E} \cdot \pdv{\mathbf{E}}{t}
	\]
	We can then combine this with Faraday's laws and some product rules, finally giving us:
	\[
		\dv{W}{t} = -\dv{t} \int_{\mathcal{V}} \frac{1}{2}\left( \epsilon_0 E^2 + \frac{1}{\mu_0}B^2 \right)
		\diff \tau - \frac{1}{\mu_0} \oint_{\mathcal{S}} (\mathbf{E} \times \mathbf{B}) \cdot \diff \mathbf{a}
	\]
	This is Poynting's theorem, which is essentially the "work energy theorem" of electrodynamics. The first
	integral represents the energy stored in the fields, and the second integral represents the energy
	flowing along the boundary of \( \mathcal{V} \) (notice also that it can be rewritten as \( \int u \diff
	\tau\)). This second term is important, because it tells us of
	the "flow" of energy in and out of \( \mathcal{V} \), and the direction of this vector tells us the
	direction of energy flow. This is called the Poynting vector \( \mathbf{S} \equiv
	\frac{1}{\mu_0}(\mathbf{E} \times \mathbf{B}) \).        

	In the case where no work is done, then we have: 
	\[
		\int_{\mathcal{V}} \pdv{u}{t} \diff \tau = -\oint \mathbf{S} \cdot \diff \mathbf{a} =
		-\int_{\mathcal{V}} (\nabla \cdot \mathbf{S}) \diff \tau
	\]
	so, we have the energy continuity equation coming out:
	\[
		\pdv{u}{t} = -\nabla \cdot \mathbf{S}
	\]
	\question{There is a remark about "electromagnetic energy" is not conserved, but all of that is captured
	in \( \mathbf{S} \) right?} 
	\subsection{Momentum}
	\subsubsection{Newton's Third Law}
	In electrodynamics, Newton's third law is not directly conserved when considering charges only. Consider
	the case of two charges \( q_1, q_2 \) moving along the \( x \) and \( y \) axes respectively, 
	which are moving toward the origin \((0, 0)\). The electrostatic force is directly repulsive, and pushes
	the two charges away equally. On \( q_1 \), the magnetic field produced by \( q_2 \) points out of the
	page, so the force on \( q_1 \) is upward; on \( q_2 \) a similar logic applies, and the force is exerted
	to the right. While the magnitude of the forces are equal, the direction is not, which is a violation of
	Newton's third law. In electrodynamics, Newton's third law does \textbf{not} hold.   

	If the third law does not hold, what happens to conservation of momentum? Ordinarily, conservation of
	momentum is also violated, until you remedy it by allowing the fields to carry momentum, under which we
	restore the conservation.    

	\subsubsection{Maxwell Stress Tensor}
	With a volume \( \mathcal{V} \) of charge, it is possible to calculate the total force on the charges:
	\[
		\mathbf{F} = \int_{\mathcal{V}}(\rho \mathbf{E} + \mathbf{J} \times \mathbf{B}) \diff \tau
	\]
	per unit volume, we can write \( \mathbf{f} = \rho \mathbf{E} + \mathbf{J} \times \mathbf{B} \), which is
	the integrand. Now, we can write this entirely in terms of the fields, and using clever vector calculus
	we end up with the equation: 
	\[
		\mathbf{f} = \epsilon_0\left[ (\nabla \cdot \mathbf{E}) \mathbf{E} + (\mathbf{E} \cdot \nabla)\mathbf{E}
		\right]  + \frac{1}{\mu_0}\left[ (\nabla \cdot \mathbf{B})\mathbf{B} + (\mathbf{B} \cdot
		\nabla)\mathbf{B} \right] - \frac{1}{2}\nabla\left( \epsilon_0 E^2 + \frac{1}{\mu_0}B^2 \right) -
		\epsilon_0 \pdv{t}(\mathbf{E} \times \mathbf{B})
	\]
	This expression can be made simpler by introducing the maxwell stress tensor, which takes care of the
	first two terms:
	\[
		T_{ij} \equiv \epsilon_0 \left( E_i E_j - \frac{1}{2}\delta_{ij}E^2 \right) + \frac{1}{\mu_0}\left(
		B_iB_j - \frac{1}{2}\delta_{ij}B^2 \right)
	\]
	The indices \( i,j  \) refer to the \( x,y \) and \( z \) directions, so:
	\begin{align*}
		T_{x x} &= \frac{1}{2}\epsilon_0\left( E_x^2 - E_y^2 - E_z^2 \right) + \frac{1}{2}\mu_0\left( B_x^2 -
		B_y^2 - B_z^2\right)\\
		T_{xy} &= \epsilon_0(E_x E_y) + \frac{1}{\mu_0}(B_x B_y)
	\end{align*}
	Note that \( T_{x x} \) has the \( E_y \) and \( E_z \) terms because the second term in both parentheses
	is \( E^2 = E_x^2 + E_y^2 + E_z^2\) and likewise for \( B \). The tensor behaves like a generalized
	vector, so you can take the dot product with a vector. However, the difference lies in that with a
	tensor, it is possible to take the dot product on the left and the right:
	\[
		\left(\mathbf{a} \cdot \mathbf{T}\right)_j = \sum_{i}a_i T_{ij} \quad \left( \mathbf{T} \cdot
		\mathbf{a} \right)_j = \sum_i T_{ji}a_i
	\]
	These two need not be the same; in fact, most of the time they are not the same, but they are both
	vectors nonetheless. Because you can take the dot product, you can also take its divergence (and curl
	too). The divergence of the Maxwell stress tensor equals the first three terms, and we can use the
	Poynting vector to simplify the last term:
	\[
		\mathbf{f} = \nabla \cdot \mathbf{T} - \epsilon_0\mu_0 \pdv{\mathbf{S}}{t}
	\]
	Thus, the total electromagnetic force can be written as:
	\[
		\mathbf{F} = \oint_{\mathcal{S}} \mathbf{T} \cdot \diff \mathbf{a} - \epsilon_0 \mu_0 \dv{t}
		\int_{\mathcal{V}}\mathbf{S} \diff \tau
	\]
	In statics, there is no flow of charge, so \( \mathbf{F} \) becomes:
	\[
		\mathbf{F} = \oint_{\mathcal{S}}\mathbf{T} \cdot \diff \mathbf{a}
	\]
	In this case, you can see why we call it a stress tensor: the stress tensor is effectively a
	representation of the internal forces (or stress) on the current charge configuration. \( T_{ij} \)
	represents the force in the \( i \)-th direction acting on a surface oriented in the \( j \)th direction;
	the diagonal elements \( T_{x x}, T_{yy}, T_{zz} \) are pressures, the other elements are shear forces,
	exactly analogous to the stress tensor in a solid object. 

	\begin{example}
		Determine the net force on the "northern" hemisphere of a uniformly charged solid sphere of radius \(
		R\) and charge \( Q \).

		\begin{solution}
			There are many ways to approach this problem: you can compute the stress tensor using a surface
			that is the hemisphere itself, but this requires quite a bit of work. Alternatively -- and here's
			the key point -- \textit{any} volume that encloses all of the charge in the system will work as a
			volume. Thus, you can also choose the region of \( z > 0 \), which allows for some
			simplifications since \( E = 0 \) at infinity. Precisely, the only boundary we need to care about
			is the \( xy \) plane, since all other boundaries are at infinity where there is zero
			contribution.

			From Gauss's law, we deduce:
			\[
				\mathbf{E} = \frac{1}{4 \pi \epsilon_0}\frac{Q}{R^2} \hat{\mathbf{r}}
			\]
			There is no \( B \) field (since this is static, and no external fields are present), so we have:
			\begin{align*}
				T_{zx} &= \epsilon_0 E_z E_x = \epsilon_0\left( \frac{Q}{4 \pi \epsilon_0 R^2} \right)^2 \sin
				\theta \cos \theta \cos \phi \\ 
				T_{zy} &= \epsilon_0 E_z E_y \left( \frac{Q}{4 \pi \epsilon_0 R^2} \right)^2 \sin \theta \cos
				\theta \sin \phi\\ 
				T_{zz} &=  \frac{\epsilon_0}{2}\left( E_z^2 - E_x^2 - E_y^2 \right) =
				\frac{\epsilon_0}{2}\left( \frac{Q}{4 \pi \epsilon_0 R^2} \right)^2(\cos^2 \theta - \sin^2
				\theta) 
			\end{align*}
			Along the \( xy \)-plane, there are two regions we need to consider: within the disk, and outside
			the disk. In both cases, due to symmetry, the force is in the \( z \) direction, and the Maxwell
			stress tensor gives:
			\[
				T_{zz} = -\frac{\epsilon_0}{2}\left( \frac{Q}{4\pi \epsilon_0} \right)^2 \frac{1}{r^{4}}
			\]
			Here, we know it's in the \( z \) direction so we take the previous equation and set \( \theta =
			\pi / 2\). We then have \( d \mathbf{a} = -r \diff r \diff \phi \mathbf{\hat{z}}\) for a disk, so
			therefore:
			\[
				\left( \mathbf{T} \cdot \diff \mathbf{a} \right)_z = \frac{\epsilon_0}{2}\left( \frac{Q}{4
				\pi \epsilon_0} \right)^2 \frac{1}{r^3} \diff r \diff \phi
			\]
			The contribution is then this integral over the interval \( \phi \in [0, 2\pi] \) and \( R \in
			[R, \infty) \), so:
			\[
				\frac{\epsilon_0}{2}\left( \frac{Q}{4\pi \epsilon_0} \right)^2 2\pi
				\int_R^{\infty}\frac{1}{r^3} \diff r = \frac{1}{4 \pi \epsilon_0}\frac{Q^2}{8R^2}
			\]
			For the interior of the disk, we have the same integral except we have \( R\in [0, R] \), so:
			\[
				F_\text{disk} = \frac{1}{4\pi \epsilon_0}\frac{Q^2}{16R^2}
			\]
			And therefore the combined force is given by:
			\[
				F = \frac{1}{4\pi \epsilon_0}\frac{3Q^2}{16R^2}
			\]
			There are two things to take away from this problem: one, even though we were computing the force
			on a solid object, the stress tensor with a surface integral was able to determine the internal
			forces without ever having to compute a volume integral. Second, that we don't necessarily need
			to choose the volume which exactly describes the charge configuration; any volume that encloses
			all the desired charge will suffice. 
		\end{solution}
	\end{example}

	\subsubsection{Conservation of Momentum}
	From the equation of the total force, from Newton's second law we can write:
	\begin{equation}\label{pmech}
		\dv{\mathbf{p}_\text{mech}}{t} = - \epsilon_0 \mu_0 \dv{t} \int_{\mathcal{V}} \mathbf{S} \diff \tau +
		\oint_{\mathcal{S}} \mathbf{T} \cdot \diff \mathbf{a}
	\end{equation}
	Because the left hand side is a representation of the momentum change over time, the right hand side must
	as well. In particular, the first term can be thought of as the momentum stored in the fields (here we
	have a derivative because the left hand side has one), and the second term is the momentum flowing in and
	out from the surface of \( \mathcal{V} \). This statement allows for momentum to be stored inside the
	fields (the first term), and is what allows the restoration of Newton's third law, as mentioned from
	earlier. 

	With this equation, it is also possible to write down the momentum density, written as
	\[
		\mathbf{g} = \mu_0 \epsilon_0 \mathbf{S} = \epsilon_0(\mathbf{E} \times \mathbf{B})
	\]
	and the momentum flux is given by \( \mathbf{T} \cdot \diff \mathbf{a} \). It's the same situation as
	with all fields. As such, if we were to simply represent the flow in and out of a surface, we may write:
	\[
		\pdv{\mathbf{g}}{t} = \nabla \cdot \mathbf{T}
	\]
	which is the continuity equation in electrodynamics. 

	\begin{example}
		A long coaxial cable, of length \( l \), consists of an inner conductor (radius \( a \)) and an outer
		conductor (radius \( b \)). It is connected to a battery at one end and a resistor at the other. The
		inner conductor carries a uniform charge per unit length \( \lambda \), and a steady current \( I \)
		to the right; the outer conductor has the opposite charge and current. What is the electromagnetic
		momentum stored in the fields?

		\begin{solution}
			Using Gauss' and Ampere's law, we have:
			\[
				\mathbf{E} = \frac{1}{2 \pi \epsilon_0} \frac{\lambda}{s} \hat{\mathbf{s}} \quad 
				\mathbf{B} = \frac{\mu_0}{2\pi}\frac{I}{s}\hat{\boldsymbol{\phi}}
			\]
			The poynting vector \( S = \frac{1}{\mu_0}(\mathbf{E} \times \mathbf{B}) \), so:
			\[
				\mathbf{S} = \frac{\lambda I}{4 \pi^2 \epsilon_0 s^2}\mathbf{\hat{z}}
			\]
			The momentum in the fields, which we can calculate as \( \mathbf{p} = \mu_0 \epsilon_0 \int
			\mathbf{S} \diff \tau \) (the first integral in eq. \ref{pmech}), gives:
			\[
				\mathbf{p} = \mu_0 \epsilon_0 \int \mathbf{S} \diff \tau = \frac{\mu_0 \lambda I}{4\pi^2}
				\mathbf{\hat{z}}\int_a^b \frac{1}{s^2}l 2\pi s \diff s = \frac{IVl}{c^2}\mathbf{\hat{z}}
			\]
			So there is momentum in the fields, even though the fields are static! The momentum here comes
			purely from the transfer of energy from the battery to the resistor, and the flow of energy,
			according to relativity, carries momentum. This momentum is then stored in the fields, which is
			perfectly allowed despite the fields being static.    
		\end{solution}
	\end{example}
\end{document}




