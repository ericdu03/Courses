\section{Subsets}
Let's look back at what we've defined so far: 
	\begin{itemize}
		\item Sets, fields, vector spaces: these are groups of objects that follow a certain pattern, 
			becoming increasingly specific.
		\item We also defined the notion of a subset: given two sets \( T_1 \subset S \) and \( T_2 \subset S \), 
			then we also investigated the union \( T_1 \cup T_2 \) and intersection \( T_1 \cap T_2 \) 
			of \( T_1 \) and \( T_2 \). 
		\item Within these definitions, we've implicitly defined the idea of a "sub" -- the idea that 
			a set can be contianed within another set.
	\end{itemize}
Sets on their own have no real structure, so it's fairly difficult to really study them. Instead, 
we will study sets with \textit{some} structure, such as a field. How would we define subsets for a
field \( F \)? Well, we'd define them to have the same structure as \( F \): being euipped with two 
operations \( +, \cdot \), and that it's closed under these operations. 

An example of a subfield is the reals: \( \R \subset \C \). Complex numbers are of the 
form \( a + bi \), whereas reals are written as \( a + 0i \), a subset of \( \C \). The reals 
also satisfies the notion of a field, since the sum and product of two real numbers is also a 
real number. 

Our interest is to define the notion of a \textit{subspace}, for vector spaces. It's defined as follows:
\begin{definition}
	Let \( V \) be a vector space over \( F \). A subset \( U \subset V \) is called a \textit{subspace} of 
	\( V \) if it is preserved by the operations of addition and scalar multiplication.
\end{definition}
Explicitly, this means:
\begin{align*}
	\forall v, w, \in U, v + w \in U\\
	\forall v \in U, \lambda \in F, \lambda v \in U
\end{align*}
Consequently, this means that \( U \) is also a vector space over \( F \) (the same \( F \)), with respect 
to these operations. For instance, first the set of real tuples 
\( \R^2 = \{(x_1, x_2) \vert x_1, x_2 \in \R \}  \). Now, consider the set \( U = (x_1, 0) \vert x_1 \in \R\). 
It's easy to show \( U \) is preserved under addition and scalar multiplication, then \( U \) is indeed a vector space.
Further, we can also show that \( U \subset  R \), so therefore \( U  \) is a valid vector subspace of \( \R^2 \). 

Also, trivially, given a vector space \( V \), then \( V \) is also its own subspace. 

\begin{theorem}
	A subset \( U \subset V \) (\( V \) is the same as defined earlier) is a subspace if and only if the following 
	three conditions are satisfied:
	\begin{enumerate}[label=\alph*)]
		\item The zero element \( 0 \in V \) is also in \( U \). 
		\item For any \( u,v \in U \), then \( u + v \in U \). 
		\item For all \( v \in U \) and \( \lambda \in F \), then \( \lambda v \in U \). 
	\end{enumerate}
\end{theorem}
How is this statement different from the definition stated earlier? The latter two conditions are the same, but the 
first condition of the zero element is the distinction. Earlier, to prove a vector subspace, what we'd have to do is 
prove its closedness, then prove all six axioms of a vector space. However, we no longer need to do all that -- all we 
need now is just to show that the zero element exists within \( U \).
