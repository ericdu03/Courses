\documentclass[10pt]{article}
\usepackage{../../local}
\urlstyle{same}

\newcommand{\classcode}{Math 110}
\newcommand{\classname}{Solutions to Selected Axler Problems}
\renewcommand{\maketitle}{%
\hrule height4pt
\large{Eric Du \hfill \classcode}
\newline
Prof. Edward Frenkel \Large{\hfill \classname \hfill} \large{\today}
\hrule height4pt \vskip .7em
\small{Header styling inspired by CS 70: \url{https://www.eecs70.org/}}
\normalsize
}
\linespread{1.2}
\renewcommand{\R}{\mathbf R}
\newcommand{\F}{\mathbf F}
\newcommand{\range}{\mathrm{range \ }}
\renewcommand{\null}{\mathrm{null \ }}
\newenvironment{problem}{\textbf{Problem:}}{}
%\renewcommand{\familydefault}{\sfdefault}
\begin{document}
	\maketitle

	\section{Linear Maps}
	\subsection{Vector Space of Linear Maps}
	\begin{problem}
		Suppose \( b, c \in \mathbf R \). Define \( T:\mathbf R^3 \to \mathbf R^2 \) by 
		\[
		T(x, y, z) = (2x - 4y + 3z + b, 6x + cxyz)
		\] 
		Show that \( T \) is linear if and only if \( b = c = 0 \).
	\end{problem}

	\begin{solution}
		We first show that if \( b = c = 0 \), then \( T \) is linear. Recall the facts of linearity:
		\[
		T(u + v) = Tu + Tv \quad T(\lambda v) = \lambda (Tv)
		\] 
		for all \( v \in V \). If \( b = c = 0 \), then we can define \( T \) as:
		\[
		T(x,y,z) = (2x - 4y + 3z, 6x)
		\] 
		Now suppose we have two vectors \( u = (x_1, y_1, z_1), v = (x_2, y_2, z_2) \). Then:
		\[
		Tu + Tv = (2x_1 - 4y_1 + 3z_1, 6x_1) + (2x_2 - 4y_2 + 3z_2, 6x_2) = (2(x_1 + x_2) - 4(y_1 + y_2) + 3(z_1 + z_2)
		, 6(x_1 + x_2)) = T(u + v)
		\] 
		Now for homogeneity:
		\[
		T(\lambda u) = T(\lambda x_1, \lambda y_1, \lambda z_1) = (2 \lambda x_1 - 4 \lambda y_1 + 3 \lambda z_1, 
		6\lambda x_1) = \lambda (Tu)
		\] 
		Therefore, both conditions are satisfied, indeed \( T \) is linear. Now we show that if \( T \) is linear, 
		then \( b = c = 0 \) is necessary. Consider what we had earlier:
		\[
		Tu + Tv = (2(x_1 + x_2) - 4(y_1 + y_2) + 3(z_1 + z_2) + 2b, 6(x_1 + x_2) + cx_1y_1z_1 + cx_2y_2z_2)
		\] 
		This is only equal to \( T(u + v) \) if \( 2b = 0 \) and \( c(x_1y_1z_1 + x_2y_2z_2) = 0 \), since 
		they are the only nonlinear terms. Thus, if \( T \) is linear, then \( b = c = 0 \). 
	\end{solution}

	\begin{problem}
		Suppose that \( T \in \mathcal L(V, W) \) and \( v_1, \dots, v_m \) is a list of vectors in \( V \) such that 
		\( Tv_1, \dots, Tv_m \) is a linearly independent list in \( W \). Prove that \( v_1, \dots, v_m \) is 
		linearly independent. 
	\end{problem}

	\begin{solution}
		We return to the definition of linear independence: a set of vectors \( v_1, \dots v_m \) is linearly 
		independent the solution to the equation:
		\[
		a_1v_1 + \cdots + a_m v_m = 0
		\] 
		is that \( a_1, \dots, a_m = 0\). Since we know that the list \( Tv_1, \dots, Tv_m \) is linearly independent
		in \( w \), then the solution to the equation:
		\[
		a_1Tv_1 + \dots + a_mTv_m = 0
		\] 
		is \( a_1 = \cdots = a_m = 0 \). Now, apply the rules of \( T \) being a linear map:
		\[
		a_1Tv_1 + \cdots +  a_mTv_m = T(a_1v_1) + \cdots + T(a_m v_m) = T(a_1v_1 + \cdots a_m v_m) = 0
		\] 
		Now, we use the fact that since linear maps take 0 to 0, this implies that \( a_1 v_1 + \cdots + 
		a_m v_m = 0 \). Further, since the only values of \( a_i \) that satisfy this equation is 
		\( a_1 = \cdots = a_m = 0 \), then this satisfies the condition that \( v_1, \dots, v_m \) is linearly 
		independent.
	\end{solution}

	\begin{problem}
		Show that every linear map from a one-dimensional vector space to itself is multiplication by some 
		scalar. More precisely, prove that if \( \dim V =  1\) and \( T \in \mathcal L(V) \), then there 
		exists \( \lambda \in \F \) such that \( Tv = \lambda v \) for all \( v \in V \). 
	\end{problem}

	\begin{solution}
		Since \( V \) is one-dimensional, this implies that there is only one basis vector, \( v_1 \). Therefore, 
		for all vectors \( v \in V \), then \( v = \alpha v_1 \) for some \( \alpha \in \F \). Then, becuase 
		\( T \in \mathcal L(V) \), then \( T \) must map every vector \( v \in V \) to another vector in \( V \), 
		which must be expressed as a scalar times \( v \). Thus, \( Tv = \lambda v \) is the only option for 
		a linear map on this space. More precisely:
		\[
		Tv = T(\lambda v_1) = \lambda Tv_1 = \alpha \lambda v_1 = \lambda (\alpha v_1) = \lambda v
		\] 
		as desired. 
	\end{solution}

	\begin{problem}
		Give an example of a function \( \varphi: \R^2 \to \R \) such that 
		\[
		\varphi(av) = a \varphi(v)
		\] 
		for all \( a \in \R \) and all \( v \in \R \) but \( \varphi \) is not linear. 
	\end{problem}

	\begin{problem}
		Suppose \( U \) is a subspace of \( V \) with \( U \neq  V \). Suppose \( S \in \mathcal L(U, W) \) and 
		\( S \neq 0 \) (which means that \( Su \neq 0 \) for some \( u \in U \)). Define 
		\( T:V \to W \) by 
		\[
		Tv = \begin{cases}
			Sv & \text{if \( v \in U \)}\\
			0 & \text{if \( v \in V \) and \( v \not \in U \)}
		\end{cases}
		\] 
		Prove that \( T \) is not a linear map on \( V \). 
	\end{problem}

	\begin{solution}
		A linear map must satisfy \( T(\lambda v) = \lambda (Tv) \)  for all \( v \in V \). However, consider 
		some \( \lambda \) such that \( v \in U \) but \(\lambda v \not\in U \). Then:
		\[
		T(\lambda v) = 0 \quad \lambda (Tv) = \lambda Sv \neq 0 
		\] 
		Hence, \( T \) is not linear on \( V \). Alternatively, we could define \( v \in V \) and \( w \in V \) 
		but \( w \not \in U \), then we have:
		\[
		T(v + w) = 0
		\] 
		But:
		\[
		Tv + Tw = Sv \neq 0
		\] 
		so this also violates linearity.  
	\end{solution}

	\begin{problem}
		Suppose \( v_1, \dots, v_m \) is a linearly dependent list of vectors in \( V \). Suppose also that \( W \neq 
		\{0\} \). Prove that there exist \( w_1, \dots, w_m \in W \) such that no \( T \in \mathcal L(V, W) \) 
		satisfies \( Tv_k = w_k \) for each \( k = 1, \dots, m \). 
	\end{problem}

	\begin{solution}
		Since \( v_i \) is linearly dependent, then we can write \( v_i = \sum_{j \neq i} a_j v_j \) for some 
		set of \( a_j \). Then, consider some nonzero  \( w \in W \), and set the \( w_k \)'s as follows:
		\[
		w_k = \begin{cases}
			w & k = i\\
			0& \text{else}
		\end{cases}
		\] 
		Then, suppose for all \( j \neq i \), that \( Tv_j = w_j = 0\). Then, let's write \( Tv_i \):
		\[
			Tv_i = T\left( \sum_{j \neq i}a_j v_j \right) = \sum_{j \neq i}a_j Tv_j \sum_{j \neq i}a_j w_j = w_i
		\] 
		But since all \( w_j = 0 \), this means that \( w_i = 0 \), but we set \( w_i \neq 0 \) purposefully, 
		therefore there is no \( T \) that would stasify this. 
	\end{solution}
	\subsection{Null Spaces and Ranges}
	\begin{problem}
		Suppose \( S, T \in \mathcal L(V) \) are such that \( \range S \subseteq \null T\). Prove that \( (ST)^2 = 0 \).
	\end{problem}

	\begin{solution}
		Consider a vector \( v \in V \). Then:
		 \begin{align*}
			 (ST)^2 v &= (STS)(Tv) \\
			 &= ST(Sv') 
		\end{align*}
		Now, \( Sv' \) will exist within \( \range S \), and since we know that \( \range S \subseteq \null T \), 
		then this impleis that \( T(Sv') = 0 \). Finally, \( S(0) = 0 \), so hence \( (ST)^2 v = 0 \), so 
		\( (ST)^2 = 0 \). 
	\end{solution}

	\begin{problem}
		Suppose \( v_1, \dots, v_m \) is a list of vectors in \( V \). Define \( T \in \mathcal L(\F^{m}, V) \) 
		by 
		\[
		T(z_1, \dots, z_m) = z_1v_1 + \cdots + z_m v_m
		\] 
		\begin{enumerate}[label=\alph*)]
			\item What property of \( T \) corresponds to \( v_1, \dots, v_m \) spanning \( V \)?

				\begin{solution}
					If \( T \) is surjective, then \( v_1, \dots, v_m \) spans \( V \). This is because if the range 
					of \( T \) is \( V \), then the set of vectors \( T \) applies a linear combination to 
					must span \( V \). 
				\end{solution}
			\item What property of \( T \) corresponds to the list \( v_1, \dots, v_m \) being linearly 
				independent?

				\begin{solution}
					The set \( v_1, \dots, v_m \) is linearly independent if and only if \( T \) is injective, since 
					linear independence means that there is only one way to express every vector (i.e. 
					the solution to \( T(z_1, \dots, z_m) = 0 \) is \( z_1 = \cdots = z_m  = 0 \)). 
				\end{solution}
		\end{enumerate}
	\end{problem}

	\begin{problem}
		Suppose \( T \in \mathcal L(V, W) \) is injective and \( v_1, \dots, v_n \) is linearly independent in \( V \).
		Prove that \( Tv_1, \dots, Tv_n \) is linearly independent in \( W \). 
	\end{problem}

	\begin{solution}
		The proof of this is very similar to the one we did earlier. If \( v_1, \dots, v_n \) is linearly dependent, 
		then this means that
		\[
		a_1v_1 + \cdots + a_nv_n = 0
		\] 
		is solved by setting \( a_i = 0 \). Then, now let's consider the list  \( Tv_1, \dots, Tv_n \):
		\[
		a_1Tv_1 + \cdots + a_n Tv_n = T(a_1v_1 + \cdots + a_nv_n) = 0
		\] 
		where the last equality we obtain from the fact that linear maps map 0 to 0. This implies that 
		the only solution to this equation is \( a_i = 0 \), hence the list \( Tv_1, \dots, Tv_n \) is 
		linearly independent.
	\end{solution}

	\begin{problem}
		Suppose that \( V \) is finite-dimensional and that \( T \in \mathcal L(V, W). \) Prove that there 
		exists a subspace \( U \) of \( V \) such that 
		\[
		U \cap \null T = \{0\}  \quad \text{and} \quad \range T = \{Tu :u \in U\}.
		\] 
	\end{problem}

	\begin{solution}
		We can define \( U = \{0\} \cup (V \setminus \null T) \). This way, \( \null T \cap U = \{0\}  \) by 
		definition, and \( \range T = \{Tv: v\in V\}  \), but since \( U \) defines the same set of vectors 
		(since we only get rid of the null space), then \( \range T = \{Tu: u \in U\}  \), as desired. 
	\end{solution}

	\begin{problem}
		Suppose \( T \) is a linear map from \( \F^{4} \) to \( \F^{2} \) such that 
		\[
		\null T = \{(x_1, x_2, x_3, x_4) \in \F^{4}: x_1 = 5x_2 \ \text{and} \ x_3 = 7x_4\} 
		\] 
		Prove that \( T \) is surjective.
	\end{problem}

	\begin{solution}
		We know that \( T \in \mathcal L(\F^{4}, F^{2}) \). Because the null space can be determined by two variables
		only (\( x_1, x_3 \)), so \( \dim \null T = 2 \). This implies that since  \( \dim V = 4 \), then 
		\( \dim \range T = 2 \), and since this equals the dimension that \( T \) maps to \( \F^2 \), then 
		this implies that \( T  \) is indeed surjective by (3.19).
	\end{solution}

	\begin{problem}
		Suppose \( V \) and \( W \) are both finite-dimensional. Prove that there exists an injective map 
		from \( V \) to \( W \) if and only if \( \dim V \le \dim W \). 
	\end{problem}

	\begin{solution}
		Recall the definition of injectivity: a linear map \( T \in \mathcal L(V, W)  \) is injective if and only if 
		\( Tu = Tv \) implies that \( u = v \), or equivalently that \( \null T = \{ 0 \}  \). 

		We prove the forward case: if \( \dim V \le \dim W \), we show that there exists 
		an injective map from \( V \) to \( W \). Let \( v_1, v_2, \dots, v_n \) and \( w_1, w_2, \dots, w_m \) 
		be the basis vectors of \( V \) and \( W \) respectively, and \( n \le m \). Then, define a map
		\( Tv_i = w_i \) for all \( i = 1, \dots, n \). Then, \( \dim \range T = \dim V \), implying 
		that \( \dim \null T = 0 \) from FTLM, as desired. 

		Now we prove the reverse: we want to show that if there exists an 
		injective map from \( V \) to \( W \), then \( \dim V \le \dim W \). This is trivial by contradiction:
		if \( \dim V > \dim W \), then by (3.22) this is impossible, so we're done.   
	\end{solution}

	\begin{problem}
		Suppose \( V \) and \( W \) are finite-dimensional and \( U \) is a subspace of \( V \). Prove that 
		there exists \( T \in \mathcal L(V, W) \) such that \( \null T = U \) if and only if 
		\( \dim U \ge  \dim V - \dim W \). 
	\end{problem}
	

	\begin{solution}
		We prove the forward case: there exists a \( T \) such that \( \null T = U \) if 
		\( \dim U \ge \dim V - \dim W \). Let \( \{u_i \} \) be a basis of \( U \), and \( \{ w_i \} \) be a basis 
		of \( W \). Then, define a linear map \( T \) as follows:
		\[
		Tv_i = \begin{cases}
			\vec 0 & v_i \in \{u_i\} \\
			w_i & v_i \not \in \{u_i\} 
		\end{cases}
		\] 
		One can check very easily that this is linear, with \( \null T = U \). Here, \( \dim \null T = \dim U\)
		and \( \dim \range T = \dim W - \dim U \), since the basis vectors that map to a nonzero vector in \( W \) 
		are those that do not form a basis of \( U \). Therefore, \( \dim V = \dim U + (\dim W - \dim U) = \dim W
		\le \dim U + \dim W\), so the inequality is satisfied. 


		Now we prove the reverse case: if such a \( T \) exists, then \( \dim U \ge \dim V - \dim W \). 
		From the fundamental theorem of linear maps, we know that 
		\( \dim V = \dim \null T + \dim \range T \). Now suppose we have a \( T \) such that \( \null T = U \). 
		Then, we have \( \dim V = \dim U + \dim \range T \), and since \( \dim \range T \le \dim W \), then 
		\( \dim V \le  \dim U + \dim W  \), which is the inequality we wanted to satisfy. 
	\end{solution}

	\begin{problem}
		Suppose \( W \) is finite-dimensional and \( T \in \mathcal L(V, W) \). Prove that \( T \) is 
		injective if and only if there exists \( S \in \mathcal L(W, V)  \) such that \( ST \) is the identity 
		operator on \( V \).
	\end{problem}


	\begin{solution}
		We show the forward case: \( T \) is injective if there exists an \( S \in \mathcal L(W, V) \) such that 
		\( ST  \) is the identity. Suppose \( T \) is not injective, so there exists two vectors \( u \neq v \) 
		such that \( Tu = Tv \). Then, acting \( S \) on the left of both sides gives:
		\( STu = STv \) and since \( ST \) is the identity, then we're left with \( u = v \), which is a contradiction. 

		Now we show the reverse case: if \( T \) is injective, there exists an \( S \in \mathcal L(W, V) \) such that 
		\( ST \) is the identity. Since  \( T \) is injective, then we know that for any two vectors 
		\( v_1, v_2 \in V \), if \( Tv_1 = Tv_2 \) then \( v_1 = v_2 \). Now, let \( \{w_1, w_2, \dots, w_m\}  \)
		be a basis for \( W \) and \( \{v_1, v_2, \dots, v_n\}  \) be a basis for \( V \). 
		Then, let \( S \in \mathcal L(W, V)\) be defined as:
		\[
		S w_i = \begin{cases}
			v_i & i \in \{1, 2, \dots, n\}\\
			0 & \text{else}
		\end{cases}
		\] 
		\( S \) is clearly linear, and take any vector \( v = \sum_i \alpha_i v_i \). Then:
		\[
		STv = S\left( \sum_i \alpha_i w_i \right) = \sum_i \alpha_i v_i = v
		\] 
		so therefore \( ST  \) is indeed the identity.

		As an aside, we also should prove that 
		\( T \) transforms in a way such that \( Tv_i = w_i \), or in other words \( T \) transforms 
		each basis vector in \( V \) to a basis vector in \( W \). This needs to be true since FTLM
		says that \( \dim V = \dim \null T + \dim \range T \), and since \( \dim \null T = 0 \), then 
		\( \dim \range T = \dim V \). 
	\end{solution}
	\begin{problem}
		Suppose \( \phi \in \mathcal L(V, \F) \) and \( \phi \neq 0 \). Suppose \( u \in V \) is not in 
		\( \null \phi \). Prove that
		\[
		V = \null \phi \oplus \{au : a \in \F \} 
		\] 
	\end{problem}

	\subsection{Matrices}

	\begin{problem}
		Suppose \( T \in \mathcal L(V, W) \). Show that with respect ot each choice of bases of  \( V \) and \( W \), 
		the matrix of \( T \) has at least \( \dim \range T \) nonzero entries. 
	\end{problem}
	
	\begin{problem}
		Suppose \( v_1, \dots, v_n \) is a basis of \( B \) and \( w_1, \dots, w_m \) is a basis of \( W \). 
		\begin{enumerate}[label=\alph*)]
			\item Show that if \( S, T \in \mathcal L(V, W) \) then 
				\( \mathcal M(S + T) = \mathcal M(S) + \mathcal M(T) \) 
			\item Show that if \( \lambda \in \F \) and \( T \in \mathcal L(V, W) \), then \( \mathcal M(\lambda T)
				= \lambda \mathcal M(T)\). 
		\end{enumerate}
	\end{problem}


\end{document}
