The ninth lecture of Physics 5C was held on  \textbf{Tuesday, September 18}. 

\section{Last Lecture: The Equipartition Theorem}

Last time, we dicsussed the equal partition theorem - namely, the fact that the energy of a particular gas is spread eqully across all degrees of freedom that have quadratic dependence on some variable. More specifically, we found that each degree of freedom gives us a factor of $k_BT/2$. Now, we will develop a framework which allows us to calculate any state variable given any starting point. 

In order to accomplish this, the first thing we need is to choose two variables that will uniquely define our system. Here, we will use the temperature represented by $T$, and what we will call the \textit{partition function}, denoted by $Z$. The partition function is defined as follows: 

\[ Z = \sum_{\alpha} e^{-\beta E_\alpha}\] 

Where $E_{\alpha}$ refers to all the energy levels, and $\beta = \frac{1}{k_BT}$. Effectively, this defines a system with parameters $(T, Z)$, from which we will calculate all other state variables. 

\begin{example}{Partition Function Practice}{}
    Suppose we have a system where there are an infinite number of allowed energy levels, each separated by an energy difference $\Delta$. (In the case of a quantum harmonic oscillator, we hvae $\Delta = \hbar \omega$.) Our partition function sums over all possible energies, so therefore: 

    %[INSERT TIKZ HERE]

    \begin{align*}
        Z = \sum_i e^{-\beta E_i} &= 1 + e^{-\beta \Delta} + e^{-2\beta \Delta} + \cdots\\
        &= e^{-\beta} + e^{-\beta \Delta} + \left( e^{-\beta \Delta}\right)^2 + \cdots \\
    \end{align*}

    But now notice that this is an infinite geometric series! This series converges since $e^{-n\beta \Delta} < 1$, and working out the geometric series gives:

    \[ Z = \frac{1}{1-e^{-\beta \Delta}}\]
\end{example}

\subsection{Calculating Functions of State}

\subsubsection*{Internal Energy $U$}

Now we move on to calculating the functions of state, starting with the internal energy. We know that the internal energy of a system is given by 

\[ U = \frac{\sum_i E_i e^{-\beta E_i}}{\sum_i e^{-\beta E_i}}\] 

Note the denominator here is exactly equal to our partition function $Z$. However, how do we cleanly express the numerator in terms of $Z$? Here, we use a clever bit of mathematics: notice that: 

\[ E_i e^{-\beta E_i} = - \frac{\partial}{\partial \beta} e^{-\beta E_i}\] 

and since partial derivatives are linear, then we get that 

\[ \sum_i E_i e^{-\beta E_i} = -\frac{\partial}{\partial \beta} Z\] 

And so therefore, we can express $U$ as:

\[ U = \frac{-\frac{\partial}{\partial \beta} Z}{Z}\] 

\subsubsection*{Entropy $S$}

Here, we will use the definition of $S = -k_B \sum_i P_i \ln P_i$ instead of our other expression $k_B \ln \Omega$, mainly becuase $\Omega$ is difficult to define for a general system. Recall that $P_i$ is defined as

\[ P_i = \frac{e^{-E_i/k_BT}}{\sum_i e^{-E_i/k_BT}} = \frac{e^{-\beta E_i}}{Z}\] 

Adn so therefore, we have the relation $\ln P_i = -\beta E_i - \ln Z$ using the relations with logarithms. Thus, we can write: 

\begin{align*}
    S = -k_B \sum_i P_i (-\beta E_i - \ln Z) &= -k_B \left[-\beta \underbrace{\sum_i E_i P_i}_{\mean E = U} - \underbrace{\sum_i P_i}_{= 1} \ln Z\right]\\
    &= -k_B \left( -\frac{1}{k_BT} U\right) + k_B \ln Z\\
    &= \frac{U}{T} + k_B \ln Z
\end{align*}

which is, all things considered, a relatively simple expression for such an abstract concept like entropy. 

\subsubsection*{Combination of 1st and 2nd law}

We know that the firat law can be expressed as $dU = T dS - pdV$, but these variables aren't exactly the easiest thing to calculate because we don't really know how to quantify $dS$. Instead, we define the \textit{Helmholtz function} $F$, defined as $F = U - TS$

\begin{insight*}{}
    Because $U$, $T$ and $S$ are all well defined functions of state, then this makes $F$ also an equally well defined function of state! This is important to know because it means that for any thermodynamic system, we will be able to define an $F$. 
\end{insight*}

