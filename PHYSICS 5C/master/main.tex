%%PACKAGE INCLUSIONS%%
\documentclass{book}
\usepackage[letterpaper, margin=1in]{geometry}
\usepackage[pdftex]{graphicx}
\usepackage[utf8]{inputenc}
\usepackage{tikz, pgfplots, wrapfig, amssymb, array, mathtools, enumitem, circuitikz, physics, parskip, hyperref, chemformula}
\usepackage{tkz-euclide}
\usepackage{titlesec}
\usepackage{lipsum}
\usepackage[english]{babel}
\usepackage{amsmath, amsthm}
\usepackage{fancyhdr}
\usepackage{xcoffins}
\usepackage{dirtytalk}
\usepackage{shortcuts}

\theoremstyle{plain}
\usetikzlibrary{calc,patterns,angles,quotes}
\setlength{\parskip}{0pt}
\usetikzlibrary{calc,patterns,angles,quotes}
\makeatletter
\pagestyle{fancy}
\setlength{\parskip}{1ex plus 0.5ex minus 0.2ex}
\pgfplotsset{compat=1.17}
\setcounter{tocdepth}{1}
% \linespread{1.3}
%%%%%%%%%%%%%%%%%%%%%%%%%%%%%%%%%%%%%%%%%%%%%%%%%%%%%%%%%%%%%%%%%%%%%%%%%%%%%%%%%%%%%%%%%%%%%%%%%%%%%%%%%%%%%%%
%%SECTIONING EDITS%%
%
%DEFINITION FOR CHAPTER HEADERS%
\def\thickhrulefill{\leavevmode \leaders \hrule height 1ex \hfill \kern \z@}
\def\@makechapterhead#1{
  {\parindent \z@ \centering \reset@font
        \thickhrulefill\quad
        \scshape Lecture \thechapter
        \quad \thickhrulefill
        \par\nobreak
        \vspace*{10\p@}%
        \interlinepenalty\@M
        \hrule
        \vspace*{10\p@}%
        \huge \bfseries #1\par\nobreak
        \par
        \vspace*{10\p@}%
        \hrule
    \vskip 20\p@
  }}
%%%%%%%%%%%%%%%%%%%%%%%%%%%%%%%%%%%%%%%%%%%%%%%%%%%%%%%%%%%%%%%%%%%%%%%%%%%%%%%%%%%%%%%%%%%%%%%%%%%%
%PART STYLING%
\titleclass{\part}{top} % make part like a chapter
\titleformat{\part}
[display]
{\centering\normalfont\huge\bfseries}
{\vspace{200pt}\titlerule[5pt]\vspace{3pt}\titlerule[2pt]\vspace{3pt}\MakeUppercase{Part} \thepart}
{0pt}
{\titlerule[2pt]\vspace{20pt}\huge\MakeUppercase}

\titlespacing*{\part}{0pt}{0pt}{20pt}
%%%%%%%%%%%%%%%%%%%%%%%%%%%%%%%%%%%%%%%%%%%%%%%%%%%%%%%%%%%%%%%%%%%%%%%%%%%%%%%%%%%%%%%%%%%%%%%%%%%%
%SECTION STYLING%
\def\section{\@ifstar\unnumberedsection\numberedsection}
\def\numberedsection{\@ifnextchar[%]
    \numberedsectionwithtwoarguments\numberedsectionwithoneargument}
\def\numberedsectionwithoneargument#1{\numberedsectionwithtwoarguments[#1]{#1}}
\def\numberedsectionwithtwoarguments[#1]#2{%
    \ifhmode\par\fi
    \removelastskip
    \vskip 5ex\goodbreak
    \refstepcounter{section}%
    \hbox to \hsize{\hss\vbox{\advance\hsize by 0cm
        \noindent
        \leavevmode\Large\bfseries\raggedright
        \thesection\space$\bigg\vert$\hskip -1ex $\bigg\vert$\space   \
        #2\par
        \vskip -2ex
        \noindent\hrulefill
        \vskip -3ex
        \noindent\hrulefill
        }}\nobreak
    \vskip 2ex\nobreak
    \addcontentsline{toc}{section}{%
    \protect\numberline{\thesection}%
    #1}%
}
%%%%%%%%%%%%%%%%%%%%%%%%%%%%%%%%%%%%%%%%%%%%%%%%%%%%%%%%%%%%%%%%%%%%%%%%%%%%%%%%%%%%%%%%%%%%%%%%%%%%
%SUBSECTION STYLING%
\def\subsection{\@ifstar\unnumberedsubection\numberedsubsection}
\def\numberedsubsection{\@ifnextchar[%]
    \numberedsubsectionwithtwoarguments\numberedsubsectionwithoneargument}
\def\numberedsubsectionwithoneargument#1{\numberedsubsectionwithtwoarguments[#1]{#1}}
\def\numberedsubsectionwithtwoarguments[#1]#2{%
    \ifhmode\par\fi
    \removelastskip
    \vskip 5ex\goodbreak
    \refstepcounter{subsection}%
    \hbox to \hsize{\hss\vbox{\advance\hsize by 0cm
        \noindent
        \leavevmode\large\bfseries\raggedright
        \thesubsection\space$\bigg\vert$\space\
        #2\par
        \vskip -2ex
        \noindent\hrulefill
        }}\nobreak
    \vskip 2ex\nobreak
    \addcontentsline{toc}{subsection}{%
        \protect\numberline{\thesubsection}%
        #1}%
    }

    \makeatother
%%%%%%%%%%%%%%%%%%%%%%%%%%%%%%%%%%%%%%%%%%%%%%%%%%%%%%%%%%%%%%%%%%%%%%%%%%%%%%%%%%%%%%%%%%%%%%%%%%%%%%%%%%%%%%%
\begin{document}

    %TITLE PAGE%
    \begin{titlepage}
        \centering
        \vspace*{\baselineskip}\vspace{200pt}
        \rule{\textwidth}{1.6pt}\vspace*{-\baselineskip}\vspace*{2pt}
        \rule{\textwidth}{0.4pt}\\[\baselineskip]
        {\Huge \bfseries PHYSICS 5C NOTES}\\[0.2\baselineskip]
        \rule{\textwidth}{0.4pt}\vspace*{-\baselineskip}\vspace{3.2pt}
        \rule{\textwidth}{1.6pt}\\[\baselineskip]
        \scshape
        Typeset notes for Physics 5C: Introductory Thermodynamics and Quantum Mechanics \\
        \par
        \vspace*{2pt}
        {\Large Andrew Binder and Eric Du}\\
        {\large University of California, Berkeley\par}
        {\scshape Fall 2022} \\
        \normalsize
    \end{titlepage}

    \tableofcontents
    \newpage
    \setcounter{chapter}{-1}
%%%%%%%%%%%%%%%%%%%%%%%%%%%%%%%%%%%%%%%%%%%%%%%%%%%%%%%%%%%%%%%%%%%%%%%%%%%%%%%%%%%%%%%%%%%%%%%%%%%%%%%%%%%%%%%
%%%%%%%%%%%%%%%%%%%%%%%%%%%%%%%%%%%%%%%%%%%%%INTRODUCTION%%%%%%%%%%%%%%%%%%%%%%%%%%%%%%%%%%%%%%%%%%%%%%%%%%%%%%
    \chapter{Introduction}
      This course is titled \textbf{Introductory Thermodynamics and Quantum Mechanics}. Eric and Andrew both took this course in Fall 2022.

      \section{Basic Syllabus Info}

      Some basic information regarding the syllabus for this course as it was taught in Fall 2022.

      \subsection{Professor and GSI Info}

      \begin{itemize}
          \item \textbf{Professor:} Feng Wang (fengwang76@berkeley.edu) OH: Tuesdays 11-12pm, 361 Birge
          \item \textbf{GSI 1:} Sami Kaya (samikaya@berkeley.edu) (TBD)
          \item \textbf{GSI 2:} Neel Modi (neel\_modi@berkeley.edu) (TBD)
      \end{itemize}

      \subsection{Discussion Info}
      \begin{itemize}
          \item \textbf{Lectures:} Tu/Th 9:30-11:00, Physics Building 2
          \item \textbf{Discussions:} W 10:00-12:00 in Evans 2 or Tu 17:00-19:00 in Etcheverry 3119
      \end{itemize}


      \subsection{Grading Breakdown}

      \begin{itemize}
          \item \textbf{Homework:} One assignment per week, due Friday at 17:00, lowest score is dropped, total 30\%
          \item \textbf{Midterm:} Taken on Thursday, Oct. 13, in class. Total 25\%
          \item \textbf{Final:} Taken on Tuesday, Dec. 13 from 15:00-18:00
      \end{itemize}

      \subsection{Course Materials}

      Two textbooks are recommended for this course: \textit{Concepts in Thermal Physics} by Blundel \& Blundel, and \textit{Introduction to Quantum Physics} by French \& Taylor. Other possible resources include \textit{Introduction to Quantum Mechanics} by Griffiths and the online Feynman lectures.

      \subsection{Lecture Topics}

      This course covers the principles of thermodynamics as well as introductory quantum mechanics. The Thermodynamics portion will deal primarily with gases and their properties, and overall is the study of how we model systems with many particles. This portion is taught over the span of approximately 6 weeks. The Quantum Mechanics portion of this course deals primarily with the strange behavior of microscopic particles, and has an approximate length of 8 weeks.
%%%%%%%%%%%%%%%%%%%%%%%%%%%%%%%%%%%%%%%%%%%%%%%%%%%%%%%%%%%%%%%%%%%%%%%%%%%%%%%%%%%%%%%%%%%%%%%%%%%%%%%%%%%%%%%
%%%%%%%%%%%%%%%%%%%%%%%%%%%%%%%%%%%%%%%%%%%%THERMODYNAMICS%%%%%%%%%%%%%%%%%%%%%%%%%%%%%%%%%%%%%%%%%%%%%%%%%%%%%
  \part{Thermodynamics}

%%%%%%%%%%%%%%%%%%%%%%%%%%%%%%%%%%%%%%%%%%%%%%LECTURE 1%%%%%%%%%%%%%%%%%%%%%%%%%%%%%%%%%%%%%%%%%%%%%%%%%%%%%%%%
    \chapter{Lecture 1 (8/25)}
        The first lecture of Physics 5C was held on  \textbf{Thursday, August 25}. It covered the basics of the course syllabus and an introduction of large numbers.
        \section{Introduction of Large Number Systems}
          The biggest issue regarding dealing with thermodynamics and quantum mechanics will be reconciling the scale of the very big and the scale of the very small. Atomic scales are, as intuition suggests, tiny  compared to our normal human scale. So how can we jump between them in a way that makes sense? To connect the different size scales, we are going to need some tools, which we will be exploring in depth here.

          \subsection{Avogardo's Number $N_A$}
            Firstly, we need to introduce an important number that will be the jumping-off point for the rest of our exploration.
            \begin{definition}{Avogadro's Number}{}
               Denoted $N_A$; The number of carbon atoms in 12 grams of carbon material.
            \end{definition}
            Why 12? This is a number chosen fairly arbitrarily, but it is a nice number, since we are dealing mostly with Carbon-12 (6 protons and 6 neutrons), the most common form of carbon. This number is defined as roughly \fbox{$6.023*10^{23}$}, but we will forego writing out this number in favor of the much simpler $N_A$. From Avogadro's number, we get a few more definitions that will help count quantities on the atomic scale:
            \begin{definition}{Mole}{}
              The quantity of matter that contains $N_A$ objects. Denoted mol.
            \end{definition}
            For example, 1 mol of water contains $N_A$ $H_2O$ molecules.
            \begin{definition}{Molar Mass}{}
              The mass of 1 mol of objects (or $N_A*m_{atom}$, where $m_{atom}$ refers to the mass of the atom in question).
            \end{definition}
            These concepts and definitions now will help us jump between the two scales (human and atomic) much more easily. But, since individual atoms are so small, we really can't track the individual behavior of these atoms, so what do we do? Do we give up? As a matter of fact, no. So how do we proceed?
          \subsection{Thermodynamic Limit (Large Number Limit)}
            For this, it's effective to completely disregard individual behavior of atoms in favor of considering \underline{average behavior}, as this will help us describe the \underline{general behavior}. One key concept with the Thermodynamic Limit (or essentially in layman's terms the large number limit) is the following:
            \begin{insight*}{}{}
              Fluctuations decrease with large numbers.
            \end{insight*}
            This means that the average value for a quantity in a system with very many objects is extremely accurate in describing the system's individual components. In other words, we need not worry that the exact behavior may not match our approximations: those differences are, for all our intents and purposes, negligible. Consider the following example as a motivation:
            \begin{example}{Container of Gas Atoms}{}
              To illustrate the idea of fluctuations decreasing with large numbers, consider a container with a single gas molecule in it, and we have a device which allows us to measure the force on one of the sides of the wall. In this case, we'd observe that on occasion (let's say, 1\% of the time), when the particle collides with the wall, a force is observed, but otherwise the force vs. time graph remains zero. Here, we see the fluctuations in the force are rather large (one might even say, infintely large!), and this is simply due to the fact that there is only a single particle.


                % Insert TikZ diagram here w/ graph

              Now, suppose that instead of 1 molecule, we now have $10^4$ molecules instead. What would our force vs. time diagram look like? Well, certainly it would be very rare if ever got a reading of zero, since statistically speaking we should always expect at least one particle colliding with each side of the box. In this case, the force should average at about $10^4 \cdot 1\%$ or roughly 100 units of force. If we say that the force fluctuates by approximately 10 units above or under the average, it's easy to see how these fluctuations are much smaller compared to the earlier case.


              The underlying principle here is, as the number of molecules we are concerned in increases, it's natural that the relative fluctuations decrease in size. If we took $10^8$ atoms, then the fluctuations would be even smaller. Now remember that generally in thermodynamics we're talking about amounts of gas on the order of $10^{23}$ molecules, so hopefully this argument can convince you that at those scales, the fluctuations are effectively zero.
            \end{example}
            In order to make our arguments concrete, let's define $\mean{F}$ to be the average force, and $\delta F$ be the absolute magnitude of the fluctuations. From statistical mechanics, we know that the former is proportional to $N$, and the latter is proprotional to $\sqrt{N}$. Now we define the following:
            \begin{definition}{Relative Fluctuation}{}
                Defined as

                \[ \frac{\delta F}{\mean F} \propto \frac{\sqrt{N}}{N}  = \frac{1}{\sqrt N}\]

            \end{definition}
            From this definition, it's then also easy to see why the relative fluctuation decreases as $N$ increases, since the expression becomes smaller as $N$ increases. In thermodynamics, we also sometimes like to define the pressure $P$ as $\frac{F}{A}$, which is known as a state variable.

            \section{State Variables}
              State variables are, as their name suggests, variables that describe the state that a particular gas is in. For instance, the pressure of the gas is a state variable, so is the average velocity $\mean {\vec v}$, $\mean {\vec p}$, $\mean {\vec{r}}$. However, it turns out that the last three all equal to zero, since we assume some sort of isotropic environment for gases. Nevertheless, we have other state variables that are useful:
              \begin{itemize}
                \item Average total energy $U$
                \item Average pressure $P$
                \item Average temperature $T$
                \item Average volume $V$
              \end{itemize}
              Note that in these definitions of the average, we've dropped the $\langle$ and $\rangle$ symbols, but we are still referring to averages in this case. In fact, since we will \textit{always} be talking about these quantities in terms of average values, these symbols will be continuously dropped throughout the rest of the course material. We also split these state variables into two kinds, based on how they change. Take a box full of gas, and split it down the middle. Now in each half of the box, we have

              \begin{align*}
                U^\star &= \frac{U}{2}\\
                V^\star &= \frac{V}{2}\\
                P^\star &= P\\
                T^\star &= T
              \end{align*}

              Because the first two change when we split it in half, we call these quantities \textit{extensive variables} and the other two \textit{intensive variables}:
              \begin{definition}{Extensive and Intensive Variables}{}
                Two different types of variables:
                \begin{itemize}
                  \item{\textbf{Extensive} variables are \underline{dependent on the size of the system}}
                  \item{\textbf{Intensive} variables are \underline{independent of the size of the system}}
                \end{itemize}

            \end{definition}

      \section{The Ideal Gas}
        To understand the relationship between the state parameters we defined in the previous section, we need to first consider the simplest of cases: the \textit{ideal gas}. An ideal gas is one where we \textit{ignore the interactions of individual atoms}. This is a model system (and a simple one at that), so it will help us understand the very fundamental relationships between these state parameters. This model was used by scientists in the past to conclude one important fact: pressure, volume, and temperature are closely related. Three separate scientists came up with three separate laws to describe the relationship between two of these:
        \begin{theorem}{Boyle's Law}{boyle's law}
          Fixing the Temperature, it turns out that \underline{Pressure is inversely proportional to the Volume}.
        \end{theorem}
        \begin{theorem}{Charles's Law}{charles's law}
          Fixing Pressure, it turns out that \underline{Volume is directly proportional to Temperature}.
        \end{theorem}
        \begin{theorem}{Gay-Lussac's Law}{gay-lussac's law}
          Fixing Volume, it turns out that \underline{Pressure is directly proportional to Temperature}.
        \end{theorem}
        These observations are all nice and important, but can we combine these three laws into a single, general description? As it turns out, we can.
        \begin{theorem}{Ideal Gas Law}{ideal gas law}
          Pressure and Volume are directly proportional to Temperature and the number of atoms with this relation: $$PV = k_{B}NT$$
          Here, $k_B$ is the famous, later-discussed \textit{Boltzmann constant}.
        \end{theorem}
        This is a really crucial observation that helps us relate all of these important state parameters. However, $k_B$ is very small and $N$ is very big, so it might get difficult at times to keep track. So, some scientists scale to this version of the law:
        \begin{theorem}{Scaled Ideal Gas Law}{scaled ideal gas law}
          An alternate statement of this law becomes
          $$PV = nRT$$
          Here, $n$ is the number of moles of material ($\frac{N}{N_A}$), and $R = k_BN_A$
        \end{theorem}
        These two laws are equivalent, and, as of now, empirical. It is important to remember, however, that this is \underline{idealized}: it's a good approximation for most cases, but not perfect. It'll need to be modified for very low temperatures or very high pressures, etc.

      \section{Even Larger Numbers: Combinatorial Problems}
        Consider a system containing $n$ atoms, with each atom having either energy 0 or 1. Suppose that the total energy is $r$. How many microscopic arrangements of this gas are possible?

        This question is a very familiar one for those who have had experience with combinatorics. The idea is that out of $n$ molecules to choose from, we require that $r$ of them have energy 1 and the rest of them have energy $r$. The question is, then, how many of these arrangements are possible? Since this problem can be effectively thought of as choosing which atoms to assign energy 1 to, then it makes sense that we use the choose function here. Specifically, there are $n \choose r$ possible configurations.

        To just illustrate how quickly this choose function grows, if we had $n = 100$ and $r = 40$, then ${n \choose r} \approx 10^{28}$, and if $n = 1000$ and $r = 400$, then ${n \choose r} \approx 10^{209}$. From here, you can see that with $10^{23}$ particles in a typical gas, there are significantly more combinations possible. To further illustrate how large these numbers can be, we can use Stirling's formula in order to derive an approximation for $n!$:
        \begin{theorem}{Stirling's formula}{stirling}
          Given $n$, we have that

          \[ \ln n! \approx n \ln n - n\]

          For sufficiently large $n$.
        \end{theorem}
        If we apply Stirling's formula on $10^{23}!$, we get that $\ln 10^{23}! \approx 52 \times 10^{23}$, and this means that $10^{23}! \approx 10^{22 \times 10^{23}}$, which is monstrously large! Hopefully, this gives a better idea as to why we cannot ever track every individual particle in thermodynamics.
%%%%%%%%%%%%%%%%%%%%%%%%%%%%%%%%%%%%%%%%%%%%%%LECTURE 2%%%%%%%%%%%%%%%%%%%%%%%%%%%%%%%%%%%%%%%%%%%%%%%%%%%%%%%%
    \chapter{Lecture 2 (8/30)}
      The second lecture of Physics 5C was held on  \textbf{Tuesday, August 30}. It covered the remaining basics of statistics needed to understand thermodynamics.

      \section{Last Time: Statistics Review}
        Last lecture we showed that in a typical gas, there are too many particles to track individually, and thus we must resort to using state variables to describe them as a whole instead. In this process, we defined the state variables as the mean values (e.g. mean velocity, etc.). Now, we're going to see what other statistical information we are able to obtain.

      \section{Discrete and Continuous Probability}

        \subsection{Discrete Probability Distribution}
          Discrete probability distributions refers to cases where only discrete outcomes are allowed. In other words, if we throw a die, we can get $\{1, 2, 3, 4, 5, 6\}$ as possible outcomes, but we cannot get $3.5$ as an outcome. If the dice were fair, it would also not shock you that the probability that each number appears should be $\frac{1}{6}$.

          We also have some properties of discrete probability:

          \begin{enumerate}
            \item $\sum_i P(x_i) = 1$. This is quite self-explanatory. The probability of \textit{something} happening is 1, and thus the total probability should sum to 1.
            \item The mean, or also known as the average or expected value, is defined as follows:
            \begin{definition}{Mean/Expected Value}{mean expected value}{}
              Denoted $\mean{x}$: $\mean{x} = \sum_{i} x_iP_i$.
            \end{definition}
            Note that we will use $P(x_i)$ and $P_i$ interchangeably when we come to continuous probability distribution. A useful special case involves taking the average of $x^2$ (something that'll come up a lot in the future):
            $$\mean{x^2} = \sum x_i^2P_i$$

            More generally, we have:

            \[ \mean{f(x)} = \sum f(x_i) P_i\]

            One small note about this is that the mean value need not be obtainable in our set of discrete values; it is simply a representation of the \textit{average} value of the system.
          \end{enumerate}

      \section{Continuous Probability Distribution}
        Continuous probability distributions are used when our variable $x$ can take on a continuous range of values. Here, the notion of asking \say{what is the probability that $x = 1$?} doesn't really mean much, since $x$ is a continuous variable.\footnote{There is an answer to this, $P(x = 1) = 0$ since there are an infinite number of values $x$ could take.} Instead, we look at the probability of $x$ taking on a range of values, say on the interval $(x, x + \Delta x)$. In this case, the probability that such an event occurs is equal to the probability multiplied by the length of the interval:

        \[ P(x + \Delta x) = P(x) \cdot \Delta x\]

        In the limit that $\Delta x \to 0$, then we substitute $\Delta x$ for $dx$.

        Just like discrete probability, continuous distributions also have properties, which are similar to their discrete counterpart:

        \begin{enumerate}
          \item The total probability of something occurring is 1. Just like Discrete probabilities, this property holds, and is written as:
            \[ \int_{All} P(x) dx = 1\]
          \item The mean is defined as follows:
            \begin{definition}{Mean/Average value for continuous variables}{}
              Denoted with $\mean{x}$, we write it as:
              \[ \int_{All} x P(x) dx\]
            \end{definition}
            Again, more generally, if we have an arbitrary function $f(x)$, then
              \[ \mean{f(x)} = \int_{All} f(x)P(x) dx \]
        \end{enumerate}

      \section{Linear Transformations}
        Now that we've seen how to compute the average value given a probability distribution, let's look at how probability distributions work under a linear transformation. That is, if we have the average for a variable $x$, let's see how we compute the average for $y = mx + b$. Given that $y = mx + b$ we have:
        \begin{align*}
          \mean y = \mean{ax + b} &= \int (ax + b) P(x) dx\\
          &= \int axP(x) dx + \int bP(x) dx\\
          &= a\underbrace{\int xP(x) dx}_{= \mean x \text{ by definition}}+ b \underbrace{\int P(x) dx}_{= 1 \text{ by definition}}\\
          &= a \mean x + b
        \end{align*}
        This gives us  a nice result and confirms that the mean is linear:
        \begin{theorem}{Linearity in the Mean}{linearity in the mean}
          Given that $y = mx + b$ we have:
          $$\mean{y} = \mean{ax + b} = a\mean{x} + b$$
        \end{theorem}

      \section{Fluctuation (or Spread) Over the Average Value}
        Now that we've defined our important expressions, we can start analyzing the statistics around these expressions. Specifically, we'll start with our \textit{fluctuation over the average}, which is simply defined as the deviation $x - \mean{x}$.

        \subsection{Variance and Standard Deviation}
          Firstly, let's consider the average value of $x - \mean{x}$:
          $$\mean{x - \mean{x}} = \mean{x} - \mean{x} = \fbox{$0$}$$
          This obviously makes sense, since we expect the average deviation from the mean to come out to be $0$. However, this does make the average of $x - \mean{x}$ a relatively meaningless quantity: we want something that won't always be $0$. So, let's consider the absolute value $|x - \mean{x}|$. This is a well-defined, positive-definite function, but it's not analytical. In other words, we can't really work with it in any elegant manner. What we need is a positive-definite analytical function that behaves roughly like this absolute value. What better candidate than $(x - \mean{x})^2$? This expression has its own name:
          \begin{definition}{Variance}{}
            Denoted $\sigma_x^2$: $\sigma_x^2 = \mean{(x - \mean{x})^2}$
          \end{definition}
          This is also sometimes called the \textit{mean square deviation}. From here, we also define another expression everyone has most likely seen before:
          \begin{definition}{Standard Deviation}{}
            Denoted $\sigma_x$: $\sigma_x = \sqrt{\mean{(x - \mean{x})^2}}$
          \end{definition}
          These are nice new expressions of meaningful values. Let's compute (or rather simplify) the variance:
          \begin{align*}
            \sigma_x^2 &= \mean{(x - \mean{x})^2} = \mean{x^2 - 2x\mean{x} + \mean{x}^2} = \mean{x^2} - \mean{2x\mean{x}} + \mean{x}^2 = \mean{x^2} - 2\mean{x}^2 + \mean{x}^2 = \fbox{$\mean{x^2} - \mean{x}^2$}
          \end{align*}
          From this, we see then that the standard deviation becomes \fbox{$\sigma_x = \sqrt{\mean{x^2} - \mean{x}^2}$}.

      \section{Linear Transformation of Variance}
        With the newly-defined variance and standard deviation, let's subject them to a linear transformation and see what we can glean from them:
        \begin{align*}
          y = ax + b &\implies \mean{y} = a\mean{x} + b &&\text{(from before)}\\
          \therefore \sigma_y^2 &= \mean{y^2} - \mean{y}^2 = \mean{(ax+b)^2} - (a\mean{x} + b)^2 = \\
          &= \mean{a^2x^2 + 2abx + b^2} - a^2\mean{x}^2 + 2a\mean{x}b - b^2 = \\
          &= a^2\mean{x^2} + 2ab\mean{x} + b^2 - a^2\mean{x}^2 + 2ab\mean{x} - b^2 = \\
          &= a^2\mean{x^2} - a^2\mean{x}^2 = a^2(\mean{x^2} - \mean{x}^2) = \fbox{$a^2\sigma_x^2$}
        \end{align*}
        As we can see, $\sigma_y^2$ only depends on $a$ and not on $b$!

      \section{Independent Variables}
        Suppose we have two properties $u$ and $v$, which are continuous and independent variables. That is, the outcome of $u$ does not affect the outcome of $v$. If we wanted to calculate the probability that a particle exhibits both properties $u$ and $v$ atthe same time, we simply need to multiply the probability that $u$ occurs with the probability that $v$ occurs:
        \[ \mean{uv} = \iint uv P_u(u) du \ P_v(v) dv\]
        And since $u$ and $v$ are independently determined, we can split the integral here:
        \[ \mean{uv} = \int uP_u(u) du \cdot \int vP_v(v) dv\]
        Leading us to the property of separation of variables:
        \begin{theorem}{Separation of Independent Variables}{separation of independent variables}
          If $u$ and $v$ are independent variables, then:
          \[ \mean{uv} = \mean{u} \mean{v}\]
        \end{theorem}

      \section{Generalization to \textit{n} variables}
        Now let's suppose that we have $n$ variables $\{x_1, x_2, \dots, x_n\}$, each having mean $\mean x$ and same variance $\sigma_x^2$. Now let
        \[ Y = \sum x_i\]
        What is the mean and variance of $Y$? Well to start, we know that since $Y$ is defined as the summation of $x_i$, then by the rules of linear transformation, the mean of $Y$ would be the summation of the mean of each $x_i$. And since they are defined to be the same, we have:
        \begin{align*}
          \mean Y &= \mean {x_1} + \mean {x_2} + \dots + \mean {x_n} \\
          &= n\mean x
        \end{align*}
        To calculate the variance, we use the property that $\sigma_Y^2 = \mean{Y^2} - \mean{Y}^2$. First, we calculate $\mean{Y}^2$:
        \[ \mean{Y}^2 = \left[ \mean {x_1} + \mean {x_2} + \dots + \mean {x_n}\right]^2 = \mean{x_1}^2 + \mean {x_2}^2 + \dots + \mean{x_1}\mean{x_2} + \dots + \text{(very many cross terms)}\]
        Similarly, calculating $\mean{Y^2}$:

        \begin{align*}
          \mean{Y^2} &= \mean{(x_1 + x_2 + \dots + x_n)^2} = \mean{x_1^2 + x_2^2 + \dots + x_1x_2 + x_2 x_3 + \dots}\\
          &= \mean{x_1^2} + \mean{x_2^2} + \dots + \mean{x_1}\mean{x_2} + \dots + \text{(very many cross terms)}
        \end{align*}
        Note that we can change $\mean{x_1x_2} = \mean{x_1}\mean{x_2}$ because $x_1$ and $x_2$ are independent variables. Now subtracting the two, we see that the cross terms $\mean{x_1}\mean{x_2}$ disappear, and we're left with:

        \begin{align*}
          \sigma_Y^2 &= \underbrace{\mean{x_1}^2 - \mean{x_1^2}}_{\sigma_{x}^2} + \underbrace{\mean{x_2}^2 - \mean{x_2}^2}_{\sigma_x^2} + \dots \\
          &= n \sigma_x^2
        \end{align*}
        Note that this simplification is only possible due to our initial assumption that all our independent variables have the same $\mean{x}$ and $\sigma_x$.

      \section{Random Walks}
        Now, let's pivot focus to an important example in statistics: random walks. In fact, let's consider a special case: Brownian motion. Brownian motion describes the seemingly random motion of pollen when floating on a liquid. As we came to understand, this motion is not the pollen particle simply vibrating for no apparent reason, but it's actually due to many water molecules randomly bouncing off of the pollen, propelling it in random directions. So, let's consider a simple case of Brownian motion with discrete movements.

        Suppose that, at each step, a particle will either have $+a$ or $-a$ movement. After very many steps (say on the order of a million), what will our final position look like?

        \subsection{Scaling Standard Deviation}
          Considering one step, our scenario is simple to understand: $\mean{x} = 0$ since the values cancel, and $\sigma_x^2 = \mean{x^2} - \mean{x}^2 = a^2$. But what about $n$ steps, for a large $n$? Consider $Y = x_1 + x_2 + \dots + x_n$. In this case, $\mean{Y} = 0$ from before, and then $\mean{Y^2} = na^2$. As we can see, this mean \textit{scales with $n$}. In other words, we have the following insight:
          \begin{insight*}{}{}
            Standard deviation scales with $\sqrt{n}$: $\sigma_y = a\sqrt{n}$
          \end{insight*}
          Our particles are all independent, so adding more particles will scale the fluctuation and the force proportionally, the former with $\sqrt{n}$ and the latter with $n$.

      \section{Transitioning to Physics}
        We've spent a lot of time these past two lectures talking about mathematical introductions, so it's about time for us to start talking about physics. We will do this in the next lecture, where we will need to understand \textbf{heat and temperature}.

%%%%%%%%%%%%%%%%%%%%%%%%%%%%%%%%%%%%%%%%%%%%%%LECTURE 3%%%%%%%%%%%%%%%%%%%%%%%%%%%%%%%%%%%%%%%%%%%%%%%%%%%%%%%%
    \chapter{Lecture 3 (9/1)}
      The third lecture of Physics 5C was held on  \textbf{Thursday, September 1}. It covered the quantitative definition of heat and temperature.

      \section{Defining Heat}
        To start, people have observed that doing work on objects generates heat, and this set the foundation for the idea that heat is some form of energy transfer. Specifically, heat can be measured in units of energy, or $[J]$. It was also observed that some objects seemed to heat up faster than others - that is, given the same energy $E$, some items would heat to a higher temperature $T$ than others. This led to the definition of heat capacity, or essentially the proportionality constant between how much energy is absorbed into an object relative to how much energy was put in:
        \begin{definition}{Heat Capacity}{Heat Capacity}
          Defined as:
          \[ C = \frac{\Delta Q}{\Delta T} = \frac{dQ}{dT}\]
        \end{definition}
        This is an \underline{extensive variable}, as it depends on the volume and mass of the object we are trying to heat. In order to remove this dependence on mass, we also define the \textit{specific heat capacity}, which is the heat capacity per unit mass:
        \begin{definition}{Specific Heat Capacity}{}
          Heat capacity per \textbf{volume}:
          \[ c = \frac{C}{m} = \frac{1}{m} \cdot \frac{dQ}{dT}\]
        \end{definition}
        We also define the molar heat capacity as the heat capacity per mole:
        \begin{definition}{Molar Heat Capacity}{}
          Heat capacity per \textbf{mole}:
          \[ c = \frac{C}{n} = \frac{1}{n} \cdot \frac{dQ}{dT}\]
        \end{definition}
        These definitions for specific and non-specific heat capacity only hold true for solids and liquids, but they do not hold for gases. This is because with a gas, there are more parameters which are affected when we heat up a gas. For instance, the pressure of the gas could change, but so could the volume. Because of this, we need to be more careful about our definition for heat capacity. To resolve this, we instead define two different types of specific heat capacity, one measured at constant volume, and the other measured at constant pressure:
        \begin{definition}{Heat Capacity for Gases}{}
          \[ C_v = \left(\frac{\partial Q}{\partial T}\right)_V, \ C_p = \left(\frac{\partial Q}{\partial T}\right)_p\]
          $C_v$ refers to heat capacity when gas is held at \textbf{constant volume}, $C_p$ defined as heat capacity when gas is held at \textbf{constant pressure}.
        \end{definition}
        In future lectures we will show that $C_p > C_v$ holds true for any gas.

      \section{Thermalization}
        Alongside specific heat capacity, people have also noticed that when a hot object is placed next to a cold one, then the cold one warms up, while the hotter one decreases in temperature. After a long time, the two objects reach some final temperature $T_f$, and no more heat transfer occurs. This process is called \textit{thermalization}, and it is a core principle of thermodynamics.
        \begin{insight*}{}{}
          As we can see from the example, thermalization is an irreversible process! Once two objects have reached equilibrium, we can no longer reverse this flow of energy and restore the original temperatures of the two objects. While this is obvious with heat, this fact also holds for energy transfer, but this is not so obvious as it may seem.
        \end{insight*}
        This principle of thermalization is so important that we have a law named after it:
        \begin{theorem}{Zeroth Law of Thermodynamics}{}
          Two systems, each in thermal equilibrium with a third, must be in thermal equilibrium with each other.
        \end{theorem}
        Again, while this may seem obvious with day-to-day life occurrences, this is not a trivial result in thermodynamics. However, this does point to the fact that temperature is a well-defined concept, and that it is very closely tied with thermal equilibrium.

      \section{Defining Temperature}
        In our everyday life, we define temperature on a Celsius scale, where $0^\circ$ represents the freezing point of water and $100^\circ$ represents the boiling point of water. Then, we define every tick in between as a unit increase in the volume of water. So by definition,

        \[ 1^\circ = \frac{\Delta V}{100}\]

        This definition for temperature is actually quite problematic $-$ the expansion rates for different materials is different, and so we don't have a consistent scale. Therefore, we need a new definition for temperature.

      \section{Microstates and Macrostates}
        Let's go back to the \say{flipping a coin} example. Say instead of one coin, we flip 100 coins simultaneously, then count the distribution. A microstate, in this context, would be the specific result of each coin (either heads or tails). A macrostate, however, refers to \textit{how many heads} there are. We can easily show that there are significantly more microstates than macrostates.

        Since each coin has 2 outcomes, then by a combinatorial argument we can see that there are $2^{100}$ possible microstates, but there are only 101 possible macrostates. This should also give you a sense as to why we always talk in terms of state variables as opposed to specific configurations. We can also calculate the number of microstates in a specific macrostate of $X$ $-$ this is the same problem as we've done earlier with the particles of energy 1 or 0, and we showed that there are $n \choose x$ possible microstates. So a macrosatate of 0 has only 1 corresponding microstate, but a macrostate of 50 corresponds to $4 \times 10^{27}$ possible microstates. As a result, even though each microstate has equal probability, each macrostate does not!

        \subsection{States with Temperature}
          Now we apply the concepts of a macro and microstate to temperature. Here, our microstates are defined as the specific position, velocity, energy of each particle, and the macrostates refer to the state variables $-$ pressure, volume, energy, etc.. To simplify things, we only look at energy for now, and call $\Omega(E)$ the number of microstates with energy $E$.

          Boltzmann had the following insight: suppose we have two boxes, with energies $E_1$ and $E_2$. We call $E$ the total energy, so $E = E_1 + E_2$, and we set it to be a fixed value. In each box, since they have energies $E_i$, then we have $\Omega_1(E_1)$ microstates in the first box and $\Omega_2(E_2)$ states in the second box. 
          
          %[INSERT TIKZ HERE] 
          
          Since these boxes are independent of each other then we have that $\Omega_1(E_1) \Omega_2(E_2) = \Omega$. Now notice that the number of microstates is the highest when $E_1 = E_2$, from our combinatorial argument. Thus, it is reasonable to conclude that thermal equilibrium occurs when $\Omega_1(E_1)\Omega_2(E_2)$ is maximized.

          Doing so is quite easy, we take $\frac{d\Omega}{dE} = 0$, and this is simply product rule. What we eventually get is the following:

          \[ \frac{\partial \ln \Omega_1}{\partial E_1} = \frac{\partial \ln \Omega_2}{\partial E_2}\]

          Since these terms have something to do with temperature, they must be proportional to some power of $T$. It turns out that the correct proportionality is $T^{-1}$, so we arrive at the following:

          \begin{theorem}{Quantitative Definition of Temperature}{Quantitative Definition of Temperature}
            \[\frac{\partial \ln(\Omega)}{\partial E} = \frac{1}{k_B T}\]
            Where $k_B$ represents the Boltzmann constant.
          \end{theorem}
%%%%%%%%%%%%%%%%%%%%%%%%%%%%%%%%%%%%%%%%%%%%%%LECTURE 4%%%%%%%%%%%%%%%%%%%%%%%%%%%%%%%%%%%%%%%%%%%%%%%%%%%%%%%%
    \chapter{Lecture 4 (9/6)}
      The fourth lecture of Physics 5C was held on  \textbf{Tuesday, September 6}. It covered a review of the derivations from the previous lecture, as well as ensembles and physical examples.

      \section{Last Time: Microstates, Macrostates, and Temperature}
        Last lecture, we defined and discussed \textbf{microstates} and \textbf{macrostates}, and formulated a statistical definition of temperature. Recall that we have very many microstates for a given system, but we can group them into more manageable macrostates which are far easier to count.
        \subsection{Notation and Boltzmann}
          For a given macrostate $E$, we defined the number of microstates associated with it as $\Omega(E)$. Then, Boltzmann made a key insight: assuming each microstate is equally likely, then the probability must be proportional to the number of microstates for a given macrostate:
          $$P \propto \Omega(E)$$
          Then, Boltzmann considered two connected systems, one with $\Omega(E_1)$ microstates and the other with $\Omega(E_2)$ macrostates. The total energy in the system was then fixed at $E = E_1 + E_2$. This gives freedom for $E_1$ and $E_2$ to fluctuate, but fixes the total energy. 
          
          %[INSERT TIKZ HERE]

          
          Then, at thermal equilibrium, we want to maximize $\Omega(E_1)*\Omega(E_2)$, keeping in mind that $\frac{d\ln\Omega}{dE}$ must be constant given our thermal equilibrium condition. So, we find that:
          $$k_B\frac{d\ln\Omega}{dE} = \frac{1}{T}$$
          Now, we need to get some examples of how to calculate the macroscopic configurations to get a better understanding of this expression, and see if we can simplify it to something that looks a bit nicer.

        \section{Ensembles}
          Now that we went over what happened, let's dive into some examples to further motivate these results. To do this, we need to introduce something important.
          \subsection{Gibbs Ensembles}
            Physicist Gibbs has an idea: why not try to do some \say{mental} experiments. In other words, let's repeat an experiment to measure a property again and again. To do this, let's define a new important concept:
            \begin{definition}{Ensemble}{Ensemble}
              Defined as a large number of \say{mental photocopies} of a system, where each represents a possible state.
            \end{definition}


             There are also different types of ensembles. 
              \begin{itemize}
                \item An ensemble of a total system with \underline{fixed energy} is called a \textbf{microcanonical ensemble}
                \item An ensemble that \underline{can exchange energy} but \underline{cannot exchange particles} is called a \textbf{canonical ensemble}
                \item An ensemble that \underline{can exchange energy as well as particles} is called a \textbf{grand canonical ensemble}
            \end{itemize}
            These definitions will be useful later on when we want to refer to a collection of molecules and its properties.

            

        \section{Deriving Probability Distribution}

        Imagine two systems, one with a single particle with energy $\epsilon$, and another system at energy $T$, comprised of a large number of particles. These systems are connected in the same fashion that the two systems in the previous lecture were connected, and have a total energy $E$. Just like in the previous lecture, the number of total states is equal to the product:

        \[ \Omega(E) = \Omega(\epsilon) \Omega(E - \epsilon)\]

        Since the container with energy $\epsilon$ is a single particle, then $\Omega(\epsilon) = 1$. Thus $\Omega(E) = \Omega(E - \epsilon)$. Then, from Boltzmann's insight, we know that $P(E) \propto \Omega(E) = \Omega(E - \epsilon)$. And since $\epsilon \ll E$, then we can Taylor-expand:

        \begin{insight*}{}{}
            Whenever we have one quantity significantly smaller than the other, a Taylor expansion will almost always simplify our lives by allowing us to ignore higher-order terms. For a function $F(a + \epsilon)$, we have:

            \[ F(a + \epsilon) = F(a) + \epsilon\frac{dF}{dx}\Biggr|_{a} + \frac{\epsilon^2}{2} \frac{dF}{dx}\Biggr|_{a} + \dots \]

            Generally, since $\epsilon \ll a$, we only need to keep the first two terms. 
        \end{insight*}

        We Taylor-expand on $\ln \Omega$, so we can use our quantitative definition of temperature (theorem \ref{th:Quantitative Definition of Temperature}):

        \begin{align*}
            \ln \Omega(E - \epsilon) &= \ln \Omega(E) + \frac{\partial \ln \Omega(E)}{\partial x}\bigg|_E (-\epsilon)\\
            &= \ln \Omega(E) - \frac{1}{k_BT} \epsilon
        \end{align*}

        Combining this with our previous relation:

        \[ P(\epsilon) \propto \Omega(E) \cdot \exp{-\frac{\epsilon}{k_BT}} \implies P(\epsilon) \propto \exp{-\frac{\epsilon}{k_BT}}\]

        Now we need to normalize this such that $\int P(\epsilon) d\epsilon = 1$:

        \[ A = \frac{1}{\int \exp{-\frac{\epsilon}{k_BT}} d\epsilon }\]

        And this leads to the famous Boltzmann distribution:

        \begin{theorem}{Boltzmann Distribution}{Boltzmann Distribution}
            The probability that a particle has energy $\epsilon_i$:

            \[ P(\epsilon_i) = \frac{\exp{-\frac{\epsilon_i}{k_BT}}}{\int \exp{-\frac{\epsilon}{k_BT}} d \epsilon}\]

            There's also the discrete case, where we sum over all $i$ to get our normalization constant. This also works for any particle system, since $\Omega(E)$ had no constraints.
        \end{theorem}


        \begin{example}{Two-Energy System}{}
            To illustrate this distribution, consider a two-energy system. That is, a system with two allowed energies, 0 and $\epsilon$. Using the Boltzmann distribution, we can calculate the probability that a particle inhabits each energy:

            \begin{align*}
                P(0) &= \frac{1}{e^{-\frac{\epsilon}{k_BT}} + 1}\\
                P(\epsilon) &= \frac{e^{-\frac{\epsilon}{k_BT}}}{1 + e^{-\frac{\epsilon}{k_BT}}}
            \end{align*}

            Thus, we can calculate the expected value for the energy:

            \begin{align*}
                \mean{E} &= \sum P_iE_i = 0 \cdot P(0) + \epsilon P(\epsilon)\\
                &= \frac{\epsilon e^{-\frac{\epsilon}{k_BT}}}{1 + e^{-\frac{\epsilon}{k_BT}}}
            \end{align*}

            To see if our answer makes sense, we check the limits. As $T \to 0$, we can see that $\mean{E}$ goes to zero, meaning that only the lowest energy is populated. This makes sense, since as temperature gets lower, the probability that a particle has energy $\epsilon$ should decrease. As $T \to \infty$, $\mean{E} = \frac{1}{2} \epsilon$, meaning that both energies are equally populated. This also makes sense, since we expect that as temperature increases the population of energy levels becomes more uniform.
        \end{example}

        \section{Gas Velocity Distributions}

        Now let's take a look at the gas velocity distribution for 1D gases, then we extend this into three dimensions later. So, to find the gas velocity distribution of a gas, we have:

        \[ \mean{v_x} = \infint v_x g(v_x) dv_x\]

        Where $g(v_x)$ represents the probaiblity distribution of $v_x$ in our gas. From the previous section, we know that $g(v_x) \propto \exp{-\frac{mv_x^2}{2k_BT}}$, so:

        \[ 1 = \infint Ae^{-\frac{mv_x^2}{k_BT}} dv_x\]

        Now you could do this integral by hand, but we're physicists so you're allowed to consult an integral table for this. After integrating, we get that our normalization constant:

        \[ A = \sqrt{\frac{m}{2\pi k_BT}} \implies g(v_x) = \sqrt{\frac{m}{2\pi k_BT}}\exp{-\frac{mv_x^2}{2k_BT}}\]

        Since this probability is a Gaussian, then we know that $\mean{v_x} = 0$, which also intuitively makes sense since we should expect that our gas is an isotropic environment. What's more useful, is if we calculate $\mean{v_x^2}$:

        \[ \mean{v_x^2} = \infint v_x^2 g(v_x) dv_x = \frac{k_BT}{m}\]

        Again, we use an integral table here to get this result. Now, we are ready to extend to three dimensions. Specifically, let's find the energy of a gas system, by combining our classical definition for kinetic energy and using our previous result:

        \begin{align*}
            \mean{E_k} &= \mean{\frac{1}{2}mv^2}\\
            &= \frac{1}{2} m\mean{v_x^2 + v_y^2 + v_z^2}\\
            &= \frac{1}{2} m\left(\frac{3k_BT}{m}\right)\\
            &= \frac{3}{2} k_BT
        \end{align*}

        This is an interesting, if not surprising result of thermodynamics: the energy of a system only depends on its temperature!

        \section{Pressure}

        Now that we've looked at the energy of a gas, let's also define pressure. We can model pressure was the collisions of particles on a wall, exchanging their momentum:

        \[ F = \frac{\Delta p}{\Delta t} = \frac{\sum 2mv_x}{\Delta t}\]

        Note that we have $v_x$ instead of a general $v$ term since a collision on a wall only affects the particle's momentum in one direction. Now we look at a single wall with area $A$ with particles $v_t$ a width $\Delta t$ away from the wall. 
        
        
        %[INSERT TIKZ HERE]
        
        We can see that after a time $\Delta T$, all particles within that volume of gas would have collided with the wall, so thus: 

        \[ \Delta p = \int g(v_x) dv_x v_x \Delta t A \cdot 2mv_x \]

        And since pressure is defined as $P = \frac{\Delta p}{\Delta t A}$, we have:

        \begin{align*}
            P &= \int g(v_x) (2mv_x^2) dv_x\\
            &= n_0 k_BT
        \end{align*}

        Where $n_0$ represents the density of the gas. 

        
%%%%%%%%%%%%%%%%%%%%%%%%%%%%%%%%%%%%%%%%%%%%%%LECTURE 5%%%%%%%%%%%%%%%%%%%%%%%%%%%%%%%%%%%%%%%%%%%%%%%%%%%%%%%%
    \chapter{Lecture 5 (9/8)}
    The fifth lecture of Physics 5C was held on  \textbf{Thursday, September 8}. It introduced the First law of Thermodynamics, as well as the concept of a thermodynamic process.
    
    
    \section{Last time: Heat and Boltzmann Distribution}

    Last time, we discussed the importance of the Boltzmann distribution and how it relates to specific state variables such as the energy distribution of particles in a gas. In this lecture, we will take a short break from that framework and discuss how state variables change in a thermodynamic process. 


    \section{Defining a Thermodynamic System}

    In order for us to define a thermodynamic system, we must introduce some constraints to our system, so that we may use appropriate mathematics to describe them. Firstly, we will say that the system has reached \textbf{thermoequilibrium}

    \begin{definition}{Thermoequilibrium}{}
        When the macroscopic state variables stop changing. In other words, state variables such as $V, P,$ and $T$ are now constants. Note that these state variables are average values. 
    \end{definition}

    We will also claim that the state variables that describe the system does not depend on the history of the system. In other words, the state variables describe the state of the system itself, and it makes no connection to what those values were at an earlier point in time.

    \subsection{Requirements for State Variables}

    There are a couple of requirements for state variables. Let a system be described by parameters $\{x_1, x_2, \dots, x_n\}$, with each $x_j$ representing a state variable. Now, let some of the state variables change from $x_i \to x_f$. If we can write the following:

    \[ \Delta f = \int_{x_i}^{x_f} df = f(x_f) - f(x_i)\] 

    Then we call $df$ an \textbf{exact differential.}


    \begin{definition}{Exact Differential}{}
        An exact differential is a differential which can be expressed as the differential of a function rather than parts of a differential.
    \end{definition}
    To give a concrete example, the differential $x dy + ydx$ is an exact differential, since we can rewrite this as $d(xy)$. However, the same cannot be done for a differential such as $x dy$. Exact differentials are also path independent, whereas inexact differentials are not path independent. We can illustrate this with an example:

    \begin{example}{Exact vs. Inexact Differentials}{}
        Consider the function:

        \[ f = xy, \ df = xdy + ydx\]

        Then any path integral we write:

        \[ \Delta f = \int_{(0,0)}^{(1,1)} xdy + ydx = \int_{(0,0)}^{(1, 1)} df = f(1, 1) - f(0, 0)\] 

        And thus this demonstrates that an exact differnetial is path independent. However, consider $dg = ydx$. Now if we try to compute the same integral, we will see that it is path dependent. There are multiple ways to do this, here we will choose two different paths: 
        \begin{itemize}
            \item \textbf{First Path:} $(0,0) \to (0, 1)$ and $(0, 1) \to (1, 1)$. 
            \item \textbf{Second Path:} $(0, 0) \to (1, 0)$ and $(1, 0) \to (1, 1)$.
        \end{itemize}

        \begin{align*}
            \Delta g &= \int_{(0, 0)}^{(0, 1)} dg + \int_{(0, 1)}^{(1, 1)} dg\\
            &= 0 + 0 = 0\\
            \Delta g &= \int_{(0, 0)}^{(1, 0)} dg + \int_{(1, 0)}^{(1, 1)} dg\\
            &= 0 + 1 = 1
        \end{align*}

        Since these two integrals yield different results, then it follows that this integral is path dependent, and thus $dg$ is not an exact differential. 
    \end{example}


    \subsection{Choosing Equations of State}

    In a container of gas, there are generally four state variables that are used to describe the system: $P, V, T, U$. In the case where we're dealing with an ideal gas (which will be most of the time), the ideal gas equation $PV = nRT$ and the equation for the average energy $U = \frac{3}{2} nRT$ mean that of these four variables, we can choose two of them to be independent ones. Depending on what's given in the problem, we will choose a different set of two variables to describe the system. With this complete, we're now ready to explore the first law of Thermodynamics.

    \section{First Law of Thermodynamics}

    Her'es the statement, then we'll talk about its significance:

    \begin{theorem}{First Law of Thermodynamics}{First Law of Thermodynamics}
      If we want to increase the energy of a system, we may choose to heat up the object or do work on it. More mathematically,

      \[ \Delta U = \Delta W + \Delta Q\] 

      where $\Delta U$ represents the total energy $\Delta W$ represents the work and $\Delta Q$ represents the heat transferred to the system. A differential form of this equatino also exists: 

      \[ \dd Q = \db Q + \db W\]

      which is the same expression as the previous but with deltas replaced by an infinitesimally small change.
    \end{theorem}

    As complex as this law is, it really is just a statement about the conservation of energy that we've already learned before $-$ if you want to heat something up, some energy must be expended to do that. However, the first law also categorizes the energy into \textit{heat} and \textit{work}, which are two different things. We will explore the differences of these two concepts in a later lecture.

    There is also a differential form of the first law: $dU = \db Q + \db W$. Here we use $\db$ to define an inexact derivative, whereas the normal $d$ represents a total derivative. But how do we calculate $dU$ for a gas? Well, first, we need to define $\db W$ in terms of quantities we know - state variables. 

    Suppose we have a container of gas with a piston on one end, at equilibrium. Then, the piston is slowly pushed to compress the gas - in fact, so slowly the piston feels no acceleration! We will also assume no friction.

    [INSERT TIKZ HERE]


    Now we ask: how much work is being done to the system? We know that the force exerted is $F = P \cdot A$, so

    \begin{align*}
      \db W &= \vec F \cdot d\vec{x}\\
      &= P \cdot \underbrace{A dx}_{-dV}\\
      &= -P dV\\
      \therefore dU = \db Q - P dV
    \end{align*}

    Which we summarize in the theorem below: 

    \begin{theorem}{work done by gas}{work done by gas}
      In a thermodynamic process, the work done by the gas is always calculated as 

      \[ \db W = -P \dd V\]
    \end{theorem}


    \section{Heat Capacity for Gases}

    We know from earlier lectures that $c = \frac{\db Q}{dT} = \frac{dU + p dV}{dT}$. We know that

    \begin{align*}
      dU &= \left(\frac{\partial U}{\partial T}\right)_V dT + \left(\frac{\partial U}{\partial V}\right)_T dV\\
      &= \left(\frac{\partial U}{\partial T}\right)_V dT + \left[ \left( \frac{\partial U}{\partial V}\right)_T + p\right] dV
    \end{align*}

    Notice that the specific heat capacity of gases is not so simple, as discussed earlier. When we change the temperature of a gas, there are multiple variables that can change, and as a result we get a more complex expression for specfic heat capacity. We split this into specific heat at constant volume $c_v$ and constant pressure $c_p$:

    \begin{align*}
        c_p &= \left(\frac{\db Q}{dT}\right)_V = \left(\frac{\partial V}{\partial T}\right)_V\\
        c_v &= \left(\frac{\partial V}{\partial T}\right)_V + \left[\left(\frac{\partial U}{\partial V}\right)_T + p \right]\left(\frac{\partial V}{\partial T}\right)_P
    \end{align*}

    Notice that in this formula, $c_p > c_v$ for any gas, since $c_p = c_v + \text{more terms}$. For an ideal gas, $c_v = \frac{3}{2} nR$, and $c_p = \frac{5}{2} nR$. From this, we can obtain a useful constant for ideal gases: 

    \[ \frac{c_p}{c_v} = \frac{5}{3} \equiv \gamma\]

    $\gamma$ is used in many textbooks to denote this constant, so we will do so here as well. 


    \section{Processes}

    In the next lecture, we will discuss thermodynamic processes and how they alter the state variables in a system. We will define them here so that we may use them later:

    \begin{definition}{Isothermal Process}{Isothermal Process}
      An isothermal processes is a process that occurs at fixed temperature. 
    \end{definition}


    \begin{definition}{Adiabatic Process}{Adiabatic Process}
      An adiabatic process is one where no heat exchange occurs with the outside environment.
    \end{definition}


    \begin{definition}{Reversibility}{Reversibility}
      Reversibility points to the idea that we perform a process sufficiently slowly so that the gas remains in equilibrium throughout the process. This is also sometimes called a quasi-static evolution.
    \end{definition}

%%%%%%%%%%%%%%%%%%%%%%%%%%%%%%%%%%%%%%%%%%%%%%LECTURE 6%%%%%%%%%%%%%%%%%%%%%%%%%%%%%%%%%%%%%%%%%%%%%%%%%%%%%%%%
    \chapter{Lecture 6 (9/13)}
    The sixth lecture of Physics 5C was held on  \textbf{Tuesday, September 13}. It introduced thermodynamic processes and relationships between state variables in these proceses. 


\section{Last Time: First Law of Thermodynamics}

Last time, we discussed the First law of Thermodynamics, and there we established that heat and work are forms of energy. Furthermore, we also introduced two reversible processes: the \textbf{isothermal} and \textbf{adiabatic} processes. In this lecture, we will discuss these proceses in greater detail. 

\section{The Isothermal Process}

An isothermal process, as discussed in definition \ref{def:Isothermal Process} is one where the overall temperature of a gas does not change. There is only one way in which this can be accomplished: when the volume increases, the pressure must decrease in order to account for the unchanging temperature. We can see this in the following $P$-$V$ diagram:

[INSERT TIKZ GRAPH OF DIAGRAM + GRAPHS]

Note that we have $U = \frac{3}{2}nRT = k_BT$, so if temperature remains constant, we come to a very important conclusion about isothermal processes: 

\begin{theorem}{constant isothermal energy}{constant isothermal energy}
  In a thermodynamic system, the energy is defined as $U = \frac{3}{2}nRT$. So if $T$ is to remain constant, as is in isothermal processes, then 

  \[ \Delta U = 0\]
\end{theorem}

Now if we look at the First law of Thermodynamics: 

\begin{align*}
  \dd U &= \db Q + \db W \\
  \therefore \db Q &= -\db W
\end{align*}

In other words, there is a \textit{direct} conversion of work into heat in an isothermal process! Note the negative sign in $-\db W$, which denotes that the system does work \textit{to} the outside. And since we know from theorem \ref{th:work done by gas} that $\db W = -P \dd V$, then we know that 

\[ \db Q = P \dd V \implies \Delta Q = \int_{V_1}^{V_2} P \dd V\]

And if we want to take this one step further, we know from the ideal gas law (definition \ref{th:ideal gas law}) that $P = \frac{nRT}{V}$, then 

\[ \Delta Q = \int_{V_1}^{V_2} \frac{nRT}{V} \dd V  = nRT \ln\left(\frac{V_2}{V_1}\right)\]

From here, we conclude that the amount of work done by a gas on the outside is proportional to $\ln \left(\frac{V_2}{V_1}\right)$.

\section{The Adiabatic Process}

Following from definition \ref{def:Adiabatic Process}, we know that an adiabatic process has the property that $\Delta Q = 0$. That is, there is no heat exchange between the system and the outside. Generally, this is described in problems as an \textbf{isolated system.} 

Now let's to back to first law. Since $\Delta Q = 0$, then our law becomes $\dd U = \db W$. Furthermore $\dd U = \frac{3}{2} nR \dd T$, so we can write:

\begin{align*}
\frac{3}{2} nR \dd T &= -\frac{nRT}{V} \dd V\\
\therefore \frac{\dd T }{T} &= -(\gamma - 1) \frac{\dd V}{V}
\end{align*}

So now we can integrate: 

\begin{align*}
\int_{T_1}^{T_2} \frac{1}{T} \dd T &= -\int_{V_1}^{V_2} (\gamma -1) \frac{\dd V}{V}\\
\ln \left(\frac{T_2}{T_1}\right) &= -(\gamma -1) \ln\left(\frac{V_2}{V_1}\right)
\end{align*}

Rearranging, 

\begin{align*}
\ln\left(\frac{T_2}{T_1}\right) + \ln\left(\frac{V_2}{V_1}\right)^{\gamma -1} &= 0\\
\ln\left(\frac{T_2V_2^{\gamma -1}}{T_1V_1^{\gamma -1}}\right) &= 0\\
\therefore T_2V_2^{\gamma -1} &= T_1V_1^{\gamma -1}
\end{align*}

Here, we can see that $TV^{\gamma - 1}$ is a constant, which is an important result for adiabatic processes: 

\begin{theorem}{Constant in an Adiabatic Process}{TV constant for adiabat}
In an adiabatic process, the quantity $TV^{\gamma -1}$ remains constant throughout all time:

\[ T_1V_1^{\gamma -1} = T_2V_2^{\gamma - 1}\]
\end{theorem}

Now you might be wondering: what is the point of defining different processes if we can just reverse one to the other? If this were true, then certainly be no use in choosing a specific way to achieve a result, since all results should be the same. However, as we will discover, \textit{different forms of energy are not equal}. And with that, let's take a look at the second law of thermodynamics. 

\section{Second Law of Thermodynamics}

In essence, the Second law of Thermodynamics shows us that not all forms of energy are equal. In other words, there are some forms of energy loss which cannot be recovered. There are two formulations for the second law of thermodynamics, which we will show later to be equivalent: 

\begin{theorem}{Second Law of Thermodynamics}{Second Law of Thermodynamics}
The two formulations for the second law of thermodynamics are as follows: 

\begin{itemize}
  \item \textbf{Clausius:} No process is possible whose \underline{sole result} is the transfer of heat from cold to hot.
  \item \textbf{Lord Kelvin:} No process is possible whose \underline{sole result} is heat conversion from heat to work.
\end{itemize}
\end{theorem}

Well we've just said a minute ago that an isothermal system directly converts heat to work! How is that possible? The answer lies in the fact that the state of the gas must change in order to do work, so the second law is not in fact violated. However, this isn't really something that we need to worry about too much going forward. Furthermore, we will show in the next lecture that these two statements, despite their differences, are in fact equivalent.

\section{Processes} 

In this section, we will study cyclic systems - that is, systems where the initial and final states are the same. These are also called \textit{engines}, which is the term we will use moving forward. To start, we will study the simplest engine: the Carnot cycle. 

The carnot cycle is one which is formed by two adiabatic processes and two isothermal processes. On a PV diagram, it looks like the diagram below, with points $A$, $B$, $C$ and $D$ denoting the points where the type of process changes:

[INSERT TIKZ HERE]

We know how adiabatic and isothermal processes work, so we can write out the following set of equations:

\begin{align*}
A &\to B: \ \Delta Q = nRT \ln \left(\frac{V_B}{V_A}\right)\\
B &\to C: \ \Delta Q = 0 \implies T_hV_B^{\gamma -1} = T_LV_C^{\gamma -1} \\
C &\to D: \ \Delta Q = -nRT\ln\left(\frac{V_D}{V_C}\right) = nRT \ln \left(\frac{V_C}{V_D}\right)\\
D &\to A: \ \Delta Q = 0 \implies T_LV_D^{\gamma -1} = T_hV_A^{\gamma -1}
\end{align*}

Here, a positive $\Delta Q$ denotes absorbing heat. If we combine the equations $B \to C$ and $D \to A$, then we get:

\[ \left(\frac{V_C}{V_B}\right)^{\gamma -1} = \left(\frac{V_D}{V_A}\right)^{\gamma -1}\]

Which implies $V_AV_C = V_DV_B$ and consequently $\frac{V_C}{V_D} = \frac{V_B}{V_A}$. This relation is important, becuase by combining the equations $A \to B$ and $C \to D$, we get:

\begin{align*}
\frac{Q_h}{Q_L} &= \frac{nRT_h \ln \left(\frac{V_B}{V_A}\right)}{nRT_L \ln \left(\frac{V_C}{V_D}\right)}\\
&= \frac{T_h}{T_L}
\end{align*}

This is a surprisingly simple relationship for such a complicated system! As a reuslt, this is also a very powerful result.

%Possibly insert stuff about schematic diagrams here

\section{General cyclic processes}

For any general cyclic process, we have $\Delta U = 0$ by definition of being cyclic, since the initial and final states must be identical. Thus, this means that $\db Q + \db W = 0$, so 

\[ \Delta Q = \frac{Q_h}{Q_L} = \frac{T_h}{T_L} = W\]

Further, the efficiency $\eta$ is defined as 

\[ \eta \equiv \frac{W}{Q_h} = \frac{Q_h - Q_L}{Q_h} = 1 - \frac{T_L}{T_h}\]

Just to give a sense of real engines, they normally operate at around $T_h \approx 800$K and $T_L \approx 300$K, so thier efficiency $\eta_{max} \approx 60\%$.
%%%%%%%%%%%%%%%%%%%%%%%%%%%%%%%%%%%%%%%%%%%%%%LECTURE 7%%%%%%%%%%%%%%%%%%%%%%%%%%%%%%%%%%%%%%%%%%%%%%%%%%%%%%%%
    \chapter{Lecture 7 (9/15)}
    The seventh lecture of Physics 5C was held on  \textbf{Thursday, September 13}. It is a continuation of our study of thermodynamic engines and a conclusion to our discussion on the Second law of Thermodynamics.

    \section{Last Time: The Second Law and Engines}

    Last lecture, we derived some fairly interesting equations for the Carnot engine: 

    \[ \frac{Q_h}{Q_L} = \frac{T_h}{T_L} \implies \frac{W}{Q_h} = 1 - \frac{T_L}{T_h}\] 

    In this lecture, we will continue our discussion of the Carnot engine and also discuss the equivalence of Clausius and Kelvin's statement regarding thermodynamic processes.

    \section{Carnot's Theorem}

    After proposing the Carnot engine, he also proposed the following theorem: 

    \begin{theorem}{Carnot's Theorem}{}
      Of all heat engines working between two temperature $T_h$ and $T_L$, none are more efficient than the Carnot engine.
    \end{theorem}

    The proof of this statement is rather clever: consider a more efficient engine $E$ with an efficiency $\eta_E > \eta_{carnot}$. Then, this means that 

    \[ \frac{W}{Q_h'} > \frac{W}{Q_h} \implies Q_h > Q_h'\]

    Now consider a new engine which consists of our new engine connected to a Carnot engine run in reverse, as shown in the following schematic: 

    [INSERT TIKZ HERE]

    And since $Q_h > Q_h'$, then this means that there is a net flow from $T_l \to T_h$ to balance out the heat flow, but this necessarily means that there is flow of heat from $T_l$ to $T_h$, in direct violation of the Second law of Thermodynamics. Therefore, engine $E$ cannot exist.

    \begin{insight*}{}
      Note that the only fact that we've used in this entire proof is the fact that the Carnot engine is reversible. Therefore, this proof actually holds for any reversible engine! In fact, this hints at a deeper property of Carnot engines: that they all have the same efficiency $\eta_{carnot}$! 
    \end{insight*}

    \section{Equivalence of Kelvin and Clausius} 

    Last lecture we intrduced Clausius and Kelvin's statements about the Second law of thermodynamics. Here, we aim to show that the are in fact equivalent statements. To do this, we can equivalently show that whenever Kelvin's statement is violated, then so is Clausius' statement. 

    Now suppose we have an engine $E$ which solely converts work to heat (recall that this is in direct violation of Kelvin's statement). Then, we connect this engine to the Carnot engine, to create the following combined engine: 

    [INSERT TIKZ HERE] 

    But then if we compute the efficiency of this engine: 

    \[ \eta_M = \frac{W}{Q_h} = 100\%\]

    which is more efficient then a Carnot engine, which is in direct violation of Clausius' statement. Similarly, to prove the reverse direction, we construct an engine $E$ which transfers heat from a cold reservoir $T_L$ to $T_h$ in direct violation of Clausius' statement. Now if this were possible, then it means that we could connect this to a Carnot engine in the following way:

    [INSERT TIKZ HERE] 

    Since $Q_l$ is the same on both sides, this means that the $T_l$ medium does not change, and therefore all the energy from $Q_h$ to $Q_l$ is doing work, which is in direct violation of Kelvin's statement. Therefore, this engine cannot exist either.


    \section{Clausius' Theorem} 

    Now we come to another important theorem by Clausius. Consider the equation relating the heat transfer in a Carnot cycle

    \[ \frac{Q_h}{T_l} = \frac{T_h}{T_l}\] 

    We can rearrange this equation into

    \[ \frac{Q_h}{T_h} - \frac{Q_l}{T_l} = 0\] 

    The first term in this equation refers to the heat absorbed, and the second term refers to the heat released (written as negative to denote that heat is leaving the engine). Therefore, we can also rewrite this as: 

    \begin{equation}\label{integral for carnot engine}
      \oint \frac{\dd Q}{T} = 0
    \end{equation} 

    Now what if we have a general cycle? If the cycle is reversible, we can imagine this cycle as being built out of a bunch of mini Carnot engines:

    [INSERT TIKZ HERE] 

    And since each Carnot cycle satisfies the relation \ref{integral for carnot engine}, then this is true for any general cycle as well!

    \begin{theorem}{Clausius' Theorem}{Clausius' Theorem}
      A general reversible cycle can be written as the sum of many Carnot cycles: 

      \[ \oint \frac{\dd Q}{T} = \sum_i \oint_i \frac{\dd Q}{T}\]

      And since each integral is zero because they're Carnot cycles, then:

      \[ \oint \frac{\dd Q}{T} = 0\]
    \end{theorem}

    Taking this one step further, consider a irreversible process and some reversible process. Then this means that 

    \[ \frac{\Delta Q}{T_h} - \frac{\Delta Q}{T_l} < 0 \implies \oint \frac{\dd Q}{T} < 0\]

    So therefore, while all reversible processes have $\oint \frac{\dd Q}{T} = 0$, all irreversible processes have $\oint \frac{\dd Q}{T} < 0$!

    \section{Entropy}

    Now let's take a look at the equation we derived in the previous section for reversible processes:

    \[ \oint \frac{\dd Q}{T} = 0\] 

    This implies that the quantity $\frac{\dd Q}{T}$ is an exact differential, so let's define it as such:

    \[ dS \equiv \frac{\dd Q}{T}\] 

    and we call this new quantity $\dd S$ as \textit{entropy}. Notice immediately some consequences of this: during an adiabatic process, we have $\dd Q = 0$ by definition. Therefore, an adiabatic process does not cause a change in entropy! Further, let's look at an irreversible process: 

    [INSERT TIKZ HERE]

    In order to calculate the change in entropy, we introduce a reversible process which goes from $A$ to $B$, which creates a hypothetical \say{cycle}. Further, we know that the entropy change for a reversible process from state $A \to B$ is the same as that from $B \to A$, so we can write: 

    \[ \oint \frac{\dd Q}{T} = \int_A^B \frac{\dd Q}{T} - \int_{A}^B \frac{\dd Q_{rev}}{T} \le 0 \]

    So therefore

    \[ \int_A^B \frac{\dd Q}{T} \le \int_A^B \frac{\dd Q_{rev}}{T} = \Delta S\]

    And so therefore 

    \[ \Delta S \ge \int_A^B \frac{\dd Q}{T}\]

    In other words, $\Delta S = 0$ only when the process is reversible - otherwise, $\Delta S > 0$.

    \subsection{Discussion on Entropy}

    Let's now consider an isolated system where $\Delta Q = 0$, but this need not be an adiabatic process. Then, this means that $\Delta S \ge 0$ all the time, so entropy only increases!

    In order for entropy to not increase, then it directly follows that all processees inside the the system must be reversible. Thus, if we think of the universe as a thermodynamic system (we can do this, why not?), $\Delta S_{\text{universe}} > 0$, since there are many irreversible processes. But this also direclty implies that the universe was created with very low entropy? That's more a philosophical question rather than a physical one to ponder about.

%%%%%%%%%%%%%%%%%%%%%%%%%%%%%%%%%%%%%%%%%%%%%%LECTURE 8%%%%%%%%%%%%%%%%%%%%%%%%%%%%%%%%%%%%%%%%%%%%%%%%%%%%%%%%
    \chapter{Lecture 8 (9/20)}

    The ninth lecture of Physics 5C was held on  \textbf{Thursday, September 13}. It discusses entropy in greater detail, and also redefines the first law in terms of new quantities. 

\section{Last time: Second Law and Entropy} 

Last time, we introduced the quantity of Entropy

\[ S \equiv \frac{\dd Q_{rev}}{T}\]

If anythinng, entropy should strike us as a strange variable, since for any isolated system, the entropy always increases: 

\[ \dd Q = 0 \implies \dd S \ge 0\]

\section{Entropy and heat flow} 

Imagine two systems, one of which is a reservoir and the other is a small system. The large reservoir has temperature $T_R$ and the small system has temperature $T_s$, as shown below: 

[INSERT TIKZ HERE]

We let $T_R > T_s$. Calculating the entropy in systems $R$ and $S$: 

\[ \Delta S_R = \int \frac{\dd Q}{T_R} = -\frac{\Delta Q}{T_R}, \phantom{aaa} \Delta S_S = \int_{T_s}^{T_R} \frac{\dd Q}{T}\]

And since we have $\frac{\dd Q}{\dd T} = c$ so $\dd Q = c \  \dd T$, then we have 

\[ \Delta S_s = \int_{T_s}^{T_R} \frac{c\dd T}{T} = c\ln \left(\frac{T_R}{T_s}\right)\]

By that same token 

\[ \Delta S_R = -\frac{c}{T_R}(T_R - T_s)\]

And so calculating the total entropy, 

\begin{align*}
    S_{\text{total}} &= \Delta S_R + \Delta S_s\\
    &= c\left[\ln \frac{T_R}{T_s} - \frac{1}{T_R}(T_R - T_s)\right] \ge 0
\end{align*}

The last step is true because $T_R > T_s$ is a vital assumption that we've made at the beginning of the derivation. Since this is true, then the second law is not violated. 

\section{Redefining First Law}

Recall from previous derivations that $\dd U = \dd Q + \dd W$, and specifically for a reversible process, $\dd W = -p \dd V, \dd Q = T \dd S$, the latter of which we just derived. Therefore, we can now write the first law as

\[ \dd U = T \dd S - p \dd V\]

which gives us a relation between state variables $V$ and $S$, so if we know $\dd V$ and $\dd S$ then we can calculate $\dd U$. Further, based on simple differential properties, 

\[ \dd U = \left(\frac{\partial U}{\partial S}\right)_V \dd S + \left(\frac{\partial U}{\partial V}\right)_S \dd V\] 

And so we get the relations 

\[ T = \left(\frac{\partial U}{\partial S}\right)_V \phantom{aa} -p = \left(\frac{\partial U}{\partial V}\right)_S\]

Which also gives us 

\[ \frac{-p}{T} = \frac{-\left(\frac{\partial U}{\partial S}\right)_S}{\left(\frac{\partial U}{\partial S}\right)_V} = -\left(\frac{\partial U}{\partial V}\right)_S\left(\frac{\partial S}{\partial U}\right)_V\] 

\begin{insight*}{}
    Alternatively, we could have written $S(U, V)$ to get the relations

    \[ \frac{p}{T} = \left(\frac{\partial S}{\partial V}\right)_V \implies \left(\frac{\partial S}{\partial V}\right)_U = -\left(\frac{\partial U}{\partial V}\right)_S\left(\frac{\partial S}{\partial U}\right)_V\] 

    And if we do the final combination of $V(S, U)$, we can get a similar relation. If we combine all three relations together, we get

    \[ 1 = -\left(\frac{\partial U}{\partial V}\right)_S \left(\frac{\partial S}{\partial U}\right)_V \left(\frac{\partial V}{\partial S}\right)_V\] 

    Which is a quite a special relation! In fact, this is a special relation about all multivariable systems. 
\end{insight*}

Now let's return to our definition of entropy. Based on physical experiments, we've discovered that $S$ is maximized for an equilibrium state. Recall that Boltzmann theorized that the equilibrium state is given by the maximum value of $\Omega$, the number of microstates. Is there a connection between these two? Let's explore this a bit further. 

Recall that the number of microstates is given by the relation:

\begin{equation}\label{boltzmann temperature equation}
    \frac{\dd \ln \Omega}{dE} = \frac{1}{k_BT}
\end{equation}

which also happens to be our formal definition of temperature. Now remember our new definition of the first law: 

\[ \dd U = T \dd S - p \dd V\] 

From which we can extract 

\[ T = \left(\frac{\partial U}{\partial S}\right)_V\] 

We can do this by taking $\dd S$ on both sides, then hold $V$ constant in the equation so $\frac{\dd V}{\dd S} = 0$. Now, the total energy $E$ and the quantity $U$ are actually related (since they both represent total energy in some capacity), therefore we can rewrite equation \ref{boltzmann temperature equation} as: 

\[ \frac{1}{k_BT} = \left(\frac{\dd \ln \Omega}{\dd U}\right)_V\] 

Now we can combine the two expressions by substituting in $\dd U$ and get:

\[ S = k_B \ln \Omega\] 

Which leads us to a very famous equation derived by Boltzmann, described in brief below: 

\begin{theorem}{Boltzmann Entropy Relation}{Boltzmann Entropy Relation}
    Given a thermodynamic system, its entropy is related to the number of microstates whose energy equals $E$ by 

    \[ S = k_B \ln \Omega\]
\end{theorem}

As an aside, this equation was so influential at the time that it is literally imprinted on Boltzmann's grave. Also note that this formulation does not violate the second law of thermodynamics, since the number of microstates $\Omega$ is always at least 1. Now, let's take a look at a couple of examples relating to calculating entropy. 

\begin{example}{Free Expansion of Gas}{Free Expansion of Gas}
    Suppose we have an isolated system consisting of two boxes of volume $V_0$ connected by a valve. The valve is initially closed and gas is contained in the left box, and the right box remains empty. Now suppose we open the valve and let the gas freely expand. 

    Clearly, this process is not reversible, and since the system is isolated, there is no change in energy of the system: $\Delta U = 0$. If this is an ideal gas, then the change in temperature must then also be zero. Therefore, this is an isothermal process. 

    If we write out equations of state:

    \begin{align*}
        P_iV_0 &= nRT_i\\
        P_v(2V_0) &= nRT_f
    \end{align*}

    And since $T_i = T_f$, then $P_F = \frac{1}{2} P_i$. Now that we've established the characteristics of the initial and final state, how do we calculate the change in entropy? The idea is to hypothesize that this system is reversible, and reverse it back to the initial state. The change in entropy in this process is well known, and because by adding this path we've created a cycle, we know that the total change in entropy is zero. 

    \[ \dd S = \frac{\dd Q}{T} = \frac{p \dd V}{dT}\]

    Note here that $\dd Q$ is positive because work is being done \textit{to} the system, not \textit{by} the system. Therefore, 

    \begin{align*}
        \Delta S &= \int_{V_0}^{2V_0} \frac PT \dd V\\
        &= \int_{V_0}^{2V_0} \frac{nRT}{V} \frac{1}{T} \dd V\\
        &= nR \ln \left(\frac{2V_0}{V_0}\right)\\
        &= nR \ln 2
    \end{align*}

    This is an interesting result since it implies that the change in entorpy does not depend on common state state variables we expect it to! In fact, the only state variable in this equation is $n$.
    
\end{example}

While this result is rather shocking, it also makes sense even according to our previous framework. Consider the same system as in example \ref{ex:Free Expansion of Gas} except we consider a single particle. 

[INSERT TIKZ HERE]

Initially the particle has $\Omega_0$ states, and once the valve is open, it now has $2\Omega_0$ states, since the particle can either be in the left or right chamber. Thus, for $N$ molecules, the number of microstates is $\Omega_0^N$ and the final state is $(2\Omega_0)^N$, so we get:

\begin{align*}
    S_i &= k_B \ln \Omega^N\\
    S_f &= k_B \ln (2 \Omega)^N
\end{align*}

So calculating the change in entropy $\Delta S = S_f - S_i$: 

\begin{align*}
    \Delta S &= S_f - S_i\\
    &= k_B \ln (2\Omega_0)^N - k_B \ln \Omega_0^N\\
    &= k_B \ln \left(\frac{2 \Omega_0}{\Omega_0}\right)^N\\
    &= Nk_B \ln 2\\
    &= nR \ln 2
\end{align*}

And so we achieve the same result. 

\section{Particle Mixing Problem} 

Here we've come to a fundamental dilemma of free expansions. Consider the containers in \ref{ex:Free Expansion of Gas}, except this time we have two gases, gas 1 and gas 2 in either chamber. The question we want to answer is what is the entropy change when the gases are mixed? 

The way to think about this problem is to imagine it as two free expansions - in other words replace the valve with a special membrane that only allows one gas through on one side, and one gas through on the other side. In this way, the change in entropy is just twice that of a singular free expansion

\[ \Delta S = 2nR \ln 2\]

However, what if gas 1 and gas 2 are the same? Then, there is no way to distinguish between a particle in the left chamber and the right chamber, and so therefore the change in entropy is zero! In other words, this hints at a very deep fact about entropy: it only increases if we can \textit{detect} that a change has occurred, such as in the case with two gases. If it's the same gas, the system looks identical regardless of what we do with it, and so there is no observed entropy change.




%%%%%%%%%%%%%%%%%%%%%%%%%%%%%%%%%%%%%%%%%%%%%%LECTURE 9%%%%%%%%%%%%%%%%%%%%%%%%%%%%%%%%%%%%%%%%%%%%%%%%%%%%%%%%
    \chapter{Lecture 9 (9/22)}

    The ninth lecture of Physics 137A was held on \textbf{Wednesday, September 14}. It explored the solutions to the \schrodinger equation given some potential $V(x)$. 

\section{Last time: Solutions to the Schrodinger Equation}

Last time, we talked about how we should interpret solutions to the \schrodinger equation, and specifically we derived how energy eigenstates evolve in time: 

\[ \psi_E(\vec{r}, t) = \psi_E\left(\vec r\right) e^{-iEt/\hbar}\] 

where $\psi_E$ satisfies the equation $\hat H \psi_E = E \psi_E$. Further, we also now know that a general state can be written as a superposition of eigenstates: 

\[ \psi(\vec{r}, t) = \sum c_E \psi_E\left(\vec r\right) e^{-iEt/\hbar}\] 

Here, we use the generalized position vector $\vec r$: the \schrodinger equation is valid for all space (since we're not limited to 1D quantum mechanics in the most general case), as we'll see later.

\section{Different Types of Energies}

Now let's look at how $V(x)$ changes the nature of our solutions. Consider the following potential 

[insert tikz here]

Now recall the 1D \schrodinger equation: 

\[ \left[ -\frac{\hbar}{2m} \frac{d^2}{dx^2} + V(x) \right] \psi_E(X) = E\psi_E(x)\]

There are four different cases that we need to explore here for a given energy $E$: 

\begin{itemize}
    \item Case 1: $E < V_{min}$
    \item Case 2: $E > V_{min}$, $E < V_-$ 
    \item Case 3: $E > V_-$, $E < V_+$
    \item Case 4: $E > V_+$
\end{itemize}

\subsubsection*{Case 1}

We rearrange the \schrodinger equation to see this a little better: 

\[ \frac{d^2}{dx^2} \psi(x) = \frac{2m}{\hbar} \left(V(x) - E\right) \psi(x)\] 

And so here, we get that $V(x) - E > 0$, and so therefore $\frac{d^2\psi}{dx^2}$ and $\psi(x)$ now have the same sign! But this means that there aren't any normalized functions that have that property (as they are exponentials), so we conclude that no such solutions can exist.

\subsubsection*{Case 2}

We know from the previous case that any time $V > E$, then we get exponential solutions. This is still true in our second case, except there are now regions where $E > V$. Here, as we've seen in the case of the free particle, will generate oscillatory solutions. To combine these two together, we enforce the condition that $\psi$ must be a continuous quantity for all $x$. 

\begin{insight*}{}{}
    This restriction of continuity actually means that only specific functions solve the \schrodinger equation, and these functions correspond to specific energies (we'll see this later). This is the \textit{fundamental} reason why energy levels are quantized!
\end{insight*}

%%%%%%%%%%%%%%%%%%%%%%%%%%%%%%%%%%%%%%%%%%%%%%LECTURE 10%%%%%%%%%%%%%%%%%%%%%%%%%%%%%%%%%%%%%%%%%%%%%%%%%%%%%%%
    \chapter{Lecture 10 (9/27)}

    The ninth lecture of Physics 5C was held on  \textbf{Tuesday, September 18}. 

\section{Last Lecture: The Equipartition Theorem}

Last time, we dicsussed the equal partition theorem - namely, the fact that the energy of a particular gas is spread eqully across all degrees of freedom that have quadratic dependence on some variable. More specifically, we found that each degree of freedom gives us a factor of $k_BT/2$. Now, we will develop a framework which allows us to calculate any state variable given any starting point. 

In order to accomplish this, the first thing we need is to choose two variables that will uniquely define our system. Here, we will use the temperature represented by $T$, and what we will call the \textit{partition function}, denoted by $Z$. The partition function is defined as follows: 

\[ Z = \sum_{\alpha} e^{-\beta E_\alpha}\] 

Where $E_{\alpha}$ refers to all the energy levels, and $\beta = \frac{1}{k_BT}$. Effectively, this defines a system with parameters $(T, Z)$, from which we will calculate all other state variables. 

\begin{example}{Partition Function Practice}{}
    Suppose we have a system where there are an infinite number of allowed energy levels, each separated by an energy difference $\Delta$. (In the case of a quantum harmonic oscillator, we hvae $\Delta = \hbar \omega$.) Our partition function sums over all possible energies, so therefore: 

    %[INSERT TIKZ HERE]

    \begin{align*}
        Z = \sum_i e^{-\beta E_i} &= 1 + e^{-\beta \Delta} + e^{-2\beta \Delta} + \cdots\\
        &= e^{-\beta} + e^{-\beta \Delta} + \left( e^{-\beta \Delta}\right)^2 + \cdots \\
    \end{align*}

    But now notice that this is an infinite geometric series! This series converges since $e^{-n\beta \Delta} < 1$, and working out the geometric series gives:

    \[ Z = \frac{1}{1-e^{-\beta \Delta}}\]
\end{example}

\subsection{Calculating Functions of State}

\subsubsection*{Internal Energy $U$}

Now we move on to calculating the functions of state, starting with the internal energy. We know that the internal energy of a system is given by 

\[ U = \frac{\sum_i E_i e^{-\beta E_i}}{\sum_i e^{-\beta E_i}}\] 

Note the denominator here is exactly equal to our partition function $Z$. However, how do we cleanly express the numerator in terms of $Z$? Here, we use a clever bit of mathematics: notice that: 

\[ E_i e^{-\beta E_i} = - \frac{\partial}{\partial \beta} e^{-\beta E_i}\] 

and since partial derivatives are linear, then we get that 

\[ \sum_i E_i e^{-\beta E_i} = -\frac{\partial}{\partial \beta} Z\] 

And so therefore, we can express $U$ as:

\[ U = \frac{-\frac{\partial}{\partial \beta} Z}{Z}\] 

\subsubsection*{Entropy $S$}

Here, we will use the definition of $S = -k_B \sum_i P_i \ln P_i$ instead of our other expression $k_B \ln \Omega$, mainly becuase $\Omega$ is difficult to define for a general system. Recall that $P_i$ is defined as

\[ P_i = \frac{e^{-E_i/k_BT}}{\sum_i e^{-E_i/k_BT}} = \frac{e^{-\beta E_i}}{Z}\] 

Adn so therefore, we have the relation $\ln P_i = -\beta E_i - \ln Z$ using the relations with logarithms. Thus, we can write: 

\begin{align*}
    S = -k_B \sum_i P_i (-\beta E_i - \ln Z) &= -k_B \left[-\beta \underbrace{\sum_i E_i P_i}_{\mean E = U} - \underbrace{\sum_i P_i}_{= 1} \ln Z\right]\\
    &= -k_B \left( -\frac{1}{k_BT} U\right) + k_B \ln Z\\
    &= \frac{U}{T} + k_B \ln Z
\end{align*}

which is, all things considered, a relatively simple expression for such an abstract concept like entropy. 

\subsubsection*{Combination of 1st and 2nd law}

We know that the firat law can be expressed as $dU = T dS - pdV$, but these variables aren't exactly the easiest thing to calculate because we don't really know how to quantify $dS$. Instead, we define the \textit{Helmholtz function} $F$, defined as $F = U - TS$

\begin{insight*}{}
    Because $U$, $T$ and $S$ are all well defined functions of state, then this makes $F$ also an equally well defined function of state! This is important to know because it means that for any thermodynamic system, we will be able to define an $F$. 
\end{insight*}



%%%%%%%%%%%%%%%%%%%%%%%%%%%%%%%%%%%%%%%%%%%%%%LECTURE 11%%%%%%%%%%%%%%%%%%%%%%%%%%%%%%%%%%%%%%%%%%%%%%%%%%%%%%%
    \chapter{Lecture 11 (9/29)}

%%%%%%%%%%%%%%%%%%%%%%%%%%%%%%%%%%%%%%%%%%%%%%LECTURE 12%%%%%%%%%%%%%%%%%%%%%%%%%%%%%%%%%%%%%%%%%%%%%%%%%%%%%%%
    \chapter{Lecture 12 (10/4)}

%%%%%%%%%%%%%%%%%%%%%%%%%%%%%%%%%%%%%%%%%%%%%%LECTURE 13%%%%%%%%%%%%%%%%%%%%%%%%%%%%%%%%%%%%%%%%%%%%%%%%%%%%%%%
    \chapter{Lecture 13 (10/6)}

%%%%%%%%%%%%%%%%%%%%%%%%%%%%%%%%%%%%%%%%%%%%%%LECTURE 14%%%%%%%%%%%%%%%%%%%%%%%%%%%%%%%%%%%%%%%%%%%%%%%%%%%%%%%
    \chapter{Lecture 14 (10/11)}

%%%%%%%%%%%%%%%%%%%%%%%%%%%%%%%%%%%%%%%%%%%%%%%%%%%%%%%%%%%%%%%%%%%%%%%%%%%%%%%%%%%%%%%%%%%%%%%%%%%%%%%%%%%%%%%
%%%%%%%%%%%%%%%%%%%%%%%%%%%%%%%%%%%%%%%%%%%QUANTUM MECHANICS%%%%%%%%%%%%%%%%%%%%%%%%%%%%%%%%%%%%%%%%%%%%%%%%%%%
  \part{Quantum Mechanics}
  
%%%%%%%%%%%%%%%%%%%%%%%%%%%%%%%%%%%%%%%%%%%%%%LECTURE 15%%%%%%%%%%%%%%%%%%%%%%%%%%%%%%%%%%%%%%%%%%%%%%%%%%%%%%%
    \chapter{Lecture 15 (10/18)}

%%%%%%%%%%%%%%%%%%%%%%%%%%%%%%%%%%%%%%%%%%%%%%LECTURE 16%%%%%%%%%%%%%%%%%%%%%%%%%%%%%%%%%%%%%%%%%%%%%%%%%%%%%%%
    \chapter{Lecture 16 (10/20)}

%%%%%%%%%%%%%%%%%%%%%%%%%%%%%%%%%%%%%%%%%%%%%%LECTURE 17%%%%%%%%%%%%%%%%%%%%%%%%%%%%%%%%%%%%%%%%%%%%%%%%%%%%%%%
    \chapter{Lecture 17 (10/25)}

%%%%%%%%%%%%%%%%%%%%%%%%%%%%%%%%%%%%%%%%%%%%%%LECTURE 18%%%%%%%%%%%%%%%%%%%%%%%%%%%%%%%%%%%%%%%%%%%%%%%%%%%%%%%
    \chapter{Lecture 18 (10/27)}

%%%%%%%%%%%%%%%%%%%%%%%%%%%%%%%%%%%%%%%%%%%%%%LECTURE 19%%%%%%%%%%%%%%%%%%%%%%%%%%%%%%%%%%%%%%%%%%%%%%%%%%%%%%%
    \chapter{Lecture 19 (11/1)}

%%%%%%%%%%%%%%%%%%%%%%%%%%%%%%%%%%%%%%%%%%%%%%LECTURE 20%%%%%%%%%%%%%%%%%%%%%%%%%%%%%%%%%%%%%%%%%%%%%%%%%%%%%%%
    \chapter{Lecture 20 (11/3)}

%%%%%%%%%%%%%%%%%%%%%%%%%%%%%%%%%%%%%%%%%%%%%%LECTURE 21%%%%%%%%%%%%%%%%%%%%%%%%%%%%%%%%%%%%%%%%%%%%%%%%%%%%%%%
    \chapter{Lecture 21 (11/8)}

%%%%%%%%%%%%%%%%%%%%%%%%%%%%%%%%%%%%%%%%%%%%%%LECTURE 22%%%%%%%%%%%%%%%%%%%%%%%%%%%%%%%%%%%%%%%%%%%%%%%%%%%%%%%
    \chapter{Lecture 22 (11/10)}

%%%%%%%%%%%%%%%%%%%%%%%%%%%%%%%%%%%%%%%%%%%%%%LECTURE 23%%%%%%%%%%%%%%%%%%%%%%%%%%%%%%%%%%%%%%%%%%%%%%%%%%%%%%%
    \chapter{Lecture 23 (11/15)}

%%%%%%%%%%%%%%%%%%%%%%%%%%%%%%%%%%%%%%%%%%%%%%LECTURE 24%%%%%%%%%%%%%%%%%%%%%%%%%%%%%%%%%%%%%%%%%%%%%%%%%%%%%%%
    \chapter{Lecture 24 (11/17)}

%%%%%%%%%%%%%%%%%%%%%%%%%%%%%%%%%%%%%%%%%%%%%%LECTURE 25%%%%%%%%%%%%%%%%%%%%%%%%%%%%%%%%%%%%%%%%%%%%%%%%%%%%%%%
    \chapter{Lecture 25 (11/22)}

%%%%%%%%%%%%%%%%%%%%%%%%%%%%%%%%%%%%%%%%%%%%%%LECTURE 26%%%%%%%%%%%%%%%%%%%%%%%%%%%%%%%%%%%%%%%%%%%%%%%%%%%%%%%
    \chapter{Lecture 26 (11/24)}

%%%%%%%%%%%%%%%%%%%%%%%%%%%%%%%%%%%%%%%%%%%%%%LECTURE 27%%%%%%%%%%%%%%%%%%%%%%%%%%%%%%%%%%%%%%%%%%%%%%%%%%%%%%%
    \chapter{Lecture 27 (11/29)}

%%%%%%%%%%%%%%%%%%%%%%%%%%%%%%%%%%%%%%%%%%%%%%LECTURE 28%%%%%%%%%%%%%%%%%%%%%%%%%%%%%%%%%%%%%%%%%%%%%%%%%%%%%%%
    \chapter{Lecture 28 (12/1)}

%%%%%%%%%%%%%%%%%%%%%%%%%%%%%%%%%%%%%%%%%%%%%%LECTURE 29%%%%%%%%%%%%%%%%%%%%%%%%%%%%%%%%%%%%%%%%%%%%%%%%%%%%%%%
    \chapter{Lecture 29 (12/6)}

%%%%%%%%%%%%%%%%%%%%%%%%%%%%%%%%%%%%%%%%%%%%%%LECTURE 30%%%%%%%%%%%%%%%%%%%%%%%%%%%%%%%%%%%%%%%%%%%%%%%%%%%%%%%
    \chapter{Lecture 30 (12/8)}

%%%%%%%%%%%%%%%%%%%%%%%%%%%%%%%%%%%%%%%%%%%%%%%%%%%%%%%%%%%%%%%%%%%%%%%%%%%%%%%%%%%%%%%%%%%%%%%%%%%%%%%%%%%%%%%
  \part{Miscellaneous}
    \chapter{Closing Remarks}


  \end{document}
