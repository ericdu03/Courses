\documentclass[10pt]{article}
\usepackage{../../local}


\newcommand{\classcode}{Physics 105}
\newcommand{\classname}{Analytic Mechanics}
\renewcommand{\maketitle}{%
\hrule height4pt
\large{Eric Du \hfill \classcode}
\newline
\large{HW 08} \Large{\hfill \classname \hfill} \large{\today}
\hrule height4pt \vskip .7em
\normalsize
}
\linespread{1.1}
\begin{document}
	\maketitle
	\section*{Collaborators}
	I worked with \textbf{Adarsh Iyer, Aren Martinian} and \textbf{Andrew Binder} to complete this homework. 
	\section*{Problem 1}
	Revisit Problem 4 on Homework 4, with the following modifications. The chain is now a massless, ideal string.
	The ends of the string are wrapped around and fixed to two uniform-density cylindrical shafts of mass $M_1$
	and $M_2$, respectively, as shown in the figure. The string can slide along the wedge without friction. 
	Gravity acts downwards. Find the tension in the string and the (linear) acceleration of each mass.

	\begin{solution}
		First, let $x_1$ and $x_2$ denote the positions of the masses down their respective slopes. Then, writing
		down the Lagrangian, we get:
		\[
		\mathcal L = T - U = \frac{1}{2}M_1\dot x_1^2 + \frac{1}{2}M_2 \dot x_2^2 + \frac{1}{2}I_1 \omega_1^2
		+ \frac{1}{2}I_2\omega_2^2 + M_1gx_1\sin \alpha + M_2gx_2\sin \beta
		\] 
		Here, I use the top of the wedge as the point of zero potential. Now, we have the constraint equation:
		\[
		x_1 + x_2 = L + R_1 \theta_1 + R_2 \theta_2 \implies f = x_1 + x_2 - R_1 \theta_1 - R_2 \theta_2 = \text{
		const.}
		\] 
		Therefore, our full Lagrangian with constraint is:
		\[
		\mathcal L = T - U = \frac{1}{2}M_1\dot x_1^2 + \frac{1}{2}M_2 \dot x_2^2 + \frac{1}{2}I_1 \omega_1^2
		+ \frac{1}{2}I_2\omega_2^2 + M_1gx_1\sin \alpha + M_2gx_2\sin \beta + \lambda(x_1 + x_2 - R_1\theta_1
		- R_2 \theta_2)
		\] 
		Then, we can write down the Euler-Lagrange equations for each coordinate, for which we have 4 (since 
		$\omega = \dot \theta$:
		\begin{align*}
			\dv{\mathcal L}{x_1} &= \dv{t}\dv{\mathcal L}{\dot x_1} \implies \lambda + M_1g \sin \alpha = M_1
			\ddot x_1\\
			\dv{\mathcal L}{x_2} &= \dv{t}\dv{\mathcal L}{\dot x_2} \implies \lambda + M_2g \sin \beta = M_2
			\ddot x_2\\
			\dv{\mathcal L}{\theta_1} &= \dv{t}\dv{\mathcal L}{\dot \theta} \implies -R_1 \lambda = I_1 \ddot 
			\theta_1 \\
			\dv{\mathcal L}{\theta_2} &=  \dv{t} \dv{\mathcal L}{\dot \theta_2} \implies -R_2 \lambda = 
			I_2 \ddot \theta_2 
		\end{align*}
		Finally, we can use the second derivative of the time constraint to get a final equation: 
		\[
		\ddot x_1 + \ddot x_2 - R_1\ddot \theta_1 - R_2 \ddot \theta_2 = 0
		\] 
		Using all five of these equations, it's possible to solve for $\ddot x_1$ and $\ddot x_2$ in terms of 
		known quantities. I did this computation via a computer, which gave:
		\begin{align*}
			\ddot x_1 &= \frac{3gM_1 \sin \alpha + 2gM_2 \sin \alpha - M_2g\sin \beta}{3(M_1 + M_2)} \\
			\ddot x_2 &=  \frac{3gM_2 \sin \beta + 2gM_2 \sin \beta - M_1g \sin \alpha}{3(M_1 + M_2)}  \\
			\lambda &=  -g\frac{M_1M_2}{M_1 + M_2}\frac{\sin \alpha + \sin \beta}{3} 
		\end{align*}
	\end{solution}

	\pagebreak

	\section*{Problem 2}
	A heavy, uniform rod AB moves without friction inside a cylindrical hole as shown below, remaining in the 
	vertical plane passing through $O$. The initial position of the rod is as shown in the figure, with $A$ at 
	the cusp. Gravity acts downwards. Find the angular velocity of the rod at the moment the rod becomes 
	horizontal. 

	\begin{solution}
		To do this problem, we first notice that the angular velocity of the center of mass $M$ around the point
		$O$ is equal to the angular velocity of the rod about its center of mass due to geometry. Therefore, 
		if we can find the angular velocity of the rod at the moment the rod becomes horizontal, then we 
		have solved the problem. First, consider the bottom of the hole as 0 potential. 

		From geometry alone, we find that the potential energy of the rod when it is horizontal is equal to:
		\[
			U_{\text{bottom}} = Mga\left(1 - \sqrt{a^2 - \frac{l^2}{4}}\right)
		\] 
		Likewise, we can solve for the potential energy of the rod in the initial position as:
		\[
			U_{\text{top}} = Mgy_0
		\] 
		To calculate $y_0$, we use similar triangles:
		\[
			\frac{\sqrt{a^2 - \frac{l^2}{4}} }{a} = \frac{\frac{l}{2}}{y_0} \implies y_0 = \frac{al}{2\sqrt{
			a^2 - \frac{l^2}{4}} }
		\] 
		The last quantity we need to calculate to solve is the moment of inertia of the rod about the point $O$.
		To do this, we use parallel axis theorem. Since we know that the moment of inertia of the rod about 
		its center of mass is $\frac{1}{3} mL^2$, then:
		\[
			I_{\text{center}} = \frac{1}{3}ML^2 + M\left( a^2 - \frac{l^2}{4} \right) 
		\] 
		Finally, we can use conservation of energy:
		\begin{align*}
			\frac{1}{2}I_{\text{center}} \omega^2 &= U_{\text{top}} - U_{\text{bottom}}\\
			\frac{1}{2}\left( \frac{1}{3}ML^2 + M\left( a^2 - \frac{l^2}{4} \right)  \right) \omega^2 &= \frac{mgal}{2\sqrt{a^2 - \frac{l^2}{4}} } - mga\left( 1 - \sqrt{a^2 - \frac{l^2}{4}}  \right)
		\end{align*}
		First, let the quantity $\sqrt{a^2 - \frac{l^2}{4}}$ be denoted as $k$. Therefore, we have the equation:
		\[
		\frac{1}{2}\left( \frac{1}{3}ML^2 + Mk^2  \right)\omega^2 = \frac{mgal}{2k^2} - mga(1 - k)
		\] 
		We can now solve for $\omega$:
		\begin{align*}
			\omega^2 &= \frac{\frac{2mgal}{2k^2} - 2mga(1 - k)}{\frac{1}{3}ML^2 + Mk^2} \\
			\therefore \omega &= \sqrt{\frac{\frac{mgal}{k^2} - 2mga(1-k)}{\frac{1}{3}ML^2 + Mk^2}} 
		\end{align*}
		Again, remember that $k = \sqrt{a^2 - \frac{l^2}{4}}$. Returning the substitution, I'd argue, doesn't
		really simplify the expression too much, so I'll leave the final expression in terms of $k$.
	\end{solution}
\end{document}
