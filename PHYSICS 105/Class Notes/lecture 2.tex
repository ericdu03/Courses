\chapter{Lecture 2 (01/)}

This lecture was held on \textbf{January 19th, 2023}. It covered damped and driven oscillators.

\section{Last time: The Free Oscillator}

On Tuesday we explored oscillatory mechanics where there were no other forces besides the restoring force. However, in most systems, we will always have some kind of \textit{daming force} which impedes motion. This doesn't always have to be the case, but we will first explore a damping force which is proportional to the velocity: 

\[ \vec f = b \vec v\]

Under this, we now have the restoring force and the damping force, so Newton's second law now reads: 

\[ m \ddot x + b \dot x + kx = 0\] 

And so if we let $\beta = \frac{b}{2m}$ (we'll see later why this substitution is useful), then we can write

\[ \ddot x + 2\beta x + \omega_0^2 x = 0\]

The nature of these differential equations is that due to their linearity, if we find two independent solutions $x_1(t)$ and $x_2(t)$, then in general their solution will be a linear combination of the two: 

\[ x(t) = C_1x_1(t) + C_2x_2(t)\] 

We saw that exponentials worked before, let's have that as our main guess. Let 

\[ x(t) = e^{rt} \implies \dot x(t) = re^{rt}, \ddot x(t) = r^2e^{rt}\] 

Plugging this in, we get: 

\begin{align*}
    r^2e^{rt} + 2\beta r e^{rt} + \omega_0^2e^{rt} &= 0\\
    \therefore r^2 + 2\beta r + \omega_0^2 &= 0
\end{align*}

This is quadratic in $r$, so therefore we have solutions $r = -\beta \pm \sqrt{\beta^2 - \omega_0^2}$. Now, we can then write

\begin{align*}
    r_1 &= -\beta + \sqrt{\beta^2 - \omega_0^2}\\
    r_2 &= -\beta - \sqrt{\beta^2 - \omega_0^2}
\end{align*}

so the general solution now becomes: 

\[ x(t) = C_1e^{r_1t} + C_2e^{r_2t} = e^{-\beta t} \left( C_1e^{\sqrt{\beta^2 - \omega_0^2} t } + C_2e^{-\sqrt{\beta^2 - \omega_0^2}t}\right)\]

This equation makes sense intuitively, since a large value of $\beta$ generates a faster decay, which makes sense since $\beta$ refers to the damping constant. 

Now we have 3 cases that we want to analyze: 

\begin{enumerate}[label = (\alph*)]
    \item Underdamped: $\omega_0^2 > \beta^2$ 
    \item Critical damping: $\omega_0^2 = \beta^2$
    \item Overdamped: $\omega_0^2 < \beta^2$
\end{enumerate}

As it will turn out, only the overdamping will give us oscillatory motion.

\subsection{Case 1: Underdamped Oscillation} 

Here we look at the case where $\omega_0^2 > \beta^2$. If this is the case, then we can write $\sqrt{\beta^2 - \omega_0^2} = i\sqrt{\omega_0^2 - \beta^2} = i\omega_1$, with $\omega_1 = \sqrt{\omega_0^2 - \beta^2}$

\begin{insight*}{}{}
    if $\beta = 0$, then we exactly recover the solution that we got last time: 

    \[ x(t) = C_1e^{\sqrt{-\omega_0^2}t} + C_2e^{-\sqrt{-\omega_0^2}t} = C_1e^{i\omega_0t} + C_2e^{-i\omega_0t}\]

    This is a good way to check that what we're doing still makes sense.
\end{insight*}




