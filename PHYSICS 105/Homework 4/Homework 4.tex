\documentclass[10pt]{article}
\usepackage{../local}

\newcommand{\classcode}{Physics 105}
\newcommand{\classname}{Analytic Mechanics}
\renewcommand{\maketitle}{%
\hrule height4pt
\large{Eric Du \hfill  \classcode}
\newline
\large{HW 04} \Large{\hfill  \classname \hfill } \large{\today}
\hrule height4pt \vskip .7em
\normalsize
}
\linespread{1.1}
\begin{document}
    \maketitle
    \section*{Collaborators}
    \section*{Problem 1}
    Consider a vertical plane in a constant gravitational field (which is also vertical).
    A particle of mass $m$ moves on the plane, and it is also under the influence of a force
    $F = -Ar^{\alpha - 1}$, where $A$ and $\alpha$ are constants ($\alpha \neq 0, 1$). Here, $r$
	is the distance from the origin, which is some fixed point on the plane. After choosing suitable 
	generalized coordinates, find the Lagrangian and the equations of motion. Is the angular momentum
	conserved? Explain.
	
	\begin{solution}
			Since we're given the force, we actually also know the potential: 
			\[ -\pdv{r} \left( \frac{A}{\alpha} r^{\alpha}\right) = F \implies U = \frac{A}{\alpha} r^{\alpha}\]
			and so therefore we can write the Lagrangian as: 
			\[ \mathcal L = \frac 12 m(\dot r^2 + r\dot \theta^2) - mgr \sin \theta - \frac A\alpha r^\alpha\]
			So we can just solve using the normal method:
			\begin{align*}
					\pdv{\mathcal L}{r} - \dv{t} \pdv{\mathcal L}{\dot r} &= 0\\
					\pdv{\mathcal L}{\theta} - \dv{t} \pdv{\mathcal L}{\dot \theta} &= 0
			\end{align*}
			And so solving this system, we get the following two sets of equations:
			\begin{align*}
					m\ddot r + Ar^{\alpha -1}-mg \sin \theta &= 0\\
					r(\ddot \theta + \cos \theta) - \dot r \dot \theta &= 0
			\end{align*}
			Angular momentum is written as:
			\[ \pdv{\mathcal L}{\dot \theta} = mr \dot \theta\] 
			And since this quantity is not constant, then we know that angular momentum is not conserved 
			in this model.
	\end{solution}	

	\pagebreak

	\section*{Problem 2}

	Use the method of Lagrange multipliers to find the tension in the string of the double Atwood machine
	given in Homework 2 Problem. 

	\begin{solution}
			Our constraint equation is $f(x_1, x_2, X) = x_1 + x_2 + X = L$, where we 
			interpret $L$ as the length of the string. From Homework 2, we know that our Lagrangian 
			is:
			\[\mathcal L = \frac 12 m\dot x_1^2 + \frac 12 m \dot x_2^2 + \frac 12 m\dot x^2 - m_1gx_1
			- m_2gx_2 - Mgx\]
			and so now we can use the modified Lagrange equations with constraints to solve: 
			\[ \pdv{\mathcal L}{x_i} + \lambda(t) \pdv{f}{x_i} = \dv{t} \pdv{\mathcal L}{\dot x_i}\] 
			And so therefore we have the equations: 
			\begin{align*}
					-m_1g + \lambda(t) &= m_1 \ddot x_1\\
					m_2g + \lambda(t) &= m_2\ddot x_2\\
					-Mg + 2\lambda(t) = M\ddot x
			\end{align*}
			And so solving the last equation, we get:
			\[ \lambda(t) = \frac{M}{2}(\ddot x + g)\]
			Substituting this relation into the first equation, we get: 
			\begin{align*}
					m_1 \ddot x_1 + m_1g &= \frac{M}{2} g - \frac{M}{4}\ddot x_1 - \frac{M}{4}\ddot x_2\\
					\ddot x_1\left(m_1+\frac{M}{4}\right) + \frac{M}{4} \ddot x_2 &= g\left( \frac{M}{2} - 
							m_1\right)
			\end{align*}
			Likewise for the second equation, we can also obtain
			\begin{align*}
					m_2\ddot x_2 + m_2g &= \frac{M}{2}(\ddot x + g)\\
					&= \frac{M}{2}\left( g - \frac{\ddot x_1 + \ddot x_2}{2}\right)\\
					&= \frac{M}{2}g - \frac{M}{4}\ddot x_1 - \frac{M}{4} \ddot x_2
			\end{align*}
			And so therefore we obtain:
			\[ \ddot x_2\left( m_2 + \frac{M}{4} \right) + \frac{M}{4}\ddot x_1 = g\left( \frac{M}{2} - 
			m_2\right)\]
			To find the tension in the rope, we can then use the relation that 
			\[ \lambda \pdv{f}{y} = F^{cstr} = T\]
			And so therefore: 
			\[ T = \frac{M}{2}(\ddot x + g)\]

	\end{solution}
	\pagebreak

	\section*{Problem 3}
	Consider a particle of mass $m_2$ constrained to move on the surface of a cone which is placed vertex-up.
	There is a massless, ideal string of length $\ell$ connecting this mass to another mass $m_1$ which hangs
	inside the cone, as in the figure below. The cone has an opening angle of $2\alpha$.

	\begin{enumerate}[label=\alph*)]
			\item Define generalized cylindrical coordinates suitable to this system, and write down the 
					Lagrangian.

					\begin{solution}
						$m_1$ is constrained to move in the $\hat{z}$-direction, so we can write the coordinates
						of $m_1$ in terms of strictly $z_1$. For $m_2$, we can write its position in terms of 
						$(x_2, \phi, z_2)$, so therefore we can write the kinetic energy as:
						\begin{align*}
								T &= \frac 12 m (\dot x_2^2 + x_2^2 \dot \phi^2 + \dot z_2^2) + 
						\frac 12 m\dot z_1^2\\
								  &= \frac 12 m z^2 + \frac 12 m \left(\dot x^2 + x^2 \dot \phi^2 + 
								  \frac{\dot x^2}{\tan^2\alpha}\right)
						\end{align*}
						Here, I've dropped the indices since none of the variables interfere with one another.
						We can also write the potential as:
						\[ U = mgz + mgx\cos \alpha\]
						And so our full Lagrangian $\mathcal L$ is:
						\[ \mathcal L = T - U = \frac 12 m_1 z^2 
								+ \frac 12 m_2 \left(\dot x^2 + x^2 \dot \phi^2 + 
						\frac{\dot x^2}{\tan^2\alpha}\right) - m_1gz - m_2g \frac{x}{\tan \alpha}\]
					\end{solution}
			\item Find the equations of motion; use the constraint, so that you only have two equations.

				\begin{solution}
						Right now we have three coordinates: $x$, $z$ and $\phi$. The constraint tells us that 
						\[ x = (\ell-z)\sin \alpha\]
						So now we can substitute and get:
						\begin{align*}
								\mathcal L &= \frac{1}{2}m_1\dot z^2 + 
								\frac{1}{2}m_2 \left[\dot z^2 \sin^2 \alpha
								+ (\ell-z)^2\sin^2 \alpha \dot \phi^2 + \dot z^2 \cos^2\alpha\right]
								- m_1gz - m_2g(l-z)\cos \alpha\\
										   &= \frac{1}{2}m_1\dot z^2 + \frac{1}{2}m_2\left[\dot z^2 +
										   (\ell-z)^2 \sin^2 \alpha \dot \phi^2\right] - m_1gz - 
										   m_2g(\ell-z)\cos \alpha
						\end{align*}
						Now we have an equation in terms of two variables, for which we can solve using 
						the standard Euler-Lagrange equations, which gives us:
						\begin{align*}
								0 &= (\ell- z)^2\ddot \phi - 2(\ell- z)\dot \phi \dot z\\
								0 &= -2(\ell-z)\sin^2\alpha \dot \phi^2 - m_1g+m_2g\cos \alpha - 
								\ddot z(m_1 +m_2)
						\end{align*}
				\end{solution}
		 	\item Notice that the equation for $\varphi$ can be separated and integrated to obtain a solution for
					$\dot \varphi$. Do so, and plug this equation in to the equation for $z$

					\begin{solution}
							Notice that our equation for $\phi$ can be written as
							\[ \frac{\ddot \phi}{\dot\phi} = \frac{2 \dot z}{\ell - z}\]
							which we can integrate to get:
							\begin{align*}
									\int \frac{1}{\dot \phi}\dv{\dot \phi}{t} \ dt &= 
									\int \frac{2\dot z}{\ell -z } \ dt\\
									\ln \dot \phi &= -2\ln(\ell - z)\\
									\therefore \dot \phi &= \frac{1}{(\ell - z)^2}
							\end{align*}
							So now we can plug this into the equation for $z$: 
							\begin{align*}
									0 &= -2(\ell-z) \sin^2 \alpha \left(\frac{1}{(\ell-z)^2}\right)^2 -
										\ddot z(m_1 + m_2)\\ 
									&= -\frac{2\sin^2\alpha}{(\ell-z)^3} - m_1g +m_2g \cos \alpha 
										- \ddot z(m_1 + m_2)
							\end{align*}
					\end{solution}
			\item You should now have a differential equation for $z$ only. Show that this equation of
					motion corresponds to a potential of the form 
			\[ U(z) = \frac{A}{(\ell - z)^2} + Bz\]

			\begin{solution}
					So looking at the potential, we can get the force by taking the derivative:
					\begin{align*}
							F &= -\dv{U}{z}\\
							  &= -\left(\frac{2A}{(\ell-z)^3} + B\right)\\
							  &= -\frac{2A}{(\ell-z)^3} - B
					\end{align*}
					Then by Newton's second law, we can write:
					\begin{align*}
							-\frac{2A}{(\ell-z)^3}-B &= (m_1+m_2)\ddot z\\
							-\frac{2A}{(\ell-z)^3}-(m_1+m_2)\ddot z&= B
					\end{align*}
					Comparing this with our solution from above, we can identify: 
					\begin{align*}
							A &= \sin^2\alpha & B &= m_1g-m_2g\cos\alpha
					\end{align*}
			\end{solution}
			\item When is $B$ positive or negative? What does this correspond to physically? Analyze 
					the motion in both cases, $B < 0$ and $B > 0$.

					\begin{solution}
						From the previous part, we know that $B = m_1g - m_2g\cos \alpha$, so $B$ is positive
						when $m_1g > m_2g\cos \alpha$ and $B < 0$ when $m_1g < m_2g\cos \alpha$. Physically,
						from a force perspective, it compares whether the downward force exerted by 
						$m_1$ is larger or that of $m_2$ or the other way around.

						Therefore, $B$ is positive when the downward force from $m_1$ exceeds that of $m_2$, 
						and so $m_2$ moves up the cone. Likewise, if $B$ is negative then the force from 
						$m_2$ exceeds that of $m_1$, and so $m_2$ moves down the cone.

						To analyze the motion, we first make the substitution that $u = \ell - z$, so 
						therefore we have the equation: 
						\[ -\frac{2A}{u^3} +(m_1 +m_2)\ddot u = \frac{B}{m_1+m_2}\]
						And so therefore we can write:
						\[ \ddot u - \frac{2A}{(m_1 +m_2)u^3} = \frac{B}{m_1+m_2}\]
						I'm not really sure where to go from here, since this differential equation is nearly 
						impossible to solve.

					\end{solution}
			\item Use the method of Lagrange multipliers to determine the tension in the string in 
					terms of $z$ (and the initial conditions)

					\begin{solution}
						I wasn't able to carry through the previous part so I couldn't explicitly get an 
						expression, but what I can do is describe how I would do it. Essentially, we just need
						to setup the expression for Lagrange multipliers:
						\[ \pdv{\mathcal L}{x_{i}} + \lambda \pdv{f}{x_i} = \dv{t} \pdv{\mathcal L}{\dot x_{i}}\]
						And once we solve for $\lambda$, we can then write:
						\[ F^{cstr} = T  = \lambda \pdv{f}{x_i}\]
						which would get us the expression for the tension.
					\end{solution}
	\end{enumerate}

	\pagebreak

	\section*{Problem 4}
	
	Consider a uniform chain of length $\ell$ and mass $m$ which is free to move without friction on a
	triangular block as in the figure below. Suppose that at $t = 0$, the chain has length $x_0$ lying over
	the edge of the side of the prism. Find and solve the equation of motion.

	\begin{solution}
			First, let $\lambda = \frac{m}{\ell}$. Then we can find the potential. To do so, we split up the 
			integral based on which side of the block our chain is:
			\[ U(x) = -\int_0^x (\lambda d\ell) g\ell \sin \alpha - 
			\int_x^\ell \lambda \ d\ell g \ell \sin \beta = \lambda g \left( \sin \alpha \frac{x^2}{2} - 
			\sin \beta \left( \frac{l^2 -x^2}{2}\right)\right)\]
			and so our Lagrangian can be written as:
			\[ \mathcal L = \frac{1}{2}m\dot x^2 - \lambda g \sin \alpha \frac{x^2}{2} - \lambda g \sin \beta 
			\left(\frac{\ell^2-x^2}{2}\right)\]
			And so solving for the equation of motion, we get: 
			\begin{align*}
					x(\lambda g\sin \beta - \lambda g \sin \alpha) - m\ddot x &= 0\\
					\ddot x - \frac{\lambda g(\sin \beta - \sin \alpha)}{m}x &= 0
			\end{align*}
			This gives solutions of the form: 
			\[x(t) = Ae^{kt} + Be^{-kt}, \ \ k = \sqrt{\frac{\lambda g(\sin \beta - \sin \alpha)}{m}} \]
			We can then solve for the initial condition that $x(0) = x_0$, which gives us
			\[ x(0) = x_0 = A + B\]
			We can't really solve for $A$ and $B$ explicitly without knowing more information (such as 
			knowing $\dot x(0)$, so this is the most general solution that we can have: 
			\[ x(t) = Ae^{kt}+ Be^{-kt}\]
			where $k = \sqrt{\dfrac{\lambda g(\sin \beta - \sin \alpha)}{m}}$
	\end{solution}	




	

	
\end{document}
