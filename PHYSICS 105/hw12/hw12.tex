\documentclass[10pt]{article}
\usepackage{../../local}


\newcommand{\classcode}{Physics 105}
\newcommand{\classname}{Analytic Mechanics}
\renewcommand{\maketitle}{%
\hrule height4pt
\large{Eric Du \hfill \classcode}
\newline
\large{HW 12} \Large{\hfill \classname \hfill} \large{\today}
\hrule height4pt \vskip .7em
\normalsize
}
\linespread{1.1}
\begin{document}
	\maketitle
	\section*{Collaborators}
	I worked with \textbf{Adarsh Iyer, Aren Martinian} and \textbf{Andrew Binder} to complete this homework.
	\section*{Problem 1}
	Calculate the solid angles subtended by the Moon and by the Sun, both as seen from the Earth.

	\begin{solution}
		From the textbook, we know that the solid angle is given by $\Delta \Omega = \frac{A}{r^2}$, so 
		all we need to do is find the cross sectional area of each celestial body (given by $\pi R^2$), and 
		divide that by the mean distance between the Earth and the Sun/Moon. Plugging in numbers for the Sun:
		\[
			\Delta \Omega_{\text{sun}} = 6.79 \times 10^{-5} \text{ sr}
		\] 
		And then for the moon:
		\[
			\Delta \Omega_{\text{moon}} = 6.4\times 10^{-5} \text{ sr}
		\] 
		This actually makes sense, since we see that the sun and moon look approximately the same size, despite 
		the sun being much bigger, because the Sun is also much farther away than the moon. 
	\end{solution}

	\pagebreak
	\section*{Problem 2}
	One can set up a two-dimensional scattering theory, which could be applied to puck projectiles sliding on 
	an ice rink and colliding with various target obstacles. The cross section would be the effective width
	of a target, and the differential cross section $d\sigma/d\theta$ would give the number of projectiles 
	scattered in the angle $d\theta$.

	\begin{enumerate}[label=\alph*)]
		\item Show that the two-dimensional analog of Eq. (14.23) is $d\sigma/d\theta = |db/d\theta|$. (Note
			that in two-dimensional scattering it is convenient to take $\theta \in [-\pi, \pi]$. 

			\begin{solution}
				The small patch of particles that scatter into the solid angle $d\theta$ is now 
				given by a small patch of vertical height $db$, so therefore $d\sigma = db$. 
				Further, since the system is symmetric for $\pm \theta$, then we can combine this 
				information and solve for the absolute value instead. Therefore:
				\[
					\dv{\sigma}{\theta} = \left| \dv{b}{\theta}\right|
				\] 
			\end{solution}
		\item Now consider the scattering of a small projectile off a hard ``sphere'' (actually a hard
			disk) of radius $R$ pinned down to the ice. Find the differential cross section.

			\begin{solution}
				We follow a very similar argument to that of the hard ball scattering. Consider a particle
				with an impact parameter $b$. Due to the geometry of the problem, we see that 
				$b = R \sin \alpha$. Further, due to momentum conservation, this scattering must obey the law
				of reflection, which implies that $\theta = \pi - 2\alpha$ (see diagram). Therefore, 
				we can write
				\[
				b = R \left|\sin (\frac{\pi - \theta}{2})\right|
				\] 
				Now, taking the derivative with respect to $\theta$, we get: 
				\[
					\left|\dv{b}{\theta}\right| = \left|\frac{R}{2}\cos(\frac{\pi - \theta}{2})\right|
				\] 
			\end{solution}
		\item By integrating your answer to part (b), show that the total cross section is $2R$ as expected.

			\begin{solution}
				Here, we integrate with respect to $\theta$, from the bounds $-\pi$ to $\pi$:
				\[
					b = \frac{R}{2}\int_{-\pi}^\pi \left| \cos\left( \frac{\pi - \theta}{2} \right) \right| d\theta = \frac{R}{2}(4) = 2R
				\] 
				as desired. 
			\end{solution}
	\end{enumerate}

	\pagebreak
	\section*{Problem 3}
	One of the first observations that suggested his nuclear model of the atom to Rutherford was that several 
	alpha particles got scattered by metal foils into the backward hemisphere, $\pi/2 \le \theta \le \pi$ - an
	observation that was impossible to explain on the basis of rival atomic models, but emerged naturally 
	from the nuclear model. In an early experiment, Geiger and Marsden measured the fraction of incident alphas
	scattered into the backward hemisphere off a platinum foil. By integrating the Rutherford cross section
	\[
		\dv{\sigma}{\Omega} = \frac{\sigma_0(E)}{\sin^4(\theta/2)}, \text{    with     } \sigma_0(E) =
		\frac{k^2q^2Q^2}{16E^2}
	\] 
	over the backward hemisphere, show that the cross section for scattering with $\theta \ge \pi/2$ should be 
	$4 \pi \sigma_0(E)$. Using the following numbers, predict the ratio $N_{sc}(\theta\ge \pi/2)/N_{in}$: 
	thickness of platinum foil $\approx$ 3 pm, density = 21.4 gram/$\text{cm}^3$, atomic weight = 195, atomic
	number = 78, energy of incident alphas = 7.8 MeV. Compare your answer with their estimate that 
	``of the incident $\alpha$ particles about 1 in 8000 was reflected". (that is, scattered into the backward 
	hemisphere). Small as this fraction is, it was still far larger than any rival model of the atom could 
	explain.

	\begin{solution}
		The integral we want to take is 
		\[
			\sigma = \int \frac{\sigma_0(E)}{\sin^4(\frac{\theta}{2})} d\Omega
		\] 
		Since $\sigma_0(E)$ is a constant, we can take it out of the integral. Therefore, we now have:
		\[
			\sigma = \sigma_0(E) \int \frac{1}{\sin^4(\frac{\theta}{2})} d\Omega
		\] 
		We can do two things here: first, we rewrite $d\Omega = \sin \theta d\theta d\phi$. Then, we note our
		bounds of integration are $\theta \in [\pi/2, \pi]$, and 
		$\phi \in [0, 2\pi]$. Since everything in the integral is independent of $\phi$, we can just multiply by
		$2\pi$, leaving us with:
		\[
			\sigma = 2\pi \sigma_0(E) \int_{\frac{\pi}{2}}^\pi \frac{\sin \theta}{\sin^4(\frac{\theta}{2})} 
			d\theta
		\] 
		Then, we can perform a $u$-substitution, letting $u = \frac{\theta}{2}$, so therefore $du = \frac{1}{2}
		d\theta$. This turns our integral into: 
		\begin{align*}
			\sigma &= 2\pi \sigma_0(E) \int_{\frac{\pi}{4}}^{\frac{\pi}{2}} \frac{\sin 2u}{\sin^4u} (2 du)\\
				   &= 4\pi \sigma_0(E) \int_{\frac{\pi}{4}}^\frac{\pi}{2}\frac{2 \cos u}{\sin^3 u} du \\ 
				   &= 8 \pi \sigma_0(E) \int_{\frac{\pi}{4}}^{\frac{\pi}{2}} \frac{\cos u}{\sin^3 u} du
		\end{align*}
		From here, it's a simple $u$-substitution again: let $v = \sin u$, so $dv = \cos u du$. Therefore:
		\begin{align*}
			\sigma &= 8\pi \sigma_0(E) \int_{\frac{1}{\sqrt{2} }}^1 \frac{dv}{v^3}\\
				&= 8\pi \sigma_0(E) (\frac{1}{2})\\
				&= 4\pi \sigma_0(E) 
		\end{align*}
		as desired. Plugging in numbers, we get:
		\[
			\frac{N_{\text{sc}}}{N_{\text{in}}} = 1.292 \times 10^{-4}
		\] 
		this gives a frequency of about 1/8333, which is approximately what Rutherford got.
	\end{solution}


	\pagebreak

	\section*{Problem 4}
	The derivation of the Rutherford cross section was made simpler by the fortuitous cancellation of the factors
	of $r$ in the integral, Eq. (14.30). Here is a method of finding the cross section which works, in principle,
	for any central force field: The general appearance of the scattering orbit is as shown in Figure 14.11. It 
	is symmetric about the direction $\mathbf u$ of closest approach. If $\psi$ is the projectile's polar 
	angle, measured from the direction $\mathbf u$, then $\psi \to \pm \psi_0$ as $t \to \pm \infty$ and the 
	scattering angles $\theta = \pi - 2\psi_0$, as in the following figure depicting nuclear scattering:

	[insert tikz here] 

	The angle $\psi_0$ is equal to $\int_{t_0}^\infty \dot \psi(t) dt$ taken from the time of closest approach
	$t = t_0$ to $t = \infty$. We can rewrite this as
	\[
		\int_{t_0}^\infty \frac{\dot \psi(t)}{\dot r(t)}\dot r(t) dt = \int_{r_0}^\infty
		\frac{\dot \psi(r)}{\dot r(r)} dr
	\] 
	where now $r_0$ is the distance of closest approach. Next rewrite $\psi$ in terms of the angular momentum 
	$\ell$ and $r$, and rewrite $\dot r$ in terms of the energy $E$ and the effective potential $U_{\text{eff}}$. Having
	done all this you should be able to prove that 
	\[
	\theta = \pi - \frac{2}{b} \int_{r_0}^\infty \frac{(b/r)^2}{\sqrt{1 - (b/r)^2 - U(r)/E}} 
	\] 
	Provided this integral can be evaluated, it gives $\theta$ in terms of $b$, and hence the cross section. Fill
	in the details of this calculation to prove this formula. 

	\begin{solution}
		First, we start by noticing that $\dot \psi$ is the polar angle, so $\dot \psi$ represents
		the angular velocity of the particle. Therefore, we can write the angular momentum as $\ell = 
		mr^2 \dot \psi$, which rearranges to
		\[
		\dot \psi = \frac{\ell}{mr^2}
		\] 
		Next, we know that at large distance, we have $\ell = r \times p = rp \sin \theta$, and since $r \sin 
		\theta = b$, then we have $l = p \cdot b$, and using the expression that $E = \frac{p^2}{2m}$ 
		(conservation of energy), we get $\ell =b \sqrt{2mE}$, so therefore:
		\[
		\dot \psi = \frac{\sqrt{2mE} b}{mr^2} = \sqrt{\frac{2E}{m}} \frac{b}{r^2}
		\] 
		Now, we have to handle $\dot r$. To do this, consider the conservation of energy
		\[
			E = \frac{1}{2}m \dot r^2 + U_{\text{eff}}(r) = \frac{1}{2}m \dot r^2 + U(r) + \frac{\ell^2}{2m
			r^2}
		\] 
		The $\frac{\ell^2}{2mr^2}$ term can be simplified directly using our substitution of $\ell = \sqrt{
		2mE} b$, giving us 
		\[
		\frac{\ell^2}{2mr^2} = \frac{Eb^2}{r^2}
		\] 
		Next, solving for $\dot r$:
		\[
		\dot r = \sqrt{\frac{2}{m}\left( E - U(r) - \frac{Eb^2}{r^2} \right) } = \sqrt{\frac{2E}{m}
		\left( 1 - \frac{U}{E} - \frac{b^2}{r^2} \right) } 
		\]
		Therefore, putting it all together: 
		\[
			\frac{\dot \psi}{\dot r} = \frac{\sqrt{\frac{2E}{m}} \frac{b}{r^2}}{\sqrt{\frac{2E}{m}\left( 1 - 
			\frac{U}{E} - \frac{b^2}{r^2} \right) }} = \frac{b/r^2}{\sqrt{1 - \frac{U}{E} - \frac{b^2}{r^2}}}
		\] 
		Factoring out a $b$, we have:
		\[
			\frac{1}{b}\frac{\dot \psi}{\dot r} = \frac{(b/r)^2}{\sqrt{1 - \frac{U}{E} - \frac{b^2}{r^^2}}}
		\] 
		Now we integrate from $r_0$ to $\infty$, so we get:
		\[
			\theta = \pi - 2 \int_{r_0}^\infty \frac{(b/r^2) dr}{\sqrt{1 - (b/r)^2 - U(r)/E} }
		\]
		as desired.
	\end{solution}
	 
	\pagebreak


	\section*{Problem 5}
	Use the formula obtained in the previous problem to answer the following problems. You may use a computer
	to evaluate integrals.

	\begin{enumerate}[label=\alph*)]
		\item Consider hard sphere scattering. What should you take to be $U(r)$? Find $b(\theta)$, and then 
			$d\sigma/d\Omega$, and finally $\sigma$. Does your result for $\sigma$ make sense?

			\begin{solution}
				For a hard sphere, we know that $U(r) = 0$ for $r > r_0$ (the radius of the sphere), and the location of closest approach is
				$r = r_0$, so therefore our integral for $\theta$ becomes:
				\[
					\theta = \pi - \frac{2}{b}\int_{r_0}^\infty \frac{(b/r)^2}{\sqrt{1 - (b/r)^2}} dr
				\] 
				we substitute $u = \frac{b}{r}$ so this integral becomes $\sin^{-1}(u)$, so then we get:
				\[
					\theta = \pi - 2\sin^{-1}\left( \frac{b}{r_0} \right) 
				\]
				Therefore, we have: 
				\[
				b = r_0\cos\left( \frac{\theta}{2} \right) 
				\] 
				To find $\dv{\sigma}{\Omega}$, we then integrate this using 
				\[
				\dv{\sigma}{\Omega} = \frac{b}{\sin \theta} \left|\dv{b}{\theta}\right| = \frac{r_0^2}{4}
				\] 
				Now finally, we integrate this from $\theta \in [0, \pi]$ and $\phi \in [0, 2\pi]$:
				\[
				\int \frac{r_0^2}{4} d\Omega = (4\pi) \frac{r_0^2}{4} = \pi r_0^2
				\] 
				this makes sense, since it's equal to the cross sectional area of the hard ball. 
			\end{solution}
		\item Consider Rutherford scattering with $F = kqQ/r^2$. Find the values of $b(\theta)$, 
			$d\sigma/d\Omega$ and $\sigma$. 

			\begin{solution}
				Here, we need to solve:
				\[
					\theta = \pi - \frac{2}{b} \int_{r_0}^\infty \frac{(b/r)^2}{\sqrt{1 - \frac{b^2}{r^2} -
					\frac{kqQ}{rE}} }
				\] 
				First, we perform a $u$ substitution of $u = \frac{b}{r}$, which simplifies the integral to:
				\[
					\theta = \pi + 2 \int_{\frac{b}{r_0}}^{0} \frac{u \ du}{\sqrt{ 1 - u^2 - \frac{kqQu}{bE}} }
				\] 
				We can then plug this into Mathematica, which gives us: 
				\[
					\theta = \pi - 2\tan^{-1}\left( \frac{2bE}{kqQ} \right) 
				\] 
				Rearranging for $b$, we have: 
				\[
					b = \frac{kqQ}{2e} \tan\left( \frac{\pi - \theta}{2} \right)  =
					\frac{kqQ}{2e}\cot\left( \frac{\theta}{2} \right) 
				\] 
				Now we can get $\dv{\sigma}{\Omega}$:
				\[
					\dv{\sigma}{\Omega} = \frac{b}{\sin \theta} \left| \dv{b}{\theta}\right| = -\frac{b}{\sin \theta}\frac{kqQ}{4E}\csc^2\left( \frac{\theta}{2} \right) 
				\] 
				Finally, integrating this over $d\Omega$, we get:
				\[
					\sigma = -\frac{kqQb}{4E}\int \frac{\csc^2\left( \frac{\theta}{2} \right)}{\sin \theta} 
					d\Omega = -\frac{kqQb}{4E}(2\pi) \int_0^\pi \frac{\csc^2\left( \frac{\theta}{2} \right) }
					{\sin \theta} d\theta
				\] 
				which does not converge. 
			\end{solution}
	\end{enumerate}

\end{document}
