\documentclass[10pt]{article}
\usepackage{../../local}


\newcommand{\classcode}{Physics 105}
\newcommand{\classname}{Analytic Mechanics}
\renewcommand{\maketitle}{%
\hrule height4pt
\large{Eric Du \hfill \classcode}
\newline
\large{HW 05} \Large{\hfill \classname \hfill} \large{\today}
\hrule height4pt \vskip .7em
\normalsize
}
\linespread{1.1}
\begin{document}
    \maketitle
	\section*{Collaborators}
	I worked with \textbf{Andrew Binder, Adarsh Iyer} and \textbf{Aren Martinian} to complete this assignment.
	\section*{Problem 1}
	\begin{enumerate}[label=\alph*)]
			\item Use Kepler's laws to weigh the sun. Use $G = 6.67 \times 10^{-11} \mathrm m^3 \mathrm{kg}^{-1}
					\mathrm s^{-2}$ and $R = 1.50 \times 10^{11}$ m (the average radius of Earth's orbit)

					\begin{solution}
							Here we use Kepler's relation that
							\[
							\tau^2 = \frac{4\pi^2a^3}{GM}
							\] 
							where $M$ here represents the mass of the sun. Therefore, solving for $M$, we get:
							\[
							M = \frac{4\pi^2a^3}{G\tau^2}
							\] 
							Computing this with the known values,  we get: $1.979 \times 10^{30}$ kg
					\end{solution}
			\item For which $n$ do stable circular orbits exist for the potential
					\[ V(r) = -\frac{A}{r^n}\]
					Think graphically!

					\begin{solution}
						This would imply an effective potential of 
						\[
								V_{eff} = -\frac{A}{r^n} + \frac{\ell^2}{\mu r^2}
						\] 
						This solution only has a bounded orbit when $n = 1$, when we plot it on a graphing 
						calculator. This also makes sense, since for $n =1$ our $V_{eff}$ is written as: 
						\[
								V_{eff} = -\frac{A}{r} + \frac{\ell^2}{\mu r^2}
						\] 
						which is the equation for planetary motion, which we know has bounded orbits. On the 
						other hand, we can also just graph $V_{eff}$ for different values of $n$:
						\begin{center}
								\begin{tikzpicture}[domain=0.3:5, samples=100]
										\draw[thick] (-1, 0) -- (5, 0);
										\draw[thick] (0, -2) -- (0, 3);
										\draw[color=red] plot (\x, {-2.5/\x + 1/(\x^2)}) node[right] {$n = 1$};
										\draw[color=blue, yscale=0.2] plot(\x, {-2.5/(\x^2) + 1/(\x^2)}) node[right]{$n = 2$};
										\draw[color=green!40!black, yscale=0.02] plot(\x, {-2.5/(\x^3) + 1/(\x^2)}) node[above] {$n>2$};
								\end{tikzpicture}
						\end{center}
						as shown in the diagram above, bounded orbits exist only when $n = 1$.

						

					\end{solution}
			\item \begin{enumerate}[label=(\roman*)]
				\item A puck of mass $m$ on frictionless ice is attached to a horiziontal string which is fed 
						through a thin hole in the ice of radius $R$. Initially, the piece of string above 
						the ice has length $l$. With the free end of the string in place, the puck is given 
						a kick so that it circles around the hole with initial speed $v_0$ at radius $l$. But 
						at $t = 0$, the free end of the string is pulled downwards beneath the ice at a constant
						rate $\lambda$, so that the length of string \textit{above the ice} is $l(t) = l - 
						\lambda t$. Eventually, the string has been pulled all the way in, and the puck falls 
						into the hole, upon reaching $r =R$. 
						What quantity is preserved during the motion? What is the puck's speed right before it 
						falls through the hole? Explain how this is consistent with conservation of energy
						and conservation of angular momentum.

						\begin{solution}
								Angular momentum of the system must be conserved because there is no net torque
								on the system. Therefore, we have the equation:
								\[
								mv_0l = mv_f R \implies v_f = \frac{v_0l}{R}
								\] 
								The energy of the system is not conserved here, because we are actively pulling
								on the string, which does work on the system. While the tangential velocity of 
								the puck does not change, its radial velocity changes (since we are puling it),
								and thus energy is being put into the system so the energy is not conserved.
						\end{solution}
				\item Another puck of mass $m$ on frictionless ice is attached by a horizontal string of length
						$l$ to a very thin vertical pole of radius $R$. The puck is given a kick and circles
						around the pole with initial speed $v_0$. The string wraps around the pole, and the puck
						gets drawn in and eventually hits the pole. What quantity is preserved during the 
						motion? What is the puck's speed right before it hits the pole?

						Answer the same questions from (i)

						\begin{solution}
								Here, the energy is conserved, because the tension in the string always acts
								perpendicular to the motion of the puck, so no work is done to the puck. 
								However, the angular momentum here is not conserved, because the path of the 
								puck spirals into the pole, and in order for this to occur a torque must have
								caused that motion. Therefore, using conservation of energy, we get:
								\[
								\frac{1}{2}m v_i^2 = \frac{1}{2} m v_f^2 \implies v_f = v_i
								\] 
						\end{solution}
			\end{enumerate}	
	\end{enumerate}

	\pagebreak

	\section*{Problem 2}
	Suppose you have a satellite orbiting the earth (mass $M$) at a radius $R$. We want to move the satellite
	to a higher orbit at radius $R' > R$. One way to make such a transfer is by the following maneuver, 
	illustrated and described in the figure overleaf. Calculate the boosts $\Delta v$ and $\Delta v'$ needed.
	It'll be useful to consider when energy and momentum are conserved. 
	
	\begin{solution}
			We can calculate the initial velocity of the satellite can be calculated as 
			\[ \frac{mv^2}{R} = \frac{GMm}{r^2} \implies v_1 = \sqrt{\frac{GM}{R}} \]
			I will stick to using $v_1$ for now, to simplify the algebra. We know that the energy right after 
			the boost is:
			\[
			E_1 = \frac{1}{2}m(v_1 + \Delta v)^2 - \frac{GMm}{R}
			\] 
			And since angular momentum is conserved, then we also know that $m(v_1 + \Delta v)R = mv_2R'$, 
			where $v_2$ is the velocity of the satellite just before the second boost. This gives: 
			\[
			v_2 = \frac{(v + \Delta v)R}{R'}
			\] 
			Writing out the energy at this location: 
			\[
			E_1 = \frac{1}{2}m v_2^2 - \frac{GMm}{R'}
			\] 
			And since energy is conserved, these two quantities are equal to each other (I've suggestively 
			labeled them as I've used $E_1$ in both equations, but the real principle is that energy is 
			conserved). Therefore: 
			\begin{align*}
					\frac{1}{2}m \left[ \frac{(v_1 + \Delta v)R}{R'} \right]^2 - \frac{GMm}{R'} &= 
					\frac{1}{2}m (v_1 + \Delta v)^2 - \frac{GMm}{R}\\
					\frac{(v_1 + \Delta v)^2 R^2}{R'^2} - \frac{2GM}{R'} &=  (v_1 + \Delta v)^2 - \frac{2GM}{R} \\
					(v_1 + \Delta v)^2 &=
					\frac{2GM\left( \frac{1}{R'} - \frac{1}{R} \right) }{\frac{R^2}{R'^2} - 1} \\
									   &=  \frac{2GMR'}{R(R + R')} \\
					\therefore \Delta v &= \sqrt{\frac{2GMR'}{R(R + R')}} - \sqrt{\frac{GM}{R}} 
			\end{align*}
			Since the second boost is circular, then we can also write:
			\[
			v_f = \sqrt{\frac{GM}{R'}} 
			\] 
			And since we know $\Delta v' = v_f - v_2$, we can just solve for $\Delta v$ explicitly: 
			\begin{align*}
					\Delta v &= v_f - v_2 \\
					&= \sqrt{\frac{GM}{R'}}- \frac{R}{R'}\left( \sqrt{\frac{2GMR'}{R(R + R')}}  \right) 
			\end{align*}
			
	\end{solution}
	
	\pagebreak
	\section*{Problem 3}
	For a particle subject to an inverse square force, written $\mathbf F = -\frac{km}{r^2} \mathbf{\hat{r}}$, 
	we define the vectors
	\[ \mathbf{h = r \times \hat{r}}\]
	and 
	\[ \mathbf e = \frac{\mathbf{\dot r \times h}}{k} - \hat{\mathbf{r}}\]
	The vector $\mathbf e$ is known as the \textit{Runge-Lenz vector}. It can be used to show that particles 
	follow conic sections when acted upon by an inverse square force. For this problem, it may be useful to 
	consult \url{https://en.wikipedia.org/wiki/Triple_product}
	\begin{enumerate}[label=\alph*)]
			\item Show that $\mathbf h$ is constant, and hence that under arbitrary conditions the particle
					will be confined to move in a plane passing through the origin.

					\begin{solution}
							To show that $\mathbf h$ is constant, we show instead that $\dot{\mathbf h} = 0$. 
							Taking the derivative, we get:
							\begin{align*}
									\dot{\vec h} &= \vec r \times \dot{\vec r} + \vec{r} \times \ddot{\vec{r}}\\
									&= \vec r  \times \left( -\frac{k}{r^2}\hat{r} \right)  \\
									&= -\frac{k}{r^2}(\vec r \times \hat{r}) = 0
							\end{align*}
					\end{solution}
			\item Show that $\mathbf e$ is also constant.
					
					\begin{solution}
							We do the same thing as the previous problem: 
							\begin{align*}
									\dot{\vec e} &= \frac{1}{k}\left( \ddot{vec r} \times \vec h
									 + \dot{\vec r} \times \dot{\vec h} \right) - \dot{\hat{r}} \\
												 &= \frac{1}{k}\left( -\frac{km}{r^2}\hat{r}\times \vec{h} 
												 \right)  \\
												 &= \frac{1}{r^2}(\hat{r}\times (\vec{r} \times \dot{\vec r}))
												 - \dot \theta \hat{\theta}\\
												 &= -\frac{1}{r^2}\left( \vec{r}(\hat{r}\cdot \dot{\vec r}) -
												 (\hat{r} \cdot \vec r) \dot{\vec r}\right)  - 
												 \dot \theta \hat{\theta}\\
												 &= -\frac{1}{r^2}\left( \hat{r}\dot{r} - \dot{\vec r} \right) -
												 \dot{\theta} \hat{\theta}\\
												 &= -\frac{1}{r}\left( \hat{r}\dot r - (\dot r \hat{r} 
												 + r\dot \theta \hat{\theta}) \right) -
												 	\dot \theta \hat{\theta} \\
												 &= \dot{\theta} \hat{\theta} - \dot \theta \hat{\theta} = 0 \\
							\end{align*}
					\end{solution}
			\item Show that 
					\[ er \cos \theta = \frac{h^2}{k}-r\]
					where $e = |\mathbf e|$ and $h = |\mathbf h|$ and $\theta$ is the angle between $\mathbf r$
					and $\mathbf e$. Since $\mathbf h$ and $\mathbf e$ are constant, we can solve for 
					$r(\theta)$ to deduce that the particle will follow a conic section trajectory.

					\begin{solution}
							We compute $\vec{e} \cdot \vec{r}$ so we get: 
							\begin{align*}
									\vec{e} \cdot \vec{r} = er\cos \theta &=
									\left( \frac{\dot{\vec{r}} \times \vec{h}}{k} -
									\hat{r} \right)\cdot \vec{r}  \\  
									&= \frac{(\dot{\vec{r}} \times \vec{h})\cdot \vec{r}}{k} -
									\frac{\vec{r}}{r} \cdot \vec{r} \\
									&= \frac{(\vec{r}\times \dot{\vec{r}}) \cdot \vec{h}}{k} - \frac{r^2}{r} \\
									&= \frac{\vec{h} \cdot \vec{h}}{k}-r  \\
									&= \frac{h^2}{k} - r 
							\end{align*}
							as desired. 
					\end{solution}

	\end{enumerate}

	\pagebreak

	\section*{Problem 4}
	Suppose we have a particle in a central force potential
	\[ V(r) = -\frac{A}{r} + \frac{B}{2mr^2}\]
	where $A$ and $B$ are constants, and $m$ is the mass of the particle (the factor of $2m$ appears for 
	convenience only). 
	\begin{enumerate}[label=\alph*)]
			\item Show that the equation of orbit has an exact solution that can be put in the form 
					\[ \frac{r_0}{r} = 1 + \epsilon \cos (\alpha \theta)\]
					\textit{(Hint: repeat the orbit derivation with the new potential)}

					\begin{solution}
							As per the hint, we solve the orbit equation:
							\[
									\dv[2]{u}{\phi} + u(\phi) = -\frac{m}{\ell^2} \frac{1}{u^2}F(u)
							\] 
							We find the force using 
							\begin{align*}
									F(r) &= -\dv{u}{r}\\
									&= -\frac{A}{r^2} + \frac{B}{r^3} \\
									&= -Au^2 + \frac{B}{m}u^3 \\
							\end{align*}
							So now we solve the differential equation: 
							\begin{align*}
									\dv[2]{u}{\phi} + u(\phi) &= -\frac{m}{\ell^2}\frac{1}{u^2}\left(-Au^2 +
									\frac{B}{m}u^3\right) \\
									-\frac{\ell^2u^2}{m}\dv[2]{u}{\phi} &= -Au^2 +
									u^3\left( \frac{\ell^2 + B}{m} \right) \\	
									-\frac{\ell^2}{m}u''(\phi) &= -A + u\left( \frac{\ell^2 + B}{m} \right) \\
									\therefore u''(\phi) &= \frac{Am}{\ell^2} - u\left( 1 +
									\frac{B}{l^2} \right)  \\
							\end{align*}
							Now let $k^2 = 1 + \frac{B}{\ell^2}$:
							\[
								u''(\phi) + k^2 u = \frac{Am}{\ell^2}
							\] 
							Now let $w(\phi) = k^2u -\frac{Am}{\ell^2}$, this gets us $w''(\phi) = k^2u''(\phi)$,
							so this gets us: 
							\[
							w''(\phi) + k^2w(\phi) = 0
							\] 
							which is the equation for simple harmonic motion. Therefore, we get the solution 
							\[
							w(\phi) = C\cos(k\phi-\delta)
							\] 
							Here, we can let $\delta = 0$ for an appropriate choice for $\phi = 0$, so we get
							\[
							w(\phi) = C\cos(k\phi)
							\] 
							As the general solution. Substituting back $u$, we get:
							\begin{align*}
									k^2 u(\phi) - \frac{Am}{\ell^2} &=  C\cos(k\phi) \\
									\therefore u(\phi) &= \frac{C}{k^2} + \frac{Am}{k^2\ell^2} \\
							\end{align*}
							Finally, we return $u = \frac{1}{r}$:
							\begin{align*}
									\frac{1}{r}&= \frac{C}{k^2}\cos (k\phi) + \frac{Am}{k^2\ell^2} \\
									&= \frac{Am}{k^2\ell^2}\left(1 + \frac{C\ell^2}{Am}\right) \\
							\end{align*}
							We let $k = \alpha$ to match with the problem, so we get: 
							\[
							\frac{1}{r} = \frac{Am}{\alpha\ell^2}\left( 1 + 
							\frac{C\ell^2}{Am}\cos(\alpha \phi) \right) 
							\] 
							setting $\alpha = \frac{k^2l^2}{Am}$ and $\epsilon = \frac{C\ell^2}{Am}$, we get the
							equation: 
							\[
							\frac{r_0}{r} = 1 + \epsilon \cos (\alpha \phi)
							\] 
					\end{solution}
			\item The trajectory in (a)  is an ellipse of $\alpha = 1$, but if $\alpha \neq 1$, the ellipse 
					\textit{precesses}, which may be described in terms of the precession of the apsides 
					(turning points). For $B$ small, derive an approximate expression for the precession rate.
					You may express your answer in terms of the period of the orbit where $B = 0$ - call it $T$.

					\begin{solution}
							The apsides are the furthest and nearest point, which are described by
							$\cos(\alpha\phi) = 1$, or equivalently when $\alpha \theta = 2\pi n$. This allows 
							us to solve for $\theta$: 
							\[
							\theta = \frac{2\pi n}{\alpha}
							\] 
							So the deviation per orbit can be calculated is 
							\[
							\Delta \theta = \frac{2 \pi (n+1)}{\alpha} - \frac{2\pi n}{\alpha} =
							\frac{2\pi}{\alpha}
							\] 
							So per one orbit, the deviation from $\theta$ from a singular orbit is:
							\[
							\delta \theta = 2\pi - \Delta \theta = 2\pi\left(1 - \frac{1}{\alpha}\right) 
							= 2\pi\left[ 1 - \left( 1 + \frac{B}{\ell^2} \right)^{-1/2}\right] 
							\] 
							Taylor expanding this quantity, we get: 
							\begin{align*}
									2\pi \left[ 1 - \left( 1 + \frac{B}{\ell^2} \right) ^{-1/2}\right] &= 
									2\pi\left[1 - \left( 1-\frac{B}{2\ell^2} \right) \right]\\
									&= 2\pi\left( \frac{B}{2\ell^2} \right)  \\
									&= \frac{\pi B}{\ell^2} \\
							\end{align*}
							For small $B$, where we've dropped the higher order terms. Therefore, the period of
							precession is: 
							\[
									\frac{\delta \theta}{T} = \frac{\pi B}{T\ell^2}
							\] 
					\end{solution}
	\end{enumerate}
\end{document}

