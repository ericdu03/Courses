\documentclass[10pt]{article}
\usepackage{../../local}
\usepackage{listings}
\lstdefinestyle{tt}{basicstyle=\small\ttfamily,keywordstyle=\bfseries,language=[LaTeX]{TeX}}



\newcommand{\classcode}{CS 170}
\newcommand{\classname}{Efficient Algorithms and Intractable Problems}
\renewcommand{\maketitle}{%
\hrule height4pt
\large{Eric Du \hfill \classcode}
\newline
\large{Notes} \Large{\hfill \classname \hfill} \large{\today}
\hrule height4pt \vskip .7em
\small{Header styling inspired by CS 70: \url{https://www.eecs70.org/}}
\normalsize
}
\linespread{1.1}

\DeclareMathOperator*{\pre}{pre}
\DeclareMathOperator*{\prev}{prev}
\DeclareMathOperator*{\post}{post}
\DeclareMathOperator*{\dist}{dist}
\DeclareMathOperator*{\cost}{cost}
\DeclareMathOperator*{\size}{size}
\DeclareMathOperator{\capacity}{capacity}
\DeclareMathOperator*{\Var}{Var}
\DeclareMathOperator*{\len}{len}

%\newcommand{\question}[1]{\textcolor{red}{#1}}
%\newcommand{\answer}[1]{\textcolor{green!80!black!}{#1}}
%\renewcommand{\comment}[1]{\textcolor{blue!50}{#1}}
\begin{document}
	\maketitle
	\section{Introduction}
\begin{itemize}
	\item Let's go back to 1974, e
	\item No classical theory permits this gradual energy loss, except in General Relativity!
	\item Speaking of General relativity, one thing it predicted was the present of gravitational waves (GW), and 
		this was where the lost energy was going. Specifically, we can calculate its power:
		\[
		P = -\frac{2}{5}\frac{G^{4}M^{5}}{R^{5}c^{5}}
		\] 
	\item We commonly think of \( G = 6.67 \times 10^{-11}\), but later in the course we're going to work in units 
		where \( G = 1 \), to simplify things. 
	\item In electrodynamics, an charge \( q \) that experiences an acceleration also emits electromagnetic waves. This
		is called synchrotron radiation:
		\[
			P = -\frac{2}{3} \left( \frac{q^2}{4\pi \epsilon_0} \right) \frac{a^2}{c^3}
		\] 
		To get a sense of the scale of this power, the amount that our solar system is losing due to the sun and 
		Jupiter is around 200 W. But Hudson and Taylor found a power of \( P = -7 \cdot 10^{34} \) W!
	\item In 1983, they measured this, in 1993 they won the Nobel prize for their indirect detection of 
		gravitational waves. In 2015, LIGO detected these waves directly, and found a power 
		\( P = 3.6 \times 10^{49} \) W. In 2017, they won the Nobel prize for this discovery. 
\end{itemize}
\subsection{Why General Relativity?}
\begin{itemize}
	\item In Newton's gravity, we have the equation \( \vec F_i = \dv[2]{\vec r_i}{t} \), and for universal 
		gravitation, we had:
		\[
		\vec F_i = \sum_{j\neq i}\frac{G m_i m_j}{|\vec r_i - \vec r_j|^3}(\vec r_j - \vec r_i)
		\] 
		\comment{Note that it's only formatted like this so that we can talk about vectors.}
	\item One problem with this interpretation is that things are instantaneous: this is an issue because objects
		don't react instantaneously to changes (information 
		can't travel faster than the speed of light), which Newton's equations seem to imply. 

		We can say the same about Coulomb's law: and the solution there was to replace the notion of a force 
		with \textit{fields}. Now, the force can be written as: 
		\[
		\vec F_i = q(\vec E + \frac{1}{c}\vec v \times \vec B)
		\] 
		\comment{We'll use Gaussian units, mainly because \( \vec E \) and \( \vec B \) now have the 
		same units. With this,
		\begin{align*}
			\vec \div E &= 4 \pi \rho \\
			\vec \div B &=  0 \\
			\vec \curl E &= -\frac{1}{c}\pdv{\vec B}{t}\\
			\vec \curl B &= \frac{1}{c}\pdv{\vec E}{t} + \frac{4\pi}{c}\vec J 
		\end{align*}		}
		So the question then becomes: why didn't we do this for Gravitation? Well, this is a thing, but it's 
		only an approximation. 
	\item Let's talk about energy conservation: in E\&M, the energy is written as:
		\[
		\mathcal E = \frac{1}{8\pi}\int (\vec E^2 + \vec B ^2) dv + \sum_i k_i
		\] 
		What happens when we change the sign on everything to accomodate for gravitation? Then, we introduce instability 
		into the system! 
\end{itemize}

	\section{Divide and Conquer I, Asymptotics}
\begin{itemize}
	\item Last time, we talked about the motivations for studying algorithms: designing ways to solve problems efficiently. 
	\item We also talked about addition and multiplication, and the latter in particular we saw two algorithms for it, but 
		couldn't break the \( O(n^2) \) runtime. Today, we'll try to beat this.
\end{itemize}

\subsection{Karatsuba's Algorithm}
\begin{itemize}
	\item The main issue we ran into with the divide and conquer algorithm is that when we broke a problem down, we didn't 
		actually simplify our life at all -- we just created subproblmes for ourselves. What if we can create 
		fewer than 4 subproblems? This is the key idea with divide and conquer: if we can use the results of subproblems 
		to simplify computation at a given layer, we can generate an overall speedup.
	\item In Karatsuba's case, if we can write the term \( \text{P2} + \text{P3} \) in terms of P1 and P4, then we can 
		simplify the number of computations needed. 
	\item Karatsuba's trick is as follows: let's compute only three things:
		\begin{itemize}
			\item Q1: \( a \times c \) 
			\item Q2: \( b \times d \) 
			\item Q3: \( (a + b)(c + d) \)
		\end{itemize}
		Then, the idea is that the middle term \( \text{P2} +  \text{P3} \) can be written in terms of these smaller subproblems:
		\[
		a\times d + c\times b = (a + b)(c + d) - ac - bd
		\] 
		Therefore, our multiplication now looks like:
		\begin{align*}
			x \times y &= \left( a \times 10^{n / 2} + b \right) \left( c \times 10^{n /2} + d \right) \\
					   &= \underbrace{(a \times c)}_{\text{Q1}} 10^{n} + \underbrace{(a \times d + c \times b)}_{\text{Q3} - 
					   \text{Q1} - \text{Q2}} 10^{ n / 2} + \underbrace{(b \times d)}_{\text{Q3}}
		\end{align*}
	\item With this algorithm, what is the runtime of this algorithm? The problem is almost the same, except we now only have 
		3 subproblems instead of 4. Therefore, if we employ the tree method (just as last time), then we'll see that 
		we have \( \log_2(n) \) layers, which means we have \( 3^{\log_2 n} = n^{\log_2 3} \approx n^{1.6} \), which is 
		where we get our runtime from.
		\begin{itemize}
			\item The Toom-3 algorithm mentioned last lecture also uses divide and conquer, but instead it reduces 
				9 problems into 5 subproblems, which gives a better bound. 
		\end{itemize}
	\item Note that Karatsuba's algorithm also doesn't care about the base we're working in: this works with any base. 
		Note that the factor of \( 2^{n /2} \) that will appear in the division step is also just adding zeros, but in binary! 
\end{itemize}
\subsection{Asymptotic Notations (Formally)}
\begin{itemize}
	\item Suppose an algorithm takes \( T(n) = 5n^2 + 20n \log n + 7 \) microseconds. Then, we say that \( T(n) \in O(n^2) \), or 
		also sometimes written as \( T(n) = O(n^2) \).
	\item Why do we employ this \( O(\cdot) \) notation? It's because constants like 5, 20, 7 usually depend on the computer 
		(say your computer is a year newer and has a faster CPU inside), so getting rid of these constants make comparing algorithms
		much easier. It's also often the case that the constants can be improved via some other smaller, less important 
		optimizations. 
	\item The formal definition of \( O(\cdot) \) is as follows: 

		Let \( T(n) \) and \( g(n) \) be functions of positive integers. Think of \( T(n) \) as a runtime, so it's positive and 
		increasing with \( n \) (usually). Then, we say "\( T(n) \) is  \( O(g(n)) \)" if and only if for some large enough 
		\( n \), \( T(n) \) is at most some constant multiple of \( g(n) \). Mathematically:

		There exists \( c \) and \( n_0 > 0 \) such that for all \( n \ge  n_0\), \( T(n) \le  c \cdot g(n) \). 

		Note that this \( g(n) \) also isn't unique! If a function is in \( O(n^2) \), then it's also \( O(n^3) \), and it's 
		also in \( O(2n^2) \). However, we generally ask for the simplest and the tightest bound, so while these are 
		all technically correct answers, \( O(n^2) \) is the "most correct". 
	\item As an example, we can prove that \( T(n) = 2n^2 + 2 \in O(n^2)\). Here, we can choose \( n_0 = 1 \), \( c = 4 \), 
		so that we have: 
		\[
		2n^2 + 2 \le 4n^2
		\] 
		All we have to do is prove that this inequality holds for all \( n \ge  n_0 \). We can do this via derivatives, or other 
		equivalent methods. 
	\item There's an equivalent definition for a lower bound: we say that "\( T(n) \in \Omega(g(n)) \)" if and only if 
		there exists \( c \) and \( n_0 > 0 \) such that for all \( n \ge n_0 \), \( c \cdot g(n) \le T(n) \). Note that this 
		inequality is reversed from the previous one. 

	\item To test asymptotics, one way that's particularly efficient is using limits: 
		\[
			\lim_{n \to \infty} \frac{T(n)}{g(n)} = \begin{cases}
				0 & T(n) \in O(g(n))\\
				c \in \mathbb R & T(n) \in \Theta(g(n))\\
				\infty & T(n) \in \Omega(g(n))
			\end{cases}
		\] 
	\item The asymptotics of the geometric series is quite important for runtime analysis. Take any constant \( r \) and 
		a function \( T(n) = 1 + r + r^2 + \cdot + r^{n} \). We have that:
		\[
		T(n) = \begin{cases}
			\Theta{r^{n}} & \text{if \( r > 1 \) }\\
			\Theta(1) & \text{if \( r < 1 \)}\\
			\Theta(n) & \text{if \( r = 1 \)}
		\end{cases}
		\] 
		\textit{Proof:} Recall that for a goemetric series with \( r \neq 1 \), then we have:
		\[
		1 + r + r^2 + \cdots + r^{n} = \frac{r^{n+1} - 1}{r-1}
		\] 
		For \( r > 1 \), then this right hand side roughly evalutes to \( \frac{r^{n + 1}}{r} = r^{n} \), hence the 
		\( \Theta(r^n) \) bound. For \( r < 1 \), \( r^{n + 1} \) is going to be very small, and hence \( r^{n + 1} - 1 < 0\). 
		Overall, this means 
		\[
			\frac{r^{n + 1} - 1}{r - 1} \approx \frac{1}{1 - r}
		\] 
		and since \( r \) is a constant we have \( T(n) = \Theta(1) \). For \( r = 1 \), then we have \( T(n) = n \in \Theta(n)\). 
\end{itemize}
\subsection{Formal Proof of Karatsuba's}
\begin{itemize}
	\item Now we formally look at Karatsuba's algorithm runtime. At each layer, we have 3 subproblems, each of size \( n / 2 \).  
		At every layer, we have to do a bunch of things: finding Q1, Q2, Q3, additions, and other stuff. However, all of 
		this stuff runs in \( O(n) \) time. To use a specific number, we'll say that the work is \( 20n \). Therefore, 
		we have the following formula for \( T(n) \): 
		\[
		T(n) = 3T\left( \frac{n}{2} \right)  + 20n
		\] 
		This is a \textbf{recurrence relation.} We should also have a base case: \( T(1) = O(1) \). Now, our goal is to find a 
		closed form relation to \( T(n) \). 

		Now we look at this layer by layer. At the first layer, we have 1 problem, so that has \( 20n \) units of work. At 
		the second layer, we have 2 subproblems, each of size \( n / 2 \), so we have \( 3 \times 20 \times \frac{n}{2} \) 
		amount of work from this layer. In general, we have:
		\[
		\text{work} = (\text{number of subproblems}) \times 20 \times (\text{subproblem size})
		\] 
		This translates into the equation
		\[
		3^{t} \cdot 20 \left( \frac{n}{2^{t}} \right) 
		\] 
		We now need to sum this for all \( t \), so we have:
		\begin{align*}
			T(n) &= \sum_{t = 0}^{\log_2 n}3^{t} \cdot 20 \cdot \left( \frac{n}{2^{t}} \right) \\
			&= 20n \sum \left( \frac{3}{2} \right)^{t} 
		\end{align*}
		Now recall the geometric series we had from earlier: since \( r = \frac{3}{2}> 1 \), then the summation 
		is \( \Theta((3 / 2)^{\log_2 n}) \), so overall:
		\[
		T(n) = 20n \left( \Theta\left( \frac{3}{2} \right)^{\log(n)} \right) = 
		O\left(n \left( \frac{3}{2} \right) ^{\log 3 - \log 2}\right) = O(n^{\log 3}) = O(n^{1.6})
		\] 
\end{itemize}
\subsection{The Master Theorem}
\begin{itemize}
	\item The tree method is useful, but slightly annoying to deal with sometimes. There's a theorem, called the Master Theorem, 
		that tells us the runtime of \( T(n) \), if we have a recurrence relation of the following form:
		\[
		T(n) = a \cdot T\left( \frac{n}{b} \right) + O(n^{d})
		\] 
		Then, we have:
		\[
		T(n) = \begin{cases}
			O(n^{d}) & \text{if \( a < b^{d} \)}\\
			O(n^{d} \log(n)) & \text{if \( a = b^{d} \)}\\
			O(n^{\log_b(a)}) & \text{if \( a > b^{d} \)}
		\end{cases}
		\] 
	\item The Master Theorem only tells us the runtime given this very specific recurrence relation. In fact, as you'll notice 
		with the \( O(n^{d}) \) term, it only works if the work at every step is polynomial. Also, if \( n / b \) is not an 
		integer, we can force it to be an integer by enforcing a recurrence relation of the form:
		\[
		T(n) = a \cdot T\left( \left\lceil \frac{n}{b} \right\rceil  \right) + O(n^{d})
		\] 
		Loosely, this is because constants don't matter, so small shifts (of less than 1) in the subproblem size doesn't really 
		change the recurrence relation at all. 
\end{itemize}


	\section{Characterization Continued}
\subsection{Step Response}
\begin{itemize}
	\item The step response function is the function \( y_{\text{step}}(t) \) when a step function 
		\( u(t) \) is fed into the system. In discrete-time: we feed \( u[n] \) into the system, and get 
		\( y_{\text{step}}[n] \) as an output.
	\item For instance, for the moving average filter defined earlier, we have the following result:
		\begin{center}
			\begin{tabular}{c|c}
				\( n \) &  \( y_{\text{step}}[n] \)\\
				\hline 
				-2 & 0\\
				-1 & 1/3\\
				0 & 2/3\\
				1 & 1\\
				2 & 1
			\end{tabular}
		\end{center}
		Note that this resembles a ramp function, and is called a ramp-step function.
	\item \textbf{Harmonic Response:} The harmonic response is the response by the system when presented with 
		a harmonic function, of the form \( Ae^{i \omega t}\). 

		In discrete time, we feed in \( Ae^{i \omega n} \) where \( n \) is an integer. 
	\item For the moving average filter, let's write out \( y[n] \) :
		\begin{align*}
			y[n] &= \frac{1}{3}\left(Ae^{i \omega (n - 1)} + Ae^{i \omega n} + Ae^{i \omega (n + 1)}\right)\\
			&= \frac{1}{3}\left( e^{-i \omega} + 1 + e^{i \omega} \right)  \\
			&= \frac{1}{3}(2 \cos \omega + 1) Ae^{i \omega n} \\
		\end{align*}
		The interesting thing here is that when given a harmonic function, the system response just scales the signal 
		by a constant amount!
\end{itemize}
\subsection{LCCDE} 
\begin{itemize}
	\item In this class, we will deal with lots of differential equations, so it's going to be very useful to 
		look at their form, and how to solve them.   
	\item There are two solutions to any differential equation: 
		\begin{itemize}
			\item \textbf{Particular Solution:} \( y_p(t) \) is called a particular solution if it satisfies:
				\[
					\sum_{k = 0}^{N}a_k \dv[k]{y_p(t)}{t} = \sum_{k = 0}^{N}b_k \dv[k]{x(t)}{t}
				\] 
			\item \textbf{Homogeneous Solution:} \( y_h(t) \) is called a homogeneous solution if it satisfies:  
				\[
						\sum_{k = 0}^{N}a_k \dv[k]{y_p(t)}{t} =	0
				\] 
		\end{itemize}
	\item In general, the solution will be a linear combination of the two: 
		\[
		y(t) = y_p(T) + ay_h(t)
		\] 
		the value of \( a \) is generally going to be given by some initial condition. 
	\item For the homogeneous solution, an ansats of the form \( Ae^{st} \) where \( s \) is an undetermined constant 
		will solve the differental equation. We can then determine the value of \( s \) by solving the resulting 
		polynomial.

		To determine the value of \( A \), these are determined by the initial conditions, and depending on the 
		number of initial conditions given, that would correspond directly to the number of distinct values of \( A \). 
\end{itemize}


	\section{Polynomial Multiplication}
\begin{itemize}
	\item Earlier, we talked about multiplying numbers and also matrices quickly. Today, we'll talk about manipulating polynomials.
	\item Given two polynomials \( p(x) \) and \( q(x) \), what are the fastest algorithms that add and 
		multiply the two polynomials together?
\end{itemize}
\subsection{Representing Polynoimals}
\begin{itemize}
	\item Typically, we'd write a degree \( n-1 \) polynomial as 
		\( p(x) = p_0 + p_1x + p_2x^2 + \cdots + p_{n-1}x^{n-1} \). This is called
		the \textbf{coefficient representation}, represented by an array of numbers \( (p_0, p_1, \cdots, p_{n-1}) \). 
		Note that \( p_i \) 
		 are \textit{real numbers}, not necessarily integers.  
	 \item We will think of \( n \) as being very large, say \( n = 10^{10} \), while thinking that \( p_0, \cdots, p_{n-1} \)
		 to be very small. So we'll imagine that all these arithmetic operations are going to take \( O(1) \) time.  
	 \item Goal: measure runtime as a function of \( n \), not on the coefficients themselves.    
\end{itemize}
\subsection{Adding Polynomials}
\begin{itemize}
	\item Given two polynomials \( p(x) \) and \( q(x) \), we want to output \( r(x) = p(x) + q(x) \), in its coefficient 
		representation. 
	\item How fast can we do this? To find the coefficients of \( r(x) \), we just add \( r_i = p_i + q_i \), and since each 
		takes constant time, there are \( n \) additions, hence \( O(n) \). 
	\item This is like adding integers, but even simpler, since there is no carry over term!
\end{itemize}
\subsection{Evaluating Polynomials}
\begin{itemize}
	\item Given an input \( p(x) = p_0 + p_1x + p_2x^2 + \cdots + p_{n-1}x^{n-1} \), and a real number \( \alpha \in \mathbb R \). 
	\item We want to output \( p(\alpha) = p_0 + p_1\alpha + p_2\alpha^2 + \cdots + p_{n-1}\alpha ^{n-1} \in \mathbb R \). 
	\item How fast can we do this? There are three algorithms that take \( O(n^2) \), \( O(n \log n) \), and \( O(n) \) 
		respectively. We'll first take a look at the \( O(n^2) \) and the \( O(n) \) ones. 
		\begin{itemize}
			\item Algorithm 1: We compute the terms individually and add them together: 
				\begin{align*}
					&\phantom{+a} p_0 & \text{0 multiplications}\\
					&+ p_1 \cdot \alpha & \text{1 multiplication}\\
					&+ p_2 \cdot \alpha \cdot \alpha & \text{2 multiplications}\\
					&+ p_3 \cdot \alpha \cdot \alpha \cdot \alpha & \text{3 multiplications}\\
					&\vdots\\
					&+ p_{n-1} \cdot \alpha \cdot \alpha \cdots \alpha & \text{\( n-1 \) multiplications}\\
					&\rule{4cm}{0.6pt}&\\
					&p(\alpha) & \text{\( O(n^2) \) multiplications}
				\end{align*}
			Notice the repeated computation we have here: when we compute \( \alpha^3 \), we don't actually need to multiply
				\( \alpha \) three times, since we've already computed \( \alpha^2 \) in the previous step. 
			\item Algorithm 2: Initialize an array \( A \), and set \( A[i] = \alpha \cdot A[i-1] \) for each 
				\( i = 1, 2, \dots \). Therefore,  \( A = [1, \alpha, \alpha^2 , \alpha^3, \cdots] \). So every step here 
				takes 1 multiplication, so to compute the whole array \( A \) takes \( O(n) \) steps. Now, if we were to evaluate 
				now:
				\begin{align*}
					&\phantom{+a} p_0 \cdot A[0]& \text{1 multiplication}\\
					&+ p_1 \cdot A[1]& \text{1 multiplication}\\
					&+ p_2 \cdot A[2] & \text{1 multiplication}\\
					&+ p_3 \cdot A[3] & \text{1 multiplication}\\
					&\vdots\\
					&+ p_{n-1} \cdot A[n-1]& \text{1 multiplication}\\
					&\rule{4cm}{0.6pt}&\\
					&p(\alpha) & \text{\( O(n) \) multiplications}
				\end{align*}
				Therefore, this will take \( O(n) \) total steps, hence an \( O(n) \) runtime.  

				\comment{Note here that \( n \) (the problem size) refers to the length of the polynomial, and not the size 
					of the coefficients \( p_i \). We assume these to be small for our analysis, but this is not true in practice, 
				and these computations will add up.}
		\end{itemize}
\end{itemize}
\subsection{Multiplying Polynomials}
\begin{itemize}
	\item Given two polynomials \( p(x) \) and \( q(x) \), we want to output \( p(x) \cdot q(x) \). For instance, \( p(x) = 
		(7 - 5x), q(x) = (1 + 3x + 2x^2)\). If we were to do this by hand, we find that \( p(x) \cdot q(x) = 
		7 + 26x + 29x^2 + 10x^3\).
	\item How fast can we do this? \( O(n^2) \). This is because for every coefficient in \( p \) we need to perform 
		\( n \) multiplications (one for every coefficient in \( q \)), and since \( p \) has \( n \) coefficients then there are 
		\( n^2 \) total multiplications. Hence, the runtime is \( O(n^2) \). 

		Note also that in doing so, we also increase the degree of the product, with it being a degree \( 2n-2 \) polynomial.

		Our goal for the rest of today's lecture is to improve this to \( O(n \log n)  \) time.  
	\item To do this, we use the fact that \( n \) points determine a degree \( n-1 \) polynomial (recall this from 
		cs70 notes) In other words, 
		if \( p(x) \) is degree 1, then 2 points suffice; if \( p(x) \) is degree 2, then we need 3 points, and so on.
	\item This is useful because instead of representing polynomials 
		by their coefficients, we can instead represent them by values given certain inputs. So given some points \( \alpha_1, 
		\alpha_2, \dots, \alpha_m \in \mathbb R\), the value representation of \( p(x) \) is given by 
		\( (p(\alpha_1), p(\alpha_2), \dots, p(\alpha_m)) \). 
	\item Using the previous fact, as long as \( m \ge n  \), then the polynomial is uniquely determined. For us, 
		a typical choice of \( m \) is \( m = O(n) \). 
\end{itemize}
\subsubsection{Adding and Multiplying with Value Representation}
\begin{itemize}
	\item Given two polynomials in their \textit{value representation}, how would we add these two polynomials together? We can 
		just add these two values together, and output \( (p(\alpha_1) + q(\alpha_1), \dots, p(\alpha_m) + q(\alpha_m) \). This 
		takes \( O(n) \) time. 
	\item For multiplication, we output the product of the values: \( (p(\alpha_1) \cdot q(\alpha_1), \dots, 
		p(\alpha_m) \cdot q(\alpha_m) \). This is also \( O(n) \). However, one thing to note with multiplication is because
		the degree of the product changes, we need \( m \ge 2n-1 \) in order for the result to uniquely specify the product. 
	\item The takeaway is that multiplication is much faster in the value representation compared to the coefficient representation!
\end{itemize}
\subsection{Fast Polynomial Multiplication}
\begin{itemize}
	\item The last section motivates a scheme where we multiply polynomials using their value representation rather than their 
		coefficient representation. Therefore, we need the following scheme:
		\begin{center}
			\begin{tikzpicture}[every text node part/.style={align=center}]
				\foreach \x in {0, 7}
				\foreach \y in {0, 3}
				{
					\draw (\x, \y) -- (\x+3, \y) -- (\x+3, \y+1.5) -- (\x, \y+1.5) -- cycle;
				}
				\draw node at (1.5, 0.75) {Coefficients of \\ \( p \cdot q \)};
				\draw node at (1.5, 3.75) {Coefficients of \\ \( p \) and \( q \) };
				\draw node at (8.5, 0.75) {Values \\ \( p(\alpha_i) \cdot q(\alpha_i) \) };
				\draw node at (8.5, 3.75) {Values \\ \( p(\alpha_1), \dots, p(\alpha_m) \) \\ \( q(\alpha_1), \dots, q(\alpha_m) \) };
				\draw[-stealth, red, thick] (3.5, 3.75) --node[midway, above] {Evaluation} node[midway, below] 
					{\( O(n \log n) \) } (6.5, 3.75); 
				\draw[-stealth, red, thick] (8.5, 2.7) -- node[midway, right] {\( O(m) = O(n) \) } (8.5, 1.8);
				\draw[-stealth, red, thick] (6.5, 0.75) -- node[midway, above] {Interpolation} node[midway, below] 
					{\( O(n \log n) \) } (3.5, 0.75);
			\end{tikzpicture}
		\end{center}
		If we can do this whole sequence, then that gives us an efficient multiplication algorithm. However, the evaluation step 
		still takes \( O(m \cdot n) = O(n^2) \) time, so how can we possibly get this down to \( O(n \log n) \)? The secret 
		lies in how we pick \( \alpha_1, \dots, \alpha_m \) -- it is possible to pick them in such a way that the evaluation 
		step takes \( O(n \log n) \) time. Same goes for interpolation. 

		How do we pick \( \alpha_1, \dots, \alpha_m \)? We use complex numbers!
\end{itemize}

\subsection{Complex Numbers}
\begin{itemize}
	\item A complex number is any number of the form \( a + bi \), where \( i = \sqrt{-1}  \). \( a \) represents the real part, and 
		\( b \) represents the imaginary part of the number. We can add complex numbers:
		\[
			(1 + 2i) + (3 + 4i) = (1 + 3) + (2 + 4)i = 4 + 6i
		\] 
		so the real parts add and the imaginary part also adds. To multiply:
		\[
			(1 + 2i) \cdot (3 + 4i) = 1 \cdot 3 + 1 \cdot 4i + 2i \cdot 3 + 2i \cdot 4i = 3 + 10i - 8 = -5 + 10i
		\] 
		Recall that since \( i = \sqrt{-1}  \), then  \( i^2 = -1 \). 
	\item We can also represent complex numbers on the complex plane. A number \( a + bi \) corresponds to  \( (a, b) \) on the 
		complex plane. We can also represent it using polar coordinates, using a radius \( r \) and an angle \( \theta \)
		\begin{center}
			\begin{tikzpicture}
				\draw (-3, 0) -- (3, 0) node[above] {real};
				\draw (0, -3) -- (0, 3) node[above right] {imaginary};
				\draw[red] (0, 0) -- node[midway, above] {\( r \) } (1.8, 1.4) node[above right, black] {\( (a, b) \) }; 
				\filldraw[red] (1.8, 1.4) circle (2pt); 
				\draw[red] (1, 0) arc (0:37.87:1) node[midway, right]{\( \theta \) };
			\end{tikzpicture}
		\end{center}
		With this construction, we can relate polar to cartesian with the relations
		\begin{align*}
			a &= r\cos\theta \\
			b&= r \sin \theta 
		\end{align*}
		For today, we'll only consider points with \( r = 1 \). This means that we basically forget about \( r \), and only 
		worry about \( \theta \). So with \( r = 1 \), then we have:
		\begin{align*}
			a &=  \cos \theta \\
			b &=  \sin \theta 
		\end{align*}
		And since \( r = 1 \), we'll be dealing with points on the \textbf{unit circle}.
	\item Consider two complex numbers, the first specified by \( \theta_1 \), and the other specified by \( \theta_2 \) (both 
		having \( r = 1 \) ). Then, we define the product of the two to be specified by \( \theta_1 + \theta_2 \). To multiply, 
		we just add the angles.
\end{itemize}
\subsubsection{Roots of Unity}
\begin{itemize}
	\item The \( n \)-th root of unity is a solution to the equation \( x^{n} = 1 \). So the second roots of uhnity are solutions 
		to the equation \( x^2 = 1 \), which are \( \pm 1 \). The 4-th roots of unity are solutions to \( x^{4} = 1 \), 
		which are \( \{1, -1, i, -i\}  \). 
	\item In general, the \( n \)-th roots of unity will have \( n \) distinct solutions. 
	\item Graphically, the \( n \)-th roots of unity correspond to \( n \) equally spaced points placed on the unit circle.   
	\item When we talk about the roots of unity, we will use \( \omega_0 \) through \( \omega_{n-1} \) to label them. In particular, 
		we'll focus on \( \omega_1 \).
		\begin{itemize}
			\item Note that \( \omega_1 \) always sits at an angle of \( \frac{2\pi}{n} \) for the \( n \)-th roots of unity, due 
				to the even spacing.
			\item Note that \( \omega_2 = \omega_1 \cdot \omega_1 \), since multiplying is equivalent to adding the angles together.
				Therefore, we have the relation that \( \omega_i = \omega_1^{i} \), which we will call the \textbf{Generator Fact}. 
		\end{itemize}
	\item So this gives us a nice formula for the \( n \)-th roots of unity: they will always sit at angles \( k\theta \), where 
		\( \theta = 2\pi / n \). As angles, this is represented as the set: \( \{\cos(k\ell) + i\sin(k\ell) | \ell= 0, 1, \dots, 
		n - 1\}  \).  
\end{itemize}
\subsubsection{Square Roots}
\begin{itemize}
	\item When we take a square root, remember that they always come in pairs of \( \pm \sqrt{a}  \). So to get the second 
		roots of unity, we find the square roots of 1, which are \(  \pm 1 \). To get the 4-th roots of unity, then we just need 
		to take the square roots of the previous roots of unity. 

		In general, if we take the square roots of the \( n \)-th root of unity, then we generate the \( 2n \)-th root of unity. 
		Conversely, if we square the roots of unity then we get the \( n / 2 \)-th roots of unity. 

		This is the magical fact that we will leverage for the following lecture: squaring the \( n \)-th roots gives us 
		the \( n / 2 \)-th roots of unity. This is not true for most numbers! For instance, squaring the set \( \{1, 3, 5, 7\}  \)
		gives \( \{1, 8, 25, 49\}  \), which still contains the same number of elements as the original set!
\end{itemize}
\subsection{Fast Polynomial Multiplication Algorithm (Preview)}
\begin{itemize}
	\item Recall the evaluation step in our polynomial multiplication: \( p \cdot q \) is degree \( 2n - 2 \), so we need 
		\( m \ge 2n - 1 \). Let \( m \) be the first power of 2 such that \( m \ge  2n - 1 \). Then, we will evaluate \( p \) 
		and \( q \) on the \( m \)-th roots of unity in time \( O(m \log m) = O(n \log n)\), 
		using the \textbf{Fast Fourier Transform}.  
\end{itemize}

	\section{Quantum Key Distribution}
\begin{itemize}
	\item Qubits used for communication are usually photons, which have a momentum \( \vec k \), and 
		an electric field \( E_x \) and \( E_y \) that propagates in the plane perpendicular to \( \vec k \). 
		The \( E_x \) vector is denoted as \( \ket*{v} \), and  \( E_y \) as \( \ket*{H} \), and this means that 
		the general electric field \( \overline E = \alpha \ket*{v} + \beta \ket*{H} \). 
	\item We can pass these photons thorugh polarizers, which only transmit light with specific oscillations.  
	\item So as a quantum circuit, it's written as:
		\begin{center}
			\begin{quantikz}
				\lstick{\( \ket*{\psi} \) } & \gate[2]{pol} & & \rstick{\( |x| \)}\\
											& & &  \rstick{\( |y| \) }
			\end{quantikz}
		\end{center}
		Only one of these channels can be measured,  
\end{itemize}
\subsection{Distributed Entanglement}
\begin{itemize}
	\item One of the ways to do quantum communication and computation
	\item It includes: 
		\begin{itemize}
			\item teleportation
			\item Secure QKD: communication
			\item Distributed quantum computation -- quantumgatsby teleportation
			\item "Blind quantum teleportation" 
		\end{itemize}
\end{itemize}
\subsubsection{QKD Secureness}
\begin{itemize}
	\item The way QKD works is a server \( \ket*{\psi} = \ket*{HV} + \ket*{VH} \), and the first qubit is sent to 
		Bob, and the second is sent to Alice: 
		\begin{center}
			\begin{tikzpicture}
				\node (A) at (1, 0) {\(  B \)};
				\node (B) at (-1, 0) {\( A \) }; 
				\draw (-0.5, 2) rectangle node{\( S \) } (0.5, 3);
				\draw[-stealth] (0, 2) -- (A); 
				\draw[-stealth] (0, 2) -- (B);
				\draw node at (1, -0.5) {Qubit 2};
				\draw node at (-1, -0.5) {Qubit 1};
			\end{tikzpicture}
		\end{center}
	\item Classically, if an observer were to say, measure the second qubit, then send an identical copy through 
		that channel, then Alice and Bob won't be able to tell at all that the state has been measured. 

		However, if the system was quantum, this measurement is now impossible. 
	\item The proof of this is called the No cloning theorem, whose proof is below: 

		\begin{proof}
			Suppose we have an unknown state   \( \ket*{\phi} = \alpha \ket*{0} + \beta \ket*{1} \).  Now suppose 
			there is a \( U_{cl} \) (a "cloning matrix") which can clone \( \ket*{\phi} \). That is:
			\[
			\ket*{\phi}\ket*{0} \mapsto \ket*{\phi}\ket*{\phi} = \alpha^2 \ket*{00} 
			+ \beta \alpha \ket*{10} + \alpha \beta \ket*{01} + \beta^2 \ket*{11}
			\] 
			But if we do this on the initial state \( \ket*{\phi} \) : 
			\[
				(\alpha \ket*{0} + \beta \ket*{1}) \ket*{0} \mapsto \alpha \ket*{00} +\beta \ket*{11}
			\] 
			\comment{We are cloning this exactly based on what we want: we clone the information of the second 
				qubit onto the first qubit, but we see that even if we could "copy", we don't get the desired 
			product state.} 

			But this is not equal to the copied state that we should expect. Therefore, no such \( U_{cl} \) 
			can exist.
		\end{proof}
\end{itemize}
\subsection{Quantum Algorithms}
\begin{itemize}
	\item The Deutsch-Josza is a \textit{promise problem}: we are given a function \( f(x) \), and 
		it's one of two types: 
		\begin{itemize}
			\item \( f(x) \) is either constant for all \( x \) : it is either always 0 or always 1.
			\item \( f(x) \) is balanced: it is 0 half the time, \( f(x)  \) is 1 half the time. 
		\end{itemize}

		More generally, we can write \( f: \{0, 1\} ^{n} \mapsto \{0, 1\}  \), and we ask whether \( f \) is 
		constant or balanced. 
	\item For a function on \( n \) bits, this implies that the total domain space is of size \( 2^{n} \). We need to 
		measure a little more than half, or \( 2^{n} / 2 + 1 = 2^{n - 1} + 1 \) measurements in order to determine 
		the identity of \( f \).

		Quantumly, we only need a single measurement!
	\item The quantum circuit is as follows:
		\begin{center}
			\begin{quantikz}[slice all, slice titles = $\ket{\psi_\col}$ ]
				\lstick{\( \ket*{0}^{\otimes n} \)} \slice{\( \ket*{\psi_0} \) } & \gate{H^{\otimes n}} & \gate[2]{ U_f } & \gate{H^{\otimes n}} & \\
				\lstick{\( \ket*{1} \) } & \gate{H} & & & 
			\end{quantikz}
		\end{center}
		Initially, the state is in \( \ket*{\psi_0} = \ket*{0}^{\otimes n} \ket*{1} \). Then, after passing through 
		both Hadamard gates, we have: 
		\[
		\ket*{\psi_1} = \frac{1}{\sqrt{2}}\sum_x \ket*{x} \otimes \frac{1}{\sqrt{2} }(\ket*{0} - \ket*{1})
		\] 
		To explain what's happening here, here's a convenient way to denote \( H \) : 
		\[
		H = \frac{1}{\sqrt{2} }\sum_{x, y \in \{0, 1\} } (-1)^{xy}\ket*{y}\bra*{x}
		\] 
		(check for yourself that this does indeed generate the correct Hadamard matrix). Therefore, the general 
		\( n \)-qubit Hadamard gate \( H^{\otimes n} \) :
		\[
		H^{\otimes n} = \frac{1}{\sqrt{2^{n}} }\sum_{x, y \in \{0, 1\} ^{n}}(-1)^{x \cdot y}\ket*{y}\bra*{x}
		\] 
		Therefore, we can write:
		\[
			\ket*{\psi_1} = H^{\otimes n}\ket*{0^{\otimes n}} \otimes \frac{1}{\sqrt{2} }(\ket*{0} -\ket*{1})
			= \frac{1}{\sqrt{2^{n}} }\sum_x \ket*{x} \otimes \frac{1}{\sqrt{2} }(\ket*{0} - \ket*{1})
		\] 
	\item Now we send \( \ket*{\psi_1} \) through \( U_f \). What it does is it sends \( \ket*{y} \) 
		to \( \ket*{y \oplus f(x)} \). If \( y = 0 \), then we just output \( f(x) \), and if \( y = 1 \), thne 
		we output the \textit{complement} of \( f(x) \), since if \( f(x) = 1 \) then the addition modulo 2 
		would return us 0, and vice versa. Therefore, we can write \( \ket*{\psi_2} \) as: 
		\[
			\ket*{\psi_2} = \frac{1}{\sqrt{2^{n}} }\sum_x \ket*{x }\otimes \frac{1}{\sqrt{2} }
			(\ket*{f(x)} - \ket*{\overline{f(x)}})
		\] 
		If \( f(x) = 0 \), then the ancilla (the last qubit) is \( \ket*{0} - \ket*{1} \), and if \( f(x) = 1 \), 
		then the ancilla is  \( \ket*{1} - \ket*{0} = -(\ket*{0} - \ket*{1}) \). So in general, the ancilla is 
		\( (-1)^{f(x)}(\ket*{0} - \ket*{1})  \). Therefore, we can write:
		\[
		\ket*{\psi_2} = \frac{1}{\sqrt{2^{n}} } \sum_x (-1)^{f(x)}\ket*{x} \otimes \frac{1}{\sqrt{2} }(\ket*{0} - \ket*{1})
		\] 
		\comment{We can write \( (-1)^{f(x)} \) because when \( f(x) = 0 \) then \( (-1)^{f(x)} = 1 \), which doesn't
		change the product at all, but it changes when \( f(x) = 1 \), which is what we want.}
	\item Finally, we act \( H^{\otimes n} \) on the data register. Note that \( H^{\otimes n}\ket*{x} = 
		\sum_y (-1)^{xy}\ket*{y}\), so this gives us:
		\[
		\ket*{\psi_3} = \frac{1}{\sqrt{2^{n}} }\sum_y\sum_x (-1)^{f(x) + (x \cdot y)}\ket*{y} \otimes 
		\frac{1}{\sqrt{2} }(\ket*{0} - \ket*{1})
		\] 
	\item Now we measure all \( n \) qubits. If \( f(x) \) is constant, 
\end{itemize}

	\section{Lecture 6}
\subsection{Clarification on BIBO Stability}
\begin{itemize}
	\item When we say a "bounded" signal, we mean that the amplitude of the signal is bounded at all times:
		\[
		|x(t)| < \infty \ \forall t \in \R
		\] 
		The same definition follows for discrete-time signals. 
	\item For LTI systems, we call the system BIBO stable if and only if its impulse \( h(t) \) is absolutely 
		integrable:
		\[
		\int_{-\infty}^{\infty} |h(t)| \diff t < \infty
		\] 
\end{itemize}
\subsection{Cross-Correlation}
\begin{itemize}
	\item The cross correlation between two signals \( r_{xy}(t) = r_{yx}(-t) \). To show this explicitly, we 
		look at the cross-correlation equation:
		\begin{align*}
			r_{xy}(t) &= \int_{-\infty}^{\infty} x(\tau)y(t + \tau) \diff \tau \\
			r_{yx}(t) &= \int_{-\infty}^{\infty} y(\tau) x(t + \tau) \diff \tau  
	\end{align*} 
	But for the second equation, we can define a \( \tau' = t + \tau\), so then we get: 
	\[
	r_{yx}(t) = \int_{-\infty}^{\infty} y(\tau' - t)x(\tau') \diff \tau' = \int_{-\infty}^{\infty} 
	x(\tau') y(-t + \tau') \diff \tau'
	\] 
	This looks like the first equation except we have \( -t \) instead of \( t \). Therefore, we have 
	\( r_{xy}(t) = r_{yx}(-t) \). The same works for discrete time: \( r_{xy}[n] = r_{yx}[-n] \). 
\end{itemize}

\subsection{More Convolution Properties} 
\begin{itemize}
	\item \textbf{Differentiation property:} Given \( y(t) = x(t) * h(t) \), then:
		\[
			\dv{t} y(t) = x(t) * \dv{h(t)}{t} = \dv{x(t)}{t} * h(t)
		\] 
	\item \textbf{Intergration Property:} Given \( y(t) = x(t) * h(t) \), we have:
		\[
		\int_{-\infty}^{t'} y(t) \diff  t = x(t) * \int_{-\infty}^{t'} h(\tau)\diff \tau 
		\] 
\end{itemize}
\subsection{Fourier Transform} 
\begin{itemize}
	\item The fourier transform came from the study of the heat equation, written as:
		\[
			c \rho \pdv{t} u(x, y, z, t) = k \left( \pdv[2]{x} + \pdv[2]{y} + \pdv[2]{z} \right) 
			u(x, y, z, t)
		\] 
		Fourier then claimed that the solution can be expanded in a series of sines with multiples of the variable. 
		In other word,s the solution is of the form:
		\[
		f(x) = \frac{1}{2a_0} + (a_1 \sin(x) + b_2 \cos(x)) + (a_2\sin(2x) + b_2\cos(2x)) + \cdots 
		\] 
	\item Recall the frequency response of an LTI system:
		\begin{center}
			\begin{tikzpicture}
				\node (A) at (-3, 0) {\( x(t) \) };
				\node (B) at (4, 0) {\( y(t) \) };
				\draw[-stealth] (A) -- (0, 0);
				\draw (0, -0.5) rectangle node {\( h(t) \) } (1, 0.5);
				\draw[-stealth] (1, 0) -- (B);
			\end{tikzpicture}
		\end{center}
		Recall that we can characterize \( y(t) \) via a convolution: 
		\[
		y(t) = \int_{-\infty}^{\infty} x(\tau) h(t - \tau) \diff  \tau 
		\] 
		If we do this with our input \( e^{j 2 \pi ft} \), then we get:
		\[
		y(t) = H(f) e^{j 2 \pi ft} = H(\omega) e^{j \omega t }
		\] 
		Here, \( H(\omega) \) is defined to be the Fourier transfrm of the impusle response \( h(t) \):
		\[
			H(\omega) = \int_{-\infty}^{\infty} e^{- j \omega t} h(t) \diff t 
		\] 
		Alternatively, written in frequency language:
		\[
		H(f) = \int_{-\infty}^{\infty} e^{-j 2 \pi ft}h(t) \diff t 
		\] 
	\item Formally, the Fourier transform is defined as:
		\[
		H(f) \equiv \mathcal F \{h(t)\} \equiv \int_{-\infty}^{\infty} h(t) e^{-j 2\pi ft }\diff  t 
		\] 
		This transforms the signal \( h(t)  \) from the time domain into the frequency domain. The reason for this 
		is becuase the Fourier transform is a definite integral, which kills off any \( t \) dependence entirely. 
		In terms of angular frequency, we have:
		\[
		H(\omega) \equiv \mathcal F \{h(t)\} = \int_{-\infty}^{\infty} h(t) e^{-j \omega t}\diff t 
		\] 
	\item The inverse Fourier transform is:
		\[
			h(t) = \mathcal F^{-1} \{H(f)\} = \int_{-\infty}^{\infty} H(f) e^{j 2\pi ft}\diff  f 
		\]
		Since the Fourier transform takes objects from the time domain to the frequency domain, the inverse 
		Fourier transform takes things from the frequency domain to the time domain. 

		In terms of angular frequency, we have:
		\[
		h(t) = \mathcal F^{-1} \{H(\omega)\} = \frac{1}{2\pi}\int_{-\infty}^{\infty} H(\omega) 
		e^{j \omega t}\diff \omega 
		\] 
		This is also sometimes called the "synthesis equation", since we basically create \(  x(t) \) out of 
		\( H(\omega) \). 
	\item We can also provably show that the Inverse fourier transform does indeed invert the Fourier transform, 
		albeit with a lot of algebra. See lecture slides for the full derivation.  
	\end{itemize}

	\section{Lecture 7}
\subsection{DTFT and Convergence}
\begin{itemize}
	\item Not all functions have a Fourier transform, and the problem of whether a function has a Fourier 
		integral is an incredibly complex problem with no simple statement. 
	\item However, we know that there are several sufficient (but not necessary) conditions. Firstly, 
		we know that \( x(t) \) must be absolutely integrable. That is, 
		\[
		\int_{-\infty}^{\infty} |x(t)| \diff t < \infty
		\] 
\end{itemize}
\subsection{Fourier Transform Pairs}
\begin{itemize}
	\item There are several pairs of Fourier transforms that are useful to memorize.
	\item The Delta function:
		\[
			x(t) = \delta(t - t_0) \leftrightarrow X(f) = e^{-j 2\pi ft_0}
		\]
		This actually has strong implications about the nature of the Fourier transform -- there is an 
		"uncertainty principle" that manifests itself here. A signal cannot be both localized in time and frequency at
		the same time.
	\item Complex exponentials:
		\[
		x(t) = e^{j \omega_0 t} = e^{j 2 \pi f_0 t} \leftrightarrow X(f) = \delta(f - f_0)
		\] 
		This is the same as the previous point, except now we're going backwards.   
	\item Cosine functions:
		\[
		x(t) = \cos(2 \pi f_0 t) \leftrightarrow X(f) = \frac{1}{2}\delta(f - f_0) + \delta(f + f_0))
		\] 
		This makes sense: a plane wave is a composition of a left and right travelling wave.  
	\item Sine functions:
		\[
		x(t) = \sin(2 \pi f_0 t) \leftrightarrow X(f) = \frac{1}{2j}(\delta(f - f_0) - \delta(f + f_0))
		\] 
		Note that the only difference here is the minus sign, as a result of the conversion of sine into 
		complex exponentials.
	\item Shah function:
		\[
		x(t) = III(t) \leftrightarrow X(f) = III(f) \text{ or } X(\omega) = \frac{1}{2\pi}III(\omega)
		\] 
	\item Rect function:
		\[
		x(t) = \sqcap(t) \leftrightarrow \sinc(f)
		\] 
\end{itemize}

	\section{Paths in Graphs}

	\subsection{Single Source Shortest Path (SSSP)}
	\begin{itemize}
		\item We want to compute the distances from a source $s \in V$ to other nodes in $V$. 
		\item We don't use DFS here because DFS might explore much longer paths first, so it might be very 
			inefficient.
		\item Solution: use \textbf{Breadth-First Search (BFS)}
			\begin{itemize}
				\item Analogous to a bird's eye perspective, where we explore successively outward in 
					``neighbourhoods.''
				\item Start at exploring from distance 1, then when everything at distance 1 is explored, 
					continue to explore at distance 2, etc. 
			\end{itemize}
		\item The type of BFS that we use depends a lot on what kind of graph we're dealing with:
			\begin{itemize}
				\item Unweighted graphs: Ordinary BFS works
				\item Positive Weights: Dijkstra's algorithm
				\item Negative Weights allowed: Bellman-Ford Algorithm
			\end{itemize}
		\item Going down this list makes the graph more general, but they are less efficient than the ones 
			above. 

			\question{Would a more correct statement be that BFS works if all the edges have the same weight?}
	\end{itemize}

	\subsection{Breadth First Search (BFS)}
	\begin{itemize}
		\item Start at $s$, and add all the neighbours of $s$ to a queue. For every vertex in 
			the queue, we visit all the unvisited nodes from that vertex, and add it to the queue. Repeat
			until all nodes have been visited.
	\end{itemize}

	\subsubsection{Runtime of BFS}
	\begin{itemize}
		\item 	We enqueue and deque every node exactly once if the node is connected, otherwise we don't do it 
			at all. This takes $O(1)$ time. 
		\item Once an item is dequeued, we need to check all the neighbours of a graph, costing $O(\deg(u))$ 
			time.
		\item In total, our runtime is:
			\[
				\sum_{u \in V} O(1 + \deg(u)) = O(n + m)
			\] 
			This is the same runtime as DFS, which is not a coincidence! DFS and BFS are actually related, 
			except the queue is replaced by a stack.
		\item We didn't implement it as a stack in lecture, but the idea is the same.
	\end{itemize}

	\subsection{Weighted Graphs}
	\begin{itemize}
		\item BFS doesn't work here because it ignores the weights of the graph. It is possible that a graph 
			ends up being shorter but goes through more nodes, a possibility that BFS doesn't catch.
		\item \textbf{Useful Fact:} Any sub path of a shortest path is also a shortest path. This is rather 
			obvious.
		\item So what we should think about is that to build the shortest path, we build the shortest path from 
			other, shotest paths but add in the shortest edge. This guarantees that our shortest path 
			remains the shortest.
	\end{itemize}

	\subsection{Dijkstra's Algorithm}
	\begin{itemize}
		\item Let $K$ denote the set of ``known'' nodes where the length of shortest path is computed. To 
			determine node we should add to $K$, we should select the vertex that gives the smallest 
			$\dist(s, u) + \ell(u, v)$. Visually:
			\begin{center}
				\includegraphics[scale=0.5]{dijkstra.png}
			\end{center}
		\item The red region is the set of nodes that we look at.
		\item We don't need to recompute all distances at every iteration - instead we can just store 
			the distances as we go along. Initial overestimates are fine, since eventually we will explore 
			the shortest path, and its distance will eventually be updated. 
		\item If we find a shorter path later on, we can update $\dist(s, u)$ to reflect that.
		\item If we want to find the shortest path from $S$, then we can add a new variable that stores 
			the previous node in the sequence from $S$ to $u$. Therefore, when we want to find the 
			shortest path, then we are continually looking backward until we get back to $S$. 
	\end{itemize}

	\subsubsection{Runtime of Dijkstra's}
	\begin{itemize}
		\item The runtime of Dijkstra's depends on the kind of data structure we used to keep track 
			of the distances:
			\begin{center}
				\begin{tabular}{c|c|c|c|c}
					\textbf{Implementation} & \textbf{Insert} & \textbf{Delete Min} & 
					\textbf{Decrease Key} & \textbf{Runtime}\\
					\hline 
					Array & $O(1)$ & $O(n)$ & $O(1)$ & $O(n^2 + m) = O(n^2)$\\
					Binary Heap & $O(\log n)$ & $O(\log n)$ & $O(\log n)$ & $O((n + m) \log n)$\\
					Fibonacci Heap & $O(1)$ & $O(\log n)$ & $O(1)$ & $O(n \log n + m)$
				\end{tabular}
			\end{center}
		\item The best known runtime of Dijkstra's algorithm is $O(n \log \log n + m)$.
		\item At the end of the day, this is slower than DFS, by the $\log n$ term.
	\end{itemize}
	
	\subsection{Negative Weights: Bellman-Ford Algorithm}
	\begin{itemize}
		\item Sometimes, having negative weights is possible, for instance when traversing an edge is more 
			beneficial to you in some way.
		\item Shortest paths don't really make sense if a cycle has negative length (since then we'd be 
			infinitely descending) 
		\item All we need to do is modify Dijkstra's update function!
			\begin{itemize}
				\item Call an update ``safe'' if $\dist(w)$ is an overestimate of the true shortest path 
					between $s$ and $w$. In other words, $\dist(w) \ge d(s, w)$ for all $w \in V$. 
			\end{itemize}
	\end{itemize}

	\question{see lectures for Bellman-Ford}

	\section{More on Fourier Transforms}
\subsection{Discrete Time Fourier Transforms (DTFT)}
\begin{itemize}
	\item Recall that for continuous time signals, the fourier transform is written as 
		\[
		X(\omega) = \int_{-\infty}^{\infty} x(t) e^{-j \omega t}\diff t 
		\] 
	\item For discrete time signals, the way to do this is to write the signal as a continuous time signal 
		via delta functions, and then apply CTFT:
		 \[
			 x(t) = \sum_{n=-\infty}^{\infty} x[n] \delta(t - n)
		\] 
		Then, we can write:
		\[
			X(\omega) = \sum_{n=-\infty}^{\infty} x[n] \int_{-\infty}^{\infty} \delta(t - n) e^{-j \omega t}\diff t
			= \sum_{n=-\infty}^{\infty} x[n] e^{- j \omega n }
		\] 
		So this actually just means that in discrete time, the Fourier transform of \( x[n] \) is just 
		the amplitude of the signal at that particular \( n \), multiplied by a sinusoid of a corresponding 
		frequency specified by \( n \). 
	\item In general, we have:
		\begin{align*}
			X(e^{j 2 \pi f}) &= \sum_{n=-\infty}^{\infty} x[n] e^{-j 2 \pi f n} & x[n] &=
			\int_{-\infty}^{\infty} X(e^{j 2 \pi f})e^{j 2 \pi f n}\diff f \\
			X(e^{j \omega}) &= \sum_{n=-\infty}^{\infty} x[n] e^{-j \omega n} & x[n] &= \frac{1}{2\pi}
			\int_{-\infty}^{\infty} X(e^{j \omega}) e^{j \omega n}\diff  \omega 
		\end{align*}
		Sometimes textbooks use \( \Omega \) insetad of \( \omega \), but we will use the latter.   
	\item See lectures for worked examples on how to do this.
\end{itemize}

\subsection{Characteristics of DTFT}
\begin{itemize}
	\item DTFT is generally a continuous function over \( f \) or \( \omega \), even though we know that 
		\( x[n] \) is a discrete time signal. This is due to the presence of the delta functions. 
	\item Further, DTFT is periodic with a period of 1 in frequency or  \( 2\pi \) in angular 
		frequency:
		\[
			X(e^{j 2\pi f}) = \sum_{n=-\infty}^{\infty} x[n] e^{-j 2\pi f n} = \sum_{n=-\infty}^{\infty} x[n] 
			e^{-j2 \pi f (n + 1)}
		\] 
		This is because  \( e^{-j 2 \pi f(n + 1)} = e^{-j 2 \pi f n}e^{-j 2 \pi n} \) and the latter term is 1. 
\end{itemize}
\subsection{Common DTFTs}
\begin{itemize}
	\item For a delta function \( x[n] = \delta[n] \), its Fourier transform \( X(e^{j \omega}) = 1 \). 
	\item For a constant function \( x[n] = 1 \), its Fourier transform is the Shah function: 
		\[
		X(e^{j \omega}) = 2\pi \sum_{k=-\infty}^{\infty} \delta(\omega - 2 \pi k) 
		\] 
	\item For complex exponentials \( x[n] = e^{j \omega_0 n} \), its Fourier transform is
		\[
		X(e^{j \omega}) = \sum_{k=-\infty}^{\infty} \delta(f - f_0 - k)
		\] 
		So this is basically this is the comb function, but shifted over by some constant 
		amount \( \omega_0 \). This is incredibly useful for applications such as signal modulation and other 
		applications, where we talk about a "linear phase" addition.  

		For instance, consider sending a signal to a receiver that only accepts a specific frequency. Then, in order 
		for a source to be able to send an appropriate signal, we can "modulate" the signal by a constant 
		factor \( \omega_0 \) instead of completely modifying our signal. 
	\item For sinusoids, recall the identities: 
		\[
			\cos[\omega_0 n]= \frac{1}{2}(e^{ j \omega_0 n} + e^{- j \omega_0 n})
		\] 
		and sicne we've expressed it in terms of exponentials, we can use the earlier bullet point to find DTFT 
		here. This gives us: 
		\[
		X(\omega) = \pi \sum_{k = -\infty}^{\infty}\delta(\omega - \omega_0 - 2 \pi k) 
		+ \pi \sum_{k= -\infty}^{\infty} \delta(\omega + \omega_0 - 2 \pi k)
		\] 
		For the sine function, we have
		\[
			\sin[\omega_0 n] = \frac{1}{2j}(e^{j \omega_0 n } - e^{- j \omega_0 n})
		\] 
		and we can do the same trick.
	\item For the rectangular function \( x[n] = \sqcap[n] \), its Fourier transform is bascially a restricted 
		comb function: 
		\[
		X(e^{j \omega}) = e^{ j \omega} + 1 + e^{j \omega} = 2 \cos(\omega) + 1
		\] 
		Note that here we're using the standard rectangular function that is only nonzero over \( n \in [-1, 0, 1] \).
		For the general rectangular function: \( x[n] = \sqcap\left[ \frac{n}{N} \right]  \), then 
		we have:
		\[
		X(e^{j \omega}) = \sum_{n=-N}^{N} e^{-j \omega n}
		\] 
		So it is a restricted sum over \( -N \) to \( N \). But this is just a geometric series, so this will give 
		us the formula:
		\[
		X(e^{j \omega}) = e^{j \omega N}\frac{1 - e^{-j \omega (2N + 1)}}{1 - e^{-j \omega}} = 
		\frac{e^{j \omega N} - e^{-j \omega (N + 1)}}{1 - e^{-j \omega}} =
		\frac{\sin(\omega(N + 1 /2))}{\sin(\omega / 2)}
		\] 
		This is very similar to a sinc function. 
\end{itemize}

	\section{Quanutm Gates}
\begin{itemize}
	\item The discussion here is the preamble to decoherence and why quantum computers 
		"don't work" (yet).
	\item A classical NOT gate can be implemented using a transistor and some resistors. Such a circuit cannot be 
		built quantumly, and we store information in qubits. 
	\item \textbf{Qubit:} There's two different definitions that we use:
		\begin{itemize}
			\item A two-dimensional hilbert space -- this is the theoretically ideal picture of what a qubit 
				is. 
			\item A physical objects that acts a lot like an ideal object -- this is what experimentalists deal with.
		\end{itemize}
		What we will try to investigate is the actual physics of quantum computers.  
	\item Now let's start talking about the former: let \( \mathcal H \) be a Hilbert space, spanned by 
		the set \( \{\ket*{0}, \ket*{1}\} \), which is also written as the set 
		\( \left \{\begin{pmatrix} 1\\0 \end{pmatrix}, 
		\begin{pmatrix} 0\\1 \end{pmatrix} \right\}  \). 
	\item If we were to fully characterize the quantum system, then the quantity we would like to know the most is the 
		Hamiltonian \( H \). In a 2-level system, then Hamiltonian is written as: 
		\[
		H = E_a \ket*{a}\bra*{a} + E_B \ket*{b}\bra*{b}
		\] 
		Further, the Hamiltonian is an example of an \textit{observable}. Observables have the property that 
		their eigenvalues are real, which implies that they must be Hermitian operators. 
\end{itemize}
\subsection{Time Evolution of Quantum Systems}
\begin{itemize}
	\item The time evolution of quantum systems is given by the Schr\"odinger equation, writtne as: 
		\[
			i \hbar \pdv{t} \ket*{\psi(t)} = H(t) \ket*{\psi(t)}
		\] 
		In introductory quantum mechanics, we're usually interested in the stationary states, where we're 
		given the Hamiltonian and asked to solve for the Eigenstates. Here, we'll be interested more in the 
		dynamics of the system. So, this means that we're more interested in the PDE in this equation more than 
		\( \ket*{\psi} \) itself. 
	\item Let's look at a simpler example first, the ODE counterpart to this equation (in a sense): 
		\[
			\dv{y}{t} = hy \implies y = Ce^{ht}
		\] 
		Why is this the case? Well, because for every small time step, we're saying that the amount that we increase 
		by is equal to the previous value. This implies that we are growing exponentially over time, which 
		is why we have the exponential term. Specifically, we can then write the increase as
		\[
		\lim_{N \to \infty} \left(1 + h \frac{t}{N}\right)^{N}
		\] 
		This trick also works with matrices! So, by analogy, the solution to \( \ket*{\psi(t)} \) can be written as: 
		\[
			\ket*{\psi(t)} = e^{-i H t / \hbar}\ket*{\psi(0)} = \mathcal U(t) \ket*{\psi(0)}
		\] 
		Here, the exponentiation can be resolved by taking a Taylor expansion. 
	\item We can also take functions of matrices: consider \( f(H) = f(\mathcal U \Lambda \mathcal U^\dagger) = 
		\mathcal U f(\Lambda) U^{\dagger}\). This can be proven easily by taking a Taylor expansion of \( f \).
		This is also easy to calculate, since if \( \Lambda \) is a diagonal matrix, then we have the 
		identity: 
		\[
			f(\Lambda) = \begin{pmatrix} f(a) & &&\\
			& f(b) & & \\ & & \ddots & \end{pmatrix} 
		\] 
		basically, we apply \( f \) to all the elements on the diagonal.
	\item Now, let's work with an example: consider a Hamiltonian 
		\( H = -\frac{1}{2}g \mu_B \vec B \cdot \vec \sigma \), where \( \vec B \) is the magnetic field, 
		and \( \vec \sigma \) is the vector characterized by the Pauli spin matrices. We can expand this out 
		as: 
		\[
		H = -g \mu_B (B_x \sigma_x + B_y \sigma_y + B_z \sigma_z) 
		\] 
		If our magnetic field is only in the \( x \) direction (for simplicity), then we can write
		\[
		H = -g \mu_B B_x \sigma_x
		\] 
	\item We want to find time evolution, and as discussed earlier, the unitary matrix that describes the time evolution
		is given by \( e^{-i H t} \), and given \( H \), therefore we have: 
		\[
			\mathcal U(t) = \exp{i \alpha B_x \sigma_x t}		
		\] 
		Now, we want to write this in a nicer form, so first we leverage the fact that \( \sigma_x = 
		H \sigma_z H\), and since \( H \) is Hermitian, then this satisfies the relation 
		\( f(\mathcal U \sigma_z \mathcal U) = \mathcal U f(\sigma_z) \mathcal U \), so we have: 
		\[
			\mathcal U(t) = H \exp{i \alpha B_x t \begin{pmatrix} 1 & 0\\0&-1 \end{pmatrix}} H
		\] 
		Applying the exponent to the diagonal matrix, we eventually get: 
		\[
			\mathcal U(t) = \begin{pmatrix} \cos(\alpha B_x t) & i \sin (\alpha B_x t)\\
			i \sin (\alpha B_x t) & \cos(\alpha B_x t)\end{pmatrix} 
		\] 
	\item This result allows us to conclude a much more general fact. Recall De Moivre's formula
		\( e^{ i \phi} = \cos(\phi) + i \sin(\phi) \), and more generally with matrices: 
		\[
			e^{i \phi \hat{n} \cdot \vec \sigma} = \cos(\phi) \mathbbm{1} + i \sin(\phi) \hat{n} \cdot \sigma
		\] 
	\item Now that we've found \( \mathcal U(t) \), let's see what happens when we act it in \( \ket*{\psi(0)} \) :
		\begin{align*}
			\ket*{\psi(t)} &= \mathcal U(t) \ket*{\psi(0)}\\
			&= \begin{pmatrix} \cos(\omega t) & i \sin (\omega t) \\ i \sin(\omega t) & \cos(\omega t) \end{pmatrix}  \\
			&= \begin{pmatrix} \cos(\omega t)\\ i \sin(\omega t) \end{pmatrix}  
		\end{align*}
		If we measure our state, the probabiilty that we get the state \( \ket*{0} \) is given by 
		\( \cos^2(\omega t) \), which is also known as \textit{Rabi Oscillation}. 
	\item The Hamiltonian we wrote down earlier is nice, but it's impractical because it's \textit{too} ideal. What 
		we'll look at next is how to deal with small imperfections in our Hamiltonian.
\end{itemize}
\subsection{Energy Splitting}
\begin{itemize}
	\item Now, consider a Hamiltonian \( H_0 = \frac{1}{2}\omega_0 \sigma_z = 
		\begin{pmatrix} \omega / 2 & \\ & -\omega / 2 \end{pmatrix} \). Under time evolution, the 
		basis states evolve as: 
		\begin{align*}
			\begin{pmatrix} 1\\0 \end{pmatrix} &\to \begin{pmatrix} e^{-i \omega t / 2} \\0 \end{pmatrix} \\
			\begin{pmatrix} 0 \\1 \end{pmatrix} &\to \begin{pmatrix} 0 \\ e^{i \omega t/2} \end{pmatrix} 
		\end{align*}
		So, we can write out the time evolution operator 
		\[
			\mathcal U(t) = \begin{pmatrix} e^{-i \omega t /2} & \\ & e^{i \omega t / 2} \end{pmatrix} 
		\] 
		What \( H \) does to the basis states are easily computable, but they also happen to be the most boring state
		evolutions. What's more interesting is what happens to, say, the \( \ket*{+} \) state: 
		\[
			\ket*{+} = \frac{1}{\sqrt{2} }\begin{pmatrix} 1\\1 \end{pmatrix} \to 
			\frac{1}{\sqrt{2} }\begin{pmatrix} e^{-i \omega t /2}\\e^{i \omega t /2} \end{pmatrix} 
			= \begin{pmatrix} \cos(\omega t /2) - i \sin (\omega t /2)\\
			\cos(\omega t /2) + i \sin(\omega t /2)\end{pmatrix} = \cos(\omega t /2) \ket*{+} 
			- i \sin(\omega t / 2) \ket*{-}
		\] 
		So if we measure this state in the \( \ket*{+}, \ket*{-} \) basis, then we find that there are moments 
		where the state is entirely in the \( \ket*{ + }\) state, and also moments where we're 
		entirely in the \( \ket*{-} \) state. Here, the frequency \( \omega \) is also referred
		to as the \textbf{Larmor frequency}. 
\end{itemize}

	\section{Recurrent NNs, Attention \& Transformers}
\subsection{Sequence-to-Sequence Models}
\begin{itemize}
	\item To motivate the later topics in this lecture, one question we should answer
		is how we deal with variable length inputs. In the case of protein folding,
		you have different length sequences for different proteins, so how do you get
		your neural net to predict parameters like the stability, or structure of the
		protein?

		In the most interesting case, your output is contingent on the length of your
		input. This is the case in language translation. 
	\item In general, you need a way to handle variable length inputs, and models
		that do this are called \textbf{sequence-to-sequence models}.
	\item Let's first consider only variable length inputs, so we have something
		like:
		\begin{align*}
			x_1 &= (x_{1, 1}, x_{1, 2}, x_{1, 3}, x_{1, 4})\\
			x_2 &= (x_{2, 1}, x_{2, 2}, x_{2, 3}) \\ 
			x_3 &= (x_{3, 1}, x_{3, 2}, x_{3, 3}, x_{3, 4}, x_{3, 5}) 
		\end{align*}
		because you want to have variable length inputs, you can't just feed the
		neural network the entire \( x_1 \) because then you lose the ability to have
		variable length. 
	\item One possible way you could do this is to feed each dimension of the input
		vector sequentially. That is, for \( x_1 \), we first feed it in \( x_{1, 1}
		\), then \( x_{1, 2} \), then \( x_{1, 3} \), etc. As a diagram:
		\begin{center}
			\includegraphics[scale=0.7]{images/lec11-1.png}
		\end{center}
		so, the output \( a^{3} \) hinges on the output \( a^2 \), and so on.
		For inputs with shorter length, we can just prepend a bunch of zeros, which
		does nothing to the data.
	\item In principle, this would work, but the major problem with this direct
		approach is that the number of layers in your network is equal to the length
		of the largest input. If you have very few long inputs, then the model won't
		be very good at predicting the long inputs because there isn't enough data
		out there. 

		To fix this, we tie the layer parameters together, which is called a
		recurrent neural network. It's called recurrent because each \( a^{\ell} \)
		depends recursively on \( a^{\ell - 1} \). 
	\item Something else you could do is to have an output for each input, and
		"decode" the output from that specific layer. The computation is the same,
		the only thing we've changed is that we're decoding the information at every
		step. 
	\item You could also use what's called an \textbf{autoregressive model}, which
		uses the output of the previous iteration to generate the next output. This
		is what ChatGPT does. 

		\question{How is this model different than the first one?} 

		\answer{You are getting the output at every step here, in the first type
			you're not. In other words, you're getting the output of each next word,
		which is generated based on the prevoius word.} 
	\item From this process, hopefully it's clear that there's this general setup
		going on: you have an encoder that "encodes" the input data, and a decoder
		that you run through your neural network to decode the output. That is, the
		general structure could look something like this:
		\begin{center}
			\includegraphics[scale=0.6]{images/lec11-2.png}
		\end{center}
		Note that the RNN encoder could be replaced by a CNN encoder if we were
		dealing with images. Point is, you have a neural network that
		\textit{encodes} the input data, and a decoder that \textit{decodes} the
		data. 

		The encoder translates the input into a "content vector" that describes the
		content, then the RNN decoder does the actual translation.    
\end{itemize}
\subsection{RNN Bottleneck, Attention}
\begin{itemize}
	\item One major problem with this encoder-decoder structure is the transition
		between the encoder and the decoder, called the bottleneck. In essence, if
		you have very long recurrences, then the earlier features that are passed in
		tend to get washed out the longer your recurrence goes. 

		This is the motivation for solutions that involve attention and transformers.
	\item Suppose you don't want all the information to be bottlenecked by the last
		output of the encoder, is there a way we can just translate intermediate
		outputs? The answer is yes, via attention. 
	\item Imagine the standard recurrent neural network (image below), and 
		you're trying to decode \( s_2 \). 
		\begin{center}
			\includegraphics[scale=0.5]{images/lec11-3.png}
		\end{center}
		we can generate a query vector based on the hidden state at \( s_2 \), that
		allows us to query back in time and look for matches. This is a probabilistic
		lookup, so it picks the one that has the highest similarity, based on some
		heuristic (maybe a dot product or something).     

		\comment{The intuition to have is that if you're trying to decode the subject
			of the sentence, the query vector goes back in time and looks for the
		subject of the sentence from the encoder.}
	\item Refer to the diagram below:
		\begin{center}
			\includegraphics[scale=0.5]{images/lec11-4.png}
		\end{center}
		\begin{itemize}
			\item Suppose you're at \( s_2 \), and you're looking to query.
				To compute the keys, you take the hidden input \( h_t \) at 
				every layer, and you put it through a function \( k(h_t) \). \( k \) 
				could be as simple as the identity, but it can also be a simple 
				learned function as well (e.g. \( k_t = \sigma(W_k h_t + b_k) \)). 
			\item When you're query, you put it through a query function \( q_{l} =
				q(s_l) \). \( q_l \) is a vector representing some concept, and
				we take the inner product of \( q_l \) with every input \( k_t \).
				This quantity \( e_{t, l} \) is called the \textbf{attention score}.  
			\item Intuitively, we would like this to be a database lookup, which
				would basically be the same as looking for \( \argmax e_{t, l} \).
				But, because this is not differentiable, we normally use a softmax
				instead.  That is:
				\[
					\alpha_{t, l} = \frac{\exp(e_{t, l})}{\sum_{t'} \exp(e_{t', l})}
				\]
				Now, the combination of all of these is a normalized probability
				distribution, then we take a linear combination of the attention
				score with the information in the input sequence back. That is, we
				send \( a_l = \sum_t \alpha_{t, l} h_t \) back. Basically, this
				ensures that the things which are sent back with high probability are
				the ones that correspond highly with the data.  
			\item After sending \( a_l \) back, we can then use this information
				combined with \( s_2 \) to finally get a prediction \( \hat{y}_{2}
				\).   
			\item The number of attention scores you have depends on the length of
				the input string. The computational complexity is then dependent on
				the length of your input. 
		\end{itemize}
	\item Some general examples of \( k(t) \) and \( q(t) \) are just the identity
		function:
		\[
			k_t = h_t \quad q_l = s_l
		\]
		so the decoder is just the inner product between the encoder and decoder. You
		can also use more complicated functions, such as:
		\[
			k_t = W_k h_t \quad q_l = W_q s_l
		\]
		where \( W \) is a parameter we augment by (\question{is this learned?}). On
		the decoder side, you then have:
		\[
			e_{t, l} = h_{t}^{\top} W_k^{\top} W_q s_l = h_{t}^{\top}W_e s_l
		\]
	\item You can also do weird things with the returned \( a_l \) as well. Instead
		of just taking the dot product of the attention score with each layer, you
		can also use a learned function here, so you return
		\[
			a_l = \sum_t \alpha_t v(h_t)
		\]
		where \( v(h_t) \) is some learned function.  

		\question{We keep saying this "learned function", how do you learn functions
		like this? Do you have another neural net that does this?}
	\item Attention is a \textbf{very} powerful tool! There is no longer a bottleneck
		because each layer can just use attention. This is very similar to how
		resents work -- we're shortening the gradient path, so this makes the neural
		net better. 
\end{itemize}
\subsection{Transformers}
\begin{itemize}
	\item Is attention all we need? Do we even need the recurrence or the
		autoregressive model? The answer is no, and what we can actually do is just
		rely only on attention. This is what transformers are. 
	\item The only issue you have with the current model is that you can only look at
		the key vectors for the input pairs, but not the decoding layers. To solve
		this, we implement \textbf{self-attention}. So, we basically make a key
		vector for \textit{every layer}, including the ones in the decoding network.
	\item Refer to the following diagram:
		\begin{center}
			\includegraphics[scale=0.5]{images/lec11-5.png}
		\end{center}
		\begin{itemize}
			\item The intermediate arrows are now gone, and instead it's replaced by
				these \( k_i, q_i, v_i \) values. We still have a shared weight \(
				h_t = \sigma(W x_t + b) \), but they are no longer connected to each
				other. 
			\item Now, each layer has a value \( v_t \), which is defined by \(
				v(h_t) \), just as before. The key vector is also stored, in the same
				way as before \( k_t = k(h_t) \). The query is also the same as
				before. 
			\item So basically, this is a very dense version of what we had before --
				instead of having the keys in the encoder and the queries in the
				decoder, they are now all in the same system (so to speak). 
			\item If you're querying for \( h_1 \), then you query through \( k_1,
				k_2, k_3 \), and compute the attention in the same way we did before.
				The only difference is that you're querying your own key \( k_1 \) as
				well. 
			\item \( a_l \) is defined the same way, through the softmax, etc. 
		\end{itemize}
	\item Now that we've removed the recurrence, this means that the network is now
		permutation equivariant: if you switch up the order of \( h_i \) the output
		doesn't change, which is bad.  
	\item You can also stack attention layers on top of each other in the same way
		you stack a convolutional neural network. 
	\item Some issues with our model so far:
		\begin{itemize}
			\item Lack of sequence information: by removing the recurrence we've
				removed the ordering.  

				To fix this, we just force-feed position information into \( h(t) =
				f(x_t, t)\). If you just feed it in something simple (like 1, 2, 3,
				...), it actually
				doesn't do very well. What people actually do is give it a positional
				encoding \( p_t \) which is a giant vector of sines and cosines. 
			\item Multi-headed attention: in the same way you can have multiple
				kernels for your neural network, you can also have an attention layer
				for each kernel, which is called multi-headed attention.
				
				You can do multi-headed attention in parallel, the mechanisms are
				exactly the same, except in a higher dimension because we have more
				queries, keys, and values. 
			\item Linearity: each successive layer is linear in the previous one

				Fixing this just requires you to add a some nonlinear function in
				between the attention layers. 
			\item Masked decoding: How do we prevent attention lookups into the
				future?  

				This is mainly an issue caused by addressing the first issue -- when
				we bake an ordering into the system, then we have to restrict what
				each layer can query, which we can do by modifying the dot product to
				be:
				\[
					e_{l, t} = \begin{cases}
						q_l \cdot k_t & t \leq l\\
						-\infty & \text{otherwise}
					\end{cases}
				\]
				This solves the issue because now you'll get a \( -\infty \)
				attention score for the things you're not supposed to look at. 
				
				\comment{Note that throwing this into the softmax will give you \(
				e^{-\infty} = 0 \), so it does work out as expected.}
		\end{itemize}
	\item And now we've essentially built a transformer. Nowadays, transformers
		usually just refer to this stacking of self attention layers. 
	\item Some downsides is that this is pretty slow, as it runs in \( O(n^2) \),
		compared to \( O(n) \) (roughly) due to backpropagation. It's also slightly
		harder to implement. 

		That said, it also has many benefits, which outweigh the downsides: they have
		much better long-range connections, they're much easier to parallelise, and
		you can also make them much deeper than you can with RNNs.  
\end{itemize}

	\section{Dynamic Programming I}

	\subsection{Fibonacci Numbers, revisited}
	\begin{itemize}
		\item Imagine computing Fibonacci numbers; there's a lot of repeated calculations! For instance, 
			$F(1)$ is computed $2^n$ times when we're looking for $F(n)$!
		\item To optimize this, store each successive computation of $F(n)$ into an array that we access, 
			so that we only need to compute each $F(k)$ exactly once.
		\item This is called \textbf{memoization}, where we store things in a ``memo,'' to be accessed by our 
			algorithm later on. 
	\end{itemize}
	\subsection{Elements of Dynamic Programming}
	\begin{itemize}
		\item There are a couple hallmarks of DP:
			\begin{enumerate}
				\item Subproblems, or ``optimal substructure''. Refers to the fact that large problems 
					can be broken up into smaller subproblems. For Fibonacci, this means that 
					$F(n)$ is recursively expressed in terms of smaller subproblems.
				\item Overlapping subproblems: A lot of subproblems overlap with one another. We recurse to 
					smaller subproblems, and in doing so we see that a lot of computation 
					is repeated. The solution to this is to use memoization, so that 
					each computation is done only once.
		\item There are two ways to do DP:
			\begin{itemize}
				\item Top-Down: start from the largest subproblem and recurse to smaller subproblems. 
					This often involves recursion.    
				\item Bottom-up: start from the smallest subproblems then work to larger subproblems.
					Memoization still happens; we just fill the table from the small to 
					largest problems. In this method, this doesn't need a recursive call. 
			\end{itemize}
		\end{enumerate}
		\item The mathematical runtime of top-down and bottom-up are the same.
		\item The computation structure for DP actually looks awfully similar to a DAG. 
		\item If we view every subproblem as a node in the graph: construct it in such a way that 
			an edge $i \to j$ exists if the solution to subproblem $j$ directly depends on the solution to of 
			subproblem $i$. 
		\item Consider a topological sort on this DAG: then the bottom-up solution directly follows the 
			conputation of this DAG! 
			\begin{itemize}
				\item In the top-down framework, we are filling up the memo table in topological sort order, 
					since that table is still being filled from bottom up.
			\end{itemize}
	\end{itemize}
	\subsection{Shortest Paths on DAGs}
	\begin{itemize}
		\item We're given a DAG with a source $s$. We want to find the cost of the shortest path from 
			$s$ to $u$ for all $u \in V$. We also want to do this in linear time, 
			$O(n + m)$.
		\item We can always run a topological sort on this DAG in $O(n + m)$ time. Our subproblems 
			are the distances from $s$ to $u$ for every node $u$.
		\item After ordering in topological sort, we can just go down this graph \textit{in topological 
			order!} This means that the structure of the DP tree is the same as that of the topological sort.
		\item In terms of our recurrence relation, $\dist(u) = \dist(v) + \ell(u, v)$. Here, 
			$\dist(v)$ is implied to be memoized, since it's already a solved problem.
		\item This is an $O(n +m)$ solution to this problem!
	\end{itemize}
	\subsection{DP Recipe}
	\begin{enumerate}[label=\alph*)]
		\item Identify the subproblems (i.e. find the optimal substructure)
		\item Find a recursive formulation for the subproblems: just try to solve it via recursion and 
			see where it gets you.
		\item Design the DP algorithm -- fill in a table, starting with the smallest sub-problems and 
			building up. 
	\end{enumerate}
	\subsection{Shortest Paths with $k$}
	\begin{itemize}
		\item Here we consider the same problem of finding shortest path, but we're restricted to use at most 
			$k$ edges. 

			\question{Fill this out from lecture recordings}
	\end{itemize}
	\subsection{All-Pair Shortest Paths}
	\begin{itemize}
		\item Here, instead of finding the shortest path from a singular source node, we want to find 
			it for all pairs of nodes. 
		\item Input: again a graph with no negative cycles.
		\item Naively, we can run Bellman-Ford on all nodes, but this would take $O(nm)$ a total 
			of $n^2$ times, so our total runtime could be as large as $O(n^4)$ for dense graphs.
			Therefore, we're looking for a better algorithm.
		\item Identify the subproblem: subproblem $k$: for all pairs, find the shortest $u \to v$ path 
			whose internal vertices (so the path they take) only use nodes $\{1, 2, \dots, k\}$.
			\begin{itemize}
				\item In other words, there's a collection of $k$ nodes, and the path from $u \to v$ 
					\textit{only} uses these nodes.
			\end{itemize}
		\item Recursion: When we have the set from $\{1, \dots, k\}$, we want to find the relation between 
			this set and how to expand this set. There are a couple ways that the new 
			node can be added:
			\begin{itemize}
				\item Case 1: The new node added does not lie on the path: then nothing really changes, so 
					$\dist_{k +1}(u, v) = \dist_k(u, v)$.
				\item Case 2: The shortest path uses the added node: this path can be broken into two 
					parts: the shortest $u \to (k+1)$ path and then the shortest $(k+1) \to v$ path. 
					Both of these paths are already computed (by definition of them only using 
					the set $\{1, \dots, k\} $), so we just have to add these two up. 
				\item To combine these two, we find the minimum of these two to find whether the 
					path from $u$ to $v$ has changed or not.  
			\end{itemize}
		\item Runtime: Each update is $O(1)$ time, and we have to loop over $u, v$ a total of $k$ times, 
			so overall $O(n^3)$ runtime. 
		\item This is called the \textbf{Floyd-Washall Algorithm.}
	\end{itemize}

	\section{t-SNE}
\begin{itemize}
	\item So far in PCA, the assumption is that you have a dataset \( \{x_i\} \) that
		you can find the principal components of. What if you instead are given \(
		X^{\top}X \)? 
	\item As a practical example, what if you were just given the pairwise distances
		between cities \( M \)? The solution is that you can think of \( M = X
		X^{\top} \) of some unobserved data \( X \). This is slightly different from
		PCA, where we use \( X^{\top}X \). 

		\question{Why do we assume \( XX^{\top} \) instead of \( X^{\top} X \)?}
	\item We can still use SVD, except we use the eigenvectors in \( U \) instead of
		\( V \), which is what we used for PCA. 
	\item The issue with PCA is that it is a linear mapping (using SVD), but in
		general this is not true with all the data we get. So, we need something more
		complex. 

		In particular, we need an \textit{embedding} of what we define to be
		neighbors in our dimensionality reduction. The idea is to look locally and
		make sure that things are embedded locally only, and not really care about
		what happens globally. In the case of PCA, it treats neighbors globally,
		which is why it fails to dimension-reduce some datasets.   
\end{itemize}
\subsection{Neighborhood Embedding (NE)}
\begin{itemize}
	\item The idea is basically to define what the neighbors of a point are. You can
		do this by iterating through each point, and build an "edge" between them,
		then compute the distances using this new function you've created. 
	\item Then, once you get all the pairwise distances you can now create \( M \),
		from which you can do PCA.
	\item One issue with this though is that the quality of your reduction depends
		heavily on the nearest neighbor graph you create. So, instead of
		\textit{telling} you what the neighbors are, we can build a probability
		distribution instead.  
\end{itemize}
\subsection{Stochastic NE (SNE)}
\begin{itemize}
	\item Here, we make the event that two samples are neighbors be a random
		variable. More specifically, the probability that \( x_i \) "chooses" \( x_j
		\) is proportional to:
		\[
			P_{j \leftarrow i} = \frac{\exp\left( -\|x_i - x_j\|^2
			/2\sigma_i^2\right)}{\sum_{k \neq i} \exp(-\|x_i - x_k\|^2 /2
			\sigma_i^2)}
		\]
		and we also have \( P_{j \leftarrow j} = 0 \). 

		\question{Is this just softmax?}
	\item We then symmetrize this and renormalize it, so we make \( P =
		\frac{1}{2n}(P_{j \leftarrow i} + P_{i \leftarrow j}) \). 
	\item This value of \( \sigma \) tat you pick is that you choose it differently,
		so that the entropy of anything choosing \( x_i \) is constant. That is:
		\[
			\sum_{j \neq i}P_{j \leftarrow i} \log P_{j \leftarrow i} = \text{const.}
		\]
	\item We're now going to posit that after the dimensionality reduction, we get a
		space \( Y \), and the points live in \( Y \) somehow. We can also get the
		same neighborhoods in \( Y \) in the exact same way, and we can make these \(
		Q = \{Q_{ij}\}\):
		\[
			Q_{ij} = \frac{\exp\left( -\|y_i - y_j\|^2 \right)}{\sum_{l \neq
			k}\exp\left( -\|y_l - y_k\|^2 \right)}
		\]
		Here, we don't have to mess around with \( \sigma \), because \( Y \) is in a
		space which we learn anyways. \question{what does learn mean in this case?}
	\item We now want to find a distribution of \( Y \) that best represents the
		neighborhoods in the original space \( X \), so effectively we want \( Q
		\approx P \). To do this, we want to minimize the KL-divergence between \( P
		\) and \( Q \). The KL-divergence is defined as:
		\[
			KL(P \parallel Q) = \sum_{i, j} P_{ij} \log \frac{P_{ij}}{Q_{ij}}
		\]
		This is not symmetric, so \( KL(P \parallel Q) \neq KL(Q \parallel P) \). 
	\item One of the problems with dimensionality reduction is that points tend to
		clump up, so intuitively in order to find better neighbors, the distribution
		on \( Y \) should be more sparse than the distribution on \( X \). Based on
		this intuition, we arrive at t-SNE. 
	\item In reality, t-SNE makes only one change to our approach, which is that
		instead of using a Gaussian we use a student-t distribution for \( Q_{j
		\leftarrow i} \), so we have:
		\[
			Q_{j \leftarrow i} = \frac{\left( 1 + \|y_i - y_j\|^2
			\right)^{-1}}{\sum_{i \neq k} (1 + \|y_i - y_k\|^2)^{-1}}
		\]
		The student-t distribution has the benefit that along the edges you have more
		mass, so the probability of points further away in \( Y \) being neighbors is
		higher. 
\end{itemize}



	\section{Dynamic Programming III}

We'll look at more examples today of DP.

\subsection{Knapsack (without repetition)}
\begin{itemize}
	\item Start with a recap with knapsack: had a weight capacity $W$, and a set of items with individual 
		weights $(w_i, v_i)$, and we wanted to look at the most valuable combination of items.
	\item Now, we're going to look at this problem with the with the constraint that \textit{we cannot 
		choose with repetition}
	\item To solve, look at how we solved the problem with repetition: introduced $K(c)$ which gets us 
		the best achieveable value for a capacity $c \le W$. The issue with trying the same thing 
		is that our subproblems don't track which items have already been used. Why not keep track of both? 
\end{itemize}

\subsubsection{Subproblems}
\begin{itemize}
	\item Introduce a 2D array: essentially solve the problem for smaller knapsacks and also smaller 
		capacities. Then expand in two directions: in terms of the number of items and also the 
		capacity.
	\item So keep track of all weights $c \le W$ and all items $j \le n$. Define $K(j, c)$ to be the 
		optimal solution to the knapsack for capacity $c$ and items $\{1, 2, \dots, j\} $. (It doesn't 
		need to use all the items from 1 to $j$.
	\item For each $K(j, c)$, we recurse smaller subproblems:

		\textit{Case 1:} The optimal solution on items 1 through $j$ doesn't use item $j$. 
					Here, $K(j, c) = K(j-1, c)$.

			\comment{Note that this is not equivalent to $K(j-1, c - w_j)$, since the $w_j$ could be 
			distributed among other items.}

		\textit{Case 2:} the optimal solution on items 1 through $j$ uses item $j$.
		Here, $K(j, c) = K(j-1, c - w_j) + v_j$. We add $v_j$ to $K$ since we're now using item $j$.   

		The intuition here is that we use the optimal solution without item $j$, then add in item $j$ 
		at the end. 
\end{itemize}
\subsubsection{Implementation}
\begin{itemize}
	\item So let's formalize this:
		\[
			K(j, c) = \max\{ K(j-1, c), v_j + K(j-1, c - w_j)\}
		.\]
		with base cases $K(0, c) = 0$ and $K(j, 0) = 0$. The base cases make sense since with no items our 
		optimal value is 0, and with no allowed weights then the optimal value is also 0. 

	\item Looking at $K(j, c)$ it only relies on the subproblmes $K(j-1, c)$, or $K(j-1, c - w_j)$, so we're 
		only looking at row $j-1$, and different elements in that row. This tells us about the order in which 
		we should be solving the subproblems: we could either do this row by row or column by column. 
	\item For runtime, ther eare $O(nW)$ subproblems, and in each subproblem we're doing constant 
		work (memory access), so therefore the total runtime is $O(nW)$, just like knapsack with repetition. 
	\item For space complexity, notice that each $K$ only depends on the previous row, so once we've 
		moved onto the 3rd row, we no longer need the first. We can delete this from memory, so the optimized 
		space complexity is $O(W)$. 
\end{itemize}
\subsection{Traveling Salesperson Problem}
\begin{itemize}
	\item A notoriously difficult problem, and DP helps us get a \textit{slightly} better runtime. 
	\item Input: Cities $1, \dots, n$ and pairwise distances $d_{ij}$ between cities $i$ and $j$. We want to 
		find a ``tour'' of minimum total distance (so we need to visit every city exactly once and 
		return to the city we started at).
	\item The naive brute-force algorithm basically is the one where we have to go through all possible 
		tours: there are \( n! \in O(n^n) \) possible tours, which makes this computation very expensive. 
	\item Dynamic programming gives us $O(n^2 2^n)$. (this is nearly optimal, beating $O(n 2^n)$ is theorized
		to be impossible)
		\begin{itemize}
			\item To give an illustration of the difference DP makes, if $n = 25$, then $O(n!) \approx 10^{25}$, 
				whereas $O(n^2 2^n) \approx 10^{10}$, so we're already better by 15 orders of magnitude. 
		\end{itemize}
\end{itemize}
\subsubsection{Subproblems}
\begin{itemize}
	\item One challenge of TSP is that subproblems aren't exactly solving the problem. If we just look at 
		TSP for a subset of our graph, that doesn't necessarily give us a solution to the larger problem, since 
		we're looking for cycles. Instead, we think of ``partial solutions'' to our graph. 
	\item We think of the subproblmes as starting from city 1, ends in city $j$, and passes thorugh all cities
		in a set $S$ (which includes city 1 and \( j \)). Visually:
		\[
			1 \to i_1 \to i_2 \to \dots \to j			
		\]
		So we want to formally define $T(S, j)$ to be the length of the shortest path visiting 
		all cities in $S$ exactly once, starting from 1 and ending at $j$. 
\end{itemize}
\subsubsection{Recurrence Relation}
\begin{itemize}
	\item How to compute $T(S, j)$ using smaller subproblems? Well, look at the string again:
		\[
			\overbrace{\underbrace{1 \to i_1 \to i_2 \to \dots \to i}_{T(S \setminus j, i)} \to j}^{T(S, j)}
		\]
		To actually talk about $T(S, j)$, then we need to add $d_{ij}$ onto every $T(S \setminus j, i)$. However,
		what is annoying is that we actually don't know which city is second to last, so we'll need 
		to consider every possible city $S \setminus j$. 
	\item So, we'll have to pick the minimum over all $i \in S$ such that $i \neq j$. Formally:
		\[
			T(S, j) = \min \{T(S \setminus j, i) + d_{ij} | i \in S \land i \neq  j\} 
		\]
	\item Our base cases are $T(\{1\} , 1) = 0$, this is fairly trivial. We also want that $T(S, 1) = \infty$.
		The reason we want this is because we're talking about incomplete paths, so $T(S, 1)$ is not 
		a valid non-cycle. Hence, we want to set it to $\infty$.  
	\item We're not done though, because we have to do something to get us back to a cycle! At the end of 
		the recursion step, we'll want to add the final edge $(j, 1)$ back, but adding only the minimum:
		\[
			T(S, 1) = \min_{j \neq  1}\{T(\{1, \dots, n\}, j) + d_{j 1}\}
		\] 
\end{itemize}
\subsubsection{Implementation}
\begin{itemize}
	\item Want an array of size $2^n \times n$, and start with base cases. Then work on the recursion:
		\begin{center}
			\includegraphics[scale=0.5]{TSP-code.png}
		\end{center}
	\item For runtime, there are $O(2^n \times n)$ subproblems, and on each layer we're doing $O(n)$ work, since
		we're checking the minimum across $n$ nodes every iteration. So, we have $O(n^2 2^n)$ as 
		the final runtime. 

		\question{How do we explain that \(n^2 2^n\) is the number of subproblems?}

		\answer{$O(n)$ work at every step, then $n \times 2^n$ subproblems. There are $2^n$ subsets, and in 
			each subset we can choose a $j$ to exclude, which we can upper bound by saying that there are $n$ 
		of these. So $n \times 2^n$ is a tight upper bound on the number of subproblems.}
\end{itemize}
\subsection{Independent Sets in Trees}
\begin{itemize}
	\item We're given an undirected graph \(G = (V, E)\), and want to output the largest independent 
		set of \(G\).
	\item Recall that a set \(S \subseteq V\) is considered independent if there are no 
		edges between \(u, v \in S\).
	\item This is also a notoriously hard problem, for general graphs. There isn't a polynomial time algorithm 
		that does this. But for trees, we're in luck!

		\question{Why isn't the solution just selecting every other layer?} 

		\answer{There are instances where we can pick from two consecutive layers and still not have an edge. 
		Consider the tree:
			\begin{center}
				\begin{tikzpicture}
					  \node {A}
					child {node {B}}
					child {node {C}}
					child {node {D}
					  child {node {E}}
					  child {node {F}}
					};
				\end{tikzpicture}
			\end{center}
			Our greedy algorithm would select either \(\{A, E, F\} \) or \( \{ B,C,D\} \), but the optimal set
			is actually 
		\(\{B, C, E, F\} \), so this proves that our algorithm isn't optimal.}
	\item For trees, we know that they don't have cycles, so we can pick any node and say that that is the root.
		By doing this, we can get a ``natural ordering'' of the subproblems. 
\end{itemize}

\subsubsection{Subproblems}
\begin{itemize}
	\item Let \(I(v)\) be the size of the maximum independent set in the subtree that is rooted at \(v\). 
	\item Why is this a good subproblem? Becuase it's easy to write a recursion relation for it!
	\item For the subproblems, there are two cases:

		\textit{Case 1:} \(v\) (the root of the tree) is part of the optimal independent set. This 
		means that the children aren't allowed to be part of the independent set. So if we take $v$, we can't 
		take any of the subproblems. So we need to look instead at the \textit{grandchildren} of \(v\) to join.
		Here, we'd write this as:
		\[
			I(v) = 1 + \sum_{u\  \in \text{ grandchildren}} I(u)
		\] 
		We add 1 here because we're including \(v\) now. 

		\textit{Case 2:} \( v\) is not part of the optimal independent set. Here, we would just take 
		the maximum of the children. Then:
		\[
			I(v) = \max_{u \ \in \text{ children}} \{I(u)\} 
		\] 


		So we'll take the max of these two cases:
		\[
		I(v) = \max \{1 + \sum_{u \ \in \text{ grandchildren}} I(u), \sum_{u \ \in \text{ children}} I(u)\}
		\] 
		Also, base cases is that $I(\text{leaf}) = 1$. 
\end{itemize}
\subsubsection{Implementation}
\begin{itemize}
	\item We need a data structure to store the tree easily, and also make sure that every child is processed 
		before the parents are. Well, we can iterate through the graph in post decreasing post order! 
	\item The runtime of DFS on trees is \(O(|V|)\), and each edge is looked at \(\le 2\) times -- once 
		for the children and also once for its grandchildren, so each subproblme takes constant time. 
	\item So that the total work is \(O(|E|) = O(|V|)\), since \(|E| = |V| - 1\).
\end{itemize}

	\section{DT Sampling}
\begin{itemize}
	\item Recall what we talked about last time about CT sampling:
		\[
		x_p(t) = x(t) \frac{1}{T}\text{III}\left( \frac{t}{T} \right)  = x(t) f_s \text{III}(tf_s)
		\] 
		and also the fact that we can write \( x_d[n] = x(nT) \), to change the continuous signal into 
		a discrete one. 
\end{itemize}

	\section{Network Flow}
\begin{itemize}
	\item Recently declassified (1999) document about the USSR's shipment capacity from east to west. This 
		was crucial information at the time since had a war broke out, the US could identify which 
		supply routes they could bomb.
	\item They devised a greedy algorithm called ``flooding,'' but this algorithm wasn't really optimal. It 
		was finally solved by Ford and Fulkersson, and is now called the Ford-Fulkersson algorithm.
	\item Given a directed graph $G = (V, E)$, one source vertex $s$ and a sink $t$, and for each edge $e \in E$,
		we're given a capacity $c_e$ which are integers. 
	\item We want to find the maximum amount of water from $s \to t$.  
	\item \textit{Definition:} A flow assigns a number $f_e$ to each directed edge $e \in E$ such that:
		\begin{itemize}
			\item nonnegativity: \(f_e \ge  0\) 
			\item capacity: \(f_e \le c_e\)
			\item flow in and flow out are equal: \(\sum_{u \to v} f_{u, v} = \sum_{v \to w} f_{v, w} \)
		\end{itemize}
	\item Let's also define the size of the flow $f$ to be the total quantity set from $s$ to $t$. Using 
		this definition, then the maximum flow is the one that maximizes $\size(f)$. This can be solved 
		using linear programming!
\end{itemize}

\subsection{Greedy (suboptimal) algorithm}
\begin{itemize}
	\item We'll find a path $P$ from $s$ to $t$, and send flow until it's saturated. We'll do 
		this as much as we can. We repeat this until we run out of paths.
	\item This algorithm fails on some graphs, because it uses edge $A \to B$ when that edge is suboptimal! 
		Consider the graph:
		\begin{center}
			\begin{tikzpicture}[scale=2]
				\node (s) at (0, 0) {$s$};
				\node (a) at (1, 0.7) {$A$};
				\node (b) at (1, -0.7) {$B$};
				\node (t) at (2, 0) {$t$};
				\graph[edge label = 1]{(s) ->(a) -> (t), (s) -> (b) -> (t), (a) -> (b)};
			\end{tikzpicture}
		\end{center}
		Our algorithm just looks at flow rate, so a possible path to take is $s \to A \to B \to t$, but this 
		is clearly suboptimal! Instead, we should be going from $s \to A \to t$ and $s \to B \to t$. 
\end{itemize}

\subsection{Greedy Fix}
\begin{itemize}
	\item We instead consider a residual graph, where we subtract the flow given by greedy 
		($s \to A \to B \to t$), and also generate a back edge that travels in the reverse order of the flow 
		given by greedy, so that we can backtrack if needed. 
	\item Formally, given a graph $G$ and a flow $f$ on $G$, the residual raph $G_f$ is defined as:
		For all edges \((u, v)\), if $f$ goes from $u \to v$, then the residual graph will flow from $v \to u$ 
		and the edge will have capacity $c_{u, v} - f_{u, v}$. 

		By doing this, we allow our graph to backtrack along our suboptimal path if needed.  
	\item This is the approach that Ford Fulkerson uses to find the optimal flow.
\end{itemize}

\subsection{Ford-Fulkerson Algorithm}
\begin{itemize}
	\item Find a path $P$ from $s$ to $t$ in the residual graph which is not yet saturated, and send more 
		flow along $P$. We keep repeating this until everything's saturated, and this happens 
		when all edges along one particular cut is zero. 
	\item To show that this algorithm terminates, lets' first define an $s-t$ cut is a partition of the graph
		into two sets of vertices \(L\) and \( R\) such that 
		\(s \in L\) and \(t \in R\). We define the capacity of this cut to be the sum of all capacities from 
		the edges that cross from $L$ to $R$.  
	\item Therefore, for any flow $f$ and any cut \((L, R)\), then \(\size(f) \le \capacity(L, R)\). Then, the
		flow is actually upper bounded by the minimum cut along this graph (this is our ``bottleneck'' introduced
		at the outset)
	\item Then, this means that the max flow is also given by the minimum cut, and we can show that 
		Ford-Fulkerson outputs a maximum flow by considering this relation between the flow and a cut. The proof 
		of this is outlined in lecture. 

		\question{Review the proof for this later}
\end{itemize}

\subsection{Runtime}
\begin{itemize}
	\item The number of augmenting paths must be less than $U$, where $U$ denotes the maximum flow, so 
		the update is less than $O(m + n) \cdot U$. 
	\item But what this means   
	\item There are other algorithms out there that optimizes this a little more: Edmonds-Karp gives 
		us a runtime of $O(n m^2)$, which is much better than what we have. 
	\item The best runtime was discovered last year, where we have $O(m^{1 + o(1)} \cdot \log U)$
\end{itemize}

	\section{Decision Trees, Ensemble Approaches}
\begin{itemize}
	\item Decision trees are very powerful -- in instances where we can use decision trees to perform
		classification, they almost always outperform alternative methods.  
	\item One nice thing about decision trees is that you don't really have to go through the trouble of
		hyperparameter tuning, and they are also highly interpretable, unlike neural networks.   
	\item You can really think of these as a game of "20 questions", where you ask a bunch of yes or no
		questions and decide the classification based on the answers.   
	\item The nonlinearity offered by recursive decision boundaries is very powerful, and is something linear
		regression models cannot do.
	\item \comment{Some cons: the decision boundary is not smooth, and they are relatively unstable and
		sensitive to changes in the data.}
\end{itemize}

\subsection{Structure}
\begin{itemize}
	\item The goal is that given a set of "features" \( X \), if we can predict some other feature. For
		instance, given the horsepower, weight, maker of a car (\( X \)), we can be asked to predict the miles per
		gallon, \( Y \).
	\item Mathematically, we are essentially trying to learn a class of functions \( f: X \to Y \) where \( X
		\) is the features and \( Y \) are the possible outputs.    
	\item Before we look at how the decision tree is built, we look at how we use one. Given some input data,
		the decision tree determines the order in which we query the parameters to determine a result. These
		queries may not be the same length, as sometimes a single query is enough to determine a result.    

		Each node in the decision tree is essentially a query to an attribute \( x_i \), and depending on its
		value we follow the corresponding edge. Leaves in this tree correspond to the classifications \( y
		\).  
	\item Compared to linear models like KNNs, decision tree boundaries \textit{must} be axis-aligned. This
		is an issue, because it means that the decision boundary is usually more complex than linear
		regression models, or it would take more information to replicate the same boundary.  
	\item Another benefit of decision trees is that because they can grow arbitrarily large, its limit is
		basically a truth table, so this means that you can represent any function of the input attributes
		with a decision tree.     
\end{itemize}

\subsection{Building Decision Trees}
\begin{itemize}
	\item To start, the simplest decision tree we make is one which doesn't consider the data at all, and
		just classifies all data based on the \( y \) that is most common in the training data. Obviously
		this is not interesting, so we move on. 
	\item Then, you look at a single feature, and create a node for every possible value that the feature
		could take on. Then for each of these values, find the \( y \) that is the most common, and that
		becomes your prediction.
		\begin{center}
			\includegraphics[scale=0.5]{images/lec17-1.png}
		\end{center}
		So here we first query on the number of cylinders, then based on the majority mpg (either good or
		bad) in each cylinder category we can come up with a classification that makes sense. We then
		recursively do this, and find the next best query to look at. This will be different for different
		nodes:  
		\begin{center}
			\includegraphics[scale=0.5]{images/lec17-2.png}
		\end{center}
	\item So how do we grow this tree? What feature do we split on? The goal is to make leaf nodes to be
		pure (consisting of only one class), so we want splits which increases the purity of the leaf nodes.  

		One way to quantify this purity is to quantify how surprised we are at seeing a particular result,
		and then compute the \textit{entropy}, which is the expected surprise overall. The entropy is defined
		as:
		\[
			H(Y) = -\sum_k P(Y = k) \log P(Y = k)
		\]
		The negative is there to signify the fact that rare events are more entropic than common ones. Pure
		nodes which don't have misclassifications have zero surprise, so finding the best split is the same
		as finding the classification which minimizes entropy.    
	\item Thinking of distributions more generally, high entropy tend to spread their mass over the entire
		space, whereas low entropy generally has its probability mass concentrated at a particular point.   
	\item In the training data, we can't compute teh entroyp of \( Y \), but we can control the entropy of \(
		p(Y \mid X)\), since we control the feature \( X \) that we want to look at. So, our goal is to
		reduce the \textit{conditional entropy} at each node so that we eventually reach a surprise of zero. 

		The conditional entropy is just given by:
		\[
			H(Y \mid X_{j, v}) = P(X_{j, v} = 1) H(Y \mid X_{j, v} = 1) + P(X_{j, v} = 0) H(Y \mid X_{j, v} =
			0)
		\]
		By the rules of expectation, this conclusion should be relatively obvious. Here \( X_{j, v} \) is a
		binary random variable, so we only have two terms, but you can imagine a case where we have more. The
		point here is that we split on the values that \( X_{j, v} \) could equal, then compute the entropy
		for each subcategory and weight them accordingly.   

		\comment{Some books use the base-2 logarithm here; it makes no difference versus base-10 log.}  
	\item Equivalently, we can phrase this as maximizing the \textit{information gain}, which is defined as:
		\[
			I(X_{j, v}; Y) := H(Y) - H(Y \mid X_{j, v})
		\]
		\( H(Y) \) is the entropy you start with (constant), and compute the conditional entropy for the
		different \( X_{j, v} \) labels and maximize this quantity.   

		\comment{There are no parameters and "learning" here! This is really just chugging through to find
			best possible feature to split on. For real valued features, you pick a threshold to split on
		instead.}
	\item When do you stop recursing on a node? There are two cases:
		\begin{enumerate}[label=\arabic*.]
			\item If the category \( Y \) is pure after the classification, so all results after sorting are
				of a single class.
			\item If the remaining features to split on don't improve the data at all, then the node becomes
				unexpandable. Note, this is not the same as zero information gain. In cases like these, you
				just flip a coin to decide how to predict. 

				\question{In the slides, they write this as "all input values are the same", how is the
					lack of further classification (or lack of improvement) related to the fact that all
					input values are the same? Couldn't you just have a 1-1 tie with the input values not being the
				same?} 
		\end{enumerate}
\end{itemize}


	\setcounter{section}{17}
\section{Fault Tolerance}
Suppose we have a quantum computer that encodes the following qubits:
\begin{align*}
	\ket*{\overline 0} &= \ket*{000}\\
	\ket*{\overline 1} &= \ket*{111} 
\end{align*}
As we've seen before, this corrects against \( Z \)-errors (bit flips) very well, but is susceptible to \( X \)-errors.
Now suppose that this code goes into a channel with a probability \( p \) of flipping the bit, and a proability 
\( 1-p \) of it staying the same. There are three states that exist with one error:
\[
\ket*{001}, \ket*{010}, \ket*{100}
\] 
since each one has a probability \( p \) of occurring, then the probability we end up in either one of these states 
is \( 3p \). There are also three states with two errors:
\[
\ket*{101}, \ket*{011}, \ket*{110}
\] 
these codes, when checked with our error correction, will send these states to \( \ket*{111} \), which 
results in a logical error. The probability of this happening is \( 3p^2 \), since that's the probability 
we get sent to any one of these qubits. Therefore, we say that this code sends \( p \to 3p^2\).   

\question{why is this not \( 3p^2 + p^3\)?} 


\subsection{Concatenation} 
What if we wanted to correct against more errors? We can concatenate the code 

	\section{Transfer Function of LTI Systems}
\begin{itemize}
	\item For an LTI system with the standard input and output pairs, we know that since \( y(t) = x(t) * h(t) \), 
		then this means that 
		\[
		Y(s) = X(s) H(s)
		\] 
		where \( H(s) \) denotes the Laplace transform of the impulse response, 
		\[
		H(s) = \int_{-\infty}^{\infty} h(t) e^{-st}\diff t 
		\] 
	\item For an LCCDE of the form:
		\[
			\sum_{k=0}^{N} a_k \dv[k]{y(t)}{t} = \sum_{k=0}^{M} b_k \dv[k]{x(t)}{t}
		\] 
		then we can take the Laplace transform of both sides, 
		\[
		\sum_{k=0}^{N} a_k s^{k}Y(s) = \sum_{k=0}^{M} b_k s^{k}X(s)
		\] 
		So, this means that:
		\[
		H(s) = \frac{Y(S)}{X(s)} = \frac{\sum_{k=0}^{\infty} b_k s^{k}}{\sum_{k=0}^{N} a_k s^{k}}
		\] 
\end{itemize}
\subsection{RLC Circuit}
\begin{itemize}
	\item Consider the following circuit:
		\begin{center}
			\begin{circuitikz}[american]
				\draw (0, 3) to [voltage source, l_ = \( x(t) \)] (0, 0);
				\draw (0, 3) to [R] (2, 3) to [R] (4, 3);
			\end{circuitikz}
		\end{center}
\end{itemize}
\subsection{Effect of Poles}
\begin{itemize}
	\item For non-repeated poles (as in, the poles aren't degenerate), then the transfer function can 
		be written generically as:
		\[
		H(s) = \sum_{i=1}^{N} \frac{A_i}{s - a_i}
		\] 
		If the system is causal, then recall that for an LTI system this means that \( h(t) = 0 \) for \( t < 0 \)
		\[
		h(t) = \sum_{i=1}^{N} A_i e^{-a_i t}u(t)
		\] 
	\item If \( \alpha_i \) is repeated \( m \) times, then the system response will include these terms:
		\[
			t^{m-1}e^{\alpha_i t}, \cdots,te^{\alpha_1}t, e^{\alpha_i t}
		\] 
\end{itemize}
\subsection{Stability of Causal System}
\begin{itemize}
	\item Given a causal system described by  \( H(s) \), we saw earlier that \( Y(s) = H(s) X(s) \). There's also 
		a theorem that says that the system is stable if and only if all poles of \( H(s) \)  have strictly negative 
		real parts. 
		This is because of two reasons:
		
		\begin{enumerate}[label=\arabic*.]
			\item This is because if \( H(s) \) is rational, 
				then the causality is equivalent to the region of convergence to 
				the right of the rightmost pole (on the real line)
			\item The absolute integrability condition \( \int_{-\infty}^{\infty} |h(t)|\diff t < \infty \) means 
				that the imaginary axis is within the ROC. 
		\end{enumerate}
\end{itemize}
\subsection{Connected LTI Sytems}
\begin{itemize}
	\item For systems connected in series, then \( H(s) = H_1(s)H_2(s) \), and in parallel then
		\( H(s) = H_1(s) + H_2(s) \). This is the exact same as the Fourier transform \( H(\omega) \) properties. 
	\item For feedback control systems (e.g. a compensator), then we define an error function 
		\( E(s) = X(s) - H_2(s) Y(s) \). The subtraction comes from the fact that the feedback is fed 
		as a minus sign. Then, \( Y(s) = H_1(s) E(s) \), and solving both equations 
		gives:
		\[
		Y(s) = H_1(s) X(s) - H_1(s) H_2(s) Y(s) \implies H(s) = \frac{Y(s)}{X(s)} = \frac{H_1(s)}{1 + H_1(s)H_2(s)}
		\] 
\end{itemize}
\subsection{Z-transform}
\begin{itemize}
	\item Very much the same as the Laplace transform, but for discrete-time signals. 
	\item This is very similar to the DTFT formula:
		\[
			X(e^{j \omega}) = \sum_{n=-\infty}^{\infty} x[n] e^{-j \omega n}
		\] 
		This is periodic because the signal is discrete in time. The z-transform is the generalized version of this 
		formula, written as:
		\[
			X(z) = \sum_{n=-\infty}^{\infty} x[n] z^{-n}, z \in \C
		\] 
		so really, the only difference is that we swapped \( e^{-j \omega} \) for a general complex 
		number \( z \). Because \( z \) also has a magnitude term, there's also a corresponding region of 
		convergence for \( z \). 
\end{itemize}
\subsection{Pairs}
\begin{itemize}
	\item Given \( x[n] = \delta[n] \), then:
		\[
			X(z) = \sum_{n=-\infty}^{\infty} \delta[n] z^{-n} = 1
		\] 
		as long as \( z \neq 0 \). With that caveat in mind, the ROC is the entire complex plane. 
	\item Given \( x[n] = a^{n}u[n] \):
		\[
			X(z) = \sum_{n=-\infty}^{\infty} a^{n}u[n] z^{-n} = \sum_{n=0}^{\infty} a^{n}z^{-n}
			= \sum_{n=0}^{\infty} (az^{-1})^{n} = \frac{1}{1 - (az^{-1})}
		\] 
		the last step simplifies due to geometric series. The condition for this to converge is 
		that \( |az^{-1}| < 1 \), since that's when the geometric series converges. So this simplifies 
		to \( |z| > |a| \), or \( |z| > a \) since \( a \) is real. 

		Visually, this corresponds to a boundary at infinity and one that's a circle at radius \( a \). Consequently, 
		if the unit circle exists within the region of convergence, then we know that the DTFT of the signal 
		exists. 
	\item Given \( x[n] = -a^{n}u[-n - 1] \):
		\[
			X(z) = \sum_{n=-\infty}^{\infty} -a^{n}u[-n - 1] z^{-n} = \sum_{n=-\infty}^{-1} -a^{n}z^{-n}
			= 1 - \sum_{n=0}^{\infty} (-a^{-1}z)^{n}  = 1 - \frac{1}{1 - a^{-1}z}
		\] 
		The region of convergence is defined similarly: \( |za^{-1}| < 1 \), so \( |z| < |a| \). This is just a
		filled circle up to a radius of \( |a| \). Of course, the DTFT exists only when the radius of this circle 
		is larger than 1. 
	\item Given  \( x[n] = -\left( \frac{1}{2} \right)^{n}u[-n - 1] + \left( -\frac{1}{3} \right)^{n}u[n] \), 
		since the z-transform is linear, then:
		\[
		X(z) = \frac{1}{1 - 2z } + \frac{1}{1 + z^{-1} / 3} = \cdots = \frac{6z^2 - 6z - 1}{(3z + 1)(2z - 1)}
		\] 
\end{itemize}
\subsection{Z-transform properties}
\begin{itemize}
	\item The ROC for z-transforms are circles or rings instead of planes, and that's just becuase we're 
		converting \( e^{j \omega} \) to \( z \). 
	\item See the lecture slides for a table on the properties. 
\end{itemize}
\subsection{Inverse Z-transform}
\begin{itemize}
	\item We can compute the inverse Z-transform using partial fraction expansion. The reason we keep going back 
		to this is because many Z-transforms are characterized by rational functions, so if we can find a way to 
		split the fractions up then we can find the inverse from linearity.  
\end{itemize}





	\section{Reductions II}
\begin{itemize}
	\item Recap: Two computational problems \(A\) and \(B\), and \(A\) reduces (in polynomial time) to \(B\) 
		is written as \(A \preceq_p B\). This means that if an algorithm exists to solve \(B\) in polynomial 
		time, then that same algorithm can be used to solve \(A \) in polynomial time. 

		\question{Is this restricted to only polynomial time? Shouldn't any feasible algorithm that solves 
		\(B\) also solve \(A\)?}
		
		\question{Why are we so concerned about polynomial time? Do similar problems exist if we define 
		``efficient'' to be exponential time?}
	\item Also recall the diagram we made to represent this process of reducing \(A\) to \(B\), via 
		a polynomial time reduction and recovery algorithm. Note that these two algorithms \textbf{must} execute
		in polynomial time. 
		\begin{itemize}
			\item We can prove that \(A \preceq_p B\) even if \(A, B\) are not known to be efficient.
		\end{itemize}
	\item We also saw two reductions: zero sum games reducing to LP, and Hamiltonian cycle reducing to min-TSP.
	\item Transitivity: If \(A \preceq_p B \preceq_p C\), then \(A \preceq_p C\). 
\end{itemize}
\subsection{Common mistakes in Reductions}
\begin{itemize}
	\item If we're asked to prove that \(A \preceq_p B\), we need to come up with an algorithm that takes
		\(A\) to \(B\), not \(B\) to \(A\). Make sure you check that you're proving the correct direction!
\end{itemize}
\subsection{Landscape of Problems}
\begin{itemize}
	\item We're going to use the below diagram to show the problems:
		\begin{center}
			\begin{tikzpicture}
				\draw(0, 0) circle [radius=1.5cm];
				\draw node at (0, 0.2) {P};
				\draw (0, 1) ellipse [x radius = 2.5, y radius = 3];
				\draw node at (0, 3) {NP};
				\draw[blue] node at (0, -0.5) {\small MST};
				\draw[blue] node at (0.7, 0.5) {\small LP};
				\draw[blue] node at (-0.7, 0.5) {\small APSP};
				\draw[blue] node at (1, 1.8) {\small 3-coloring}; 
				\draw[blue] node at (-1.3, 1.9) {\small Search-TSP};
				\draw[blue] node at (3, 4) {\small Min-TSP};
			\end{tikzpicture}
		\end{center}
	\item We're not going to prove this, but it has been shown that factoring reduces to the 3-coloring 
		problem. Similarly, factoring also reduces to the Rudrata Cycle problem. 
	\item It turns out that every problem in NP reduces to Rudrata cycle!
	\item These are the most difficult problems in NP, and it can be shown that every problem in NP reduces 
		to an NP-complete problem
	\item \textbf{NP-Hardness:} A problem \(A\) is NP-hard if every problem \(B\) in NP reduces to \(A\). 
	\item \textbf{NP-Completeness:} A problelm \(A\) is NP-complete if \(A \in \text{NP}\) and \(A\) is 
		NP-hard.
	\item Problems in NP that aren't NP-complete are called an \textbf{NP-intermediate} problem 
	\item \textbf{Fact:} Given two problems that are NP-complete, then \(A \preceq_p B\) and \(B \preceq_p A\). 
		So this means that you can basically think of \(A\) and \(B\) are basically equivalent problems. 

		\question{Is this a biconditional?}

		\answer{No, consider two problems in P: they can be reduced to one another, but they are not 
		in NP.} 
		\begin{itemize}
			\item  There are thousands upon thousands of NP-complete problems, and by notion of reduction, 
				they are (in some sense) the same problem.
			\item This also means that if there exists a polynomial time algorithm for any NP-problem, 
				then this would imply that P = NP.
		\end{itemize}
		\question{How is it that if P = NP then every problem becomes NP-complete?}  
\end{itemize}

\subsection{Proving NP-Completeness}
\begin{itemize}
	\item Cook-Levin Theorem: showed that every problem in NP reduces in polynomial time to a circuit SAT 
		problem.
	\item It can then be shown that circuit-SAT reduces to 3-SAT, making 3-SAT an NP-complete problem. In 
		terms of a diagram:
		\begin{center}
			\begin{tikzpicture}[every text node part/.style={align=center}]
				\foreach \x in {0, 4}
						\draw[-stealth] (\x+0.5, 0) -- (\x+2, 0);
				\draw node at (-1, 0) {Every problem \\ in NP};
				\draw node at (3.25, 0) {Circuit\\SAT};
				\draw node at (7.25, 0) {3-SAT};
				\draw [-stealth] (8, 0) -- (10, 1.5) node[right] {Independent \\ Set};
				\draw [-stealth] (8, -0.2) -- (10, -1.5) node[right] {Rudrata Cycle};
			\end{tikzpicture}
		\end{center}
		\question{Finish this Diagram Later}
	\item To show that a problem is NP-complete, we first show that \(A \in \text{NP}\), then 
		pick some problem \(B\) that is known to be NP-complete and show that \(B \preceq_p A\). 
		
		\question{What if we show that \(A \preceq_p B\)?}

		\answer{We don't need to, since \(A \preceq_p B\) is true already because \(B\) 
		is an NP-complete problem!}
\end{itemize}
\subsection{Circuit SAT}
\begin{itemize}
	\item A Boolean circuit is a directed acyclic graph with:
		\begin{itemize}
			\item Input nodes \(x_1, \dots, x_n\) 
			\item one output node, with an output \(C(x)\)
			\item gates marked OR, AND, NOT: \(\lor, \land, \neg\)
		\end{itemize}
		A possible graph is:
		\begin{center}
			\begin{tikzpicture}
				\node (c) {$\lor$}
					child [stealth-] {node (a) [left] {$\land$}  
					child {node {$x_1 = 1$}}
				}
				child [stealth-] {node {$\land$}
				  child {node (b) {\(x_2 = 1\)} }
				  child {node {\(\neg\)} 
					  child {node {\(x_3 = 1\) }}
					}
				};
			\draw[-stealth] (b) -- (a);
			\draw[-stealth] (c) -- (0, 1) node[above]  {1}; 
			\end{tikzpicture}
		\end{center}	
	\item The input to circuit SAT is a circuit \(C\) with \(n\) inputs and \(m\) referring to the number of 
		gates. We want to output an assignment of \((x_1, \dots x_n)\) such that \(x_i \in \{0, 1\}\) 
		such that \(C(x) = 1\).
	\item By the Cook-Levin theorem, circuit-SAT is NP-complete. As for a bit of intuition on why this is true, 
		you can think of every problem as basically a collection of logical inputs, which basically means 
		that every problem can be reduced to some complex circuit of logical gates.  
\end{itemize}
\subsubsection{3-SAT}
\begin{itemize}
	\item Here, we're given \(n\) Boolean variables \( x_1, \dots, x_n\) such that \(x_i \in \{0, 1\} \), and 
		\(m \le  3\) variable clauses that join the variables together. 
	\item We want to output an assignment of \(x_1, \dots, x_n\) that satisfies all the clauses. 
	\item \textbf{Theorem:} Circuit-SAT reduces to 3-SAT

		\textit{Proof:} Suppose we're given an input to a circuit-SAT problem.
\end{itemize}

	\section{Quantum Computing Platforms}
\begin{itemize}
	\item Trapped ions: qubits are single atoms, but we've removed one of the electrons so they're positively charged.
		We do this so that we have better control over them. This is the platform that's pursued by Honeywell/
		Quantinuum, AQT, among other companies.  

		\textbf{Pros:}
		\begin{itemize}
			\item These have long coherence times \( T_2 \sim 1 \) minute
			\item They operate at room temperature -- basically it's just a big vaccuum chamber sitting in a room 
				without the need for cryogenics. Nothing about the qubits require low temperature, we just happen 
				to involve cryogenics in order to achieve a better vaccuum. 
			\item Highest fidelity gates so far, and have been one of the first to discover quantum gates.  
		\end{itemize}

		\textbf{Cons:}
		\begin{itemize}
			\item Gates operate typically at 50 microseconds.  
			\item Requires lasers, optics
			\item The coulomb interaction between ions makes scaling more difficult
		\end{itemize}
	\item Neutral Ions: basically the same trapping techniques, except the atoms are neutral instead of 
		charged. We can't trap them using fields because they are neutrally charged. 

		\textbf{Pros:}
		\begin{itemize}
			\item Long qubit coherence times
			\item Room temperature operation 
			\item Inherently somewhat scalable, since neutral atoms don't interact
			\item Optical interface. 
		\end{itemize}
		\textbf{Cons:}
		\begin{itemize}
			\item Experiments usually looks like a mess (requires a lots of lasers), and much of the effort goes into 
				managing the lasers
			\item Requires an ultrahigh vaccuum (so we need cryogenics basically)
			\item Trapping is inherently more difficult and requires high laser power 
		\end{itemize}
	\item Superconducting qubits: these are man-made qubits instead of atoms. 

		\textbf{Pros}
		\begin{itemize}
			\item Chip-based architecture. It's something that we can imagine scaling up to a chip, and we interact 
				with it electronically
			\item Fast gate times (on the order of 50 nanoseconds)
			\item Previously leading the field commercially, but starting to recognize that superconducting 
				qubits might not be the way. (some comapnies are starting to invest in atomic-based approaches)  
		\end{itemize}

		\textbf{Cons:}
		\begin{itemize}
			\item Requires dilution refrigerators, so need cryogenic temperatures 
			\item Short coherence times, though this is getting much better 
			\item Control lines needed for every individual qubit -- every single qubit you want to add means an 
				extra set of lines you need to connect to your system. 
			\item Not very anharmonic. 
		\end{itemize}
	\item Quantum Dots: creating a 2D electron gas, and is possibly the most similar to the structure that we studied 
		in the previous lecture. 
		
		\textbf{Pros:}
		\begin{itemize}
			\item Semiconductor chip based architecture
			\item High fidelity, fast-qubit and two qubit gates. 
			\item Controlled by microwaves and electronics, there are no lasers required in this process. 
		\end{itemize}
		
		\textbf{Cons:}
		\begin{itemize}
			\item Requires cryogenic temperatures
			\item Short coherence times 
			\item Lots of tuning required for each device -- very sensitive architecture
			\item Scaling up to many qubits is still an open problem. We can do 2-qubit gates, but it's not clear 
				how we would scale up beyond that. 
		\end{itemize}
	\item Photonics: none of the stuff that we've talked about so far really applies here. States are single 
		photons, where the information is either stored in the polarization or which rail the photon 
		lives on. 
		
		\textbf{Pros:}
		\begin{itemize}
			\item Silicon chip based architecture
			\item Room temperature operation, what this basically means is that in principle we can do this 
				at room temperature, but the best photon detectors still require cryogenics. 
			\item Fairly easy to scale up, since photons naturally fly around
			\item "Measurement" based -- some of the gates are very easy to physically implement. For instance, 
				if you wanted to change the polarization then all you'd do is just introduce a wave plate
		\end{itemize}
		
		\textbf{Cons:}
		\begin{itemize}
			\item Gate operations are very difficult to achieve, and are inherently probabilistic. Preparing 
				the state itself is a challenge (trying again and figuring out when you succeed), then performing 
				the desired computation afterwards. 
			\item Requires identical photonic states and photonic elements
			\item Low gate fidelities
		\end{itemize}
\end{itemize}
\subsection{Trapped Ions}
\begin{itemize}
	\item Ideally we'd like to use the simplest atom possible, which in our case would be hydrogen-like 
		atoms. We can't use hydrogen itself, because transitions for hydrogen are in the UV spectrum, which poses 
		some challenges (what specifically?)
	\item Commercially, we normally use alkaline earth metals and then strip one electron off so that we end up with 
		a hydrogen-like atom. 
	\item Hydrogen-like atoms have a principal quantum number \( n \), whose energy scales with:
		\[
		E_n = -\frac{z^2 \frac{\mu}{m} E_H}{2n^2}
		\] 
		\( E_H \) is a physical constant, which is written as \( E_H = mc^2 \alpha^2 \)
	\item We also have angular momentum \( \ell \), which has possibilties between \( 0 \) to \( n - 1 \). 
	\item Transitions between these obey selection rules, which tells us that \( \Delta \ell = \pm 1 \) and 
		\( \Delta m = \pm 1 \). 
\end{itemize}

	\section{Approximation Algorithms}
\begin{itemize}
	\item Suppose that you have a problem that is NP-hard. What now? There's a couple things 
		we can still do:
		\begin{enumerate}[label=\arabic*.]
			\item Learn more about its inputs 
			\item Come up with a Heuristic
			\item Approximation Algorithm (this is what we'll study today)
		\end{enumerate}
	\item For a minimization problem, an algorithm \( A \) is an \( \alpha \)-approximation problem 
		if \( A(I) \le \alpha \cdot \text{OPT} \), where OPT is the optimal value. For a maximization 
		problem, then the inequality reads \( A(I) \ge \alpha \cdot \text{OPT} \). 
\end{itemize}
\subsection{Vertex Cover}
\begin{itemize}
	\item Given an input graph \( G = (V, E) \), and we want to return a vertex cover \( C \subseteq V \) of 
		minimal size. 
	\item A \textbf{vertex cover} is a set of vertices such that every edge \( (u, v) \) is incident on 
		one of them.
	\item This is indeed an NP-hard problem -- it reduces from Independent set.  

		\comment{The NP-hardness comes from the fact that we want \( C \) of minimal size}
\end{itemize}
\subsection{Approximations to Vertex Cover}

\subsubsection{Algorithm 1: Maximal Matching}
\begin{itemize}
	\item The question we want to ask is: what is the easiest problem we can compute that is similar to the 
		problem we ultimately aim to solve?
	\item In this case, it turns out to be the maximal matching problem. 
	\item \textbf{Definition:} A \textit{matching} in a graph \( G \) is a set of edges with no 
		overlapping vertices. The matching is then maximal if we can't add any more edges to the matching. 
	\item We can compute this using a greedy algorithm: add edges to the matching, and delete all edges adjacent 
		to it. 

		\question{Is this problem also NP-hard, but there turns out to be a greedy algorithm that does it in 
		a relatively simple manner? Also, why can't we just use a greedy algorithm on the original problem?}  

		\answer{Consider the ``star graph" consisting of a central node connected to others. The maximal matching
			would output a size of 2, but a greedy algorithm could potentially pick \( n-1 \) nodes (going 
			around the circle). Hence, it's not optimal. Even with the modification to choose the vertex 
		with the highest degree, this is still not optimal, see textbook for counterexample.}

	\item Now we convert this into a vertex cover \( C \) by outputting both endpoints of every edge found in the
		matching \( M \). We now prove that this is a valid vertex cover and that 
		\( |C| \le  2 \cdot \text{OPT} \).  
	\item We prove this in two parts:

		\textit{Claim 1:} \( C \) is a vertex cover. 

		\textit{Proof:} Assume for contradiction that \( C \) is not a vertex cover. Then, there is some 
		edge \( (u, v) \in E \) where \( u, v \not\in C \). But this is a contradiction, because then 
		the Greedy algorithm would have selected this edge as well. 

		\textit{Claim 2:} \( |C| \le 2 \cdot \text{OPT} \). 

		\textit{Proof:} Any vertex cover covers every edge in the matching \( M \). Therefore, the vertex cover 
		includes one of \( (u, v) \) or potentially both. Therefore, the worst case is if both \( u, v \) are 
		selected in the matching instance when only one was required, but this gives us an upper bound of 
		\( 2 \cdot \text{OPT} \). 
		
		\question{Can we do better? Why can't we get rid of adjacently selected vertices at the end?}

		
		\comment{There is no restriction on adjacency of vertices in vertex cover, which is why the output 
		actually works as a solution to vertex cover.}
\end{itemize}
\subsubsection{Algorithm 2: LP}
\begin{itemize}
	\item We can also formulate this problem as a linear programming problem.
	\item Assign a variable \( x_i \) for each vertex \( i \), and ideally, we want to set \( x_i \) to 1 
		if vertex \( i \) is in the vertex cover, and 0 otherwise. 
	\item To formulate this as an LP, we'd want to minimize \( \sum_i x_i \), while subject to:
		\begin{align*}
			0 \le  x_i &\le  1\\
			x_i + x_j &\ge 1 \ \forall (i, j) \in E
		\end{align*}
	\item \textit{Claim:} The optimal value for the LP \( \le  \) the optimal vertex cover.

		\textit{Proof:} The optimal value for the LP could give us an infeasible (but smaller value) solution 
		because it's allowed to give us fractional \( x_i \), but the optimal set cover is a feasible solution 
		to the LP, so the LP either finds the optimal vertex cover, or an inadmissible value.    

	\item We now want to convert this to an optimal set cover, by doing an operation called \textbf{rounding.}
	\item Let \( \{x_i^*\}  \) be the optimal LP solution, and we will round based on the following rule:
		\begin{align*}
			x_i^* &\ge  \frac{1}{2} \Longrightarrow \text{Include in \( C \)}\\
			x_i^* &< \frac{1}{2} \Longrightarrow \text{Do not include in \( C \)}
		\end{align*}
		
		\textit{Claim:} \( C \) is a vertex cover.

		\textit{Proof:} For every edge \( (i, j) \), we have a constraint that \( x_i^* + x_j^* \ge  1\), meaning
		that \( \max(x_i^*, x_j^*) \ge  \frac{1}{2} \), so our rounding table would select one of these two 
		vertices to be in \( C \). 

		\textit{Claim:} \( |C| \le 2\cdot \text{OPT} \).

		\textit{Proof:} For all vertices, the actual vertex cover pays 1 unit to some \( x_i \), while  our LP 
		rounding algorithm pays \( x_i^* \ge  \frac{1}{2} \) every time, 
		so we're never doing worse than twice the optimal 
		LP. Therefore, \( |C| \le  2 \cdot \text{LP-OPT} \le  2 \cdot \text{OPT} \).
		
		\question{What is this concept of ``paying'' here? I'm not sure I understand this argument.} 

		\answer{Paying is probably not the best way to understand it. Another way to phrase this is that the worst case 
			scenario for our LP approximation is if it assigns \( x_i^* = \frac{1}{2}\) to all nodes, in which case 
			our rounding would output all the nodes for the vertex cover. However, we know that at most 
			we only need half of these nodes in the vertex cover (one per edge), hence the approximation 
		factor of 2.} 
\end{itemize}

\subsection{Metric TSP}
\begin{itemize}
	\item Recall the TSP problem, where we have \( n \) cities and pairwise distances \( d_{ij} \). 
	\item We want to output a minimum distance tour while still visiting every node exactly once. 
	\item It's known that you can't even solve this with an approximation factor!  
	\item But now we impose the restriction of the triangle inequality: for all cities \( i, j, k \), then 
		we have:
		\[
			d_{ij} + d_{jk} \ge d_{ik}
		\] 
		and \( d_{ij} \ge 0 \ \forall i,j \). With this approximation, then we \textit{can} derive an upper bound!
\end{itemize}
\subsubsection{Algorithm: MST}
\begin{itemize}
	\item Notice that with this restriction, this is very similar to the MST problem, except MST finds 
		a tree rather than a cycle. 
	\item So we first find the MST on \( G \). Then, we know that \( \cost(\text{MST}) \le  \cost(\text{OPT}) \),
		since the TSP also visits every vertex, and MST is known to be the set of edges that has the minimum 
		weight. 

		\begin{center}
			\begin{tikzpicture}
				\draw (0, 0) node[below left] {\( A \) } -- (1, 2) node[above] {\( B \) } -- (3, 0) node[above] {\( C \) } -- (4, 1) node[right] {\( D \) };
				\draw (3, 0) -- (4, -1) node[below] {\( F \) };
				\draw (3, 0) -- (2.5, -1) node[below] {$E$};
			\end{tikzpicture}
		\end{center}

		\question{How is this even a tour?}   

		\answer{It's not, the point is that we first do a DFS here and then find a tour by removing the repeated 
		vertices.}
	\item Now we explore the tree \( T \) using DFS, starting at \( A \). Then, the DFS visits all the vertices
		then comes back to \( A \) at the end. Because it takes every edge in the MST twice, then we know that 
		\[
		\cost( \text{DFS}) = 2\cdot \cost(T)
		\] 
	\item Finally, we now skip over all the repeated vertices in the traversal: suppose we visited node \( D \), 
		and we explore \( E \), and DFS gives the path \( E \to D \to F \), then we're wasting time by revisitng 
		\( D \), and it's much more optimal to just go from \( E \to F \) by the triangle inequality.

		Then, we have the inequality:
		\[
		\cost(\text{TSP}) \le  \cost(\text{DFS}) \implies \cost(\text{output}) \le  2 \cdot \text{OPT}
		\] 
		\question{Can we do better than 2?}

		\answer{We can do \( \frac{3}{2}  \) using linear programming, and as of last year we now have a 
		\( \left( \frac{3}{2} - \epsilon \right) \)-approximation (but \( \epsilon \) is very small}
\end{itemize}

	\section{Sampling and Streaming}
\subsection{Sampling}
\begin{itemize}
	\item Suppose you have a question you want to ask like ``do you approve of the U.S. Congress?'' that 
		admits a yes (1) or no (0) answer. Our goal is to estimate the fraction of population who say 
		yes.
	\item The naive approach would be to just ask everybody what their opinions are: and we can be certain of 
		our result. 
	\item Instead of doing this, we take a sample population of \( k \) people chosen at random, and collect 
		their answers \( x_!, \dots, x_k \in \{0, 1\}  \), and output the fraction of those that say 
		yes. 
		\[
			\hat p = \frac{1}{k}\sum_{i = 0}^{k}x_i
		\] 
	\item Our goal is to pick \( k \) such that with probability of \( 1 - \delta \), then
		\[
		|\hat{p} - p| \le  \epsilon
		\] 
		where \( p \) denotes the \textbf{true value} of \( \hat{p} \). 
	\item The larger the value of \( k \), the smaller our \( \epsilon \) can be. 
	\item \textbf{Chernoff Bound:} Suppose \( x_1, \dots, x_n \in \{0, 1\}  \) are independent and 
		identically distributed random variables where \( P(x_i = 1) = p\) and \( P(x_{i} = 0) = 1-p \), then 
		\[
			P\left( \left| \frac{1}{k}\sum_{i = 1}^{k} x_i  - p \right| \ge  \epsilon \right) \le 2e^{-2 
			\epsilon^2 k}
		\] 
		\question{How does this relate to the central limit theorem?}
	\item So in order to get \( P(\text{error} \ge  \epsilon) \le  \delta \), then we choose: 
		\[
		k = \left\lceil \frac{1}{2\epsilon^2}\ln\left(\frac{1}{2\delta} \right)\right\rceil 
		\] 
		\comment{There's a mistake in the notes where it says \( \log_2 \), but it should be \( \ln \) instead.} 
\end{itemize}
\subsection{Streaming}
\begin{itemize}
	\item Suppose there's a stream and we're watching fish go by, and we want to know some things about the 
		fish in the stream:
		\begin{itemize}
			\item How many fish went by?
			\item What fraction of fish were red?
			\item How many fish species are there?
		\end{itemize}
	\item The issue is, there are too many fish in the stream to record them all (basically no access to memory).
		Further, we can never rewind the clock: once a fish goes by, it doesn't go by anymore. 
	\item Streaming algorithms are actually used in the real world! (e.g. tracking packets over the internet)
\end{itemize}
\subsection{Sampling from a Stream}
\begin{itemize}
	\item Given a stream \( s_1, \dots, s_i \) (we don't know how many elements there are beforehand), and we want
		to output a uniformly random element from the stream.
	\item One way to do this is to record all elements and output a random \( s_i \). However, this takes 
		up a lot of space and isn't preferred.
	\item If \( L \) (length of stream) is known, then we can pick a random index \( i = \{1, \dots, L\}  \) 
		and output a random \( s_i \). 
	\item Neither of these two models work for our problem, because we don't know \( L \) and the first one 
		is just downright slow.
\end{itemize}
\subsubsection{Reservoir Sampling}
\begin{itemize}
	\item We start with \( r = s_1 \). 
	\item Then, for each new stream element, we flip a biased coin with probability \( p_i = 1/i\) of heads and 
		tails with \( 1 - p_i \). If heads, replace \( r = s_i \). Otherwise, leave \( r \) as is.

		Some intuition on why we might expect \( 1 / i \) : at every step \( i \), we have to handle the case
		 where the stream stops there, in which case we want \( s_i \) to be selected with probability 
		 \( 1 / i \).
	\item When the stream stops, we output \( r \).
	\item This probability is actually uniform! In order to output \( s_i \), then the coin would have 
		flipped heads on element  \( i \), then tails at every \( i \) after that. This happens with probability:
		\[
		P = \frac{1}{i}\left( 1 - \frac{1}{i+1} \right)\left( 1 - \frac{1}{i+2} \right) \dots 
		\left( 1 - \frac{1}{L} \right) = \frac{1}{i} \cdot \frac{i}{i+1} \dots \frac{L-1}{L} = \frac{1}{L}
		\] 
		\comment{Note that the flips before \( i \) don't matter, 
		since if flip \( i \) is heads then \( r \) is replaced anyway.}   
\end{itemize}
\subsubsection{Distinct Elements}
\begin{itemize}
	\item Again, given a stream \( s_i \), we want to estimate the number of distinct elements in the stream. 
	\item To solve this, we pick a random hash function \( h: \{1, \dots, N\}  \to [0, 1] \). As the stream 
		goes by, we compute \( h(s_i) \) for each element. We only keep one value, that value being the minimum
		of \( h(s_i) \), and call it \( \alpha \). Then, when the stream ends, we output \( 1 / \alpha \). 
		
		Intuition for why \( 1 / \alpha \) should be returned: suppose there's \( k \) elements in the stream. 
		As our algorithm goes, then we're only going to see the hash values for those \( k \) elements 
		that have been seen. Call these values \( r_1, \dots, r_k \). Since the hash function is random, we 
		know that they should be (in expectation) evenly distributed on \( [0, 1] \). Therefore, the distance
		between the hash values is  \( \frac{1}{k+1} \) and since \( \alpha \) is the minimum value then 
		\( \alpha = \frac{1}{k+1} \). Therefore, \( \frac{1}{\alpha} = k+1 \), which works as an estimation. 

		\comment{You can subtract 1 off the result, but because we're just estimating it really doesn't matter.} 

		\question{How does the hash function know from \( 1 \) to \( N \)?}  

		\answer{\( N \) is given to you beforehand.}
	\item There are a couple issues with this, mainly that computers can't store arbitrary real numbers, and 
		the hash function takes a lot of memory to store. One solution (preview for next time) is to use a 
		\textit{pseudorandom hash function}.
		\begin{itemize}
			\item One way to circumvent the first problem is to have a hash function  \( h \) map from 
				\( \{1, \dots, R\}  \) where \( R  \) is some very large number. This way,  \( h(i)/R \) is 
				``random enough''.
			\item For the second issue, we make \( h \) a pseudorandom hash function. This takes less space, 
				while still guaranteeing that we get the randomness we want.
		\end{itemize}
\end{itemize}

	\section{Streaming II}
\begin{itemize}
	\item Recall the way we computed the number of distinct elements 
		using a random hash function \( h: \{1, \dots, N\} \to [0, 1] \). 
	\item There were problems with this approach! There were two:
		\begin{itemize}
			\item Computers can't store arbitrarily real numbers

				\textit{Solution:} Pick a hash function that maps \( h: \{1, \dots, R\}  \) instead of 
				\( \{1, \dots, N\}  \) so that \( h(i) / R \) is basically a random number between 0 and 1. This 
				is fairly easy to implement.
			\item If the hash function is uniformly random, then we would require \( N \log R \) bits to 
				store this -- this takes up a lot of memory, especially as \( R \) grows large! 

				\textit{Solution:} Make the hash function \( h \) ``pseudorandom''. We will define 
				a \textit{hash family} \( \mathcal H = \{h_1, \dots, h_m\}  \), and we will choose from this 
				family for our hash function. We will use \( h \sim \mathcal H \) to denote choosing a random 
				\( h_i \). This approach only requires \( \log m \) bits to store, so this is much better
				than our earlier implementation. 

				\comment{Each of \( h_i \) are basically constant functions, we've changed the randomness
				of the hash function to \textit{choosing} the hash function itself.}

				\question{So are the \( h_i \) just random bits, then how are we guaranteeing that we can still 
				get the number of distinct elements?}

				\answer{Just in the same way the hash function expects a spacing of \( \frac{1}{k+1} \), 
				we expect that the randomly generated hash family is also \( \frac{1}{k+1} \).} 
		\end{itemize}
\end{itemize}
\subsection{Pairwise Independence}
\begin{itemize}
	\item This is the way we're going to define randomness.
	\item \textbf{Definition:} A hash family \( \mathcal H = \{h_1, \dots, h_m : \{1, \dots, N\} \to 
		\{1, \dots, R\}	\} \) is \textit{pairwise independent} if:
		\begin{itemize}
			\item For all \( x \neq y \in \{1, \dots, N\}  \) and \( i, j \in \{1, \dots, R\}  \), then 
				\[
					\underset{h\sim \mathcal H}{\mathrm{Pr}}\left[ h(x) = i \text{ and } h(y) = j \right]  
					= \frac{1}{R^2}
				\] 
				where Pr denotes the probability. In other words, it means that our \( i, j \) look like 
				two independent draws from \( \{1, \dots, R\}  \). 
		\end{itemize}
	\item If this is true, it also implies that \( \underset{h \sim \mathcal H}{\mathrm{Pr}}[h(x) = i] =\frac{1}{R} \) for all \( x \) and \( i \).
\end{itemize}
\subsubsection{Generating Pairwise Independence}
\begin{itemize}
	\item Our approach will utilize modular arithmetic, and to make things easy we will do modular arithmetic 
		using a prime \( p \). 
	\item Then, for each \( a, b \in \mathbb{Z}_p = \{0, 1, \dots, p-1\}\), we will define \( h_{a, b}: 
		\mathbb Z_p \to 
		\mathbb Z_p\) such that \( h_{a, b}(x) = ax + b \pmod p\). 
	\item This makes \( \mathcal H = \{h_{a, b}\} _{a, b \in \mathbb Z_p} \) is pairwise independent. Note 
		also that there are \( p^2 \) hash functions, so \( \mathcal H \) takes up \( O(p) \) space.   

		\comment{This is very close to what you'd do with the streaming algorithm, but not exactly.}

		\textit{Proof of Independence (simplified):} Let  \( x, y, i, j \in \mathbb Z_p \) 
		such that \( x \neq y \). We want to show that 
		\[
			\underset{a, b}{\mathrm{Pr}}[ax + b = i \text{ and } ay + b = j] = \frac{1}{p^2}
		\] 
		We will do it for \( x = 0, y = 1 \) and the general case is left as an 
		exercise. In this case, we'd want to prove:
		\[
			\underset{a, b}{\mathrm{Pr}}[b = i \text{ and } a+ b = j] = \frac{1}{p^2}
		\] 
		But given this construction, we know that \( b \) and \( a + b \) is a random pair in \( \mathbb Z_p^2 \),
		so the probability is indeed \( \frac{1}{p^2} \). Note that this is not 3-wise independent: given 
		\( x =0, y = 1, z = 2 \), then \( h(z) = 2a + b = 2(a + b) - b = 2h(1) - h(0) \). Since there's 
		a way to construct \( h(z)  \), this is not independent.
\end{itemize}
\subsection{Modified Distinct Elements}
\begin{itemize}
	\item Instead of throwing our value into a hash function, we will pick a pairwise independent hash 
		function \( h: \{1, \dots, N\}  \to [0, 1] \). 
	\item We compute the \( t \)-th smallest value of \( \{h(s_1), \dots, h(s_L)\}  \), which would be 
		the \( t \)-th smallest value of \( \{r_1, \dots, r_k\}  \). 
	\item Then, we output \( t / \alpha \).
	\item The reason we want to use the \( t \)-th smallest is because want our algorithm to be tolerant to 
		outliers. 

		\question{Why not just use the \( N / 2 \)-th smallest? Shouldn't this be the most tolerant value 
		to outliers?}

		\answer{It is, you can repeat the same analysis for exceedingly large values.} 
\end{itemize}
\subsubsection{Analysis}
\begin{itemize}
	\item We have the property that (based on our algorithm):
		\[
		\mathrm{Pr}[\text{alg outputs} \ge  2k] = \mathrm{Pr}\left[\alpha \le  \frac{t}{2k}\right] = 
		\mathrm{Pr}\left[ \sum_i C(i) \ge  t \right] 
		\] 

		\question{How do we go from the 2nd to the 3rd step?}

		\answer{\( C(i) \) is defined to be the number of outliers, so the probability that we output 
		something larger than \( 2k \) is if there are more than \( t \) outliers.} 

		Here, we define \( C(i)  \) as follows:
		\[
		C(i) = \begin{cases}
			1 & r_i \le  \frac{t}{2k}\\
			0 & \text{otherwise}
		\end{cases}
		\] 
		In other words, \( C(i) \) is an indicator that counts the number of outliers. Then, we have:
		\[
			E\left[ \sum_{i} C(i) \right]  = \sum_{i} E[C(i)] = \sum_i \mathrm{Pr}\left[ r_i\le \frac{t}{2k} \right] = \sum_i \frac{t}{2k} = \frac{t}{2}
		\] 
		Where the second step is reached via linearity of expectation. This means that the expected 
		number of outliers is \( \frac{t}{2} \).

		\question{Is this a good value?}
	\item Now we compute the variance to get a good idea of the spread of \( \sum_i C(i) \):
		\[
			\Var\left[\sum_i C(i)\right] = \sum_i \Var[C(i)] 
		\] 
		Since each \( C(i) \) is associated with an \( r_i \) which is derived from \( h(s_i) \), each 
		\( C(i) \) is also pairwise independent. This allows us to take the summation out of the variance. Then:
		\[
			\Var[C(i)] \le  E[C(i)^2] = E[C(i)] = \frac{t}{2k} \implies \Var\left[ \sum_i C(i) \right] \le k 
			\cdot \frac{t}{2k} = \frac{t}{2}
		\] 
\end{itemize}

	\section{Randomized Algorithms} 
\begin{itemize}
	\item Described as an algorithm which uses random bits to solve problems. Generally, this is done 
		using some function called \( \textsc{RandomInt}(a, b) \) which returns a random integer from 
		the interval \( (a, b) \).
	\item Generally, algorithms will output a \textit{pseudorandom} number (random enough for our purposes) 
		But when we model randomness mathematically, we refer to a genuinely random number.  
	\item Here, algorithms will be allowed to fail with some probability (say 5\%).
	\item Oftentimes, randomized algorithms lead to much more elegant solutions when compared to deterministic 
		algorithms. They can also sometimes be much faster than deterministic ones!
\end{itemize}
\subsection{Background: Integer Factorization}
\begin{itemize}
	\item This is not a problem we know how to solve using a randomized algorithm. Given a number \( N \), 
		we want to find its prime factorization \( N = p_1p_2\cdots p_k \). 
	\item Naively, we could just test every single number from 1 to \( \sqrt{N}  \), but this takes 
		\( O(\sqrt{N} ) \) time (if \( N \sim 10^{500} \), then our algorithm would take longer than the number 
		of atoms in the universe to compute.)
	\item The best algorithm we know today is called a \textit{General Number Field Sieve}, which factors 
		an \( n \) bit number in time \( C^{n^{1 / 3}\log(n)^{2 / 3}} \). (This \( C \) is not very large, but 
		still not small.) 
	\item This is a very important problem! RSA (the company) literally puts out numbers on the internet with 
		cash prizes attached to them if you can factor them. RSA-250 is a 250-digit number, which was 
		recently factored in 2020. RSA-290 has a \$75,000 cash prize attached to it.
	\item \textit{Aside:} this is a problem that is easily solved by quantum computers.  
\end{itemize}
\subsection{Primality Testing}
\begin{itemize}
	\item Given a number \( N \), our only goal is to figure out whether \( N \) is prime or composite. 
	\item This is very similar to the prime factorization problem! We \textit{could} take \( N \) and factor it, 
		but this is obviously very hard. But we can solve this relatively efficiently using randomness!
	\item We use another property of prime numbers: Fermat's little theorem! It says that if \( N \) is prime, 
		then 
		\[
			a^{N-1} \equiv 1 \pmod{N}
		\] 
		for all \( a \in \{1, 2, \dots, N-1\}  \).
	\item So we define a test called \( \textsc{FermatTest}(N) \). It picks a value \( a \in \{1, 2, \dots, 
		N - 1\}  \) uniformly at random. If \( a^{N} \equiv 1 \pmod N \), then we output "prime", otherwise 
		we output "composite". 
		\begin{itemize}
			\item This algorithm will always output "prime" for prime \( N \). For values that are coprime with 
				\( N \), it's harder to see whether they pass \( \textsc{FermatTest} \), and ones that aren't 
				are certainly going to fail \( \textsc{FermatTest} \).
			\item Becuase this test relies pretty heavily on the fact that \( a \) is coprime, our test is only 
				as good as the number of coprime \( a \) that we get. 
			\item \textit{Aside:} There do in fact exist are \textit{composite} numbers that satisfy
				\( a^{N-1} \equiv 1 \pmod N \) for all \( a \) coprime
		to \( N \)! In other words, these numbers will always pass \textsc{FermatTest}. These are called 
		\textit{Carmichael numbers}, but for the purposes of this class, we will pretend like these don't exist.
		\end{itemize}
	\item \textbf{Theorem:} Suppose \( N \) is composite and not carmichael. Then, we have 
		\( P(\textsc{FermatTest}(N) = \text{composite}) \ge 1 / 2 \). 
		
		\textit{Proof:} If the number is not Carmichael, this implies the existence of some coprime \( b \)
		such that \( b^{N - 1} \not \equiv 1 \pmod N \), and thus there is a single \( b \) that fails
		\textsc{FermatTest}. 
		
		\textit{Subclaim:} Suppose some value \( a \) passes \textsc{FermatTest}. Then, \( ab \pmod N\)  will fail
		\textsc{FermatTest}. 

		\textit{Proof:} This is because \( ab \) is no longer coprime to \( N \), so this gives us a 
		1-1 correspondence between a value that passes \textsc{FermatTest} and one that doesn't. More 
		rigorously, we have:
		\begin{align*}
			(ab)^{N - 1} & \equiv a^{N - 1} \cdot b^{N - 1}\pmod N\\
						 &\equiv b^{N - 1} \pmod N\\
						 &\not \equiv 1 \pmod N
		\end{align*}
		where we use the fact that \( a^{N - 1} \equiv 1 \pmod N \). Then, the one-to-one correspondence follows.
		Then, since there are values that \textit{just fail} \textsc{FermatTest}, then this means that 
		there are more values that fail than pass. Hence, \( P(\text{fail}) \ge  1 / 2 \), as desired.   
	\item Now, let's repeat this algorithm \( k \) times. On any given test, we know that if \textsc{FermatTest}
		returns "composite" then \( N \) is definitely composite, so the only case where we get a fail is if 
		\( N \) is composite but \textsc{FermatTest} outputs "prime". This occurs with probability 
		\[
		P(\text{\( N \) composite but outputs prime}) = 1 - \frac{1}{2^{k}}
		\] 
		This means that when our test outputs "prime", there is a \( 1 - \frac{1}{2^{k}} \) chance of it being 
		a false positive. 
	\item We can add another check to detect Carmichael numbers -- we won't really delve into this here. This 
		gives the Miller-Rabin primality test, developed in 1976. 
	\item Since then, there have been improvements: the AKS Primality test (2002) gave a deterministic 
		polynomial time \( O(n^{12}) \) algorithm for primality testing.   
\end{itemize}
\subsection{Randomized Complexity Classes}
\begin{itemize}
	\item Some problems have both efficient randomized and deterministic algorithms, but there are others 
		that only have efficient randomized algorithms. (e.g. polynomial identity testing) 
	\item This implies the existence of two possible worlds:
		\begin{enumerate}[label=\arabic*.]
			\item Every efficient randomized algorithm has a corresponding deterministic solution. (P)
			\item Some problems only have efficient randomized algorithms. (BPP)
		\end{enumerate}
	\item The question of which world we live in is called the P vs. BPP problem. P is the class of efficiently 
		deterministic problems (same P as before), and BPP is the class of efficient, randomized algorithms. We 
		think that these complexity classes are equal. 
\end{itemize}
\subsection{Minimum Cut} 
\begin{itemize}
	\item Consider a (unweighted, undirected) graph \( G = (V, E) \), and we want to find the minimum cut of this graph. Here, 
		the heuristic we're using to quantify the size of the cut is the number of edges it crosses through.
	\item Recall from our max-flow problem, we found the minimum \( s-t \) cut via our flow algorithm. It turns out that 
		 we can use the flow algorithm to solve this deterministically via a reduction, but today we'll talk about another 
		 algorithm: Karger's algorithm. 
	 \item For every \( i = 1, \dots, n- 1 \), it will:
		 \begin{enumerate}[label=\arabic*)]
		 	\item Pick a uniformly random edge \( e \), and "contract" the edge.
			\item It will return the cut specified by the remaining two supervertices.  
		 \end{enumerate}
	 \item The way to contract an edge is to combine the two vertices connected by \( e \), and combine them into one supervertex.
		 It will preserve all the possible edges, which means that the graph doesn't have to remain simple in this process.  
		 Then, once only two vertices are remaining, then that's the cut that we output. Notice that because we don't delete any
		 edges in this process, then the size of the "supercut" is equal to the size of the cut in the original graph. 
	 \item This won't return the minimum cut, but it will return the minimum cut with fairly good probability.  
\end{itemize}
\subsubsection{Intuition}
\begin{itemize}
	\item So suppose we have a graph with a bunch of edges, then with one singular edge outside, like this:
		% add a figure here
	\item If Karger's algorithm is to find the minimum cut, then it should never contract edge \( e \). Basically, it leverages
		 the fact that becuase there are more edges outside of the minimum cut, then the algorithm shouldn't select the edges 
		 along the minimum cut. 
	 \item \textbf{Theorem:} Let  \( C = (S, \overline S) \) be a minimum cut of size \( k \). Then, the probability that 
		 Karger's algorithm outputs the cut \( C \) is given by:
		 \[
			 P(\text{Karger's outputs \( C \)}) \ge \frac{1}{{n \choose 2}} = \frac{2}{n(n-1)}
		 \] 
		 While this is indeed bad for a single iteration, it just means that in expectation we just need to repeat this 
		 \( n^2 \) times to get the minimum cut!
	 \item \textit{Proof:} First, we state four facts:
		 \begin{itemize}
		 	\item Let \( G_i \) be the state of the graph \( G \) at the \textit{start of the} \( i \)-th iteration. Then, because 
		 \( C \) is the minimum cut with size \( k \), then we know that the minimum cut of \( G_i \ge k \). 
			\item Further, the number of vertices in \( G_i  \) is \( n - (i - 1) = n-i+1\). 
			\item The degree of each vertex in \( G_i \ge  k \), because of the fact that the minimum cut is \( k \). Had 
				the degree been less than \( k \), then the minimum cut would be less than \( k \), since the min-cut 
				would just be to cut that vertex with degree less than \( k \). 
			\item The number of edges in \( G \) can be calculated as:
				\begin{align*}
					N &= \frac{1}{2}\sum_{v \in G_i} \deg(v)\\
					  &\ge \frac{1}{2}\sum_{v \in G_i} k\\
					  &= \frac{1}{2}k |G_i| \\
					  &= \frac{1}{2}k(n-i+1) 
				\end{align*}
		 \end{itemize}
		 Now we complete the proof. Essentially, we want to compute the probability that the algorithm never contracts a cut 
		 in \( C \). Suppose at step \( G_i \), we haven't contracted any edges in \( C \). Then, the probability 
		 that we don't contract an edge in \( C \) is:
		 \[
		 	P(\text{don't contract an edge in \( C \)} = 1-P(\text{contract an edge in \( C \)}) 
			= 1 - \frac{k}{\frac{1}{2}k (n - i + 1)}
		\]
		Thus, the probability that we never contract an edge in \( C \) is the probability that this event occurs on every 
		iteration \( i \), at the step \( G_{i - k} \) :
		\[
		P(\text{never contract edge in  \( C \)}) \ge  \left( 1 - \frac{2}{n} \right) \left( 1 - \frac{2}{n-1} \right) \dots 
		= \frac{2}{n(n-1)}
		\] 
		So therefore, the probability that we succeed is lower bounded by \( \frac{2}{n^2} \).
\end{itemize}
	\section{Quantum Algorithms}
\subsection{Brief History}
\begin{itemize}
	\item In the world of physics, there was a problem: we couldn't simulate quantum systems. What does that even mean?
	\item Suppose we have \( n \) electrons, each with spin \( \ket{\uparrow} \) or \( \ket{\downarrow} \). 
		This gives us a total of \( 2^{n} \) possible configurations. The fastest algorithm known at the time 
		was one that took \( O(2^{n}) \) time.
	\item Richard Feynman suggested in a 1981 talk that we could potentially build a computer out of 
		electrons, and instead of computing the quantum mechanics itself, we just let the electrons 
		simulate themselves.
	\item How powerful are quantum computers actually? Very powerful! A small list of algorithms that 
		are very powerful:
		\begin{itemize}
			\item Deutsch-Jozsa algorithm: solves the Deutsch-Josza problem
			\item Bernstein-Vazirani algorithm
			\item Simon's algorithm
			\item Shor's algorithm
			\item Grover's algorithm
		\end{itemize}
\end{itemize}
\subsection{Deutsch-Josza Problem}
\begin{itemize}
	\item Given an input function \( f: \{0, 1\} ^{n} \to [0, 1] \), the function is one of the following:
		\begin{enumerate}[label=\roman*)]
			\item \( f(x) = 0 \) for all \( x \) 
			\item \( f(x) = 1 \) for all \( x \) 
			\item \( f(x) = 0 \) for half of \( x \), and \( f(x) = 1 \) for the other half. 
		\end{enumerate}
	\item We proved that classical deterministic algorithms have to check at least \( 2^{n-1} +1 \) values 
		for \( f(x) \) to determine. This means that the algorithm runs in exponential time. However, 
		the quantum algorithm can do this process in \( O(n) \) time!
	\item There is also a very simple randomized algorithm for this: we can pick a subset of \( x \) at random,
		and this allows us to determine the identity of \( f(x) \) also fairly quickly.
	\item So this problem tells us the power of quantum computers over deterministic algorithms, but doesn't 
		show the power of quantum algorithms over random algorithms. 
\end{itemize}
\subsection{Bernstein-Vazirani Problem}
\begin{itemize}
	\item Given an input function \( f: \{0, 1\} ^{n} \to \{0, 1\}  \) that takes a subset of \( x \) and 
		adds the values up modulo 2, we want to figure out which bits were added in the sum.  
	\item Classically, we can look at all the possible configurations for \( f \), which would require 
		\( n+1 \) looks at \( f \). However, the quantum algorithm only needs to look at \( f \) once!
\end{itemize}
\subsection{Shor's Algorithm}
\begin{itemize}
	\item We know that if we have an \( n \)-digit number, the best classical algorithm factorizes a number in 
		\( O(e^{1.9 n^{1 / 3} \cdot \log(n)^{2 / 3}}) \), but Shor showed that we can factor numbers 
		in \( O(n^2) \) time on a quantum computer. 
	\item This is significant, because RSA encryption relies factoring large numbers and this algorithm basically 
		gives us a way to break that encryption scheme. 
\end{itemize}
\subsection{Grover's Algorithm}
\begin{itemize}
	\item This algorithm solves circuit-SAT in \( O(\sqrt{2} ^{n}) \) time on a quantum computer, wherea s
		a classical algorithm takes \( O(2^{n}) \) time. This also solves many other problems (just due to their 
		relationship to circuit-SAT).
\end{itemize}
\subsection{Speedups}
\begin{itemize}
	\item There are three main types of speedups:
		\begin{itemize}
			\item Shor-Type: these are exponentially faster than classical algorithms, but only work on 
				a very restricted subset of problems.
			\item Grover-Type: This speeds up things polynomially, but with the benefit that they work on a large
				number of problems. 
			\item Physics Simulation: Exponentially faster, but can only work for physics problems. 
		\end{itemize}
	\item One of the main reasons why quantum computers outperform classical ones is mainly because they can 
		compute very fast Fourier transforms, which are used everywhere in computation.  
	\item However, quantum computers are not able to solve NP-complete problems efficiently. 
\end{itemize}

\end{document}
