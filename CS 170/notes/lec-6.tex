\section{Graphs}
	Formal definition of undirected graphs:
	\begin{itemize}
		\item Called $G(V, E)$.
		\item has a set $V$ of vertices, and $E$ of edges.
		\item Edges are denoted as a pair of vertices $\{v_i, v_j\}$, and undirected mean that each 
			pair of $\{v_i, v_j\}$ is unique.
			\begin{itemize}
				\item e.g. Facebook friendship graphs, where the friendship must be mutual. (Has nearly 3 billion
					nodes, and each node has 330 vertices)
			\end{itemize}
		\item Directed graph is denoted in the same way, except that $\{v_i, v_j\}$ means 
			specifically that $v_i \to v_j$.
	\end{itemize}

	\subsection{Size of Graphs}
	\begin{itemize}
		\item Generally denoted as the number of vertices (denoted by $n$) and the number of edges 
			(denoted by $m$).
		\item $m$ can be at most $n^2$, in the case that every vertex is connected to every other vertex 
			(in which case we have $n(n-1)$ edges)
		\item Degree: the number of edges that a vertex has. With directed graphs, we can specify in-degrees
			and out-degrees.
	\end{itemize}

	\subsection{Representing Graphs}
	\begin{itemize}
		\item Represented either as an adjacency matrix or an adjacency list.  
			\begin{itemize}
				\item Adjacency matrix: have the vertices listed out on both row and column, put a 1 if 
					$\{v_i, v_j\}$ are connected on our graph.
				\item Adjacency list: List of length $n$, and each cell is a linked list that points 
					to all the neighbours of $v_i$.
			\end{itemize}
		\item There are tradeoffs for both ways:
			\begin{center}
				\begin{tabular}{c|c|c}
			& 	\textbf{Adjacency Matrix} & \textbf{Adjacency List}\\
			\hline
					Storage Size  & $O(n^2)$  & $O(n + m)$\\
				Checking whether $(u, v) \in E$ & $O(1)$ & $O(\deg(u))$\\
				Enumerate all of $u$'s neighbours &  $O(n)$ & $O(\deg(u))$
				\end{tabular}
			\end{center} 
		\item We will work with adjacency lists, since the storage size for adjacency matrices get
			too large too quickly. 
	\end{itemize}
	\subsection{Graph Connectedness}
	\begin{itemize}
		\item We first have to solve the problem of graph traversal. Our approach will use ``string and chalk,''
			so we mark when we've reached a dead end and the ``string'' will allow us to backtrack. 
		\item Algorithm description:

\begin{lstlisting}[style=tt]
def explore():
	visited[u] = true

	For all edges of $u$: if visited[v] = false then run explore(v)
\end{lstlisting}
		\item Explore guarantees that all the vertices that are visited by explore have a path from $u$ to 
			that vertex, and vice versa.
			\begin{itemize}
				\item We prove the other direction: if there's a path, then that node is visited.

					Assume that this is false: assume $\exists v$ that hasn't been explored but there is 
					a path from $u$ to $v$. Instead of looking at $u$ to $v$, we look at the path 
					from $u$ to $v_k$, the first node along the path from $u$ to $v$ that is unexplored. 
					This means that the algorithm reached $v_{k - 1}$, then failed to explore $v$.

					This is a contradiction, since explore must have been called on $v_{k-1}$, but failed to 
					recurse down $v_k$. 
			\end{itemize}
	\end{itemize}

	\subsection{Depth First Search (DFS)}
	Essentially calling explore, except it does it recursively on all the remaining nodes that haven't 
	been visited yet.
	\begin{itemize}
		\item We can also use DFS to find the connected components of a graph! When we explore with DFS, we are 
			implicitly exploring connected components, this uses the transitive fact of the \texttt{explore()}
			function.
		\item The edges that are visited by DFS are categorized into tree edges and back edges. Tree 
			edges are edges that are used when the algorithm runs, and back edges are ones that exist within 
			the original graph but aren't used. 
		\item There's a third class called a \textit{cross edges}, but they cannot exist for an undirected graph.
			\begin{itemize}
				\item The proof is by contradiction: imagine that it did exist, then DFS would have called
					explore on that ancestor; but it can't possibly have since $u$ and $v$ do not 
					exist in the same branch.
				\item They \textit{can} exist in a directed graph, since we only traverse through out 
					edges (so there can be an in-edge that never gets traversed).
			\end{itemize}
	\end{itemize}	

	\subsubsection{DFS Runtime}
	\texttt{Explore()} is called only once per node, and the runtime for each node for explore is 
	proportional to $\deg(u)$, so the total is:
	\[
		\sum_{u \in V} O(1 + \deg(u)) = O(n + m)
	\] 

	\subsection{DFS for Directed Graphs}
	It's the same principle, our explore still only runs recursively on unvisited neighbours, but we need 
	to also keep track of the amount of time at which we start and finish processing that node.
	\begin{itemize}
		\item Every time we enter a node, we stamp it with the time of the start clock. When we come 
			out, we stamp again with the end clock. Every time we progress the clock, we increment it by 1.
		\item The point of the clock will be revealed later on in future lectures. 
		\item The edges we keep track of here are forward, back and cross edges. Forward means we go 
			down the tree from ancestor to descendant (not its immediate child), back means we go backwards 
			(can be immediate) and cross edges are defined exactly as before.
	\end{itemize}
	The edges (pre, post, cross) are useful in determining properties of the graph $G$. 
	\begin{itemize}
		\item Cross edges can only go from a later point in one branch to earlier than the other branch
			and not the other way around, due to the way that DFS executes depth-first.
		\item Suppose $(u, v) \in E$ is a tree edge. Then, we know that $\pre(u) < \pre(v)$ (since $u$ is 
			hit first), and $\post(u) < \post(v)$. 
		\item Suppose that $(u,v)\in E$ is a back edge. Then, we have $\pre(v) < \pre(u) < \post(u) < \post(v)$.
	\end{itemize}