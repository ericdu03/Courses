\section{Network Flow}
\begin{itemize}
	\item Recently declassified (1999) document about the USSR's shipment capacity from east to west. This 
		was crucial information at the time since had a war broke out, the US could identify which 
		supply routes they could bomb.
	\item They devised a greedy algorithm called ``flooding,'' but this algorithm wasn't really optimal. It 
		was finally solved by Ford and Fulkersson, and is now called the Ford-Fulkersson algorithm.
	\item Given a directed graph $G = (V, E)$, one source vertex $s$ and a sink $t$, and for each edge $e \in E$,
		we're given a capacity $c_e$ which are integers. 
	\item We want to find the maximum amount of water from $s \to t$.  
	\item \textit{Definition:} A flow assigns a number $f_e$ to each directed edge $e \in E$ such that:
		\begin{itemize}
			\item nonnegativity: \(f_e \ge  0\) 
			\item capacity: \(f_e \le c_e\)
			\item flow in and flow out are equal: \(\sum_{u \to v} f_{u, v} = \sum_{v \to w} f_{v, w} \)
		\end{itemize}
	\item Let's also define the size of the flow $f$ to be the total quantity set from $s$ to $t$. Using 
		this definition, then the maximum flow is the one that maximizes $\size(f)$. This can be solved 
		using linear programming!
\end{itemize}

\subsection{Greedy (suboptimal) algorithm}
\begin{itemize}
	\item We'll find a path $P$ from $s$ to $t$, and send flow until it's saturated. We'll do 
		this as much as we can. We repeat this until we run out of paths.
	\item This algorithm fails on some graphs, because it uses edge $A \to B$ when that edge is suboptimal! 
		Consider the graph:
		\begin{center}
			\begin{tikzpicture}[scale=2]
				\node (s) at (0, 0) {$s$};
				\node (a) at (1, 0.7) {$A$};
				\node (b) at (1, -0.7) {$B$};
				\node (t) at (2, 0) {$t$};
				\graph[edge label = 1]{(s) ->(a) -> (t), (s) -> (b) -> (t), (a) -> (b)};
			\end{tikzpicture}
		\end{center}
		Our algorithm just looks at flow rate, so a possible path to take is $s \to A \to B \to t$, but this 
		is clearly suboptimal! Instead, we should be going from $s \to A \to t$ and $s \to B \to t$. 
\end{itemize}

\subsection{Greedy Fix}
\begin{itemize}
	\item We instead consider a residual graph, where we subtract the flow given by greedy 
		($s \to A \to B \to t$), and also generate a back edge that travels in the reverse order of the flow 
		given by greedy, so that we can backtrack if needed. 
	\item Formally, given a graph $G$ and a flow $f$ on $G$, the residual raph $G_f$ is defined as:
		For all edges \((u, v)\), if $f$ goes from $u \to v$, then the residual graph will flow from $v \to u$ 
		and the edge will have capacity $c_{u, v} - f_{u, v}$. 

		By doing this, we allow our graph to backtrack along our suboptimal path if needed.  
	\item This is the approach that Ford Fulkerson uses to find the optimal flow.
\end{itemize}

\subsection{Ford-Fulkerson Algorithm}
\begin{itemize}
	\item Find a path $P$ from $s$ to $t$ in the residual graph which is not yet saturated, and send more 
		flow along $P$. We keep repeating this until everything's saturated, and this happens 
		when all edges along one particular cut is zero. 
	\item To show that this algorithm terminates, lets' first define an $s-t$ cut is a partition of the graph
		into two sets of vertices \(L\) and \( R\) such that 
		\(s \in L\) and \(t \in R\). We define the capacity of this cut to be the sum of all capacities from 
		the edges that cross from $L$ to $R$.  
	\item Therefore, for any flow $f$ and any cut \((L, R)\), then \(\size(f) \le \capacity(L, R)\). Then, the
		flow is actually upper bounded by the minimum cut along this graph (this is our ``bottleneck'' introduced
		at the outset)
	\item Then, this means that the max flow is also given by the minimum cut, and we can show that 
		Ford-Fulkerson outputs a maximum flow by considering this relation between the flow and a cut. The proof 
		of this is outlined in lecture. 

		\question{Review the proof for this later}
\end{itemize}

\subsection{Runtime}
\begin{itemize}
	\item The number of augmenting paths must be less than $U$, where $U$ denotes the maximum flow, so 
		the update is less than $O(m + n) \cdot U$. 
	\item But what this means   
	\item There are other algorithms out there that optimizes this a little more: Edmonds-Karp gives 
		us a runtime of $O(n m^2)$, which is much better than what we have. 
	\item The best runtime was discovered last year, where we have $O(m^{1 + o(1)} \cdot \log U)$
\end{itemize}
