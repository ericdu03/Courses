\documentclass[10pt]{article}
\usepackage{../local}


\newcommand{\classcode}{Physics 137B}
\newcommand{\classname}{Quantum Mechanics II}
\renewcommand{\maketitle}{%
\hrule height4pt
\large{Eric Du \hfill \classcode}
\newline
\large{HW 01} \Large{\hfill \classname \hfill} \large{\today}
\hrule height4pt \vskip .7em
\normalsize
}
\linespread{1.1}
\begin{document}
    \maketitle
    \section*{Problem 1}

    \begin{enumerate}[(a)]
        \item Write down the Hamiltonian for two identical noninteracting particles in the infinite square well. Verify that the fermion ground state given in the example is an eigenfunction of $H$, with the appropriate eigenvalue.
        \item Find the next two excited states (beyond the ones given in the example) $-$ wave functions and energies $-$ for each of the three cases (distinguishable, identical bosons, identical fermions)
    \end{enumerate}

    \pagebreak

    \section*{Problem 2}

    Imagine two noninterating particles, each of mass $m$, in the infinite square well. If one is in the state $\psi_n$ (Equation 2.24) and the other state $\psi_m$ is orthogonal to $\psi_n$, calculate $\mean{(x_1-x_2)^2}$, assuming that they are identical bosons.

    \pagebreak

    \section*{Problem 3}

    \begin{enumerate}[(a)]
        \item Suppose you put both electrons in a helium atom into the $n=2$ state; what would the energy of the emitted electron be?
        \item Describe (quantitatively) the spectrum of the helium ion, \ch{He+}
    \end{enumerate}
\end{document}