\documentclass[10pt]{article}
\usepackage{../../local}
\usepackage{mathrsfs}

\newcommand{\classcode}{Physics 137B}
\newcommand{\classname}{Quantum Mechanics II}
\renewcommand{\maketitle}{%
\hrule height4pt
\large{Eric Du \hfill \classcode}
\newline
\large{HW 12} \Large{\hfill \classname \hfill} \large{\today}
\hrule height4pt \vskip .7em
\normalsize
}
\linespread{1.1}
\begin{document}
	\maketitle

	\section*{Collaborators}
	I worked with \textbf{Andrew Binder} to complete this homework.

	\section*{Problem 1} 

	Use the WKB approximation to find the allowed energies $(E_n)$ of an infinite square well with a ``shelf,''
	of height $V_0$, extending half-way across (Figure 7.3):
	\[
	V(x) = \begin{cases}
		V_0, & (0 < x < a/2),\\
		0, & (a/2 < x < a),\\
		\infty, & (\text{otherwise})
	\end{cases}
	\] 
	Express your answer in terms of $V_0$ and $E_n^0 \equiv (n \pi \hbar)^2/2ma^2$ (the $n$th allowed energy 
	for the infinite square well with \textit{no} shelf). Assume that $E_1^0 > V_0$, but do \textit{not} assume
	that $E_n \gg V_0$. Compare your result with what we got in Section 7.1.2, using first-order perturbation
	theory. Note that they are in agreement if either $V_0$ is very small (the perturbation theory regime) or 
	$n$ is very large (the WKB--semi-classical--regime).

	\begin{solution}
		Here we use the WKB approach, which tells us that for an infinite well, we have
		\[
			\int_0^a \sqrt{2m(E - V_0)} dx = n \pi \hbar
		\] 
		The integral on the left needs to be broken up into two integrals, from $0$ to $a/2$ and then from 
		$a/2$ to $a$:
		\begin{align*}
			\int_0^a p(x) dx &= \int_0^{a/2} \sqrt{2m(E - V_0)} dx + \int_{a/2}^a \sqrt{2mE} dx \\
				n \pi \hbar&= \frac{a}{2}\left[ \sqrt{2m(E_n - V_0)} + \sqrt{2mE_n}\right]
		\end{align*}
		Solving this equation for $E_n$ (I got lazy and threw this into mathematica), we get:
		\[
		E_n = \frac{(2 \hbar^2 n^2 \pi^2 + a mV_0)^2}{8a^2 \hbar^2 mn^2 \pi^2} = E_n^0 + \frac{V_0}{2} +
		\frac{V_0^2}{16E_n^0}
		\] 
		From Section 7.1.2, we find that we got the energy levels to be 
		\[
		E_n = E_n^0 + \frac{V_0}{2}
		\] 
		This makes sense since for small perturbations (i.e. $V_0$ small), then the third term in the WKB 
		equation $V_0^2/16 E_n^0$ would be very small, so it makes sense that this term is neglected
		when considering perturbation theory. When $n$ becomes large, we see that $E_n$ grows large, so 
		therefore this last term also goes to zero under large $n$. Therefore, we've confirmed that these
		two results agree for the two regimes we were asked to check.
	\end{solution}
	\pagebreak

	\section*{Problem 2}
	use Equation 9.23 to calculate the approximate transmission probability for a particle of energy $E$ that 
	encounters a finite square barrier of height $V_0 > E$ and width $2a$. Compare your answer with the exact 
	result (Problem 2.33), to which it should reduce in the WKB regime $T \ll 1$. 


	\begin{solution}
		Let the left side of the barrier be at $x = 0$ and the right side be at $x = 2a$, for simplicity's
		sake with the integral. We know that for tunnelling, we have that 
		\[
			T \approx e^{-2\gamma}
		\] 
		where $\gamma$ is defined as
		\[
		\gamma = \int_{x_1}^{x_2} \sqrt{\frac{2m}{\hbar^2}(V(x) - E)} dx
		\] 
		So in our particular case, we can get:
		\[
			\gamma = \int_0^{2a} \sqrt{\frac{2m}{\hbar^2}(V_0 - E)} dx = 2a \sqrt{\frac{2m}{\hbar^2}(V_0 - E)} = 
			\frac{2a}{\hbar}\sqrt{2m(V_0 - E)} 
		\] 
		Therefore, our transmission coefficient is:
		\[
			T \approx \exp{-\frac{4a}{\hbar}\sqrt{2m (V_0 - E)} }
		\] 
		From Problem 2.33, we get the exact transmission as:
		\[
		 T = \frac{1}{1 + \frac{V_0^2}{4E(V_0 - E)}\sinh^2 \gamma}
		\] 
		In the case where $T$ is small, then from our WKB approximation we can see that $\gamma$ must be large
		in order for $T$ to be small. Thus, we can use an approximation for the $\sinh$ function:
		\[
			\sinh^2 \gamma \approx \left( \frac{e^\gamma}{2}\right)^2 = \frac{e^{2\gamma}}{4}
		\] 
		which now reduces our expression for the exact transmission to:
		\[
			T \approx \frac{1}{1 + \frac{V_0^2}{4E(V_0 - E)} \frac{e^{2\gamma}}{4}} \approx \frac{1}{\frac{V_0^2e^{2\gamma}}{16E(V_0 - E)}} = \frac{16 E(V_0 - E)}{V_0^2 e^{2\gamma}}
		\] 
		In the case where $\gamma$ is very large, then we can effectively neglect the constants in front, 
		meaning that our transmission now becomes
		\[
			T \approx \frac{1}{e^{2\gamma}} = e^{-2\gamma} 
		\] 
		which matches the WKB solution.
	\end{solution}
	\pagebreak

	\section*{Problem 3}

	Use the WKB approximation to find the allowed energies of the harmonic oscillator. 

	\begin{solution}
		Here, let $E$ represent the energy of the particle. Then, since we know that $V(x) = \frac{1}{2}m
		\omega^2 x^2$, then we can easily solve for the turning points, which end up becoming:

			\[
		a = \pm \sqrt{\frac{2E}{m\omega^2}} 
		\] 
		Further, we have $p(x) = \sqrt{2m(E - \frac{1}{2}m \omega^2x^2)}$, so now we can solve the integral:
		\[
			\int_{-a}^a \sqrt{2m(E - \frac{1}{2}m \omega^2x^2)} \ dx  = 2\int_0^a \sqrt{2m(E - \frac{1}{2}m \omega^2 x^2)} 
		\] 
		This integral can be simplified by hand a little:
		\begin{align*}
			\int_{-a}^a \sqrt{2m(E - \frac{1}{2}m \omega^2 x^2)} \ dx &= \int_{-a}^a \sqrt{m^2 \omega^2\left( \frac{2E}{m \omega^2} - x^2 \right) } \ dx\\
			&=  m \omega \int_{-a}^a \sqrt{a^2 - x^2}  \ dx 
		\end{align*}
		then, we can throw this into Mathematica:
		\[
			m\omega \int_{-a}^a \sqrt{a^2 - x^2}  dx = m \omega\frac{\pi a^2}{2}
		\] 
		Now, substituting back $a$:
		\[
		m \omega \frac{\pi a^2}{2} = \frac{m \omega \pi}{2} \frac{2E}{m \omega^2} = \pi \frac{E}{\omega}
		\] 
		Now, since WKB requires that for a finite barrier, that we set $\int p dx = (n + 1/2) \pi \hbar$, then 
		we have: 
		\[
			\frac{\pi E}{\omega} = \left( n + \frac{1}{2} \right) \pi \hbar
		\] 
		which implies that 
		\[
		E_n = \left( n + \frac{1}{2} \right) \hbar \omega
		\] 
		which is the exact solution for the harmonic oscillator problem.
	\end{solution}
	\pagebreak
	\section*{Problem 4}
	In the phenomenon of \textit{cold emission}, electrons are drawn from a metal (at room temperature) by an 
	externally supported electric field. The potential well that the metal presents to the 
	free electrons before the electric field is turned on is depicted in Fig. 2.5. After application of the 
	constant electric field $\mathcal E$, the potential at the surface slopes down as shown in Fig. 7.38, 
	thereby allowing electrons in the Fermi sea to ``tunnel'' through the potential barrier. If the surface of 
	the metal is taken as the $x = 0$ plane, the new potential outside the surface is 
	\[
	V = \Phi + E_F - e\mathcal E x
	\] 
	where $E_F$ is the Fermi level and $\Phi$ is the work function of the metal.
	\begin{enumerate}[label=\alph*)]
		\item  Use the WKB approximation to calculate the transmission coefficient for cold emission.

			\begin{solution}
				The WKB approximation says that for tunnelling, we have $T \approx e^{-2 \gamma}$, and we 
				define $\gamma$ to be:
				\[
					\gamma = \int_{x_1}^{x_2} \sqrt{\frac{2m}{\hbar^2}\left(V(x) - E\right)}\  dx
				\] 
				Applying this to our situation, we can see that the bounds of integration are from $x_1 = 0$ to
				$x_2 = \Phi/e \mathcal E$, and $V(x) = \Phi + E_F - e \mathcal E x$. Further, we only consider
				electrons at the top of the Fermi level, so therefore $E = E_F$ in our case. This means that the
				integral reduces to: 
				\[
					\gamma = \sqrt{\frac{2m}{\hbar^2}} \int_{0}^{\Phi/e\mathcal E} \sqrt{\Phi + E_F
					- e \mathcal E x - E_F} \ dx = \sqrt{\frac{2m}{\hbar^2}} \int_{0}^{\Phi/e\mathcal E}
					\sqrt{\Phi - e\mathcal E x} \  dx = \sqrt{\frac{2m}{\hbar^2}} 
					\frac{2 \Phi^{3/2}}{3e\mathcal E}
				\] 
				Rewriting this a bit, we get: 
				\[
					T = \exp{-\frac{4\sqrt{2m \Phi}}{3 \hbar e \mathcal E}}
				\] 

			\end{solution}
		\item Estimate the field strength $\mathcal E$, in volt/cm, necessary to draw current density of the 
			order mA/$\text{cm}^2$ from a potassium surface. For $J_{inc}$ (see Eq. 7.108:
			$T \equiv \left|\frac{J_{\text{trans}}}{J_{\text{inc}}}\right|$) use the expression 
			$J_{inc} =
			env$, where $n$ is the electron density and $v$ is the speed of electrons at the top of the Fermi
			sea. The relevant expression for $E_F$ may be found in Problem 2.42. Data for potassium is given 
			in Section 2.3.
			\[
				E_F = \frac{\hbar^2}{2m}\left( \frac{3n}{8\pi} \right)^{\frac{2}{3}}
			\] 


			\begin{solution}
				First, note that because we want $J_{\text{trans}}$ on the order of mA/$\text{cm}^2$, then this 
				is equivalent to wanting 10 A/$\text{m}^2$. This will be useful later on so that we don't have
				to worry too much about unit conversions. By the definition of the problem statement, we first
				write $J_{\text{inc}} \cdot T
				= J_{\text{trans}}$, and substituting in our equation for $T$, we get:
				\[
					J_{\text{trans}} = J_{\text{inc}} \exp{-\frac{4\sqrt{2m\Phi}}{3\hbar e \mathcal E}}
				\] 
				Here, the only unknown quantity is $J_{\text{inc}}$, but we can solve for it in terms of 
				other variables. Firstly, we know that we need to use $J_{\text{inc}} = env$, and solving 
				for $v$ by using the energy of particles at the top of the Fermi level:
				\[
					v = \sqrt{\frac{2E_F}{m}} 
				\] 
				We can also solve for $n$ in terms of $E_F$ using the expression in Problem 2.42:
				\begin{align*}
					E_F &= \frac{\hbar^2}{2m}\left( \frac{3m}{8\pi} \right) ^{\frac{2}{3}}\\
					\therefore n &= \frac{16 \sqrt{2} \pi }{3}\left( \frac{E_F m}{\hbar^2} \right)^{\frac{3}{2}}
				\end{align*}
				therefore, now we can calculate $J_{\text{inc}}$:
				\[
					J_{\text{inc}} = env = e\left( \frac{16 \sqrt{2} \pi}{3}\left( \frac{m E_F}{\hbar^2} \right)^{\frac{3}{2}} \right) \sqrt{\frac{2E_F}{m}} = \frac{32e\pi}{3}\sqrt{\frac{E_F}{m}}\left( \frac{E_F m}{\hbar^2} \right)^{\frac{3}{2}} = \frac{32 e\pi}{3} \frac{E_F^2 m}{\hbar^3} 
				\] 
				Now we can finally just plug in values:
				\[
				J_{\text{trans}} = 10 \frac{\mathrm A}{\mathrm m^2}= J_{\text{inc}} T = \left( \frac{32 e\pi m E_F^2}{3\hbar^3} \right) \exp{-\frac{4\sqrt{2m \Phi}}{2 \hbar e \mathcal E}} 
				\] 
				So again, now we can solve for $\mathcal E$ by throwing this into Mathematica once again:
				\[
					\mathcal E = \frac{4\sqrt{2m \Phi^3}}{3e \hbar \ln \left( \frac{32 em \pi}{3 \hbar^2 J_{\text{inc}}} E_F^2 \right) }  = \frac{4\sqrt{2m \Phi^3}}{3e \hbar \ln \left( \frac{32 em \pi}{30 \hbar^2} E_F^2 \right) } 
				\] 
				Plugging this value into WolframAlpha, we get approximately $-7.1 \times 10^{-6}$ for $\mathcal 
				E$, which is a very strong electric field.   
			\end{solution}
	\end{enumerate}
\end{document}
