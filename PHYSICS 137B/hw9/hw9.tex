\documentclass[10pt]{article}
\usepackage{../../local}


\newcommand{\classcode}{Physics 137B}
\newcommand{\classname}{Quantum Mechanics II}
\renewcommand{\maketitle}{%
\hrule height4pt
\large{Eric Du \hfill \classcode}
\newline
\large{HW 09} \Large{\hfill \classname \hfill} \large{\today}
\hrule height4pt \vskip .7em
\normalsize
}
\linespread{1.1}
\begin{document}
	\maketitle

	\section*{Collaborators}
	I worked with \textbf{Andrew Binder} to complete this assignment. 
	\section*{Problem 1}
	Calculate the total cross-section for scattering from a Yukawa potential, in the Born approximation. Express your answer as a function of $E$. 

\begin{solution}
	Griffiths solves in problem 10.11 that the scattering amplitdue is given by: 
	\[
	f(\theta) = -\frac{2m\beta}{\hbar^2(\mu^2 + \kappa^2}
	\] 
	Using this result, we can now calculate the differential cross section using the relation
	$\dv{\sigma}{\Omega} = |f(\theta)|^2$, so therefore: 
	\[
		\dv{\sigma}{\Omega} = \frac{4m^2 \beta^2}{\hbar^2(\mu^2 + \kappa^2)}
	\] 
	Now we integrate this with respect to $d\Omega = \sin \theta d\theta d\phi$ in order to get the 
	total cross-section. Here, we will need to return the substitution $\kappa = 2k \sin \frac{\theta}{2}$, 
	since there is $\theta$ dependence in $\kappa$. Therefore, we integrate: 
	\[
		\sigma =\left( \frac{2m\beta}{\hbar}\right)^2 \int_0^{2\pi} \int_0^\pi \frac{\sin \theta}{\mu^2 +
		4k^2 \sin^2 \frac{\theta}{2}} d\theta d\phi
	\] 
	From here, the integral can be evaluated by hand (the integral by hand is rather tedious in my opinion)
	or via a computer, eventually we get:
	\[
	\sigma = \left( \frac{4m\beta}{\mu\hbar^2} \right)^2 \frac{\pi}{\mu^2 + \frac{8mE}{\hbar^2}}
	\] 
		\end{solution}

		\pagebreak
	\section*{Problem 2}
	For the potential in Problem 10.4, 
	\begin{enumerate}[label=(\alph*)]
		\item calculate $f(\theta)$, $D(\theta)$ and $\sigma$, in the low-energy Born approximation;
		
		\begin{solution}
			The potential in question is the delta function $V(r) = a \delta(r - a)$, and in the low energy 
			Born approximation, we compute: 
			\[
			f(\theta) = -\frac{m}{2\pi\hbar^2}\int V(r) d^3r
			\] 
			
			So therefore:
			\begin{align*}
				f(\theta) &= -\frac{m}{2\pi\hbar^2}\int_0^\infty \int_0^{2\pi} \int_0^\pi \alpha \delta(r-a)
				d\theta d\phi dr\\
					&= \frac{4\pi\alpha m}{2\pi\hbar^2}\int_0^\infty \delta(r-a)r^2 dr\\
					&= -\frac{2\alpha ma^2}{\hbar^2}
			\end{align*}
			In the last step, I used the fact that $\int f(x) \delta(x - a) = f(a)$, a well known result. The 
			differential cross section $D(\theta) = |f(\theta)|^2$, so therefore: 
			\[
			D(\theta) = \frac{4\alpha^2 m^2 a^4}{\hbar^4}
			\] 
			Thus, the total cross section integrates over $d\Omega = \sin \theta d\theta d\phi$, so:
			\begin{align*}
				\sigma &= \int D(\theta) d\Omega\\
					   &=\int_0^\pi \int_0^2\pi \frac{4\alpha^2m^2a^4}{\hbar^4} \sin \theta d\theta d\phi \\
					   &= \frac{4\alpha^2m^2a^4}{\hbar^4} (4\pi) \\
					   &= \frac{16\pi\alpha^2 m^2 a^4}{\hbar^4} 
			\end{align*}
		\end{solution}
		\item calculate $f(\theta)$ for arbitrary energies, in the Born approximation;

		\begin{solution}
			For arbitrary energies, the scattering amplitude is: 
			\[
			f(\theta) = -\frac{2m}{\hbar^2 \kappa}\int_0^\infty V(r) r \sin(\kappa r) dr
			\] 
			So plugging our $V(r)$ in: 
			\begin{align*}
				f(\theta) &= -\frac{2m\alpha }{\hbar^2 \kappa}\int_0^\infty \delta(r - a)r \sin (\kappa r) dr\\
				&= -\frac{2m\alpha}{\hbar^2 \kappa} a \sin(\kappa a) \\
				&= -\frac{ma\alpha}{\hbar^2 k \sin \frac{\theta}{2}}\sin(2k\sin \frac{\theta}{2}a)
			\end{align*}
		\end{solution}
		
		
	\end{enumerate}
	\pagebreak	
	\section*{Problem 3}

	Using the Born approximation, evaluate the differential scattering cross section for scattering of particles
	of mass $m$ and incident energy $E$ by the repulsive well with potential
	\[
	V(r) = \begin{cases}
		V_0 & 0 < r < a\\
		0 & r> a
	\end{cases}
	\]
	Exhibit $E$ and $\theta$ dependence. 

	\begin{solution}
		Recall the equation for the cross section scattering: 
		\[
			\dv{\sigma}{\Omega} = \frac{m^2}{4\pi^2 \hbar^4}|\mel{k_f}{V(r)}{k_i}|^2
		\] 
		So we need to compute the matrix element $\mel{k_f}{V(r)}{k_i}$. Under the born approximation, we assume
		that the wavefunctions are free particles, so this matrix element is the integral: 
		\begin{align*}
			\mel{k_f}{V(r)}{k_i} &= \int e^{-i \vec {k_f} \cdot \vec r} V(r) e^{i \vec{k_i}\cdot \vec r} d^3r \\
								 &= \int e^{-i(\vec k_f - \vec k_i) \cdot r}V(r) d^3r 
		\end{align*}
		To compute this integral, we first perform the usual coordinate transform, so this integral becomes: 
		\[
			\mel{k_f}{V(r)}{k_i} = V_0 \int e^{-i(2k \sin \frac{\theta}{2}) r' \cos \theta'} r;^2 \sin \theta 
			dr' d\theta' d\phi'
		\] 
		where $k = k_f - k_i$. Applying bounds to our integral, this gives: 
		\[
			\mel{k_f}{V(r)}{k_i} = 2\pi V_0 \int_0^a \int_0^\pi \int_0^{2\pi} e^{-2ik \sin \frac{\theta}{2}r' 
			\cos \theta'} r'^2 \sin \theta d\phi d\theta dr
		\] 
		The $\phi$ integral evaluates to $2\pi$ since no term has $\phi$ dependence, then the other two 
		integrals are taken by a calculator. When evaluated, they give:
		\[
			\mel{k_f}{V(r)}{k_i} = 4\pi \frac{V_0}{k'} \frac{\sin(ak') - ak' \cos (ak')}{k'^2}
		\] 
		Here, $k' = 2k \sin \frac{\theta}{2}$. Therefore, the scattering cross section is:
		\[
			\dv{\sigma}{\Omega} = \frac{m^2}{4\pi^2 \hbar^4} \frac{16 \pi^2 V_0^2}{k'^6}\left[\sin (ak') - 
			ak' \cos (ak')\right]^2 = \frac{4m^2V_0^2}{\hbar^4 k'^6}\left[\sin(ak') - (ak')\cos(ak')\right]^2
		\] 
	\end{solution}

	\pagebreak
	\section*{Problem 4}
	Using the Born approximation, obtain an integral expression for the total cross section for scattering of 
	particles of mass $m$ under the attractive Gaussian potential

	\[ 
		V(r) = -V_0 \ \mathrm{exp}\left[-\left( \frac{r}{a} \right)^2\right]
	\] 
	
	\begin{solution}
		We are just asked to come up with the integral expression, which essentially just asks for $\mel{k_f}
		{V(r)}{k_i}$. Therefore, the matrix element becomes: 
		\[
			\mel{k_f}{V(r)}{k_i} =  -V_0\int e^{-i(\vec k_f - \vec k_i) \cdot \vec r} e^{-(r/a)^2} d^3r\\
		\] 
		Just like the previous problem, this means that we're solving: 
		\[
			\int_0^\infty \int_0^{2\pi} \int_0^\pi e^{-2ik \sin \frac{\theta}{2}r' \cos \theta'} e^{-r'^2/a^2}
			r'^2 \sin \theta' d\theta' d\phi' dr'
		\] 
		The $\phi$ integral drops out as usual, and after evaluating the $\theta$ term, we're left with the 
		integral: 
		\[
			\frac{2\pi}{k \sin \frac{\theta}{2}}\int_0^\infty e^{-(r'/a)^2} r' \sin (2r'\sin \frac{\theta}{2}k)
			dr'
		\] 
		I tried plugging this into WolframAlpha, and it didn't give me an expression out. So unfortunately, 
		we're going to have to leave this integral as is. Since $d\Omega = \sin \theta d\phi d\theta$, the total
		cross section becomes: 
		\begin{align*}
			\sigma &= \frac{m^2}{4\pi^2 \hbar^4} \int_0^\pi \int_0^{2\pi} \left[\frac{2\pi}{k \sin \frac{\theta}{2}} \int_0^\infty e^{-(r/a)^2} r' \sin(2r' \sin \frac{\theta}{2} k) dr'\right]^2  \sin \theta d\phi d\theta \\
				   &= \frac{m^2}{k\hbar^4} \int_0^\pi \frac{\sin \theta}{\sin^2 \frac{\theta}{2}}\left[ \int_0^\infty  
				   e^{-(r/a)^2} r' \sin(2r' \sin \frac{\theta}{2} k) dr'\right]^2 d\theta  
		\end{align*}
	\end{solution}
\end{document}
