\documentclass[10pt]{article}
\usepackage{../local}

\newcommand{\classcode}{Physics 137B}
\newcommand{\classname}{Quantum Mechanics II}
\renewcommand{\maketitle}{%
\hrule height4pt
\large{Eric Du \hfill \classcode}
\newline
\large{HW 04} \Large{\hfill \classname \hfill} \large{\today}
\hrule height4pt \vskip .7em
\normalsize
}
\linespread{1.1}
\begin{document}
    \maketitle
    \section*{Collaborators}
    I worked with \textbf{Andrew Binder} to complete this assignment.
    \section*{Problem 1}

	Consider a particle of mass $m$ that is free to move in a one-dimensional region of length $L$ that 
	closes on itself (for instance, a bead that slides frictionlessly on a circular wire of circumference
	$L$, as in problem 2.46).
	\begin{enumerate}[label=\alph*)]
			\item Show that the stationary states can be written in the form
				\[ \psi_n(x) = \frac{1}{\sqrt{L} }e^{2\pi i n x /L}, \ \ 
					\left(-\frac{L}{2} < x < \frac{L}{2}\right)\]
				where $n = 0, \pm 1, \pm 2, \dots$ and the allowed energies are 
				\[ E_n = \frac{2}{m}\left( \frac{n \pi \hbar}{L}\right)^2\]
				Notice that -- with the exception of the ground state $(n =0)$ -- these are all doubly 
				degenerate.

				\begin{solution}
					This problem is solved in the exact same way 2.46 is solved. Solving the \schrodinger equation, we obtain the solution set: 
					\[ \psi_n(x) = Ae^{ikx} + Be^{-ikx}\] 
					from the fact that our solutions should be oscillatory, then we now impose the condition that $\psi(x + L) = \psi(x)$, since the bead is circular. Therefore we have:
					\[ Ae^{ikx}e^{ikL} + Be^{-ikx}e^{-ikL} = Ae^{ikx} + Be^{-ikx}\] 
					And specifically at $x = 0$, we have: 
					\[ Ae^{ikL} + Be^{-ikL} = A + B\]
					And for $x = \pi/2k$, we have: 
					\[ Ae^{ik \pi/2}e^{ikL} + Be^{-ik \pi/2}e^{-ikL} = Ae^{i\pi/2} + Be^{-i\pi/2} \implies Ae^{ikL} - Be^{-ikL} = A - B\] 
					Adding these two equations at $x = 0$ and $x = \pi/2k$, we get:
					\[Ae^{ikL} = A\]
					This implies solutions $A = 0$ or $e^{ikL} = 1$. If $A = 0$, then we have: 
					\[ Be^{-ikL} = B \implies kL = 2\pi n\]
					and so therefore 
					\[ k = \frac{2\pi n}{L}\]
					which works for all $n$. On the other hand, if $e^{ikL} = 1$, then we get that: 
					\[ kL = 2\pi n \implies k = \frac{2\pi n}{L}\]
					which is the same conclusion that we arrived at before. Subtracting the two equations would give us 
					\[ Be^{ikL} = B\]
					which we can then solve for a similar relation in terms of $A$. In either case, the key is that we can set one of them to equal zero, and so our overall wavefunction looks like: 
					\[\psi_n(x) = \begin{cases}Ae^{\frac{2\pi i n x}{L}} & n \geq 0\\ Be^{\frac{2\pi i n x}{L}} & n < 0\end{cases}\]
					after having solved for $k$. Normalizing this, we get:
					\begin{align*}
						\int_{0}^{L}|\psi_n|^2dx &= A^2L = B^2L = 1\\
						\therefore  A = B &= \frac{1}{\sqrt{L}}
					\end{align*}
					Then, since we can take $x$ to be symmetric about the origin, then we have:
					\[\psi_n(x) = \frac{1}{\sqrt{L}}e^{\frac{2\pi i n x}{L}}, \ \ \ -\frac{L}{2} < x < \frac{L}{2}\]
					which is exactly what we wanted. Furthermore, we can now substitute our expression for $\psi$ into the \schrodinger equation, which would give us:
					\[E_n = \frac{2n^2\pi^2\hbar^2}{mL^2} = \frac{2}{m}\left(\frac{n\pi\hbar}{L}\right)^2\]
					As desired.
				\end{solution}
		\item Now suppose we introduce the perturbation 
				\[ H' = -V_0 e^{-\frac{x^2}{a^2}}\]
				where $a \ll L$ (This puts a little ``dimple'' in the potential at $x = 0$, as though we bent 
				the wire slightly to make a ``trap''.) Find the first-order correction to $E_n$, using 
				Equation 7.33. \textit{Hint:} To evaluate the integrals, exploit the fact that $a \ll L$
				to extend the limits from $\pm L/2$ to $\pm \infty$; after all, $H'$ is essentially zero outside
				$-a < x < a$.

				\begin{solution}
					To solve this, we calculate the matrix elements for the degenerate subspace, noticing that for each energy level there is a twofold degeneracy. For the diagonal terms, we have: 
					\[ \expval{H'}{\psi_n} = \frac 1L \int_{-L/2}^{L/2} -V_0 e^{-x^2/a^2} \ dx = -\frac{V_0}{L} \int_{-\infty}^\infty e^{-x^2/a^2} \ dx =  -\frac{V_0}{L} a\sqrt{\pi}\]
					For the off-diagonal terms, we have: 
					\[ \mel{\psi_n}{H'}{\psi_{-n}} = \frac 1L \int_{-L/2}^{L/2} e^{-2\pi in x/L} (-V_0 e^{-x^2/a^2}) e^{-2\pi inx/L} \ dx = -\frac{V_0}{L} \int_{-\infty}^\infty e^{-4\pi in x/L} e^{-x^2/a^2} \ dx\]
					Plugging this result into an integral calculator, we get: 
					\[ \mel{\psi_n}{H'}{\psi_{-n}} = -\frac{V_0}{L} a \sqrt{\pi} e^{-(2 \pi n a/L)^2}\]
					From here, we have all the matrix elements, all we have to do now is to calculate the eigenvalues of this matrix. Using Equation 7.33, we find that:
					\[ E_{\pm} = \frac 12 \left[ -\frac{V_0}{L} a\sqrt{\pi} - \frac{V_0}{L} a \sqrt{\pi} \pm \frac{V_0}{L} a\sqrt{\pi} e^{-(2n\pi a/L)^2}\right]\]
					Therefore, we get: 
					\begin{align*}
						E_{n+} &= -\frac{V_0}{L} a\sqrt{\pi} \left[1 - e^{-(2 n \pi a/L)^2}\right]\\
						E_{n-} &= -\frac{V_0}{L} a\sqrt{\pi} \left[1 + e^{-(2 n \pi a/L)^2}\right]
					\end{align*}
				\end{solution}
		\item What are the ``good'' linear combinations of $\psi_n$ and $\psi_{-n}$ for this problem?
				(\textit{Hint:} use Eq. 7.27). Show that with these states you get the first-order correction 
				using Equation 7.9

				\begin{solution}
					To do this, we just look for the eigenvectors of our perturbed Hamiltonian. Solving Equation 7.27, we obtain: 
					\[ \beta = \alpha \cdot \frac{\frac{-V_0}{L} a \sqrt{\pi} e^{-(2na\pi/L)^2}}{\mp \frac{V_0}{L} a\sqrt{\pi}e^{-(2na\pi/L)^2}} = \mp \alpha \]
					And so therefore we get the linear combinations: 
					\[ \overline \psi_\pm = \alpha \psi_n \mp \alpha \psi_{-n} = \alpha \frac{1}{\sqrt{L}} \left( e^{2\pi i n x/L} \mp e^{-2\pi in x/L}\right)\]
					So normalizing this state, we get that $\alpha = \frac{1}{\sqrt{2}}$:
					\[ \overline \psi_+ = \alpha \psi_n - \alpha \psi_{-n} = \frac {1}{\sqrt{2L}} \left( e^{2\pi i n x/L} - e^{-2\pi in x/L}\right) = \frac{2i}{\sqrt{2L}}\sin\left( \frac{2\pi nx}{L}\right) = i\sqrt{\frac 2L} \sin \left( \frac{2 \pi n x}{L}\right) \]
					Likewise for the other state, we get: 
					\[ \overline \psi_- = \alpha \psi_n + \alpha \psi_{-n} = \frac{1}{\sqrt{2L}}\left( e^{2\pi inx/L} + e^{2\pi inx/L} \right) = \frac{2}{\sqrt{2L}} \cos\left( \frac{2\pi n x}{L}\right) = \sqrt{\frac 2L} \cos\left( \frac{2\pi nx}{L}\right)\]
					To verify that these give the correct perturbation, we calculate: 
					\begin{align*}
						E_{n+}^1 &= \expval{H'}{\overline \psi_+}\\
						&= -\frac{2V_0}{L} \int_{-L/2}^{L/2} \sin^2\left( \frac{2\pi nx}{L}\right) e^{-x^2/a^2}\ dx \\
						&\approx -\frac{2V_0}{L} \int_{-\infty}^\infty \sin^2\left( \frac{2\pi nx}{L}\right) e^{-x^2/a^2} \ dx
					\end{align*}
					Plugging this into an integral calculator gives us: 
					\[ E_{n+} = -\frac{2V_0}{L} \cdot \frac{\sqrt \pi a \left( 1 - e^{-(2 \pi n a/L)^2}\right)}{2} = -\frac{V_0}{L} a\sqrt{\pi}\left( 1 - e^{-(2 \pi n a/L)^2}\right)\]
					And similarly with the other solution: 
					\begin{align*}
						E_{n-} &= -\frac{2V_0}{L} \int_{-\infty}^\infty \cos^2 \left( \frac{2\ pi nx}{L}\right) e^{-x^2/a^2} \ dx\\
						&= -\frac{2V_0}{L} \frac{\sqrt{\pi} a \left(e^{-(2\pi n a/L)^2} + 1\right)}{2}\\
						&= -\frac{V_0}{L}a \sqrt{\pi} \left(1 + e^{-(2 \pi n a/L)^2}\right)
					\end{align*}
					which are the same expressions we got in part (b).
				\end{solution}
	\end{enumerate}

	\pagebreak
	\section*{Problem 2}
	
	Suppose we perturb the infinite cubical well (Problem 4.2) by putting a delta function ``bump'' at the point
	$(a/4, a / 2, 3a/4)$: 
	\[ H' = a^3 V_0 \delta(x - a / 4)\delta(y - a / 2) \delta (z - 3a / 4)\]
	Find the first-order corrections to the energy ground state and the (triply degenerate) first excited states.
	
	\begin{solution}
			The ground state is nondegenerate, so therefore the energy correction is: 
			\[ E_0^1 = \expval{H'}{111}\]
			And so this equals
			\begin{align*}
					E_0^1 = a^3V_0 \cdot \left( \frac{2}{a} \right)^3 \sin^2 \left(\frac{\pi}{4} \right) \sin^2\left( \frac{\pi}{2} \right) 
					\sin^2\left( \frac{3\pi}{4} \right) = 2V_0
			\end{align*}
			after evaluating the integral while using the property that $\int f(x) \delta(x) \ dx = f(0)$. Now
			we have to calculate the first excited states. First, notice that the states are: $\ket{2, 1, 1}$
			$\ket{1, 2, 1}, \ket{1, 1, 2}$. Now we need to calculate the matrix of the degenerate subspace,
			and find its eigenvalues. First, we can calculate the diagonal terms: 
			\begin{align*}
				\expval{H'}{112} &= 8V_0 \sin^2\left( \frac{\pi}{4} \right) \sin^2\left( \frac{\pi}{2} \right) 
				\sin^2\left( \frac{3\pi}{4} \right) = 4V_0 \\
				\expval{H'}{121} &= 8V_0 \sin^2\left( \frac{\pi}{4} \right) \sin^2(\pi) 
				\sin^2\left( \frac{3\pi}{4} \right) = 0\\
				\expval{H'}{211} &= 8V_0 \sin^2\left( \frac{\pi}{2} \right) \sin^2\left( \frac{\pi}{2} \right)
				\sin^2\left( \frac{3\pi}{4} \right) = 4V_0 
			\end{align*}
			Now for the off diagonal terms. These aren't too much more tedious, but I will only show one for the
			sake of brevity:
			\begin{align*}
					\mel{211}{H'}{121} &= \left( \frac{2}{a} \right) ^3 a^3 V_0 
					\left[\mel{2}{\delta\left( x - \frac{a}{4} \right)} {1} 
					\mel{1}{\delta\left( y - \frac{a}{2} \right) }{2} 
					\mel{1}{\delta\left( z - \frac{3a}{4} \right) }{1}\right]\\
									   &= 8V_0 \left[\sin\left( \frac{\pi}{2} \right) 
						\sin\left( \frac{\pi}{4} \right) \sin\left( \frac{\pi}{2} \right) \sin\pi 
				\sin^2\left( \frac{3\pi}{4} \right) \right]\\
									   &= 0
			\end{align*}
			A very similar process is employed to calculate the other two matrix elements, giving us 
			\begin{align*}
					\mel{211}{H'}{112} &= -4V_0\\
					\mel{121}{H'}{112} &= 0
			\end{align*}
			So now we have a matrix:
			\[ H' = 4V_0 \begin{pmatrix}1 & 0 &-1\\ 0 & 0 & 0 \\ -1 & 0 & 1   \end{pmatrix}\]
			This gives the eigenvalues $E^1 = 0, 2$ with two solutions for $E = 0$, so therefore the energy 
			correction is $\Delta E = 0, 0, 8V_0$.

			
	\end{solution}
	\pagebreak
	\section*{Problem 3}
	Use the virial theorem (Problem 4.48) to prove Equation 7.56
	
	\begin{solution}
			The virial theorem writes:
			\[ \mean{V} = 2E_n\]
			And since $V = -\frac{e^2}{4\pi \epsilon_0 r}$ and 
			\[ E_n = -\left( \frac{m}{2\hbar^2}
			\left( \frac{e^2}{4\pi \epsilon_0} \right) ^2 \right) \]
			we can then plug these in:
			\begin{align*}
					\mean{V} &= -\frac{e^2}{4\pi \epsilon_0} \mean{\frac{1}{r}}\\
					-2 \cdot \frac{m}{2\hbar^2}\left( \frac{e^2}{4\pi\epsilon_0} \right) ^2 \frac{1}{n^2} 
							 &= -\frac{e^2}{4\pi\epsilon_0}\mean{\frac{1}{r}}\\
					\therefore \mean{\frac{1}{r}} &= \frac{me^2}{4\pi \epsilon_0 \hbar^2}\frac{1}{n^2}\\
												  &= \frac{1}{a_0n^2}
			\end{align*}
			which is equation 7.56, as desired.
		\end{solution}
	\pagebreak

	\section*{Problem 4} 
	Evaluate the following commutators: 
	\begin{enumerate}[label=\alph*)]
			\item $[\mathbf L \cdot S, \mathbf L]$

			\begin{solution}
					To do this we just write out the commutator:
				\begin{align*}
					[\mathbf{L}\cdot\mathbf{S},L_x] &= [L_xS_x + L_yS_y + L_zS_z, L_x] \\
					&= S_x[L_x,L_x] + S_y[L_y,L_x] + S_z[L_z,L_x]\\
					&= S_y(-i\hbar L_z) + S_z(i\hbar L_y) \\
					&= i\hbar(L_yS_z - L_zS_y) \\
					&= i\hbar(\mathbf{L}\times\mathbf{S})_x
				\end{align*}
                The other two components will behave the same way, so combining them gives us all the components summed up altogether:
                \[[\mathbf{L}\cdot\mathbf{S}, \mathbf{L}] = i\hbar(\mathbf{L}\times\mathbf{S})\]			
			\end{solution}
			\item $[\mathbf {L \cdot S}, \mathbf S]$
			
			\begin{solution}
				Doing this is much of the same process:
                \begin{align*}
                    [\mathbf{L}\cdot\mathbf{S},S_x] &= [L_xS_x + L_yS_y + L_zS_z, S_x] \\
					&= L_x[S_x,S_x] + L_y[S_y,S_x] + L_z[S_z,S_x]\\
					&= -L_y(i\hbar S_z) + L_z(i\hbar S_y) \\
					&= i\hbar(S_yL_z - S_zL_y) \\
					&= \fbox{$i\hbar(\mathbf{S}\times\mathbf{L})_x$}
                \end{align*}
				And so just like before, combining all three components together we get:
                $$\fbox{$[\mathbf{L}\cdot\mathbf{S}, \mathbf{S}] = i\hbar(\mathbf{S}\times\mathbf{L})$}$$
			\end{solution}
			\item $[\mathbf{L \cdot S}, \mathbf J]$
			
			\begin{solution}
				Since we have that $\mathbf J = \mathbf{L + S}$, then we have: 
				\begin{align*}
					[\mathbf{L \cdot S}, \mathbf J] &= [\mathbf{L \cdot S}, \mathbf{L + S}]\\
					&= i\hbar (L \times S + S \times L)\\
					&= 0
				\end{align*}
			\end{solution}
			\item $[\mathbf{L \cdot S}, L^2]$
			
			\begin{solution}
				Now we can use our previous reuslts. Since we know that $L^2 = L_x^2 + L_y^2 + L_z^2$, and that each of these commutes with $L$ and $S$, we then have
				\[ [\mathbf{L \cdot S}, L^2]  =0\]
			\end{solution}
			\item $[\mathbf {L \cdot S}, S^2]$
			
			\begin{solution}
				Using a similar reasoning as the previous problem, we know that $S^2 = S_x^2 + S_y^2 + S_z^2$ and that each of these also commutes with $L$ and $S$, so: 
				\[ [\mathbf{L \cdot S}, S^2] = 0\]
			\end{solution}
			\item $[\mathbf{L \cdot S}, J^2]$
			
			\begin{solution}
				Since $\mathbf J = J_x^2 + J_y^2 + J_z^2 = (L_x + S_x)^2 + (L_y + S_y)^2 + (L_z + S_z)^2$ and $\mathbf{L \cdot S}$ commutes with both from the previous two parts, then we know that:
				\[ [\mathbf{L \cdot S}, J^2] = 0\]
			\end{solution}
	\end{enumerate}

		\pagebreak

	\section*{Problem 5}
	How does the threefold degenerate energy 
	\[ E = 3\hbar \omega_0\]
	of the two-dimensional harmonic oscillator separate due to the perturbation 
	\[ H' = K'xy?\]

	\begin{solution}
		We know that the energy of a 2D harmonic oscillator can be written as:
		\[ E = \left( n_x + \frac{1}{2} \right) \hbar \omega + \left( n_y + \frac{1}{2} \right) \hbar \omega 
		= \left( n_x + n_y + 1 \right) \hbar \omega\]
		And since we know that the total energy is $E = 3\hbar \omega_0$, then we know that the degenerate
		states are: $\ket{n_x, n_y} = \ket{0, 2}, \ket{1, 1}, \ket{2, 0}$. So now our goal is to see whether
		the matrix elements in the degenerate subspace are zero. To do this, we look at whether:
		\[ K' \mel{\psi_n}{xy}{\psi_n} = 0 \]
		For all states in the degenerate subspace. Expanding out $x$ and $y$ in terms of raising and lowering 
		operators, we get: 
		\begin{align*}
				K'\mel{\psi_n}{xy}{\psi_n} &= K'\frac{\hbar}{2m\omega_0} \mel{n_xn_y}{(x_+ + x_-)(y_++y_-)}
				{n_x'n_y'}\\
					 &= K' \frac{\hbar}{2m\omega_0}\mel{n_xn_y}{x_+y_+ + x_+y_- + x_-y_+ 
										   x_-y_-}{n_x'n_y'}\\
					 &= K' \frac{\hbar}{2m\omega_0} \left(\braket{n_xn_y}{n_{x+1}'n_{y+1}'}	
							 + \braket{n_xn_y}{n_{x+1}'n_{y-1}'} + \braket{n_xn_y}{n_{x-1} ' n_{y-1}'} 
							 + \braket{n_xn_y}{n_{x-1}'n_{y-1}'}\right)
		\end{align*}	
		Note here that the only states which will produce a nonzero expectation value are if $\ket{n_x, n_y} 
		= \ket{2, 0}$ or $\ket{0, 2}$, since if our state is $\ket{1, 1}$ then the raising and lowering 
		operators would alter the states so that they are orthogonal to our original one. For the state 
		$\ket{2, 0}$ the only term that survives is $\braket{n_xn_y}{n_{x+1}'n_{y-1}}$ where $\ket{n_x'n_y'} =
		\ket{1, 1}$. Therefore, the matrix element is: 
		\[ \mel{2, 0}{x_+ y_-}{1, 1} = \mel{2, 0}{\sqrt{2}}{2, 0} = \sqrt{2} \]  
		Similarly, we have: 
		\[ \mel{0, 2}{x_- y_+}{1, 1} = \mel{0, 2}{\sqrt{2}}{0, 2} = \sqrt{2}\]
		So we can build our matrix:
		\[ H' = K' \frac{\hbar}{2m\omega_0}
				\begin{pmatrix} 0 & \sqrt{2} & 0\\ \sqrt{2} & 0 & \sqrt{2} \\ 0 & \sqrt{2} &0  \end{pmatrix} \]
		which gives us the eigenvalues $\lambda = 0, \pm 2$. So therefore, our threefold degeneracy splits up
		into three states with $E_0 + 2\Delta E, E_0, E_0 - 2\Delta E$ where 
		\[ \Delta E = K'\frac{\hbar}{2m\omega_0}\]
		In other words, our threefold degeneracy splits into three states: one goes up in energy by $2\Delta E$,
		one stays the same, and the final one goes down in energy by $2\Delta E$.
	\end{solution}
\end{document}
