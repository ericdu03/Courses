\chapter{Lecture 18}

The eighteenth lecture of Phyics 137B was held on \textbf{Tuesday, March 21}. It completed our discussion on \textbf{scattering.}

Before we continue our discussion about scattering, let's first do a small math review. Consider a circle of radius $R$:

[insert tikz here]

We know that if we were to move along the circle by a distance $ds$, then we can write $ds/R = d\theta$, and $\int_0^{2\pi} d\theta = 2\pi$. In three dimensions, we look at a small ``patch'' of area on the surface of a sphere: 

[insert tikz here]

Here, we can write a small area element $da = r^2 \sin \theta d\theta d\phi$, or in other words $da/r^2 = \sin \theta d\theta d\phi$, which we define as $d\Omega$. Similarly, note that $\int d\Omega = 4\pi$. 

\section{Last time: 3D Scattering} 

Now let's continue our discussion from last time of optical scattering. Recall that in our setup, we are sending a beam of particles with a given luminoisty $\mathcal L$ at a potential $V(r)$, and as the particle is ``deflected'' from this potential. Generally, we are concerned only with how the particles within an annulus of radius $b$ are affected, this is called the ``impact parameter'', and the range of angles our particle can scatter to is called the ``solid angle''. 

If we place a detector at the solid angle, then we will be measuring the proportion of particles that are scattered by a specific angle $\theta$. We saw last time that the relationship between the number of particles is related to the luminosity by: 
\[ dN = \mathcal L d\sigma\]
which implies $N = \mathcal L \sigma$. If we consider the number of particles or in other words, we have: 
\[ \dv{N}{\Omega} = \mathcal L \dv{\sigma}{\Omega}\]
Experimentally however, we care about the quantity $\frac{d\sigma}{d\Omega}$, so rearranging for this value we get: 
\[ \dv{\sigma}{\Omega} = \frac{1}{\mathcal L} \dv{N}{\Omega}\]
Integrating across the total space $\Omega$, we get:
\[
	\sigma = \int_0^{2\pi} \int_0^\pi \left( \dv{\sigma}{\Omega} \right) d\Omega
\] 
So that covers what we already know about scattering. Let's think about how we can model this system quantum 
mechanically. To do this, let's revisit our classical system one more time: 

\section{The Quantum Picture}
Clasically, we have a beam of particles with luminosity $\mathcal L$ and momentum $\vec k_i$ that points to the
right. When the particle is deflected, it now has a new momentum vector $\vec k_f$. Translating this into the
quantum world, this is the same as modelling our particle as a plane wave with an initial state $\psi_i$ with 
momentum $\vec k_i$ to one with momentum $\vec k_f$, due to the potential $V(r)$. In other words, we went from a state: 
\[
	\psi_i = \frac{1}{L^{\frac{3}{2}}}e^{i\vec k_i \cdot \vec r}
\] 
to a final state 
\[
	\psi_f = \frac{1}{L^{\frac{3}{2}}}e^{i \vec k_f \cdot \vec r}
\] 
when we introduced a perturbation $V(r)$ to our Hamiltonian.
[insert tikz here with the squiggles]
\begin{insight*}{}{}
	We could have done this same analysis with wave packets, but those are really annoying to deal with, so 
	we'll use plane waves instead. After all, since wave packets are superpositions of plane waves, then 
	abusing the linearity of physics should intuitively tell you that the same analysis can be done with 
	wave packets. 
\end{insight*}
Now with that out of the way, let's think about what our quantum analogue for the luminosity might be. 
Classically, the luminosity refers to the number of particles per time area, so intuitively speaking, we should
look for a quantity that represents a moving density. In other words, we are looking for the \textit{probability
current}. Remember that one formula we saw in 137A for the probability current:
\[
J = \frac{\hbar}{2mi}\left( \psi^*\nabla \psi - \psi \nabla\psi* \right) 
\] 
Well, this seemingly useless formula we learned is going to come in very handy in scattering. Correspondingly, 
then, the $N$ that we measure will not be an actual \textit{particle count}, but instead a measure of the 
\textit{rate} at which the particle scatters. Therefore, the quantum mechanical analogue of our initial 
scattering expression is: 
\[
	\left( \dv{\sigma}{\Omega}\right) = \frac{1}{|J|}\dv{R}{\Omega} 
\] 
\section{Calculating the scattering} 
The rest of this lecture will be devoted to calculating this quantity. First off, let's get $J$ out of the way
because it's easy. Given a plane wave, recall that $\nabla(e^{i\vec k \cdot vec r}) =
ike^{i \vec k \cdot \vec r}$, so putting both terms together (and skipping the algebra) we get: 
\[
	J_{inc} = \frac{\hbar k_{inc}}{mL^3}
\] 
\begin{insight*}{}{}
	Notice that dimensionally, $J$ actually makes sense. Since the rate is calculated as the velocity per 
	unit volume, we see that the velocity is represented by $\hbar k/m$, and the volume is $\frac{1}{L^3}$.
	This is a good way to check that we've done our algebra correctly, and the quantities we're deriving have 
	physical meaning.
\end{insight*}
Moving on, let's calculate the scattering rate $R$. To do this, we notice that the system we're trying to 
solve is actually the same as that we saw last time, with Fermi's golden rule! To see this, remember that 
initially, our particles are in a \textit{singular} state of momentum $k_i$. Then, since we are measuring 
over some range $d\Omega$, we are actually measuring the rate at which particles scatter into a \textit{
continuum} of states defined by a cone (see image below), so Fermi's golden rule applies. 
[insert tikz here]
Recall that the formula for Fermi's golden rule we derived was: 
\[
	R = \frac{\pi}{2\hbar}|\mel{\psi_f}{H'}{\psi_i}|^2 \rho(E_i)
\] 
But this rate was for a sinusoidal perturbation $H'$. It turns out that for a constant perturbation, the 
formula for rate is: 
\[
	R = \frac{2\pi}{\hbar}|\mel{\psi_f}{H'}{\psi_i}|^2 \rho(E_i)
\] 
So now we calculate the matrix element:
\begin{align*}
	\mel{k_f}{V(r)}{k_i} &= \int d^3r \frac{1}{L^{\frac{3}{2}}} e^{-i \vec k_f \cdot \vec r} V(r) \frac{1}{L^{\frac{3}{2}}} e^{i\vec k_i \cdot \vec r}\\
						 &= \frac{1}{L^3} \int d^3r e^{-i\vec k_f \cdot \vec r} V(r) 
						 e^{i\vec k_i \cdot \vec r} \\
\end{align*}
We will leave it in this integral form for now. Now we calculate the density of states, recall that the density 
of states is calculated as: 
\[
\rho(E) dE = D_3(k) \cdot 4\pi k^2 dk
\]
First, let's calculate $D_3(k)$.  
Recall that for a line, the total volume is calculated as $\frac{2\pi}{L}$, so
in three dimensions, we just cube this result, meaning the total volume we consider is going to be $\left( \frac{2\pi}{L} \right)^3$. Furthermore, in momentum space, we are considering momenta of the form:
\[
k = \frac{\sqrt{2mE}}{\hbar}
\]
Calculating $\dv{E}{k}$ (which is done by rearranging the above equation for $E$), we get: 
\[
	\dv{E}{k} = \frac{\hbar^2 k}{m} \implies \dv{k}{E} = \frac{m}{\hbar^2 k}
\] 
Putting this all together, we get the density of states: 
\[
\rho(E) = \frac{mL^3k}{2\pi^2 \hbar^2}
\] 
Now putting the two terms together, we get: 
\begin{align*}
	\dv{\sigma}{\Omega} &= \frac{1}{|J|}\dv{R}{\Omega}\\
						 &= \frac{mL^3}{\hbar k} \frac{km}{L^3 4\pi^2 \hbar^3}\left|\int e^{-i \vec k_f \cdot 
						 \vec r} V(r) e^{i\vec k_i \cdot \vec r} d^3r \right|^2\\
						 &= \frac{m^2}{4\pi^2 \hbar^4}\left|\mel{\tilde k_f}{V(r)}{\tilde k_i}\right|^2 \\
\end{align*}
This is called the \textbf{Born approximation}.  
Here, we use $\tilde k_i$ to denote the \textit{unnormalized} wave function. We can do this because as we see
in the algebra, the normalization we imposed $|1/L^{3/2}$ actually cancels. 
Also, take a look at what this term is saying: the matrix element $\mel{k_f}{V(r)}{k_i}$ tells us precisely the probability of scattering!
\section{The Matrix element and the Fourier Transform}
Now let's take a closer look at the matrix element. Combining the exponents, we can write that term as: 
\[ \mel{\tilde k_f}{V(r)}{\tilde k_i} = \int d^3r e^{-i(\vec k_f - \vec k_i) \cdot r} V(r) = \int e^{i\vec{q} \cdot \vec r} d^3r \]
This looks suspiciously like a Fourier transform, with $\vec q = \vec k_f - \vec k_i$. Moreover, since $V(r)$ is a continuous potential, we can write it also as a fourier transform: 
\[ V(r) = \sum_{q'} V(q')e^{i \vec{q'} \cdot \vec r}\]
Putting these two together, it means that we are basically evaluating the integral 
\[ \int d^3r \sum_{q'} e^{i(\vec q - \vec q') \cdot r}\]
This integral equals zero for all $q$ except when $q = q'$. This is because roughly speaking, integrating a sine wave adds up an equal amounts of positives and negatives, so therefore we expect them all to cancel. This doesn't happen in the case where $\vec q = \vec q'$, since the exponential becomes a constant. 

This statement is actually a hint to a deeper statement: the scattering is only nonzero, when \textit{the potential V(r) has a fourier component whose momentum $\vec{q'} = \vec q = \vec{k_f} - \vec{k_i}$}. Believe it or not, this statement is the quantum mechanical analogue of conservation of momentum.  

This is where we're going to stop for now, next lecture we're going to dive deeper into this conservation statement. 

