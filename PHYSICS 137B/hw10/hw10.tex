\documentclass[10pt]{article}
\usepackage{../../local}


\newcommand{\classcode}{Physics 137B}
\newcommand{\classname}{Quantum Mechanics II}
\renewcommand{\maketitle}{%
\hrule height4pt
\large{Eric Du \hfill \classcode}
\newline
\large{HW 10} \Large{\hfill \classname \hfill} \large{\today}
\hrule height4pt \vskip .7em
\normalsize
}
\linespread{1.1}
\begin{document}
	\maketitle
	\section*{Collaborators}
	I worked with \textbf{Andrew Binder} to complete this assignment. 
	\section*{Problem 1}
	Check that Equation 10.65 satisfies Equation 10.52, by direct substitution. \textit{Hint:} $\nabla^2
	(1/r)= -4\pi \delta^3(r)$.

	\begin{solution}
		Basically the problem asks us to check that the solution 
		\[
			G(r) = -\frac{e^{ikr}}{4 \pi r}
		\] 
		satisfies the differential equation $(\laplacian + k^2)G(r) = \delta^3(r)$. Plugging this in, we get:

		\begin{align*}
			(\laplacian + k^2)G(r) &= -\frac{1}{4\pi} \laplacian \left(\frac{e^{ikr}}{r}\right) - k^2\left( \frac{e^{ikr}}{4\pi r}\right)
		\end{align*}
		The first step here is to use the product rule for Laplacians: $\laplacian (fg) = g \laplacian f + 
		f\laplacian g + 2\nabla f \nabla g$. 
		Therefore:
		\begin{align*}
			(\laplacian + k^2)G(r) &= -\frac{1}{4\pi}\left[\frac{1}{r}\laplacian e^{ikr} + e^{ikr} \laplacian \frac{1}{r} + 2\nabla e^{ikr} \nabla \frac{1}{r}\right]  - k^2\left( \frac{e^{ikr}}{4 \pi r}\right) \\
								   &= -\frac{1}{4\pi}\left[\frac{2ike^{ikr}}{r^2} - \frac{k^2 e^{ikr}}{r} +
									   e^{ikr}(-4\pi \delta^3(r)) + 2\left( -\frac{\hat{r}}{r^2} \right) \cdot 
							   ike^{ikr} \hat{r}\right] - k^2\left( \frac{e^{ikr}}{4\pi r} \right) \\
								   &= -\frac{e^{ikr}}{4\pi}\left[\left(\frac{2ik}{r^2} - \frac{k^2}{r}\right) - 4\pi \delta^3(r) - \frac{2ik}{r^2}\right] - k^2\left( \frac{e^{ikr}}{2\pi r} \right)  \\
								   &= \frac{e^{ikr}}{4\pi r}k^2 + e^{ikr} \delta^3(r) - k^2 \frac{e^{ikr}}{4\pi r}\\
								   &= e^{ikr} \delta^3(r) \\
								   &= \delta^3(r)
		\end{align*}
		This last step we use the fact that $e^{ikr} \delta^3(r) = \delta^3(r)$. This result comes from the fact that $e^{ikr}$ doesn't blow up quickly at any point, so we can safely swallow $e^{ikr}$ into the delta function. 

		Now, since this result equals the right hand side, then the solution satisfies the differential equation.
	\end{solution}
	\pagebreak
	\section*{Problem 2}
	Show that combining $f_{\vec k} (\hat{r}) = -\frac{m}{2\pi \hbar^2}\int e^{-i k \hat{r} \cdot \vec r'} V(r)
	\psi_{\vec k} (r') d^3r$ with $\psi_{\vec k}(\vec r) = e^{-i \vec k \cdot \vec r} + f_{\vec k}(\hat{r})
	\frac{e^{ikr}}{r}$ leads to the approximation in the limit of weak scattering.

	\begin{solution}
		Plugging in the desired $\psi_{\vec k}$, we get:
		\begin{align*}
			f_{\vec k}(\vec r) &= -\frac{m}{2 \pi \hbar^2}\int e^{-i k \hat{r} \cdot \vec{r}} V(r) \left( 
				e^{i \vec k \cdot \vec r'} + f_{\vec k}(r') \frac{e^{ikr'}}{r'}\right) d^3r'
		\end{align*}
		In the weak born approximation, we expect that the scattering should be small, so just like in lecture,
		the second term becomes negligible, and the scattering is predominantly dependent on the first term. 
		We now focus on the first term: 
		\begin{align*}
			f_{\vec k}(\hat{r}) &= -\frac{m}{2\pi \hbar^2}\int e^{-i \vec k \hat{r} \cdot \vec r'} V(r) 
			e^{i \vec k \cdot \vec{r'}} d^3r'\\
			\therefore |f_{\vec k}(\hat{r})|^2 &= \frac{m^2}{4\pi^2 \hbar^4}\left| \int e^{-i \vec k \hat{ r}
			\cdot \vec{r'}} V(r) e^{i \vec k \cdot \vec r'} d^3r'\right|^2
		\end{align*} 
		Notice the similarities between this and the Born approximation:
		\[
		\frac{m^2}{4\pi^2 \hbar^4}\left|\int e^{-i \vec k_f \cdot \vec r} V(r) e^{i \vec k_i \cdot \vec r}
		d^3r\right|^2
		\] 
		So really, all we need to argue is that $\vec k \hat{r} \cdot \vec{r'} = \vec k_f$, and we have solved 
		the problem since the second exponential term already matches the born approximation. This is done 
		by noticing that since $\hat{r}$ points toward the direction of the detector, then it's immediately 
		true that $k \hat{r} = \vec k_f$, and so the Born approximation follows directly as a result. 
	\end{solution}


	\pagebreak

	\section*{Problem 3}

	A molecule of a homonuclear diatomic gas may roughly be regarded as composed ot two identical spherically 
	symmetric scattering centers separated by a (vectorial) distance $\mathbf l$ (Fig. 8.15). If the scattering
	amplitude for a certiain kind of particle directed against the atom is known, what scattering cross section
	is measured for the molecules of the gas? Contrast the atomic and molecular cross sections, and neglect 
	effects of multiple scattering, that is, of particles which bounce back and forth between the two centers.

	\begin{solution}
		We use the result from the hint: 
		\[
			\dv{\sigma}{\Omega} = 4 \cos^2\left[\frac{1}{2}(\mathbf k - \mathbf k') \cdot \mathbf l\right]
			|f(\theta)|^2
		\] 
		Evaluating the dot product, we get:
		\[
			d\sigma = 4\cos^2\left[\frac{1}{2}kl\sin \frac{\theta}{2}\cos \theta'\right]|f(\theta)|^2 d\Omega
		\] 
		So now we need to integrate $d\sigma$ to get the cross section. Using the fact that $d\Omega = \sin
		\theta' d\theta' d\phi$, we get:
		\begin{align*}
			\sigma &= \int 4\cos^2\left[A(\theta) \cos \theta'\right] |f(\theta)|^2 \sin \theta' d\theta' d\phi\\
				   &= 16 \pi |f(\theta)|^2 \int \cos^2 \left[ A(\theta) \cos \theta'\right] \sin \theta' 
				   d\theta' \\
				   &= 16\pi |f(\theta)|^2 \left[ \frac{\sin (kl \sin \frac{\theta}{2}) 
					   \cos(kl \sin \frac{\theta}{2})}{kl \sin \frac{\theta}{2}} + 1\right] \\
				   &= 8\pi |f(\theta)|^2 \left[\frac{\sin (2kl \sin \frac{\theta}{2})}{kl 
				   \sin \frac{\theta}{2}} + 2\right] 
		\end{align*}
		We then need to average over the angular cross section which is $4\pi$, so we finally get: 
		\[
			\sigma_{\text{avg}} = 2|f(\theta)|^2\left[ \frac{\sin 2kl \sin \frac{\theta}{2}}{kl 
			\sin \frac{\theta}{2} } + 2\right]
		\] 
		For a single atom, we just integrate $|f(\theta)|^2$ over the same region: 
		\[
			\sigma_{\text{atom}} = 4\pi |f(\theta)|^2
		\] 
		therefore, we can see that the scattering cross section for the diatomic gas is larger than that of the
		atom in most cases, unless the scattering angle is $0$, in which the scattering cross section of the atom
		would be greater. This intuitively makes sense as well -- we can think of the diatomic gas as a larger 
		target, so therefore any scattering of the diatomic gas should naturally be larger than a smaller
		target like an atom.
	\end{solution}
\end{document}
